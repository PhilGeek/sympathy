%!TEX root = /Users/markelikalderon/Documents/Git/sympathy/perception.tex
\chapter{Sources of Sound} % (fold)
\label{cha:sources_of_sound}

\section{The Heideggerian Alternative} % (fold)
\label{sec:the_heideggerian_alternative}

On one understanding, the source of sound may be a body that possesses the power of sounding, that is, the power to engage in a sound-generating activity. On another understanding, the source of sound is simply the sound-generating activity. Audible sources, the sources disclosed in auditory experience, are sound-generating events or processes. It is the audible activities of bodies that we hear and not the bodies themselves (though perhaps we may attend to these since their audible activities constitute a dynamic aural image of them). We hear sounds and their sources. What role do sounds play in affording the perceiver with auditory awareness of their sources?

The neo-Berkelean has an answer ready to hand that many find difficult to resist. We hear the sources of sound by hearing the sounds they generate. We hear a body's audible activity by hearing the sound that activity generates. Hearing sounds afford the perceiver with auditory awareness of their sources since the immediate presentation of sound in auditory experience constitutes, in a manner yet to be explained, the mediate presentation of its source.

I do not accept the neo-Berkelean response. I believe that its central claims are at odds with the phenomenology of auditory experience. Instead, I shall refine and elaborate a Heideggerian account of the role that sounds play in affording the perceiver with auditory awareness of distal sources (see \citealt{Leddington:2014aa} for a different defence of the Heideggerian alternative and the replies by \citealt{OCallaghan:2014aa} and \citealt{Nudds:2014aa}). We do not hear sources by hearing sounds as the neo-Berkelean would have it. Rather we hear the sources of sound directly. In cases where a perceiver can hear the source of a sound, the call of a feral parrot, say, they are explicitly aware of the call and only implicitly aware of its sound. The application of Fulkerson's \citeyearpar{Fulkerson:2014ek} distinction between explicit and implicit awareness to Heidegger's \citeyearpar{Heidegger:1935uq} observation about audition is the first of the refinements. There is a sense in which we hear a source through, or in \citep{Leddington:2014aa}, the sound it produces. The sound they hear is a perceptual medium through which the audible activities of distal bodies are disclosed. And sympathy is the principle that makes possible the presentation of sources in auditory experience through the perceptual medium of sound. On the refined and elaborated Heideggerian account, the role of sounds in affording the perceiver with auditory awareness of distal sources is limited to being an audible media through which or in which their sources may be heard.

% Hearing the sources of sound is a special, if privileged, case of audition. We may hear a sound and struggle to identify its source, say. So explaining the presentation in audition of the source of a sound in terms of the principle of sympathy is not yet to offer a general explanation of auditory presentation in terms of sympathy. If however a sound that we hear is segmented from all that we hear by identifying, in audition, its source, then the presentation of sources in audition will determine the sounds that we hear, and so sympathy is operative in auditory presentation quite generally. Heidegger was right to emphasize the difficulty in listening away from the sources of sound. If one can only hear a sound by being explicitly aware of its source, then listening away from the source would involve the loss at least of the ability to segment that sound from all that was heard.

% section the_heideggerian_alternative (end)

\section{The Function of Audition} % (fold)
\label{sec:the_function_of_audition}


Begin with two claims recently defended by Nudds:
\begin{enumerate}[(1)]
	\item The function of auditory perception is to afford the perceiver with awareness of the distal sources of sound;
	\item In hearing a sound in a complex sonic environment with multiple active sources of sound, the sound the perceiver hears is segmented from all that they hear due, in part, to their auditory system identifying its source.
\end{enumerate}

Concerning the first claim Nudds writes:
\begin{quote}
	It is uncontroversial to suggest that auditory perception tells us about the sources of sounds as well as about sounds. The suggestion that I am going to develop is that the function of auditory perception is to tell us about the sources of sounds---that perceiving the sources of sounds is what auditory perception is for and that what sounds we hear we hear as a consequence of the particular way auditory perception functions to tell us about the sources of sounds. \citep[284]{Nudds:2010aa}
\end{quote}
The function of auditory perception is to afford the perceiver with awareness of the distal sources of sound. This is a teleological claim. The end of auditory perception, that for the sake of which perceivers are equipped with audition, is the presentation, in audition, of distal events in the natural environment. It is also an explanatory claim. The operation of audition adequate to its function constitutes an explanatorily relevant kind. It is also objective. The operation of audition adequate to its function constitutes an explanatorily relevant kind independently of whether anyone accepts that it does. Moreover, this objective, teleological, explanatory claim, seems naturalistically acceptable. It is, at any rate, overwhelmingly plausible to suppose that an animal's ability to hear distal events in the natural environment contributes to its fitness.

Nudds' claim about the function of audition generates a tension within the Peripatetic framework. Consider the following two claims about the proper sensibles:
\begin{enumerate}[(1)]
	\item Proper sensibles are perceptible to one sensory modality alone (for example, one can see colors, but not hear, smell, taste, or touch them)
	\item Proper sensibles are the final cause of perception (for example, sight is for the sake of seeing colors in the light and the luminous in the dark)
\end{enumerate}
The difficulty is that, at least in the case of audition, these two claims cannot be true together.

Consider the second claim first, that the proper sensibles are the final cause of perception. The proper object of sight is the visible (\emph{De Anima} 2 7 418\( ^{b} \)27) and there are two kinds of \emph{visibilia}, color which is visible in light and the luminous, such as bioluminescence or starlight, visible only in the dark (\emph{De Anima} 2 7 419\( ^{a} \)1--7; the nice example of starlight is due to Philoponus  \emph{On \emph{De Anima}} 347 11). If sight is for the sake of seeing colors in the light and the luminous in the dark (\emph{Metaphysica} \( \Theta \) 8 1050\( ^{a} \)10), then is audition for the sake of hearing sounds? Nudds denies this, claiming, instead, that the function of audition is to afford the perceiver with awareness of distal events in the natural environment. Suppose, then, that audition is for the sake of hearing distal sources. Arguably it is that in which audition's selective advantage lies. Hearing sounds would be incidental to audition, so conceived, at least relative to its end, even if one can only ever hear sources through, or in, the sounds they generate. The difficulty is that the final cause, the distal sources, are perceptible to more than one sense alone. Thus one might hear one of London's feral parrots calling as one sees that parrot calling. So there is no one thing that is audible yet perceptible to no other sensory modality and that for the sake of which we possess audition. (1) and (2) are generalizations that fail for the case of audition if we accept Nudds' claim. Perhaps sounds are audible and perceptible to one sense alone and so would make (1) true but (2) would fail---audition is not for the sake of hearing sounds but their sources. The audible sources of sound would make (2) true but (1) would fail---audible sources may be available to more than one sense.

That (1) and (2) fail to be jointly true of audition signals the breakdown of the guiding explanatory framework of \emph{De Anima} 2:
\begin{quote}
	It is necessary for the student of these forms of soul first to find a definition of each, expressive of what it is, and then to investigate its derivative properties, \&c. But if we are to express what each is, viz. what the thinking power is, or the perceptive, or the nutritive, we must go farther back and first give an account of thinking or perceiving; for activities and actions are prior in definition to potentialities. If so, and if, still prior to them, we should have reflected on their correlative objects, then for the same reason we must first determine about them, i.e. about food and the objects of perception and thought. (Aristotle, \emph{De Anima} 2 4 415\( ^{a} \)14--22\index{De Anima@\emph{De Anima}}; Smith in \citealt[26]{Barnes:1984uq})
\end{quote}
Aristotle's explanatory strategy has two parts. First, Aristotle proposes to explain perceptual capacities in terms of what they are the capacity for, perceiving. Specifically, perceptual activity is prior in account to the potential for such activity, the relevant perceptual capacity. Second, perceptual activities, the exercise of our perceptual capacities, are themselves to be partly explained in terms of their correlative objects. It will emerge that, at least with respect to perception, Aristotle means, more specifically, proper objects, understood as sensible objects perceptible in themselves and perceptible to that sense alone. 

Audition poses a challenge to the second part of Aristotle's explanatory strategy. The first part of Aristotle's explanatory strategy is simply motivated by the idea that potentialities are individuated by what, for the sake of which, they are the potential for. It is not that idea that is threatened so much as the idea that the relevant perceptual activity must be understood in terms of the presentation of a sense object proper to that sensory modality. It is not the thought that the perceptual activities that are the exercise of perceptual capacities are prior in account to these capacities that is challenged. Rather it is the Platonic idea that perceptual capacities should be understood in terms of the presentation of an object available through the exercise of that capacity alone (\emph{Theaetetus} 184 e 8--185 a 3). 

Thinking of perceptual capacities as individuated by that for the sake of which they are a potential for, allows Aristotle to think that there are exercises of our perceptual capacities that are not the presentation of the proper sensibles, notably, when they are the presentation of common or incidental sensibles. And more besides---the difference between proper objects such as color and sound are perceptible as well. In this way, Aristotle broadens the domain of the perceptible \citep{Sorabji:1971fr,Sorabji:2003fk,Kalderon:2015fr}. Sight may enable a perceiver to see colors in the light and the luminous in the dark, but it enables the perceiver to see other \emph{visibilia} such as motion, a common sensible. But the presentation of motion in sight is incidental to its operation. Sight is for the sake of seeing colors in the light and the luminous in the dark. But in continuing to understand the presentation of proper sensibles as that for the sake of which a perceiver possesses the relevant perceptual capacity, Aristotle cleaves too closely to the Platonic tradition undone by audition whose function is to afford the perceiver awareness of distal events in the natural environment that are perceptually available to other sensory modalities, such as a storm whistling in the chimney or the call of a feral parrot.

% section the_function_of_audition (end)

\section{Sources and the Discrimination of Sound} % (fold)
\label{sec:sources_and_the_discrimination_of_sound}

Audition is for the sake of hearing the sources of sounds. Hearing sound is incidental to audition, relative to its end, even if one can only ever hear sources through, or in, the sounds they generate. Perhaps, in certain contexts, one may even say that one hears the source by hearing the sound, but only in a sense unavailable to the neo-Berkelean (Nudds sometimes writes this way). In Peripatetic terminology, this is an instance of hypothetical necessity (\emph{Physica} 2 9; for useful discussion see \citealt{Charles:1988as}). The necessity is hypothetical, since the end of audition, to hear the sources of sounds, is presupposed. Given the end of audition, to hear the distal sources of sounds, it is necessary to hear the sounds that they generate.

Recall, according to the neo-Berkelean, sounds are distinguished from their sources in auditory experience in that only the former are the immediate objects of audition and that we hear the latter by hearing the former. For the neo-Berkelean, the preposition ``by'' is a place holder for the presentative function of sounds, their presenting their sources in presenting themselves in audition. What is this presentative function? One of the lessons we learned from Heidegger was that there is more to the presentative function of sound than the sources we hear being necessarily accompanied, in audition, by their sounds. Typically, neo-Berkeleans are no more forthcoming than sense-datum theorists were in giving an account of this presentative function.

According to the neo-Berkelean, sounds are the immediate objects of audition in something like the following sense. Sounds are audible. Moreover, sounds are audible in themselves. Sounds contain within themselves the power of their own audibility. So one can hear a sound without hearing any other thing.  In this sense are they at least among the immediate objects of audition. Though, of course, neo-Berkeleans typically follow Berkeley in maintaining, as well, that sounds are the only immediate objects of audition, even if they do not go so far as Berkeley in maintaining that sounds are the sole objects of audition. \emph{Pace} Berkeley, sources too are audible. However, they are not audible in themselves, but are only audible by hearing other objects that are audible in themselves, namely the sounds they generate. In this sense are they the mediate objects of audition.

So understood, sound could not be the immediate object of audition. Bracket, for the moment, worries about the, as of yet, unexplained presentative function of sound in auditory experience. Focus, instead, on the claim that sounds are audible in themselves, that sounds contain within themselves the power of their own audibility with the implication that hearing a sound does not require hearing any other audible object. Nudds' second claim, if true, suffices to establish that sounds are not audible in themselves. Specifically, Nudds claims that in hearing a sound in a complex sonic environment with multiple active sources of sound, the sound the perceiver hears is segmented from all that they hear due, in part, to their auditory system identifying its source. And, as we shall see, that is inconsistent with sounds being audible in themselves. If anything, something like the reverse is true. Sounds are audible, but not audible in themselves, but audible only insofar as one hears the sources that generate them. There is then, I shall suggest, a sense in which sounds are best thought of as audible media through which or in which sources may be heard. At the very lest, if sounds were audible media, they would be audible, but not audible in themselves, but owing their audibility to other things---the sources heard through or in the sounds.

% section sources_and_the_discrimination_of_sound (end)

\section{Sympathy and Auditory Presentation} % (fold)
\label{sec:sympathy_and_auditory_presentation}

% section sympathy_and_auditory_presentation (end)

% chapter sources_of_sound (end)
