%!TEX root = /Users/markelikalderon/Documents/Git/sympathy/perception.tex
\chapter{Sources of Sound} % (fold)
\label{cha:sources_of_sound}

\section{The Heideggerian Alternative} % (fold)
\label{sec:the_heideggerian_alternative}

On one understanding, the source of a sound may be a body that possesses the power of sounding, that is, the power to engage in a sound-generating activity. On another understanding, the source of a sound is simply the sound-generating activity, the event or process that generates the sound. Indeed, there may be sound-generating events or processes not involving the activity of bodies (though for the most part we shall ignore this possibility). Audible sources, the sources disclosed in auditory experience, are sound-generating events or processes. It is the audible activities of bodies, or at least sound-generating events or processes, that we hear and not the bodies themselves (though perhaps we may attend to these on the supposition that their audible activities constitute a dynamic aural image of them, chapter~\ref{sec:sounds_and_their_sources}). We hear sounds and their sources. What role do sounds play in affording the perceiver with auditory awareness of their sources?

The neo-Berkelean has an answer, ready to hand, that many find nearly irresistable. We hear the sources of sound by hearing the sounds they generate. We hear a body's audible activity by hearing the sound that activity generates. Hearing sounds afford the perceiver with auditory awareness of their sources since the immediate presentation of the sound in auditory experience constitutes, in a manner yet to be explained, the mediate presentation of its source.

I do not accept the neo-Berkelean answer. I believe that its central claims are at odds with the phenomenology of auditory experience. Instead, I shall refine, and elaborate, a Heideggerian account of the role that sounds play in affording the perceiver with auditory awareness of their distal sources (see \citealt{Leddington:2014aa} for a different defense of the Heideggerian alternative and the replies by \citealt{OCallaghan:2014aa} and \citealt{Nudds:2014aa}). We do not hear sources by hearing their sounds as the neo-Berkelean would have it. Rather we hear the sources of sound directly. In cases where a perceiver can hear the source of a sound, the call of a feral parrot, say, they are explicitly aware of the call and only implicitly aware of its sound. The application of Fulkerson's \citeyearpar{Fulkerson:2014ek} distinction between explicit and implicit awareness to Heidegger's \citeyearpar{Heidegger:1935uq} observation about audition is the first of the refinements. There is a sense in which we hear a source through, or in \citep{Leddington:2014aa}, the sound it produces. The sound they hear is a perceptual medium through which the audible activities of distal bodies are disclosed. And sympathy is the principle that makes possible the presentation of sources in auditory experience through the perceptual medium of sound. On the refined and elaborated Heideggerian account, the role of sounds in affording the perceiver with auditory awareness of distal sources is limited to being an audible media through which or in which their sources may be heard.

% Hearing the sources of sound is a special, if privileged, case of audition. We may hear a sound and struggle to identify its source, say. So explaining the presentation in audition of the source of a sound in terms of the principle of sympathy is not yet to offer a general explanation of auditory presentation in terms of sympathy. If however a sound that we hear is segmented from all that we hear by identifying, in audition, its source, then the presentation of sources in audition will determine the sounds that we hear, and so sympathy is operative in auditory presentation quite generally. Heidegger was right to emphasize the difficulty in listening away from the sources of sound. If one can only hear a sound by being explicitly aware of its source, then listening away from the source would involve the loss at least of the ability to segment that sound from all that was heard.

% section the_heideggerian_alternative (end)

\section{The Function of Audition} % (fold)
\label{sec:the_function_of_audition}


Begin with two claims recently defended by Nudds:
\begin{enumerate}[(1)]
	\item The function of auditory perception is to afford the perceiver with awareness of the distal sources of sound;
	\item In hearing a sound in a complex sonic environment with multiple active sources of sound, the sound the perceiver hears is segmented from all that they hear due, in part, to their auditory system identifying its source.
\end{enumerate}

Concerning the first claim Nudds writes:
\begin{quote}
	It is uncontroversial to suggest that auditory perception tells us about the sources of sounds as well as about sounds. The suggestion that I am going to develop is that the function of auditory perception is to tell us about the sources of sounds---that perceiving the sources of sounds is what auditory perception is for and that what sounds we hear we hear as a consequence of the particular way auditory perception functions to tell us about the sources of sounds. \citep[284]{Nudds:2010aa}
\end{quote}
The function of auditory perception is to afford the perceiver with awareness of the distal sources of sound. This is a teleological claim. The end of auditory perception, that for the sake of which perceivers are equipped with audition, is the presentation, in audition, of distal events in the natural environment. It is also an explanatory claim. The operation of audition adequate to its function constitutes an explanatorily relevant kind. It is also objective. The operation of audition adequate to its function constitutes an explanatorily relevant kind independently of whether anyone accepts that it does. Moreover, this objective, teleological, explanatory claim, seems naturalistically acceptable. It is, at any rate, overwhelmingly plausible to suppose that an animal's ability to hear distal events in the natural environment contributes to its fitness. And if that is right, that the function of audition is to present distal events in the natural environment is plausibly determined by evolutionary pressures.

Nudds' claim about the function of audition generates a tension within the Peripatetic framework. Consider the following two claims about the proper sensibles:
\begin{enumerate}[(1)]
	\item Proper sensibles are perceptible to one sensory modality alone (for example, one can see colors, but not hear, smell, taste, or touch them)
	\item Proper sensibles are the final cause of perception (for example, sight is for the sake of seeing colors in the light and the luminous in the dark)
\end{enumerate}
The difficulty is that, at least in the case of audition, these two claims cannot be true together.

Consider the second claim first, that the proper sensibles are the final cause of perception. The proper object of sight is the visible (\emph{De Anima} 2 7 418\( ^{b} \)27) and there are two kinds of \emph{visibilia}, color which is visible in light and the luminous, such as bioluminescence or starlight, visible only in the dark (\emph{De Anima} 2 7 419\( ^{a} \)1--7; the nice example of starlight is due to Philoponus  \emph{On \emph{De Anima}} 347 11). If sight is for the sake of seeing colors in the light and the luminous in the dark (\emph{Metaphysica} \( \Theta \) 8 1050\( ^{a} \)10), then is audition for the sake of hearing sounds? Nudds denies this, claiming, instead, that the function of audition is to afford the perceiver with awareness of distal events in the natural environment. Suppose, then, that audition is for the sake of hearing distal sources. Arguably it is that in which audition's selective advantage lies. Hearing sounds would be incidental to audition, so conceived, at least relative to its end, even if one can only ever hear sources through, or in, the sounds they generate. The difficulty is that the final cause, the distal sources, are perceptible to more than one sense alone. Thus one might hear one of London's feral parrots calling as one sees that parrot calling. So there is no one thing that is audible yet perceptible to no other sensory modality and that for the sake of which we possess audition. (1) and (2) are generalizations that fail for the case of audition if we accept Nudds' claim. Perhaps sounds are audible and perceptible to one sense alone and so would make (1) true but (2) would fail---audition is not for the sake of hearing sounds but their sources. The audible sources of sound would make (2) true but (1) would fail---audible sources may be available to more than one sense.

That (1) and (2) fail to be jointly true of audition signals the breakdown of the guiding explanatory framework of \emph{De Anima} 2:
\begin{quote}
	It is necessary for the student of these forms of soul first to find a definition of each, expressive of what it is, and then to investigate its derivative properties, \&c. But if we are to express what each is, viz. what the thinking power is, or the perceptive, or the nutritive, we must go farther back and first give an account of thinking or perceiving; for activities and actions are prior in definition to potentialities. If so, and if, still prior to them, we should have reflected on their correlative objects, then for the same reason we must first determine about them, i.e. about food and the objects of perception and thought. (Aristotle, \emph{De Anima} 2 4 415\( ^{a} \)14--22\index{De Anima@\emph{De Anima}}; Smith in \citealt[26]{Barnes:1984uq})
\end{quote}
Aristotle's explanatory strategy has two parts. 

First, Aristotle proposes to explain perceptual capacities in terms of what they are the capacity for, perceiving.  Specifically, perceptual activity is prior in account to the potential for such activity, the relevant perceptual capacity. Possessing a capacity is a way for things to be, and what it is to be that way depends upon what it is to be its exercise. So possessing audition is a way at least animals to be, and what it is to be that way depends upon what it is to hear. Thus, if capacities are powers or potentialities, as Aristotle conceives of them, then they ontologically depend upon what they are the potential for. (On ontological dependence see \citealt{Fine:1995ls}. On this reading of priority in account see \citealt{Peramatzis:2011aa}. For a contemporary defense of this claim see \citealt{Kalderon:2012fk}.) 

Second, perceptual activities, the exercise of our perceptual capacities, are themselves partly explained in terms of their correlative objects. It will emerge that, at least with respect to perception, Aristotle means, more specifically, proper objects, understood as sensible objects perceptible in themselves and perceptible to that sense alone. So what it is to hear depends upon the presentation, in auditory experience, of the proper object of audition, sound. Crucially, that is consistent with auditory experience presenting more than just sound.

Thinking of perceptual capacities as individuated by that for the sake of which they are a potential for, allows Aristotle to think that there are exercises of our perceptual capacities that are not the presentation of the proper sensibles, notably, when they are the presentation of common or incidental sensibles. And more besides---the difference between proper objects such as color and sound are perceptible as well. In this way, Aristotle broadens the domain of the perceptible \citep{Sorabji:1971fr,Sorabji:2003fk,Kalderon:2015fr}. Sight may enable a perceiver to see colors in the light and the luminous in the dark, but it enables the perceiver to see other \emph{visibilia} such as motion, a common sensible. But the presentation of motion in sight is incidental to its operation. Sight is for the sake of seeing colors in the light and the luminous in the dark. But in continuing to understand the presentation of proper sensibles as that for the sake of which a perceiver possesses the relevant perceptual capacity, Aristotle cleaves too closely to the Platonic tradition undone by audition whose function is to afford the perceiver awareness of distal events in the natural environment that are perceptually available to other sensory modalities, such as a storm whistling in the chimney or the call of a feral parrot.

% section the_function_of_audition (end)

\section{Sources and the Discrimination of Sound} % (fold)
\label{sec:sources_and_the_discrimination_of_sound}

Audition is for the sake of hearing the sources of sounds, understood as sound-generating events or processes. If audition is for the sake of hearing, not sounds, but their sources, then hearing sound is incidental to audition, relative to its end, even if one can only ever hear sources through, or in, the sounds they generate. Perhaps, in certain contexts, one may even say that one hears a source by hearing its sound, but only in a sense unavailable to the neo-Berkelean (Nudds sometimes writes this way). In Peripatetic terminology, this is an instance of hypothetical necessity (\emph{Physica} 2 9; for useful discussion see \citealt{Charles:1988as}). The necessity is hypothetical, since the end of audition, to hear the sources of sounds, is presupposed. Given the end of audition, to hear the distal sources of sounds, it is necessary to hear the sounds that they generate.

Recall, according to the neo-Berkelean, sounds are distinguished from their sources in auditory experience in that only the former are the immediate objects of audition and that we hear the latter by hearing the former. For the neo-Berkelean, the preposition ``by'' is a place holder for the presentative function of sounds, their presenting their sources in presenting themselves in audition. What is this presentative function? One of the lessons we learned from Heidegger was that there is more to the presentative function of sound than the sources we hear being necessarily accompanied, in audition, by their sounds, since sounds may lack this presentative function and remain a necessary accompaniment of the sources that we hear. Moreover, the mediate presentation of sources by the immediate presentation of their sounds is unlike more ordinary cases of perceiving one thing by perceiving another, so in what does this extraordinary case consist? Typically, neo-Berkeleans are no more forthcoming than sense-datum theorists were in giving an account of this presentative function.

According to the neo-Berkelean, sounds are the immediate objects of audition in something like the following sense. Sounds are audible. Moreover, sounds are audible in themselves. Sounds are audible in themselves in the sense that they contain within themselves the power of their own audibility. So one can hear a sound without hearing any other thing. In this sense are they at least among the immediate objects of audition. Though, of course, neo-Berkeleans typically follow Berkeley in maintaining, as well, that sounds are the only immediate objects of audition. Sounds alone have within themselves the power of their own audibility, even if neo-Berkeleans do not go so far as Berkeley in maintaining that sounds are the only objects of audition. \emph{Pace} Berkeley, sources too are audible. However, they are not audible in themselves. They do not contain within themselves the power of their own audibility but are only audible by hearing other objects that are audible in themselves, namely the sounds that they generate. In this sense are they the mediate objects of audition.

So understood, sound could not be the immediate object of audition. Bracket, for the moment, worries about the, as of yet, unexplained presentative function of sound in auditory experience. Focus, instead, on the claim that sounds are audible in themselves, that sounds contain within themselves the power of their own audibility with the implication that hearing a sound does not require hearing any other audible object. Nudds' second claim, if true, suffices to establish that sounds are not audible in themselves in the way that the neo-Berkelean requires. Specifically, Nudds claims that in hearing a sound in a complex sonic environment with multiple active sources of sound, the sound the perceiver hears is segmented from all that they hear due, in part, to their auditory system identifying its source. And, as we shall see, that is inconsistent with sounds being audible in themselves. If anything, something like the reverse is true. Sounds are audible, but not audible in themselves, but audible only insofar as one hears the sources that generate them. There is then, I shall suggest, a sense in which sounds are better thought of as audible media through which, or in which, sources may be heard. At the very lest, if sounds were audible media, they would be audible, but not audible in themselves, but owing their audibility to other things---the sources heard through, or in, the sounds.

Audition, like vision and tactile perception, involves grouping, segmentation, and recognition \citep{Bregman:1990aa}. When, upon the hill in Greenwich Park near the Royal Observatory, I witnessed the Ballardian spectacle of feral parrots traversing the skyscrapers of the City of London, the call of the feral parrots was not all that I heard. I could hear, as well, the trees rustling in the light breeze, the occasional shouts of children playing, people conversing, a bicycle braking, dogs barking. Like most public spaces, Greenwich Park is a complex sonic environment with multiple active sources of sound and the call of the feral parrots was not all that there was to hear. The patterned disturbance reaching my ears was not solely caused by the parrots calling. And yet I could hear it clearly.

A patterned disturbance, occurring in a given temporal interval, can be analyzed into frequency components, component sine waves of a given frequency and amplitude. When longitudinal pressure waves superimpose, their frequency components additively combine to produce a new complex pressure wave. Given the detected frequency components of the complex pressure wave are not solely caused by the call of the feral parrot, how does my auditory system afford me the capacity to hear the sound of the feral parrot? The auditory system would need to somehow group together the frequency components that constitute the sound of the feral parrot's call.

According to \citet{Nudds:2009sf,Nudds:2010aa}, the auditory system groups frequency components by exploiting clues as to the likely source of the sound. There are a variety of different such clues, and many can be dominated by other clues. 

Some clues are synchronic. That is, sometimes frequency components occurring at a time are related in such a way that it is unlikely that they are the products of distinct sources. For example, the vibration of a material object will determine frequency components of the patterned disturbance that are harmonically related to a fundamental frequency. So there is a tendency for the auditory system to group together frequency components at a time that are harmonically related since it is unlikely that they are produced by distinct sources. 

Some clues are diachronic. That is, sometimes frequency components occurring over time are related in such a way that it is unlikely that they are the products of distinct sources. Thus, for example, the frequency components of a sound produced by a source will have the same onset time, and they will change over time in similar ways. So there is a tendency for the auditory system to group together frequency components that are diachronically related in important ways since it is unlikely that they are produced by distinct sources. 

\citet[74]{Nudds:2009sf} observes that while the clues discussed so far are ``bottom-up'' or stimulus driven groupings, there are, as well ``top-down'' groupings, especially of sequences of frequency components. The idea is that certain frequency components are grouped together because they fit together to form a pattern recognized by the auditory system to likely be produced by a single source. Thus, for example, one might hear a bottle bouncing, as opposed to breaking, and this is likely to be due to such a top-down grouping.

Notice how the clues to grouping together frequency components constituting a heard sound are all based on features of its material source. Thus, for example, the size of an  object will determine the lowest frequency at which it will vibrate. This allows us to hear that one object dropped is larger than another object that is also dropped. How exactly the auditory system extracts information about the material source and what information it extracts from the grouped frequency components is not well understood. 

However, exactly, the auditory perception performs this feat, the important point is that in a sonically complex environment with multiple active sources of sound, an individual sound is segmented from all that is heard, in part, by identifying its source. If a likely source is not identified by the auditory system, then the frequency components will not be grouped together and the sound will not be segmented from all that is heard. If in hearing the products of multiple active sources of sound, none of the sources are discriminated, all that would be heard is a kind of undifferentiated noise. Hearing the sources of sound lends intelligibility to what is heard. 

% section sources_and_the_discrimination_of_sound (end)

\section{Sympathy and Auditory Presentation} % (fold)
\label{sec:sympathy_and_auditory_presentation}

What does it mean to describe sound as perceptual media? Just as illumination makes the visible perceptually accessible, sound makes the activities of distal bodies perceptually accessible. Without illumination, the colors of distal bodies remain unseen, without sound, the activities of distal bodies remain unheard. One sees though, or in, illuminated media, such as air or water, and thereby perceives the colors of distal bodies arrayed in the natural environment. One hears through, or in, audible media, the sound, and thereby perceives the activities of distal bodies arrayed in the natural environment.

By means of the propagation of light waves, the visible aspects of distal bodies are seen. By means of the propagation, in all directions, of the patterned disturbance through a dense and imperfectly elastic medium, that is, by means of sound, the audible activities of distal bodies are heard.

Sound, like the illuminant, is perceptible. Moreover, sound, like the illuminant, is perceptible in a certain way. Concerning the perception of the illuminant, Hilbert writes:
\begin{quote}
	Do we see how an object is illuminated or do we see the illumination itself? On phenomenological grounds the first option seems better to me. What we see as changing with the illumination is an aspect of the object itself, not the light source or the space surrounding the object. \citep[150--151]{Hilbert:2007qy}
\end{quote}
One sees the character of the illumination by seeing the way objects are illuminated. When viewing a brightly lit pantry, one sees the brightness of the pantry by seeing the brightly lit objects arranged in it. So the illuminant is visible, though not visible in itself, but owes its visibility to the objects that it illuminates. (For a comparison with Aristotle's definition of transparency, \emph{De Anima} \textsc{ii} 7 418\( ^{b} \)4--6, see \citealt[41--42]{Kalderon:2015fr}.)

Like the illuminant, sound is perceptible, though perceptible in a certain way. One hears the character of a sound by hearing the activities of its distal source. (Think of how difficult it is to describe ecological sound without describing audible aspects of its source.) So sound is audible, though not audible in itself, but owes its audibility to the distal sources it discloses.

\begin{quotation}
	The game is this. Your friend digs two narrow channels up from the side of the lake. Each is a few feet long and a few inches wide and they are spaced a few feet apart. Halfway up each one, your friend stretches a handkerchief and fastens it to the side of the channel. As waves reach the side of the lake they travel up the channels and cause the two handkerchiefs to go into motion. You are allowed to look only at the handkerchiefs and from their motions to answer a series of questions: How many boats are there on the lake and where are they? Which is the most powerful one? Which is the closer? Is the wind blowing? Has any large object been dropped suddenly into the lake?
	Solving this problem seems impossible, but it is a strict analogy to the problem faced by our auditory systems. The lake represents the lake of air that surrounds us. The two channels are our two ear canals, and the handkerchiefs are our ear drums. The only information that the auditory system has available to it, or ever will have, is the vibrations of these two ear drums. Yet it seems able to answer questions very like the ones that were asked by the side of the lake: How many people are talking? Which one is louder, or closer? Is there a machine humming in the background (Bregman 1990, pp. 5-6).
\end{quotation}

% section sympathy_and_auditory_presentation (end)

% chapter sources_of_sound (end)
