%!TEX TS-program = xelatex 

%!TEX encoding = UTF-8 Unicode
%
%  proposal
%
%  Created by Mark Eli Kalderon on 2015-09-30.
%  Copyright (c) 2015. All rights reserved.
%

\documentclass[12pt]{article} 

% Definitions
\newcommand\mykeywords{perception, sympathy, presence}
\newcommand\myauthor{Mark Eli Kalderon}

% Packages
\usepackage{geometry} \geometry{a4paper}

% XeTeX
\usepackage[cm-default]{fontspec}
\usepackage{xltxtra,xunicode}
\defaultfontfeatures{Scale=MatchLowercase,Mapping=tex-text}
\setmainfont{Hoefler Text}

% PDF Stuff
\usepackage[plainpages=false, pdfpagelabels, bookmarksnumbered, backref, pdftitle={Form Without Matter}, pagebackref, pdfauthor={\myauthor}, pdfkeywords={\mykeywords}, xetex, colorlinks=true, citecolor=gray, linkcolor=gray, urlcolor=gray]{hyperref}

%%% BEGIN DOCUMENT
\begin{document}

% % Title Page
\author{\myauthor}
\title{Book Proposal\\
\emph{Parousia}\\
Sympathy and Sensory Presentation}
\date{}

\maketitle

% Layout Settings
\setlength{\parindent}{1em}

% Main Content

\section{Overview} % (fold)
\label{sec:overview}

The present essay is an unabashed exercise in historically informed, speculative metaphysics. Its aim is to gain insight into the nature of sensory presentation.

One of the fundamental issues dividing contemporary philosophers of perception is whether perception is presentational or representational in character. To claim that perception is presentational in character is to claim that it has a presentational element irreducible to whatever intentional or representational content it may have. So conceived, the object of perception is present in the awareness afforded by the perceptual experience and is thus a constituent of that experience. Representationalists deny that perception has such an irreducible presentational element, claiming, instead, that the object of perception is exhaustively specified by its intentional or representational content. If there is indeed a presentational element to perception, then, according to the representationalist, this is because sensory presentation is either reducible to the exercise of an intentional or representational capacity or otherwise essentially involves the exercise of such a capacity. There are two aspects of this debate. On the one hand, there are arguments on one side or the other urging that perception must be conceived in presentational or representational terms. One the other hand, there is a more positive, constructive aspect, where, taking for granted one’s preferred conception, one goes on to develop detailed theoretical accounts of perceptual experience.

Representationalists have been more active in this latter task. And unsurprisingly so. For suppose one took sensory presentation to be an indispensable aspect of perceptual experience and further held, in a Butlerian spirit, that it was reducible to no other thing. What positive account could one give of sensory presentation, so conceived? Since it is irreducible, no positive account could take the form of a reduction. So no causal or counterfactual conditions on sensory representations, understood independently of perception, could be jointly necessary and sufficient for the presentation, in sensory experience, of its object. One might specify the
relational features of presentation in sensory experience, but not much insight into the nature of sensory presentation is thereby gained. The tools of contemporary analytic metaphysics would seem not to leave one much to work with, at least in the present instance. So it can seem that if one maintains that perceptual experience involves an irreducible presentational element, all that one can do is press the negative point that sensory presentation, an indispensable element of perceptual experience, is reducible to no other thing.

I believe that perception has an irreducible presentational element. And yet I hoped to learn something positive about the metaphysics of sensory presentation. If there was, in fact, anything further to be learned, I could not limit myself to the tools of contemporary analytic metaphysics. The present metaphysics is historically informed, at least in part, as a result of looking for tools more adequate to the task at hand. 

The central claim of the present essay is that sensory presentation might be understood in terms of the operation of sympathy, construed not as fellow-feeling so much as the principle at work in fellow-feeling and generalizable to other domains.
% section overview (end)

\section{The Audience} % (fold)
\label{sec:the_audience}

The present essay is primarily addressed to philosophers of perception and anyone interested in the metaphysics of experience (as well as graduate students working in these fields). It may also be of interest to historians of philosophy working on these issues as it draws upon and develops some historical issues as they are relevant to the argument of the essay.

% section the_audience (end)

\section{The Competition} % (fold)
\label{sec:the_competition}

The only book that I can think of that is comparable would be Alva No\"{e}'s \emph{Varieties of Presence}. Like No\"{e}, I am taking seriously the metaphysics of sensory presentation while admitting that it reduces to no other thing. There are substantive differences, however. No\"{e}'s work is not historically informed in the way the present work is. Moreover, No\"{e}, at least by my lights, is a skeptic about sensory presentation in a way that I am not (that, I take it, is part of the force of the qualifier ``virtual'' and No\"{e}'s emphasis on perceptual availability in accounting for virtual presence).

% section the_competition (end)

\section{Chapters} % (fold)
\label{sec:chapters}

The present essay consists of 6 chapters:

\begin{enumerate}
	\item \emph{Grasping} Discusses grasping as a mode of haptic presentation and raises a puzzle concerning the role of bodily awareness in haptic perception.
	\item \emph{Symapthy} Discusses responses to the puzzle and resolves it by accounting for haptic presentation as a species of sympathy.
	\item \emph{Sound} Defends the Wave Theory of Sound.
	\item \emph{Sources of Sound} Argues that the sources of sound are heard in or through the sounds they make and explains this in terms of sympathetic presentation.
	\item \emph{Vision} Extends the account of sensory presentation in terms of sympathy to visual presentation.
	\item \emph{Realism} Argues that sympathetic presentation may present how things are in themselves and that the phenomenal/noumenal distinction collapses.
\end{enumerate}

% section chapters (end)

\section{About the Author} % (fold)
\label{sec:about_the_author}

I am a professor of philosophy at UCL and I have written extensively about color and color perception.

% section about_the_author (end)

\end{document}
