%!TEX root = /Users/markelikalderon/Documents/Git/sympathy/perception.tex
\chapter{Sympathy} % (fold)
\label{cha:sympathy}

% \epigraph{The primitive form of physical experience is emotional---blind emotion---received as felt elsewhere in another occasion and conformally appropriated as a subjective passion. In the language appropriate to the higher stages of experience, the primitive element is sympathy, that is, feeling the feeling in another and feeling conformally with another.}{Whitehead, \emph{Process and Reality}}

\section{The Metaphysics of Haptic Presentation} % (fold)
\label{sec:the_metaphysics_of_haptic_presentation}

Tactile metaphors for perception are primordial and persistent. What makes grasping an apt metaphor for perceptual awareness, even for non-tactile modes of awareness such as vision an audition? In order to answer this question, we undertook a phenomenological investigation into the nature of haptic perception. That investigation was phenomenological in that it confined itself to perceptual appearances and not because of any methodology involved. The hope was that if we better understood how grasping or enclosure, understood as a mode of haptic perception, presents itself from within haptic experience, then we would be in a better position to understand what potentially makes grasping an apt metaphor for perception generally. We discussed three claims about the metaphysics of haptic presentation:
\begin{enumerate}
	\item Tangible qualities of the object of haptic exploration are disclosed over time and so presentation in haptic experience has duration.
	\item Haptic perception formally assimilates to the tangible qualities presented to the perceiver's haptic perspective, understood as the distinctive way they are handling the object.
	\item The formal assimilation of haptic perception to its objects is a consequence of haptic experience being constitutively shaped by the object presented to the perceiver's haptic perspective.
\end{enumerate}
Not only does the hand assimilate to the contours of the object grasped, but the haptic experience that this activity gives rise to itself formally assimilates to its object. Moreover the formal assimilation of haptic experience to its object relative to the perceiver's haptic perspective is not merely causal but constitutive. Haptic perception formally assimilates to its object, relative to the perceiver's haptic perspective, because that object constitutively shapes that perceptual experience. This, I suggested, was the basis of haptic perception's objectivity and part of what makes it an apt metaphor for perception generally. 

Beside the three metaphysical claims about haptic presentation enumerated above, I also made a further explanatory suggestion about haptic perception's objectivity. Specifically, while haptic experience assimilating to its object is a manifestation of the objectivity of haptic perception, it is not its source. The presentation of tangible qualities of objects external to the perceiver’s body is due, at least in part, to the activity of the hand in grasping and the resistance it encounters. The hand, and haptic experience in turn, only assimilate to the tangible aspects of the rigid, solid body thanks to the force of the hand’s activity in conflict with the self-maintaining forces of that constitute the categorical bases of that body’s solidity and rigidity. At least with grasping or enclosure, understood as a mode of haptic perception, perceptual realism is sustained by the force of the hand’s activity in conflict with the self-maintaining forces of the object grasped. In this way, the hand is the active wax of haptic perception.

However this last insight, if it is one, gave rise to the puzzle that arose at the end of the last chapter. That puzzle revealed no genuine incoherence in the Manifest Image of Nature. The puzzle was not meant to be the basis of skepticism about objective haptic perception so much as the basis of a how-possible question, how is objective haptic perception so much as possible? The puzzle began with haptic perception's dependence upon bodily awareness. For animals like ourselves bodily awareness is a mode of self-presentation even if there is more to our nature than is revealed in bodily awareness. But how can a mode of self-presentation disclose the presence of some other thing? How is it that bodily awareness is leveraged in haptic perception into disclosing the presence and tangible qualities of an external body? Taking on board Kilwardby's transformed insight that the presentation of tangible qualities of an external body is due, at least in part, to the activity of the hand and the felt resistance it encounters, we refined our how-possible question: How does felt resistance to the hand's activity in grasping or enclosure disclose the overall shape and volume of an external body? After all not every felt resistance is due to the tangible qualities of an external body. So how does the experienced limitation to the hand's activity allow the perceiver to feel something in something external to the perceiver's body and in conformity with it?

Reflection on this puzzle or \emph{aporia} shall be the basis for further substantive claims about the metaphysics of haptic presentation. The present chapter thus proceeds dialectically. Chief among the substantive claims to be made on this basis is the perhaps surprising claim that haptic presentation is governed by the principle of sympathy---that feeling something in another thing and in conformity with it is explicable in terms of the operation of sympathy. 

% section the_metaphysics_of_haptic_presentation (end)

\section{The Dependence upon Bodily Awareness} % (fold)
\label{sec:the_dependence_of_haptic_perception_upon_bodily_awareness}

Our puzzle began with the dependence of haptic perception upon bodily awareness. Getting clearer on the nature of that dependence should help with our puzzle's resolution. 

In a chapter devoted to discussing the nature of this dependence, \citet[chapter 4.6]{Fulkerson:2014ek} draws the distinction between implicit and explicit experiences:
\begin{quote}
	An implicit bodily experience is one that is the background or recessive. ``Background'' here can be understood as an experiential content that is not consciously attended, in the minimal sense that it does not allow its objects to be open for epistemic appraisal. Such unattended contents or experiences do not incur an additional attentional load on our conscious experiences (we can only actively attend to a limited number of items at any one time, but implicit experiences do not add to this threshold). However, they are in consciousness nonetheless, primed for attention. \citep[90]{Fulkerson:2014ek}
\end{quote}
Explicit experiences, in contrast, involve attending to, or actively thinking about, the object of that experience. 

With the distinction drawn between implicit and explicit experience, we may ask whether grasping or enclosure, understood as a mode of haptic perception, depends upon an explicit bodily experience of the hand's configuration and force, or whether the presentation of the object's overall shape and volume in haptic experience merely depends upon an implicit experience of the hand's configuration and force? If the bodily experience upon which haptic perception depends is explicit, then the perceiver consciously attends to the state and activity of the body and haptic perception of the tangible qualities of an external body depends upon this explicit bodily experience. Fulkerson calls this Strong Experiential Dependence. On the hypothesis of Strong Experiential Dependence, the haptic perception involved in grasping or enclosure, a conscious experience, depends upon another conscious experience, specifically, of the hand's configuration and force.

\citet[chapter 4.8]{Fulkerson:2014ek} argues, instead, that the dependence is best understood in terms of what he calls Informational Bodily Dependence. Though information from processes that underly proprioception and kinesthesis are integrated with afferent information, such as the information provided by cutaneous activation, these give rise to a single conscious experience. The idea is that the sensitivity exhibited by haptic perception, such as grasping or enclosure, depends upon the tactile system drawing upon functionally distinct streams of information involved in bodily awareness. Nevertheless, the percept that is thereby determined is a single conscious experience, in the case of grasping or enclosure, our feeling of the overall shape and volume of the object grasped. This contrasts with Strong Experiential Dependence where conscious haptic experience is understood to depend upon a distinct conscious experience of the body's configuration and motion. On the alternative, conscious haptic experience depends upon, not an explicit, but an implicit experience of the hand's configuration and force. \citet[91]{Fulkerson:2014ek} cites with approval \citet[137]{Gallagher:2005ag} in this regard: ``Our pre-reflexive, kinesthetic-proprioceptive experience thus plays a role in the organization of perception, but in a way that does not require the body itself to be a perceptual object.'' If we understand the perceptual object as something that is actively attended to, then haptic experience merely depends upon an implicit experience of the hand's configuration and force (see also \citealt{Bower:2013aa}).

Put another way, according to Informational Bodily Dependence, our capacity for haptic perception draws upon our distinct capacities for proprioception, kinesthesis, motor activity, and our sense of agency but its exercise is an experience that affords the perceiver awareness of the presence and tangible qualities of an object external to the perceiver's body. So understood, Information Bodily Dependence could not, by itself, be a solution to the puzzle with which the previous chapter ended. First, while haptic perception may depend upon the functionally distinct streams of information associated with the various forms of bodily awareness, it depends as well upon distinct afferent information provided by cutaneous activation. So Information Bodily Dependence fails to provide anything like a sufficient condition for the tangible qualities of the perceived body to be present in haptic experience. More importantly, that our capacity for bodily awareness, however implicit, enables haptic experience to present the tangible qualities of an external body is less an explanation than what needs explaining. Our puzzle is not completely resolved until we understand how this may be so.

Campbell cites Huang's and Pashler's \citeyearpar{Huang:2007jk} distinction, in visual attention, between selecting something out from its background and characterizing or accessing its features: 
\begin{quote}
	So a property may be used to \emph{select} the object or region. Or the property may be \emph{accessed} as a property of that object or region. Selection is what makes the object or region visible in the first place; selection is what makes it possible for the subject to focus on that object or region in order to ascertain its various properties. Access is a matter of the subject making it explicit, in one way or another, just which manifold properties the object or region has. \citep[Campbell in][54]{Campbell:2014aa}
\end{quote}
Tactile perception, like visual and auditory perception, involves grouping, segmentation, and recognition. Suppose, then, that this distinction can be drawn, not only within visual attention, but also within tactile and, specifically, haptic attention. So a property may be used to select an object or region for active attention in haptic exploration or a property may be accessed in conscious haptic experience as a property of that object or region. With Huang's and Pashler's distinction in mind, and supposing it may legitimately apply to haptic attention as well, Fulkerson's notion of an explicit experience is characterized in terms of our accessing its object---it is  consciously attended to and open for epistemic appraisal. Now suppose Campbell is right in thinking that a property may be used to select an object in visual attention but not be accessed in consciously attending to it \citep[chapter 3.2]{Campbell:2014aa}. And suppose, further, that this possibility is a consequence of the distinction Huang and Pashler introduced, so that, if it holds, as well, for haptic attention, then there should be cases of selecting an object or region for haptic attention without consciously attending to the tangible quality on the basis of which that object or region was selected. Since explicit experiences are a matter of accessing their objects, then our haptic experience of a tangible quality that selected the body or region but was not consciously attended to would be an implicit experience of that quality.

This is the basis of a worry for a further claim Fulkerson makes about implicit experiences. There is a sense in which, for \citet[91]{Fulkerson:2014ek}, implicit experiences are no experiences at all. The content of an implicit experience is merely the content of a potential, that is to say, non-actual, experience \citep[95]{Fulkerson:2014ek}. And there is an associated tendency in Fulkerson's discussion to identify conscious experience with what is attended to and accessed, with explicit experience. But if the presence of a tangible quality is the basis for the selection of an object or region in haptic exploration, and if selection is what makes the object or region tangible in the first place, then surely it contributes to the phenomenological character of the haptic experience even if it is not consciously attended to. If that same object or region were selected on the basis of a different tangible quality, the subsequent experience would differ in phenomenological character. For \citet[95]{Fulkerson:2014ek}, the objects of implicit awareness are there for ``potential directedness''. But if we can voluntarily selectively attend to something about which we are implicitly aware, it must be present already in our experience, however recessively and in the background, if it can thus be consciously and voluntarily selected.

The possibility raised by Huang's and Pashler's distinction between selection and access concerns the implicit experience of tangible qualities of external bodies. Our present, focus, however, is not on implicit experiences of external bodies but on implicit experiences of the perceiver's body. But here too it seems implausible that my awareness of my hand's configuration and force in grasping or enclosure, understood as a mode of haptic perception, while implicit, is merely potential and, thereby, non-actual. The information drawn upon from proprioception, kinesthesis, motor activity, and our sense of agency in haptic perception makes a contribution to the phenomenological character of that experience, even if there is, as Fulkerson urges, only one conscious experience (the haptic experience) in play and not two (the haptic experience and a distinct experience of the body's state and activity). The information from bodily awareness drawn upon in the exercise of our haptic capacities specifically makes a difference to the way the object of haptic awareness is presented. As I argued in chapter~\ref{sec:active_wax}, distinct exploratory activities, distinct ways of handling the object of haptic exploration, constitute distinct haptic perspectives on that object, and this perspectival relativity is manifest in the different haptic appearances presented by the constant object of haptic exploration. It is one thing to claim that bodily awareness makes no explicit contribution to haptic experience. In grasping or enclosure, understood as a mode of haptic perception, we attend only to the object grasped and its manifest tangible qualities. But it is a further, contestable claim, that bodily awareness, however implicit, contributes nothing to the phenomenological character of the haptic experience it partly gives rise to. Bodily awareness, however implicit, contributes to the variable haptic appearances in the exercise of constant haptic perception. If the phenomenological character of haptic experience were exhausted by the constant tangible qualities attended to, then no room would be left for the contribution of flux to our haptic experience. But an adequate account of perceptual constancy must determine not only the constant object of perception but its variable appearances as well. In grasping or enclosure, understood as a mode of haptic perception, haptic experience is the joint upshot of the force of the hand's activity and the self-maintaining forces of the object grasped. Constant tangible aspects are presented in haptic experience as the forces that constitute their categorical bases come into conflict with force of the grasping hand. And the variable appearances of these constant tangible aspects are a phenomenological reflection of the variable activity of the hand in haptic exploration.

% section the_dependence_upon_bodily_awareness (end)

\section{Against Haptic Indirect Realism} % (fold)
\label{sec:against_haptic_indirect_realism}

That our awareness of the hand's configuration and force is merely implicit in grasping or enclosure, understood as a mode of haptic perception, rules out at least one response to our refined how-possible question. Our question was how can an experienced limitation to the hand's activity disclose the presence and tangible qualities of an external body? And one natural suggestion might be that our puzzle merely reveals haptic presentation to be indirect. That is to say, perhaps our puzzle reveals that we are immediately presented with the hand's configuration and force and thereby mediately presented with the overall shape and volume of the object grasped. That haptic perception depends only upon an implicit awareness of the hand's configuration and force reveals this otherwise natural suggestion to be ultimately misguided.

Begin with bodily sensation, a mode of self-presentation. Among the corporeal aspects of our nature of which we may be aware are felt limitations to our body's activity, be it in the exertion and depletion of physical force---lifting something until we can no more---or in the inability to move our limbs in a certain way. Perhaps felt resistance to the hand's activity is a bodily sensation causally coordinated with tangible qualities of the object grasped. Thus the overall shape of the object grasped causes in the perceiver a certain bodily sensation, a felt resistance to the hand's activity as it is thus configured. Perhaps it is this felt resistance of the hand configured so that is immediately present in our experience. We thus come to haptically experience the tangible qualities of an external body thanks to the way in which bodily sensation is causally coordinated with them. The overall shape of the external object would be mediately presented by the characteristic bodily sensations that making an effort to mold more precisely the hand to its contours gives rise to. So, on this model, we would be immediately presented with aspects of our own body's configuration and the limits to its activity and thereby mediately presented with the tangible qualities of an external body.

% The model is structurally similar to the account of proprioception that Anscombe targets:
% \begin{quote}
% 	The idea that it is by sensation that I judge my bodily position is usually the idea that it is by other sensations, not just the ‘sensation’ of sitting cross-legged, say, that I judge that I am sitting cross-legged, i.e. by a pressure here, a tension there, a tingle in this other place; such sensations are supposed to be sensations of being in that bodily position because, perhaps, they have been found to go with that. (Anscombe, ``On Sensations of Position'' 71)
% \end{quote}

Recall the obstacle that prompted our refined how-possible question was a failure of sufficiency. Not all experienced limitations to the body's activities, not all passivities of matter, are due to the tangible qualities of an external body. How then do we distinguish those experienced limitations that are perceptions of external bodies from those that are not? Haptic indirect realism provides at least a sketch of an answer. The experienced limitations of the body's activity that are involved in the perception of an external body's tangible qualities are those that are causally coordinated with them, at least in the right sort of way. This last qualification is not insignificant, as anyone who is familiar with the problem of wayward causal chains will appreciate. This is part of why this is just a sketch of an answer. 

The problem for this envisioned haptic indirect realism, however, lies not with its being underdeveloped in this way, but rather with its claim that haptic perception depends upon an explicit awareness of the hand's configuration and force. Arguably at least, that awareness is merely implicit. An explicit experience of a limit to the hand's activity is, according to this indirect realism, the means by which we experience the external body. But the disclosure of an object's overall shape and volume in grasping or enclosure is not apparently mediated in this way. Grasping seems from within haptic experience to directly disclose corporeal aspects of its object. Moreover, explicitly attending to the hand's configuration and force in grasping or enclosure draws attentive resources away from the object of haptic perception. In grasping or enclosure, understood as a mode of haptic perception, we attend only to the object of haptic investigation and its manifest tangible qualities. It is because the tangible qualities of an external body are directly disclosed in haptic perception that grasping becomes, in the cosmology of the Giants, a touchstone for reality. Grasping, however, could not play this rhetorical role, if it were apparently mediated.

Phenomenologically, this seems apt. Haptic experience seems to present itself as the immediate, if partial, disclosure, to the perceiver's haptic perspective, of the tangible qualities inhering in a thing external to the perceiver's body. In a way, haptic indirect realism makes the converse of Fulkerson's mistake. Whereas, Fulkerson emphasizes the presence of the constant tangible object in haptic attention at the expense of its variable haptic appearance, haptic indirect realism makes these variable haptic appearances the objects of active attention. Haptic indirect realism thus involves the objectification of appearing as appearance of which Cook Wilson complained in his 1904 letter to Stout:
\begin{quote}
	And so, as \emph{appearance} of the object, it has now to be represented not as the object but as some phenomenon caused in our consciousness by the object. Thus for the true appearance (=appearing) to us of the \emph{object} is substituted through the `objectfication’ of the appearing as appearance, the appearing to us of an \emph{appearance}, the appearing of a phenomenon cause in us by the object. (\emph{Correspondence with Stout 1904}, \citealt[796]{Cook-Wilson:1926sf})
\end{quote}
But when we perceive by means of our grasping hand we attend only to what is in our grasp and not to the way that it is presented in our handling of it. Our sense of our hand's configuration and force contributes only to the pre-noetic structure of haptic experience and, at best, determines the way its object is presented therein. In making our awareness of the hand's configuration and force explicit, haptic indirect realism is thus precluded. 

However, if anything, precluding this haptic indirect realism only makes our how-possible question more urgent and more challenging. For how can an implicit awareness of a limit to the hand's activity directly disclose the overall shape and volume of an external body? What contribution can an awareness, however implicit, of the hand's configuration and force make to haptic perception that would not undermine its directly disclosing the constant object of haptic attention? If anything, recognition of the dependence of haptic perception upon an implicit bodily awareness can seem only to make matters worse.

% section against_haptic_indirect_realism (end)

\section{Sympathy} % (fold)
\label{sec:sympathy_and_haptic_perception}

How can an implicit awareness of a limit to the hand's activity contribute to directly disclosing the overall shape and volume of the object grasped? How does the pre-noetic structuring of haptic experience determined by this implicit awareness contribute to the presentation of its object? 

When our hominid ancestor reaches out and picks up a rough-hewn stone, perhaps in preparation to skirmish with a competing group of hominids, they feel the overall shape and volume of the stone in their grasp. It is not the hand's shape, the configuration of the hand in grasping or enclosure, that they haptically perceive though they may be aware of it, however implicitly. It is the stone's shape that is disclosed in their grasp. They feel the overall shape and volume in the stone, and its overall shape and volume are tangible qualities of the stone that their hand is felt to conform to. I shall make a suggestion that will be the basis for an answer to our refined how-possible question. Specifically, feeling tangible qualities in something external to the perceiver's body and feeling in conformity with them can fruitfully be understood as due to the operation of sympathy. 

Felt resistance to touch, insofar as it is the presentation of an object external to the perceiver's body, is a sympathetic response to the force that resists the hand's activity. Recall our refined version of our how-possible question was this: How is it possible for felt resistance to the hand's activity in grasping or enclosure to disclose a rigid, solid body's overall shape and volume? If feeling tangible qualities in something external to the perceiver's body and in conformity with them is due to the operation of sympathy then we have a basis for an answer. It is when the limit to hand's activity is experienced as a sympathetic response to a countervailing force, as the hand's force encountering an alien force resisting it, one force in conflict with another, like it yet distinct from it, that the self-maintaining forces of the body disclose that body's presence and tangible qualities to haptic awareness.

If felt resistance is the means by which the conflicting forces are sympathetically presented in haptic experience, then in being sympathetically presented with an external body, the perceiver is naturally attending to the external body, the object of haptic perception. In haptic perception, the perceiver is explicitly aware of the object of haptic perception. Insofar as felt resistance is sympathetically presenting an external body, the perceiver's awareness of the hand's configuration and force is, by contrast, merely implicit. Indeed actively attending to the hand's activity would erode the sympathetic presentation of what is external to the perceiver's body. 

Earlier, the initial statement of the puzzle was motivated by considering the analogy of felt temperature. We contrasted two cases. In both cases you feel warm, and you feel warm to the same degree. But in the first case, you feel warm because of a fever, and in the second case, you feel warm because because of the ambient heat. There is also, importantly, a phenomenological difference between these cases. In the second case, not only do you feel warm, but you feel, as well, the warmth in the ambient air. Indeed, the warmth you feel is in conformity with the warmth felt in the ambient air. What explains the phenomenological difference is that in the second case, but not in the first, the felt warmth is a sympathetic response to the ambient heat, to the thermal properties of something external to the perceiver's body. In sympathetically responding to ambient heat, the warmth you feel becomes a way of feeling the warmth in something located outside of your body. Moreover, in sympathetically responding to ambient heat, the warmth you feel is in conformity with the warmth felt in the air. Active attention to the warmth you feel can erode the sympathetic presentation of the ambient warmth. Focus too much on the warmth you feel, and you cease to feel the warmth in the air.

Sympathetically responding to the way the body's self-maintaining forces resist the hand's grasp is a way of presenting that body and its tangible qualities. Sympathy is what makes the extrasomatic present in haptic experience. One obstacle to appreciating this concerns our present understanding of sympathy, where sympathy is a kind of emotional response to others, a kind of fellow-feeling, akin to compassion or pity. The notion of sympathy that is being invoked as the principle governing haptic presentation is closer to the notion at work in Stoic physics, if more abstract and not at all reliant on on their vitalistic metaphysics. The present approach thus contrasts with Whitehead's \citeyearpar{Whitehead:1978zr}. Whitehead both explains perceptual prehension partly in terms of sympathy and embraces the association with emotion:
\begin{quote}
	The primitive form of physical experience is emotional---blind emotion---received as felt elsewhere in another occasion and conformally appropriated as a subjective passion. In the language appropriate to the higher stages of experience, the primitive element is sympathy, that is, feeling the feeling in another and feeling conformally with another. The separation of the emotional experience from the presentational intuition is a high abstraction of thought. Thus the primitive experience is emotional feeling, felt in its relevance to a world beyond. The feeling is blind and the relevance is vague. \citep[162-3]{Whitehead:1978zr}
\end{quote}
Whitehead's retention of the emotional associations of sympathy lead him to paradoxically portray perceptual prehension as an outgrowth of blind emotion. However, as we shall see, the principle of sympathy can be understood with sufficient generality so that it may be at work both in haptic presentation and fellow-feeling, without reducing perceptual presentation to blind emotion. Perception may not reduce to blind emotion, but that is consistent with certain natural affective responses being made possible and, indeed, partly constituted by the operation of sympathy in haptic presentation. It would have to be, if, as \citet[chapter 4]{Derrida:2005aa} insists, an adequate philosophy of touch must leave room for both blows and caresses.

The proposal is that presentation in haptic perception is governed by the principle of sympathy. There are two ways to understand this. The fist proceeds synthetically. That is, beginning with elements and principles understood independently of haptic perception, one constructs the notion of the presentation of tangible qualities of external bodies in haptic experience on their basis. So, for example, one might begin with bodily sensation and ``extend its reach'', so to speak, via the operation of sympathy to construct a notion of the presentation of tangible qualities of external bodies. So understood, haptic presentation would be the coordination of bodily sensations with the tangible qualities of external bodies via the operation of sympathy. The second way proceeds analytically. That is, beginning with the notion of the presentation of tangible qualities of external bodies in haptic experience, one analyses or decomposes that notion into constituent elements that must be present and principles that must be operative if haptic perception is so much as possible. (Compare the ``top-down'' approach that \citealt[chapter 1]{Gerson:2005aa}, attributes to Platonism in contrast with a ``bottom-up'' approach.)

The synthetic approach naturally, perhaps inexorably, motivates indirect realism about haptic perception, comparable to the indirect realism that we previously rejected. So consider again our toy model where we begin with bodily sensation and extend its reach through the operation of sympathy. Bodily sensation does not involve the presentation of tangible qualities of external bodies. It is, instead, a mode of self-presentation. Thanks to the operation of sympathy, in being presented with an aspect of our corporeal nature, we are mediately presented with the tangible quality of an external body. But haptic perception is not indirect in this way. When our hominid ancestor grasps a rough-hewn stone they feel its overall shape and volume in the stone. Moreover, the presentation of these tangible qualities in their haptic experience is not apparently mediated. Our hominid ancestor need not attend to their bodily sensations as a means of attending to the tangible qualities of external bodies, rather these are directly disclosed in haptic perception. Indeed, attending to the body and its activity draws attentive resources away from the object of tactile perception. 

% Focus too much on the warmth you feel and you cease to feel the warmth in the air.

% It is because the tangible qualities of an external body are directly disclosed in haptic perception that grasping becomes, in the cosmology of the Giants, a touchstone for reality. Grasping, however, could not play this rhetorical role, if it were apparently mediated.

The problem with the synthetic approach, at least as so far developed, is twofold. First, it posits two experience---the haptic experience and the experience of the perceiver's body---when plausibly there is only one. (These would remain two distinguishable experiences even if the experience of the perceiver's body were, in some sense, a part, or constituent, of the broader haptic experience.) And, second, the awareness of the perceiver's body is explicit rather than implicit. On the synthetic approach, the state and activity of the body are actively attended to and so are, potentially at least, the object of epistemic appraisal. Moreover, both of these features were directly involved in the subsequent indirect realism. On the alternative, analytic approach, indirect realism is simply not a possibility. One begins with an irreducible unity, the presentation of the tangible qualities of external bodies in haptic experience, and then discern what intelligible structure it must display if it is so much as possible. (On sensory presentation being a kind of unity---a ``communion'' with its object---see Ardley's \citeyear{Ardley:1958aa} unjustly neglected essay.) Thus the presentation of tangible qualities of external bodies in haptic experience could not be a construction from elements and principles understood independently of haptic perception, the way they would be if indirect realism were true.

The analytic approach to sensory presentation is comparable to Frege's approach towards thought, at least at certain stages of his career, on certain interpretations (see, for example, \citealt[essays 7 and 9]{Travis:2011qd}). Frege begins with a unity, a truth-evaluable thought, and discerns what intelligible structure it must display. Beginning with the thought, Frege analyzes or decomposes that thought into constituent elements that must be present and principles that must be operative if that thought is to be so much as truth-evaluable (which is not say that there is a unique such decomposition). Frege's position thus contrasts with recent discussions of the problem of the unity of the proposition (compare \citealt{King:2007ad}, \citealt{Gaskin:2008aa}, \citealt{Soames:2010qq}, and \citealt{King:2014ls}). The problem of the unity of the proposition simply does not arise for Frege, since he does not begin with independently understood elements and principles and tries to construct thoughts on their basis. Rather the unity of thought is explanatorily prior to the intelligible structure it must display if it is to be so much as truth-evaluable. Similarly, on the analytic approach, the unity of sensory presentation is explanatorily prior to the intelligible structure it must display if it is so much as possible.

To get a general sense of the analytic approach, consider the following plausible, if contentious, example (\citealt{Johnston:2007qy}, for one, seems to deny it). Arguably at least, any notion of sensory presentation essentially involves a subject--object distinction. If an object is present in perceptual experience then not only is there the object of perception---what is present in that experience--but there is also a perceiver that undergoes that experience---the subject to whom the object is presented. If we allow for modes of self-presentation where the subject and object are the same entity, then the subject--object distinction arguably required by the presupposed unity is merely hyperintensional. So compare Plotinus' view, in the \emph{Fifth Ennead}, that intellection, the presentation of intelligible objects, the highest form of unity short of that displayed by the hyperontic One, requires the distinction between the act of intellect and its object. Nevertheless, the Intellect apprehends only itself insofar as it is an image of the One. So the subject--object distinction required for intelligible presentation is consistent with its being a mode of self-presentation and so hyperintensional (see \citealt[chapter 3.1]{Gerson:1994aa}). 

If presentation may be self-presentation, and the intelligible distinction between subject and object may be hyperintensional, then I am genuinely uncertain about Johnston's denial of the claim that presentation intelligibly requires a subject. \citet{Johnston:2007qy} invites us to to think of ourselves as Samplers of Presence, where we access objective modes of presentations that are part of a larger reality, both accessible and inaccessible, but where our access, relative to our perspective, though ours, does not involve a subject over and above the accessed objective modes of presentations. But if the subject to whom the object is presented can be one and the same thing, then there being no subject over and above the object is not yet proof that they cannot be intelligibly distinguished. Even if there is no subject over and above the objective mode of presentation accessed from our perspective, the denial that there is no subject which accesses the objective mode of presentation is a further claim. One and the same thing, the objective mode of presentation, may be playing two roles. Just as in self-hate, where, tragically, one thing both hates and is hated, perhaps, in perception, one thing both accesses and is accessed. The present point is not to criticize Johnston, nor to defend neutral monism, but to emphasize how little may be involved in the subject--object distinction.

Intelligible presentation may be a mode of self-presentation, but Plotinus claims that the subject and object of perception must be more than hyperintensionally distinguished, they must be two things. This is a reflection of the fact that the unity presupposed in sensory presentation is a lesser unity than the unity presupposed in intelligible presentation. However, once one adopts a more naturalistic approach to embodiment than Plotinus, it is plausible to allow for forms of sensory self-presentation. Since having a fever is a condition of the body, and we are fundamentally embodied, then feeling a fever is itself a mode of self-presentation, even if there is more to one's nature than the fever one is currently suffering. (For discussion of this example and the puzzlement that results from not allowing modes of sensory self-presentation see \citealt{Yrjonsuuri:2008aa}.) If sensory presentation is partial, and primates like ourselves are fundamentally embodied, then the sensory presentation of aspects of our corporeal nature is a kind of self-presentation even if there is more to our nature than is present in bodily awareness. 

There may, however, be a sense in which Plotinus was right. The unity presupposed in sensory presentation, being partial, is a lesser unity than the unity presupposed in intelligible presentation. When the Inchoate Intellect turns, and looks, and sees only itself insofar as it is the image of the hyperontic One, thus becoming the Intellect in full actuality, this intelligibly differentiated image is wholly present in the act of intellection. An intelligible object is wholly present in the act of intellection in the way that a sensible object never is in perception since sensory presentation is invariably relative to the perceiver's partial perspective.

% The contention that the unity of haptic perception presupposes that there be an intelligible distinction, potentially hyperintensional, between the object presented in haptic experience and the subject to whom it is presented is plausible, if contentious. It is contentious because it has been contended. \citet{Johnston:2007qy}, for one, denies it.

Notice that in proceeding analytically, the subject--object distinction is not something to overcome (a characteristically modern anxiety dramatized by Cartesian skepticism). Instead we are presupposing their unity in an episode or process of sensory presentation. There is no need to bridge the gap between subject and object since we began with their unity in haptic perception and merely discern that their distinction, potentially hyperintensional, is intelligibly required. The need to bridge the gap between the subject and object constituted by their distinction only arises if their unity is not in this way presupposed. Thus bridging the gap between subject an object by having bodily sensation be coordinated with tangible qualities of external bodies via the operation of sympathy and its attendant indirect realism only arises if their unity in perceptual presentation is not presupposed but something to be constructed from elements and principles antecedently understood. 

In grasping or enclosure the overall shape and volume of the object is directly disclosed in a perceiver's haptic encounter with it. Since I believe that perception quite generally involves an irreducible presentational element, I do not believe that the haptic presentation of the tangible qualities of external bodies could be constructed out of elements and principles understood independently of haptic perception. So I am debarred from the synthetic approach. It is, at any rate, inconsistent with our implicit awareness of the hand's configuration and force in grasping or enclosure, understood as a mode of haptic perception. Thus I proceed analytically. Presupposing the unity of haptic presentation, I try to determine the intelligible structure it must display if it is so much as possible. The claim that the presentation of tangible qualities of external bodies in haptic experience involves the operation of sympathy should be understood in this light. It is not the claim that one thing, the tangible qualities of external bodies, is mediately presented by another thing, the presentation of aspects of the subject's corporeal nature in bodily sensation. Rather, it is the claim that the presentation of tangible qualities of external bodies in haptic experience is an irreducible unity that is governed by the principle of sympathy. Feeling a tangible quality in an external body and in conformity with it just is the presentation of that quality in tactile experience and can be analytically explicated in terms of the operation of sympathy.

% section sympathy (end)

\section{Sensing Limits} % (fold)
\label{sec:sensing_limits}

In grasping or enclosure, understood as a mode of haptic perception, the overall shape and volume of an external body is present in haptic experience thanks to an implicit experience of an external limit to the hand's activity. If an experienced limit to the hand's activity discloses tangible qualities of an external body, then the idea of the experience of a limit, however implicit, must be in good order. But is it really? Within the phenomenological tradition, \citet{Derrida:2005aa} has expressed his doubts. Our present purpose is not to lay this doubts to rest in a way that would persuade a determined Derridean skeptic but rather to make intelligible to ourselves what would be involved in the experience of a limit.

In a representative passage, Derrida describes an \emph{aporia} involved in the figure of touch:
\begin{quote}
	Above all, nobody, no body, no body proper has ever touched---with a hand or through skin contact---something as abstract as a limit. Inversely, however, and that is the destiny of this figurality, all one ever does touch is a limit. To touch is to touch a limit, a surface, a border, an outline. Even if one touches an inside, ``inside'' of any thing whatsoever, one does it following the point, the line or surface, the borderline of a spatiality exposed to the outside, offered---precisely---on its running border, offered to contact. \ldots\ This surface, line or point, this limit, therefore, \ldots\ finds itself to be at the same time touchable and untouchable: it is as is every limit, certainly, but also well-nigh at and to the limit, and on the exposed, or exposing, edge of an abyss, a nothing, an ``unfoundable'' unfathomable, seeming still less touchable, still more untouchable, if this were possible, than the limit it self of its exposition. \citep[103--4]{Derrida:2005aa}
\end{quote}
There is a lot to say about this passage and how the \emph{aporia} it describes may, if at all, \emph{pace} Derrida, be resolved. One thing to get clearer about is the sense in which a surface, understood as a limit, is abstract. On at least one good sense of the abstract--concrete distinction, the surfaces of material bodies count as concrete---they at least exist in space and time. But notice, as well, that the surfaces of material bodies could not themselves be material. They are not themselves material parts of the bodies whose surfaces they are. Surfaces are, in Sellars' \citeyearpar[\textsc{iv} 23]{Sellars:1956xp} apt phrase, bulgy two-dimensional particulars. They are two-dimensional in the sense that they lack thickness. But no material thing lacks thickness. This suggests an alternative understanding of the sense in which such limits are abstract. Whether it is sufficient to underwrite Derrida's \emph{aporia} is another matter. Another thing to get clearer about is whether the limit which is said to be intangible is the same limit which we must be said to touch. Perhaps like Protagorean arguments, at least on a Peripatetic diagnosis of them, the puzzle turns on a conflation. After all, limits may be said of in many ways and there may be different senses in which we may be said to touch a limit.

Notice, however, that the putatively intangible limit at work in this passage is a spatial boundary, the surface of the object of tactile perception. An external limit to the hand's activity is not a spatial boundary or a surface, though it may disclose these, if it is experienced as their sympathetic presentation. However, if there is a puzzle about how anything as abstract, on some suitable understanding, as the limit of a bounded body may be tangible, surely a limit to the hand's activity is even more abstract. After all, the limit to the hand's activity is intangible---like virtue and the being of capacity more generally, as the Eleatic Visitor instructs the Giants \citep[chapter 1]{Beere:2009vn}. Bodily awareness presents corporeal aspects of the embodied perceiver, just as tactile perception presents corporeal aspects of its object. Our question is whether anything as abstract as a limit to the hand's activity so much as could be the object of bodily awareness. Thus a variant of the Eleatic Visitor's lesson raises, as well, a question about the Giant's appeal to the phenomenologically vivid and primitively compelling experience of felt resistance to touch if it is to motivate their corporealism.

What would it take to be aware of a limit to the hand's activity? Such an awareness would have to afford the subject with a contrast between the hand's present configuration and a potential configuration that extends beyond the points at which the hand's force is resisted by the self-maintaining forces of the object grasped. Such an awareness would depend upon a psychological representation of potential motor activity, a sense of how far one's grasp may extend if unimpeded. The representation of potential motor activity need only be apparent. I may have a sense that I could reach the top shelf, but trying may reveal that I was mistaken.

A sense of the contrast between the hand's present configuration and a potential configuration beyond the limit of the grasped object's boundaries may be necessary for awareness of an external limit to the hand's activity but it is not sufficient. There is a crucial additional element involved in being aware of a limit to the hand's activity. Whenever I deliberately hold my hand in a certain configuration that is not completely outstretched, I may have a sense of potential configurations extending beyond the present one, but I do not thereby experience a limit. The relevant sense of limit involves a check or impediment to the will. So not only does an awareness of a limit to the hand's activity involve a kinaesthetic representation of potential motion, but it must also draw upon our sense of agency. Not only must one have a sense of how far one's grasp may extend if unimpeded, but one must also have a sense of an impediment to one's grasp. A sense of impediment arises out of a frustration of the will in being unable to extend one's grasp further. Moreover this second condition is related to the first. The object of the will is to extend the hand further in peripersonal space, the space of potential motor activity. The object of the will is thus represented on the kinaesthetic map. The location of the hand's configuration in the space of potential motor activity is only experienced as a limit insofar as it is the frustration and not the fulfilment of the will. The frustration arises from the inability to extend the hand's activity further in peripersonal space, the object of the will being located in the space beyond which the hand may extend its activity, and this despite a sense of effort exerted in trying to obtain the object of the will---the felt force, however implicit, of the hand's activity in conflict with the self-maintaining forces of the object grasped.

These brief remarks would be insufficient to assuage the doubts of a determined Derridean skeptic. (``What does the word \emph{effort} \ldots\ designate, appearing as it does in this singular context\ldots, where effort, precisely, stalls in \emph{making an effort}. At the point where effort \emph{meets} the limit forcing it to \emph{exert} \emph{itself} in this \emph{effort}?'' \citealt[110]{Derrida:2005aa}; see especially chapter 6, \emph{Nothing to Do with Sight: ``There's no `the' sense of touch''}.) Fortunately, however, there were not meant for such a task. Rather, the Derridean skeptic was invoked as a foil against which to sketch a couple features of the implicit awareness of a limit to motor activity in grasping or enclosure. Notice that the felt resistance to touch involved in grasping or enclosure, understood as a mode of haptic perception, exhibits considerably more structure than the haptic indirect realist (section~\ref{sec:against_haptic_indirect_realism}) allows. In taking felt resistance to touch to be the object of active attention there was a temptation to conceive of it as a sensory impression existing, somehow, within the mind, as a conscious modification of the perceiving subject, as the objectification of appearing as appearance, at least by Cook Wilson's lights. Think again of the ways in which that experience depends upon kinaesthesia and our sense of agency. Not only does felt resistance to touch involve a sense of how far one's grasp may extend if unimpeded (and so locating the hand's present configuration in a broader space of potential motor activity) but also the frustration of the will in being able to extend that grasp no further and this despite the effort exerted. This complex capacity involves the representation of potential motor activity that is not only ego-centrically structured, but also teleologically structured by the will. No conception of sensory impression available to the indirect realist displays a similar structure. Without providing anything like a full account, I hope I have said enough to render \emph{prima facie} intelligible the operative conception of an awareness of a limit to the hand's activity, the active wax constrained by the passivities of matter, not least because it is a precondition for the sympathetic presentation of the tangible qualities of external bodies in haptic experience. For it is this impediment of the will that makes the disclosure of the extrasomatic in haptic experience possible.


% section sensing_limits (end)

\section{The Stoics} % (fold)
\label{sec:the_stoics}

I observed earlier that our present conception of sympathy can be an obstacle to appreciating how feeling something in another thing and in conformity with it is itself a mode of sympathy. To overcome this limitation, as well as to introduce some claims about the operative notion of sympathy, it will be useful to consider briefly a select history. Specifically, I want to consider sympathy as a principle of action at a distance in Stoic physics in this section and Plotinus' use of the Stoic notion in explaining distal perception in vision and audition in the next. 

It is easy to be impressed, as ancient medical opinion was, with how affecting a part of an animal's body may affect another part of their body without affecting the parts between (see, for example, the Hippocratic \emph{Peri Troph\'{e}} and Galen's \emph{De Locis Affectis}). Consider how the Hippocratic author of \emph{Peri Troph\'{e}} understands symptoms: 
\begin{quote}
	Signs: tickling, ache, rupture, mind, sweat, sediments in urine, rest, tossing, condition of the eyes, imaginations, jaundice, hiccoughs, epilepsy, blood entire, sleep, from both these and all other things in accordance with nature, and everything else of a similar nature that tends to harm or help. (Hippocratic author, \emph{Peri Troph\'{e}} \textsc{xxvi}; \citealt[351]{Jones:1957aa})
\end{quote}
Symptoms are understood to be signs of underlying conditions since they are the sympathetic effects, in the case of ill health, of disturbances in parts of the animal's body without any apparent disturbance in the parts between. The nature of an animal, whether in sickness or in health, is the nature of a composite natural body whose parts are organized with reference to the function of the whole and these parts may thus sympathetically interact:
\begin{quote}
	Conflux one, conspiration one, all things in sympathy; all the parts as forming a whole, and severally the parts in each part, with reference to the work. (Hippocratic author, \emph{Peri Troph\'{e}} \textsc{xxiii}; \citealt[351]{Jones:1957aa})
\end{quote}
Thus a tickling, ache, or rupture is a sign for an underlying condition since it is the sympathetic effect of an occurrence in a complex whole. The Stoics believed that such medical phenomena were subject to a corporeal explanation, involving sympathy as its principle. And since they conceived of the cosmos as a whole as a living being, then the principle involved in that explanation, sympathy, was elevated to the status of a cosmic principle.

% Thus, a disturbance in the liver might give rise to spots on the skin without apparently affecting the intervening parts of the body.

According to the Stoics, the soul that pervades and animates a living body is composed of \emph{pneuma}, a kind of rarified mixture of air and fire (\emph{Stoicorum Veterum Fragmenta} 2 773--89). The soul, while corporeal, pervades the body. It does so not by filling interstitial spaces within the body, like water absorbed by a sponge. Rather, active \emph{pneuma} is sufficiently rarified that it  can occupy the same space as the passive matter of the body it animates, the way warmth may pervade a sun-baked stone. The \emph{pneuma} in a living body is in a state of tension. This tension in the \emph{pneuma} gives rise to a continuous wave-like motion (\emph{Stoicorum Veterum Fragmenta} 2 448, 450-7). Since the \emph{pneuma} in a living body is in a state of tensional motion, affecting some part of the body will affect the living body as a whole. Thus Sextus Empiricus reports: 
\begin{quote}
	But in the case of unified things there is a kind of sympathy; for example, when the finger is cut, the whole body shares its condition. (Sextus Empiricus, \emph{M} 9 79; \emph{Stoicorum Veterum Fragmenta} 2 1013;)
\end{quote}
Thus when a part of a living body is affected, a similar or different change may be transmitted via the tensional motion of the \emph{pneuma} to another part of the body without affecting the parts between, depending upon the disposition of its parts.

The operation of sympathy was not confined to ordinary living bodies. The sensible cosmos itself was conceived to be a living being as well, though perhaps an extraordinary one, at least by our lights. The sensible cosmos was thus conceived to possess the same kind of unity as living beings. The sensible cosmos, like all living beings, has a soul that animates it, the World-Soul. The World-Soul, like all souls, is composed of \emph{pneuma}, and the souls of ordinary living beings are, in some sense, part of the World-Soul. Like ordinary living beings, the sensible cosmos is united by an all pervading \emph{pneuma} in a state of tensional motion. Thus, according to Alexander of Aphrodisias, Chrysippus: 
\begin{quote}
	first assumes that the whole of substance is unified by a breath (\emph{pneuma}) which pervades it all, and by which the universe is sustained and stabilized and made interactive with itself (\emph{sympathes} \ldots\ \emph{auto}) (Alexander of Aprhodisias, \emph{On Mixture and Growth}, 216 14--218 6; \emph{Stoicorum Veterum Fragmenta} 2 473; \citealt[48 C]{Long:1987aa})
\end{quote}
So according to Chryssipus, disparate parts of the sensible cosmos may sympathetically interact due to the all pervasive \emph{pneuma}.  Thus sympathy was transformed, in Stoic thought, into a cosmic principle of action at a distance. While perhaps Posidonius is the most famous proponent of cosmic sympathy (Augustine, \emph{Civitas Dei} 5 2), the doctrine goes back at least as far as Chrysippus and, arguably, has roots in Plato's \emph{Timaeus} (on Stoic sympathy see \citealt{Sambursky:1959ms,Meyer:2009xp,Brouwer:2015ee}; on the \emph{Timaeus} and sympathy see \citealt{Emilsson:2015wf}). Sympathy, as a principle of action at a distance, was used to explain a variety of natural phenomena, such as the influence of the moon on the tides (Sextus Empiricus, \emph{M} 9 79; Cicero, \emph{De Divinatione} 2 34) and the efficacy of divination (Cicero, \emph{De Divinatione}, and Seneca, \emph{Naturales Quaestiones} Book \textsc{ii}). Divination was a pervasive practice in the Hellenistic period. Though we can no longer give it credence, it is important to remember that Hellenistic explanations of divination are a part of natural philosophy insofar as such practices were accepted as legitimate. (On how explanations of divination are a part of Stoic natural philosophy see \citealt{Struck:2007aa}.)

% , \emph{Stoicorum Veterum Fragmenta , \emph{Stoicorum Veterum Fragmenta

% section the_stoics (end)

\section{Plotinus} % (fold)
\label{sec:plotinus}

Plotinus appeals to sympathy, understood as a principle of action at a distance, to explain a variety of natural phenomena. Plotinus' use of sympathy has been portrayed as a Stoic borrowing \citep{Emilsson:1988uq,Ierodiakonou:2006gf}, but most likely its roots lie in Plato's \emph{Timaeus} \citep{Emilsson:2015wf}. On that hypothesis, Plotinus' use of Stoic material is confined to elaborating what is, by his lights, essentially Platonic ideas.

There are number of differences between Plotinus' use of sympathy and the Stoic's use.

First, according to Plotinus, the soul is incorporeal and so could not be composed of \emph{pneuma}, no matter how rarefied the admixture of fire and air. So the mechanism of tensional motion in an all pervading \emph{pneuma} that, on the Stoic account, explained the operation of sympathy is simply left out of Plotinus' account. Moreover, not only does Plotinus abandon the Stoic explanation of sympathy as the effect of tensional motion in an all pervading \emph{pnuema}, but he seems to offer no alternative mechanism in its place \citep[48]{Emilsson:1988uq}.

This latter fact may seem like a deficit of Plotinus' account until we realize that there is a deeper issue at work, here, than Plotinus' rejection of Stoic corporealism. As the view that Alexander attributes to Chryssipus makes clear, the all pervading \emph{pneuma} and its tensional motion is meant to unify the cosmos. So while both the Stoics and Plotinus take sympathy to only operate within a unity, the Stoics further hold that this unity is subject to explanation. There is, then, an important difference in explanatory priority that leads Plotinus to reject the Stoic explanation of sympathy in terms of the tensional motion of \emph{pneuma}. It is not the corporeal character of the Stoic explanation of that unity that leads to Plotinus' rejection, so much as unity being subject to explanation at all. The hyperontic One is the fundamental principle, or \emph{arch\'{e}}, of Plotinus' metaphysics. Thus for Plotinus, unity is an \emph{explanda} not an \emph{explanadum}. That sympathy only operates within a unity is a consequence, for Plotinus, of that unity making possible the operation of sympathy. No further mechanism is specified since, by Plotinus' lights, no further mechanism is required. (Compare how action at a distance in a system of physical events would be an intelligible effect of global constraints on that system.) This second, explanatory difference roughly corresponds to the explanatory difference between the synthetic and analytic approaches discussed earlier.

Third, Plotinus' use of sympathy is in one important respect broader than the Stoics'. Plotinus invokes the principle of sympathy, in a way that the Stoics did not, to explain a variety of psychological phenomena. Thus in a remarkable anticipation of Hume and Smith, Plotinus writes:
\begin{quote}
	Indeed the argument deriving from facts opposed [to the assumption of complete separation of souls] asserts that we do share each other's experiences (\emph{sympathein}) when we suffer with (\emph{synalgountas}) others from seeing their pain and fee happy and relaxed [in their company] and are naturally drawn to love them. For without a sharing of experience there could not be love for this reason. (Plotinus, \emph{If All Souls are One}, \emph{Ennead} 4 9 3 1--5; \citealt[433--5]{Armstrong:1984aa})
\end{quote}
Sympathy involves the sharing of experiences between distinct individual souls. It is an interpersonal principle, and so underwrites a kind of action at a distance within the social sphere. So the unity of all souls---whatever, exactly, that doctrine amounts to---makes it possible for distinct individual souls to sympathetically respond to one another and so share in one another's experiences. Not only does Plotinus use sympathy to explain fellow-feeling, but he also uses sympathy to explain the operation of our distal senses, specifically, in vision and audition (see especially the treatise, \emph{On Difficulties of the Soul \textsc{iii}, or On Sight}, \emph{Ennead} 4 5 and the supplementary work, \emph{On Sense-Perception and Memory}, \emph{Ennead} 4 6). So Plotinus understands sympathy as a principle of action at a distance that explains a variety of natural and psychological phenomenon including perception and fellow-feeling. So Plotinus provides an important historical precedent for the idea that sympathy can be understood with sufficient generality so that it may be at work both in perception and fellow-feeling without one reducing to the other (as in Whitehead's \citeyear{Whitehead:1978zr} conception of perceptual prehension as the outgrowth of blind emotion).

The main elements of Plotinus' account of sympathy are in play in the following representative passage:
\begin{quote}
	This one universe is all bound together in shared experience (\emph{sympathes}) and is like one living creature, and that which is far is really near, just as, in one of the individual living things, a nail or horn or finger or one of the other limbs which is not contiguous: the intermediate part leaves a gap in the experience and is not affected, but that which is not near is affected. For the like parts are not situated next to each other, but are separated by others between, but share their experiences (\emph{sympaschonta}) because of their likeness, and it is necessary that something which is done by a part not situated beside it should reach the distant part; and since it is a living thing and all belongs to a unity nothing is so distant in space that is not close enough to the nature of the one living thing to share experience (\emph{sympathein}). (Plotinus, \emph{On Difficulties about the Soul \textsc{ii}}, \emph{or On Sight}, \emph{Ennead} 4 4 32 14--22; \citealt[235--7]{Armstrong:1984aa})
\end{quote}
There are a number of observations to make about this passage.

First, like the \emph{Timaeus} and Stoic accounts, Plotinus thinks that the sensible cosmos has the unity of a living being. And since living beings are essentially ensouled, sympathy is based on the unity of the soul. So the unity of the ensouled living being is explanatorily prior to sympathetic interaction of its parts.

Second, the effects of sympathy may be between non-contiguous parts of the living being. The distance between the parts of a living being need not be an obstacle to their sympathetic interaction. The parts of a living being that sympathetically interact may be non-contiguous, but that is consistent with contiguous parts of the living being sympathetically interacting. The point is that sympathy is a mode of affection that does not require contact between cause and effect. While Plotinus acknowledges that there is affection by contact, he also maintains, like the Stoics before him, that there are natural phenomena that can only be explained by sympathetic affection.

Third, Plotinus links the sympathetic interaction between the parts of a living being with their similarity \citep{Emilsson:1988uq,Emilsson:2015wf}. Indeed, it is the link between sympathy and similarity that explains why a distant part may be affected without the parts between being affected. This will happen when only the distant part, but not the parts between, is suitably similar to the affecting part of the living being: ``For the like parts are not situated next to each other, but are separated by others between, but share their experiences (\emph{sympaschonta}) because of their likeness \ldots'' However, as we shall see, the unity of the ensouled living being is explanatorily prior to any likeness that may obtain between its parts.

Fourth, the similarity between the parts of the living being that may sympathetically interact must be suitably understood. Suppose that some part of the living being comes to be affected in a certain way. A potentially distant part of that same living being, because of its suitable disposition, may come to be affected in that way. Let \emph{F} be this way of being affected. The potentially distant part is initially not \emph{F}, but comes to be \emph{F}, by sympathetically interacting with the initial part's being \emph{F}. So the potentially distant part is, at the beginning of this process, only potentially like the initial part actually is. So the similarity condition should be understood, in the Peripatetic fashion, as the capacity to become like.

Finally, it is consistent with the account provided by this passage that there be considerable leeway in how the similarity condition is understood. So far, we have envisioned the initial part being \emph{F} and a potentially distant part becoming \emph{F} as a result of their sympathetic interaction. But the similarity condition might be understood more broadly than this. Perhaps because of the disposition of the parts, the initial part being \emph{F} induces in a suitably disposed, potentially distant part the affect \emph{G}, at least if \emph{G} is somehow suitably related to \emph{F}, if \emph{F} and \emph{G} are correlatives (in something like Aristotle's sense in the \emph{Categoriae}), or at least not incongruous. Think, for example, of fellow-feeling. Plotinus, like Hume and Smith after him, thinks that fellow-feeling is explained by sympathy operating between individuals. One person's suffering may, due to the operation of sympathy, cause in another the sentiment of pity, say. But the latter person's pity, even if it is like the first person's suffering in being a disagreeable sentiment, is a distinct affect. Pity may, in some sense, be the appropriate response to another person's suffering, and like it in being a disagreeable sentiment, but it is not their suffering reduplicated so much as a correlative response. There is another dimension along which the similarity condition may be generalized. Even if the subsequent affect is not correlative to the initial affect in this way, perhaps the subsequent affect may be like, if not exactly like, the affect of the initial part. There is some evidence that Plotinus himself exercised considerable leeway in understanding the similarity condition. The stars may affect the course of human affairs, but there is nothing in the stars that is very like their sublunary effects. Whatever Plotinus' considered view is, the passage, as it stands, is consistent with wider and narrower interpretations of the similarity condition, even when understood, in the Peripatetic fashion, as the capacity to become like.

Importantly, for our purposes, Plotinus uses sympathy to give an account of the distal senses, vision and audition (\emph{On Difficulties about the Soul \textsc{iii}}, \emph{or On Sight}, \emph{Ennead} 4 5, 4 6). Though that is his avowed intent, the bulk of the discussion concerns vision with Plotinus maintaining that a structurally similar account applies, as well, to audition. Vision and audition are distal senses. By means of them, the perceiver may become aware of the object of perception located at a distance. This is a remarkable fact, about which ancient thinkers devoted considerable ingenuity in explaining. An important part of what is at issue is the nature of the causal transmission between the distal object and the sensory organs of the perceiver. If that was all that was at issue, however, it would be of antiquarian interest only. We rightly believe that we have an approximately correct account of the causal transmission in distal perception involving, in the case of vision and audition, the propagation of light and sound waves. But, equally, part of what is at issue is not the causal influence of objects of perception located at a distance from the perceiver but a puzzle about their sensory presentation. As I emphasized at the outset, insofar as the distant object is present in our experience, we are tempted to say that we are in perceptual \emph{contact} with it, that we \emph{apprehend}, or \emph{grasp}, that object. However, insofar as that object is distant, we could not be in contact with it, at least not literally. So these ancient discussions concern, as well, what sensory presentation could be if it is not, indeed, tantamount to sensation by contact. In these ancient discussions, then, issues about causal transmission and sensory presentation are intertwined, which is not to say confused. The present point is important, not only for reading Plotinus on perception, but for the use I propose to put that reading. Recall, the present historical digression is in aid of the proposal that haptic presentation may be analytically explicated in terms of the operation of sympathy.

\citet[chapter 3]{Emilsson:1988uq} correctly emphasizes that sympathy, in Plotinus' account of vision, is meant to provide an account of how the distal object of vision affects the eyes. Thus the object of perception is the causal agent affecting the patient, the organ of perception. Since the object is distant, it cannot affect the sense organ by contact. And since, at least within the sensible cosmos, Plotinus views affection not involving contact to instead involve sympathy, it is natural for him to understand the distant object acting upon the organ of perception by means of sympathy. 

The principle obstacle to this line of reasoning concerns the invalidity of the inference from the object of perception not affecting the sense organs by contact to there being no affection by contact in the causing of that perception. The line of reasoning above seems to present us with a stark choice: either the object affects the sense organ by contact or by sympathetic affection. But consider just one alternative. Perhaps, as on the Peripatetic model, the object affects the sense organ only mediately, by affecting an intervening medium, that in turn affects the sense organ with which it is in contact. The Peripatetic model accepts that the distant object cannot be in contact with the perceiver's sense organ, but concludes from this, not the need to postulate a principle of action at a distance, but that causal transmission from the object of perception to the sense organ requires the existence of a suitable medium, in the case of vision, the illuminated transparent.

Plotinus is well aware of this obstacle and devotes considerable effort in criticizing accounts that postulate a medium and other alternatives (though, perhaps, not as clearly and conspicuously as Alexander of Aphrodisias criticizes his opponents). We shall not review Plotinus' critical discussion here, nor who his likely targets were (for discussion see \citealt[chapter 3.1]{Emilsson:1988uq}). However, I shall make an observation about just one of Plotinus objections:
\begin{quote}
	For if our perception resulted from the air being previously affected, when we looked at the object of sight we should not see it, but we should get our perception for the air which lay close to us, just as when we are warmed. (Plotinus, \emph{On Difficulties about the Soul \textsc{iii}}, \emph{or On Sight}, \emph{Ennead} 4 5 2 50--55; \citealt[289]{Armstrong:1984aa})
\end{quote}
Plotinus is claiming that if the affection of the perceiver's sense organ involves the intervention of the medium, then the perception that would result would present not some sensible aspect of the distal object but, rather, with some sensible aspect of the intervening medium. What is presently important is not the plausibility of Plotinus' claim (the full assessment of which would involve specifying his target and explaining his explanatory framework, something from which one may depart in varying degrees), but rather with how issues about the causal influence of the object of perception are bound up with issues about their sensory presentation. It is for this reason that I suspect that Emilsson goes too far in confining sympathy to explaining the action at a distance involved in visual perception. To be sure, sympathy provides Plotinus with such an account. But sympathy explains, as well, at least in part, how it is that we are presented with the distant visible object and not the intervening medium. Unfortunately, that explanation is never made fully explicit.

Plotinus concedes that perception would not be possible in the absence of an intermediary. But Plotinus insists that this is not because of the absence of a medium, but rather ``because the sympathy of the living being with itself and of its parts with each other'' would be disrupted (\emph{On Difficulties about the Soul \textsc{iii}}, \emph{or On Sight}, \emph{Ennead} 4 5 3 15--19). Insofar as the observation that perception is not possible in the absence of an intermediary is meant to motivate the postulation of a medium, what reason it provides should be understood on the model of inference to the best explanation. If that is right, then the fact that Plotinus has provided an equally credible alternative explanation means that the reason for the postulation of a medium is, to that extent, undermined. But why should we prefer Plotinus' alternative? To address this, Plotinus provides the following thought experiment:
\begin{quote}
	if there was another universe, that is another living being making no contribution to the life of this one, and there was an eye ``on the back of the sky'', would it see that other universe at a proportionate distance? (Plotinus, \emph{On Difficulties about the Soul \textsc{iii}}, \emph{or On Sight}, \emph{Ennead} 4 5 3 21--24; \citealt[293]{Armstrong:1984aa})
\end{quote}
 The eye on the back of the sky is an image Plotinus derives from Plato's \emph{Phaedrus}:
\begin{quote}
	When [the gods] go to feast at the banquet they have a steep climb to the high tier at the rim of heaven \ldots\ when the souls we call immortals reach the top, they moved outward and take their stand on the high ridge of heaven, where its circular motion carries them around as they stand while they gaze upon what is outside heaven. (Plato, \emph{Phaedrus} 247 b1--c2; Nehemas and Woodruff in \citealt[525]{Cooper:1997fk})
\end{quote}
Like the gods feasting at their banquet, the eye on the back of the sky is looking outward, beyond the confines of the sensible cosmos (``What is in this place is without color and without shape and without solidity \ldots'' \emph{Phaedrus} 247 c 6--7; Nehemas and Woodruff in \citealt[525]{Cooper:1997fk}). Sympathy only operates within the unity provided by the soul of a living being. Since the soul of the other living being, a sensible cosmos distinct from the one within which we reside, makes no contribution to the life of this one, understood as our sensible cosmos, the parts of that other living being cannot sympathetically affect the parts of this one. They eye on the back of the sky fails to see the other universe, a sensible cosmos, at a proportionate distance, not because of the intervening void, but because the unity that makes a sympathetic response possible does not obtain between the eye in this sensible cosmos and any of the parts in the other sensible cosmos. So the eye on the back of the sky thought experiment is meant to be a case where there is no intermediary, but sight fails, not because of the absence of a medium, but because the conditions that make possible sympathetic interaction do not obtain.

Plotinus devotes the final chapter of that treatise to elaborating the thought experiment (\emph{On Difficulties about the Soul} \textsc{iii}, \emph{or On Sight}, \emph{Ennead} 4 5 8). His discussion is compact and often obscure. So a reasonable treatment of that chapter would require a close exegesis. However, I want to draw our attention to one aspect of his discussion that bears on the explanatory priority of the unity of the soul. Specifically, Plotinus denies that the similarity between the parts of the living being are sufficient to explain their sympathetic interaction. So, on the view that Plotinus opposes, one part's being \emph{F} sympathetically causes another part to become \emph{F}, say, not because they are parts of a single ensouled living being, but because of the similarity between them, understood, in the Peripatetic fashion, as the capacity to become like. Notice that if the similarity condition alone suffices for the operation of sympathy, then the eye on the back of the sky should be able to see, at a proportionate distance, the visible aspects of that other sensible cosmos, if these are suitably similar to the visible aspects of the sensible cosmos within which we reside. Plotinus, however, doubts that the visible aspects of that other cosmos would be sufficiently similar to visible aspects of our own for a capacity to become like to ground the eye's perception of the other sensible cosmos:
\begin{quote}
	Now the objects apprehended are apprehended in this way by being like, because this soul [of the universe] has made them like, so that they are not incongruous; so that if the active principle out there is the altogether different soul [of that other universe], the objects assumed to exist there would be in no way like the soul of our universe. (Plotinus, \emph{On Difficulties about the Soul \textsc{iii}}, \emph{or On Sight}, \emph{Ennead} 4 5 8 26--31)
\end{quote}
What this passage brings out is the way in which the unity of the soul is explanatorily prior to the similarity condition. Within a single living being, because of the unity provided by the soul of that living being, parts that are suitably disposed to become like may sympathetically interact. Similarity, subject to the qualifications previously discussed, may be a condition on sympathetic affection, but is insufficient to explain that affection. And this is so because the soul, the active formative principle of the living being, makes its parts like or unlike depending upon the coherence and function of the whole. While it remains difficult to understand why, for Plotinus, there could be no duplicate cosmoi, his reasoning here clearly presupposes that the unity of the soul is explanatorily prior to the similarity between the parts of the living being that sympathetically interact.

% section plotinus (end)

\section{The Principle of Haptic Presentation} % (fold)
\label{sec:sympathy_as_the_principle_of_haptic_presentation}

In grasping or enclosure, haptic perception is the joint upshot of forces in conflict. On the one hand, there is the force exerted in molding the hand more precisely to the contours of the rigid, solid body. On the other hand, there are the self-maintaining forces of the rigid, solid body itself. Haptic perception is the joint upshot of the force exerted by the grasping hand and the self-maintaining forces of the object grasped. In resisting the force of the hand's activity, the self-maintaining forces that constitute the body's rigidity and solidity present these qualities in haptic awareness. In resisting the hand's encroachment, the hand, and the haptic experience it gives rise to, assimilates to the overall shape and volume of the the object grasped. And haptic experience's assimilation to its object, relative to the perceiver's haptic perspective, is a kind of constitutive shaping. The conscious character of that experience depends upon and derives from the qualitative character of the tangible object as presented to the perceiver's haptic perspective, an event in peripersonal space, the distinctive manner in which they are handling that object in the given circumstances of perception.

Perception places us in the very heart of things. In being present in our perceptual experience, they constitutively shape that experience, at least relative to the our partial perspective on things. It is for this reason that \citet{Ardley:1958aa} describes perception as a ``communion'' with its object. In an episode of perception, the perceiver is united with the object of perception. Perceptual presentation is a distinctive kind of unity. It follows that haptic presentation is itself a kind of unity and more distinctive still. So in feeling the overall shape and volume of the stone in their grasp, our hominid ancestor is united with tangible aspects of that external body.

Just as the Stoics thought that the unity of the sensible cosmos was explicable in terms of tensional motion in the all pervading \emph{pneuma}, the synthetic approach claims that the unity involved in haptic presentation is itself subject to further explanation. However, in proceeding analytically rather than synthetically, the unity of the perceiver and the object grasped in haptic presentation is explanatorily prior to whatever intelligible structure it must display. The analytic approach thus shares at least this much with Plotinus' account. It thus contrasts with any account that would make the unity involved in haptic presentation subject to further explanation in terms of elements and principles understood independently of haptic perception. 

So far, then, we have two important features of Plotinus' account of sympathy in play, namely, that sympathy only operates within a unity and the irreducibility of that unity. What of the similarity condition? In chapter~\ref{sec:active_wax}, we discussed how haptic perception involves a kind of formal assimilation. We observed that the hand formally assimilates to the overall shape and volume of the object grasped in the sense that the shape of the hand's interior becomes like, if not exactly like, the shape of the object grasped, and that the volume of the region that the hand encloses becomes like, if not exactly like, the volume of the object grasped. Not only does the hand formally assimilate to the object grasped, but the experience that the grasping hand gives rise to itself becomes like, if not exactly like, the tangible object presented in it, at least relative to the perceiver's haptic perspective. Moreover, the formal assimilation of the hand, and the haptic experience that it gives rise to, should be understood, like Plotinus' similarity condition, on the Peripatetic model. The hand, the mobile and elastic instrument of haptic perception, only approximates the overall shape and volume of the object grasped in grasping. It thus has the capacity to become like the object grasped in these respects. Similarly, the perceiver possesses the capacity for their haptic experience to become like whatever object is presented in it, relative to their haptic perspective, the distinctive manner in which they are handling that object, in the given circumstances of perception. 

We saw in our discussion of the eye on the back of the sky thought experiment that Plotinus understood the unity of the sensible cosmos to be explanatorily prior to the capacity for its parts to become like or unlike one another. It is not just that the unity is not subject to further explanation, but that the unity explains, as well, the similarity condition. It is because of the unity provided by the World-Soul that potentially distant parts of the sensible cosmos that are suitably disposed to become like or unlike may sympathetically interact. The parts of the living being are so arranged that their being suitably disposed to become like or unlike is explained by the function and coherence of the whole. A similar pattern of explanation is in play in the case of haptic perception. Recall, at least the formal assimilation at work in haptic perception was understood as a kind of constitutive shaping. Not only does the perceiver's haptic experience formally assimilate to its tangible object relative to their haptic perspective, in the sense that the conscious qualitative character of the experience is like, if not exactly like, the qualitative character of the tangible object present in it, but the tangible qualities present in their haptic experience constitutively shapes that experience. If, in grasping, the perceiver feels the overall shape and volume in the object, then not only is this because of the object's overall shape and volume, but its feeling that way is also constituted, in part, by the overall shape and volume felt. But the constitutive shaping of haptic experience by its object is a ``communion'' with that object---in undergoing that experience the perceiver is united, in a way, with the object of their perception. Moreover, as with Plotinus, this unity explains, in part, the similarity between the haptic experience and its tangible object. The formal assimilation of haptic perception to its object, at least relative to the perceiver's haptic perspective, is the effect of constitutive shaping, and thus its conscious character depends upon and derives from, at least in part, the corporeal character of the object grasped.

So far, then, we have seen that four key elements of Plotinus' account of sympathy are in play in the haptic case. Now let us turn to the differences. Let me focus on three. 

First, for Plotinus, like the Stoics before him, sympathy is primarily a principle of action at a distance. One of Plotinus' innovations was the application of such a principle in accounting for the distal senses of vision and audition. But haptic perception, and touch more generally, is not a distal sense, at least not in this way. Does this mean that a principle of sympathy is inapplicable in the haptic case? No. Rather, the application to the haptic case is a natural generalization. Consider one of Cicero' examples of Stoic sympathetic affection, the resonance of strings of a lyre (\emph{De Divinatione} 2 34, \emph{Stoicorum Veterum Fragmenta} 2 1211). When some strings of a lyre are struck, others resonate. The strings, however, would resonate even if they were in contact with the strings that were struck. And if we suppose, with the Stoics, that their resonance was a result of sympathetic affection when they were at a distance, then their resonance would remain the result of sympathetic affection even when in contact. So understood, sympathy is a principle that merely allows action at a distance. In a way, this is the converse point of the eye on the back of the sky thought experiment. The lesson of that thought experiment was meant to be that from the absence of perception in the absence of an intermediary, we should not infer that a medium is required for perceptual transmission. Similarly, from the presence of contact in some cases of resonance, we should not infer that contact is required for these resonant affections.

In moving from self to other, the first step is the biggest. And this remains true regardless of whether the the other is contiguous with the perceiver or located at a distance from them. Indeed, sympathy was invoked to distinguish cases where felt resistance to the hand's activity was due to an internal limitation (such as the inability to stretch one's index and middle finger past a certain point) from cases where the felt resistance was due to an external limitation (such as the self-maintaining forces that constitute the categorical bases of an external body's rigidity and solidity). It is because we were puzzled, in a way that Plotinus was not, about how the limitation to the hand's activity could disclose the presence and tangible qualities of an external body, that is was natural for us to appeal to sympathy to resolve such puzzlement. The first difference, then, is merely a generalization of the Plotinian account, though a generalization prompted by a problem that Plotinus never considered.

The second and third differences are, perhaps, more of a departure from our ancient sources. Plotinus' account, not fully described here, sympathy merely playing a role in a more complex phenomena, was intended as an alternative to the Peripatetic account, at least as he understood it. Plotinus knows well and understands Alexander of Aphrodisias' Peripatetic philosophy, but his fruitful engagement with Alexander's philosophy was nonetheless the critical engagement of a rival. The present appropriation of Plotinus' notion of sympathy in explaining the haptic presentation of an external body is not, however, a self-conscious alternative to the Peripatetic account. Rather, it is, perhaps, better understood as a neo-Platonic elaboration of what is, essentially, a Peripatetic account of perception. Specifically, insofar as the assimilation of sensible form can be understood on the model of constitutive shaping, we have retained the hylomorphic account of sensory presentation from \emph{De Anima} 2 (at least on a certain interpretation \citealt{Kalderon:2015fr}). Plotinian sympathy was only invoked to elaborate the intelligible structure of the haptic presentation of an external body and its tangible qualities. So unlike Plotinus' account, the present account is not an alternative to, but an elaboration of, what is, essentially, a Peripatetic account of perception.

The third difference is also a departure from our ancient sources. Like the Stoic account of sympathy, Plotinus' account is set in the context of a vitalistic metaphysics. However, while there may be deep, if controversial, reasons for thinking that the unity that grounds the operation of sympathy is an organic unity, I propose, instead, to simply drop the vitalist metaphysics, or, at the very least, remain agnostic about it (for a contemporary, Anglophone expression of sympathy for vitalist metaphysics see \citealt{Nagel:2012as}). What is presently important is that it is because of the unity of the perceiver with the object grasped that the felt resistance to the force of the hand's activity is a sympathetic response to the self-maintaining forces of the object grasped. So it is the unity of the perceiver and the object grasped along with the capacity for their haptic experience to become like, if not exactly, like the tangible qualities presented in that experience, relative to the perceiver's haptic perspective, that grounds the operation of sympathy in haptic perception. I simply decline to follow the Stoics and Plotinus in explicitly conceiving of that unity to be the unity of a living being.

Earlier, I mentioned how one potential obstacle to appreciating that haptic presentation is a kind of sympathetic response to an external body is the emotional associations of our contemporary conception of sympathy. Sympathy, as we nowadays tend to conceive of it, is a kind of fellow-feeling akin to compassion or pity. The Plotinian account, however, revealed that sympathy can be understood with sufficient generality to be at work in both fellow-feeling and perception. Plotinus understood the operation of sympathy to be at work in fellow-feeling and perception as well as in a number of other natural phenomena not explicable in terms of affection by contact, at least by Plotinus' lights. Thus in analytically explicating haptic presentation in terms of the operation of sympathy we need not thereby understand haptic perception as an outgrowth of blind emotion the way \citet[162--3]{Whitehead:1978zr} did. Nevertheless, in understanding haptic presentation as the sympathetic presentation of an external body and its tangible qualities by felt resistance, the present account has the resources to distinguish blows from caresses as \citet{Derrida:2005aa} recommends. Our sympathetic interaction with the object of our hatred (where sympathy, here, is understood more broadly than, as we might colloquially say, feeling sympathy for them) naturally differs from our sympathetic interaction with our beloved. Our sympathetic response to contact with an enemy will naturally differ in character from our sympathetic response to contact with the beloved. And there is a natural tendency for the character of our sympathetic response to be expressed in the haptic activities that sustain them. Our anger is expressed by the blows that present an enemy, just as our love is expressed by the caresses that present the beloved. Perception may not reduce to blind emotion, but that is consistent with certain natural affective responses being made possible and, indeed, partly constituted by the operation of sympathy in haptic presentation. So without reducing haptic perception to blind emotion, in understanding haptic presentation as the sympathetic presentation of an external body and its tangible qualities, the distinction between blows and caresses is rendered intelligible, at least in principle.

Sympathy is the principle of haptic presentation. That principle was invoked to resolve the puzzle with which the previous chapter ended. Recall, that puzzle was a failure of sufficiency. How, in the case of haptic perception, can felt resistance to the hand's activity disclose the presence and tangible qualities of an external body when not all limitations to the body's activity are due to external bodies? How is it possible for felt resistance to the hand’s activity in grasping or enclosure to disclose a rigid, solid body’s overall shape and volume? If feeling tangible qualities in something external to the perceiver’s body and in conformity with them is due to the operation of sympathy then we have a basis for an answer. It is when the limit to hand’s activity is experienced as a sympathetic response to a countervailing force, as the hand’s force encountering an alien force resisting it, one force in conflict with another, like it yet distinct from it, that the self-maintaining forces of the body disclose that body’s presence and tangible qualities to haptic awareness.

If sympathy is the principle of haptic presentation, as I suggest that it must be, at least as analytically explicated, then the perceiver's experience of felt resistance to their hand's activity could not be explicit. Explicit awareness of the hand's configuration and force would draw attentive resources away from the object grasped. If our hominid ancestor explicitly attends to the intensive sensations involved in grasping a stone, such that these are open for epistemic appraisal, then they would no longer be attending to the stone and its tangible qualities. Moreover, this would be a consequence of sympathy being the principle of haptic presentation. In order for grasping or enclosure to directly disclose the overall shape and volume of the stone, the felt resistance to the force of the hand's activity must be experienced as a sympathetic response to the self-maintaining forces that constitute the categorical bases of the stone's rigidity and solidity. In this way they feel the rigidity and solidity in an object external to their body. Consciously attending to the hand's activity would erode the sympathetic presentation of the tangible qualities of an external body. So we could not be explicitly aware of the hand's activity in grasping or enclosure, understood as a mode of haptic perception, if sympathy were the principle of haptic presentation.

But that is not to say that our hominid ancestor is unaware of their hand's activity in grasping a stone. Reflection on perceptual constancy (sections~\ref{sec:haptic_perception}, \ref{sec:active_wax}, \ref{sec:the_dependence_of_haptic_perception_upon_bodily_awareness}) revealed that the phenomenological character of their haptic experience could not be exhausted by the object of explicit awareness. An implicit awareness of the hand's configuration and force contributes, as well, to the phenomenology of their haptic experience. Our hominid ancestor's sense of their hand’s configuration and force contributes only to the pre-noetic structure of their haptic experience by determining the way its object is presented therein. So not only do they feel the overall shape and volume in the stone, but their hand is felt to conform to these tangible qualities as well. Feeling the hand to conform to the stone's rigidity and solidity may be implicit, it may be recessive and in the background, so that it does not compete for attentive resources directed toward an external body, but it contributes to the conscious character of their haptic experience by being the way in which the overall shape and volume of the stone is presented in that experience. Haptic presentation in grasping or enclosure just is feeling something in an external body and in conformity with it. And feeling something in an external body and in conformity with it just is the exercise of a sympathetic capacity.

Haptic presentation is an irreducible unity. If sensory presentation is a distinctive kind of unity, then haptic presentation is more distinctive still. What distinguishes haptic presentation as the kind of unity it is is the intelligible structure it displays. If sympathy is the principle of haptic presentation, then haptic presentation, the kind of unity that it is, is a mode of \emph{being with} (which is not to say that it is a mode of \emph{mitsein}, in Heidegger's sense). Feeling the overall shape and volume in the stone and in conformity with it is a way of being with the stone in one's grasp. Grasping or enclosure, understood as a mode of haptic perception, involves the embodied perceiver consciously being with the body in its grasp. So the mode of being with involved in haptic presentation is corporeal, a way for one body to be with another. Moreover the mode of being with involved in haptic presentation is conscious. It is a way for a particular kind of body, a conscious animate body, to be with an external body encountered in peripersonal space.

In the last chapter I claimed that while the formal assimilation of haptic experience to its object, understood on the model of constitutive shaping, was a manifestation of the objectivity of haptic perception, it was the force of the hand's activity that was its source. In focussing exclusively on the role of sympathy in Plotinus' account of perception, we have ignored a crucial aspect of that account, one that highlights the activity of the perceiver:
\begin{quote}
	It is clear in presumably every case that when we have a perception of anything through the sense of sight, we look where it is and direct our gaze where the visible object is situated in a straight line from us; obviously it is there that the apprehension takes place and the soul looks outwards. (Plotinus, \emph{On Sense-Perception and Memory}, \emph{Ennead} 4 6 1 14--18; \citealt[321]{Armstrong:1984aa})
\end{quote}
And later, Plotinus generalizes the point:
\begin{quote}
	[The soul] speaks about things which it does not possess: this is a matter of power, not of being affected in some way but of being capable of and doing the work to which it has been assigned. This is the way, I think, in which a distinction is made by the soul between what is seen and what is heard, not if both are impressions, but if they are not by nature impressions or affections, but activities concerned with that which approaches [the soul]. (Plotinus, \emph{On Sense-Perception and Memory}, \emph{Ennead} 4 6 2 1--7; \citealt[325]{Armstrong:1984aa})
\end{quote}
Plotinus thus stands at the head of a historical tradition that stresses the active nature of perception and includes Augustine, Kilwardby, Olivi, Fichte, Maine de Biran, Ravaisson, Bergson, Merleau-Ponty, and contemporary enactivists (for a partial overview of this historical tradition see the essays in \citealt{Silva:2014cl}). 

We may retain, from this tradition, an important insight. Specifically, we are now in a position to fully appreciate why if the formal assimilation of haptic experience to its object, relative to the perceiver's partial perspective, is the manifestation of the objectivity of haptic perception, being a mode of constitutive shaping, it is the force of the hand's activity that is its source. The force of the hand's activity, and the felt resistance it encounters, is a precondition for sympathy's partial disclosure, relative to the perceiver's handling, of the self-maintaining forces of an external body. It is the hand, the mobile and elastic instrument of haptic exploration, the active wax of haptic perception, whose activity must be resisted, by the passivities of matter, in order to sympathetically present the external body whose self-maintaining forces constrain that activity. The felt resistance to the hand's activity in grasping or enclosure, understood as a mode of haptic perception, is an event occurring in an ego-centrically and teleologically structured peripersonal space that partly discloses corporeal aspects of the object of haptic investigation. It is for this reason that the perceiver's handling of the object counts as a perspective on that object, albeit a distinctively haptic perspective. The hand's activity in peripersonal space constitutes, in part, the haptic perspective to which the object is sympathetically presented. Thus the activity of the hand, of which we are merely implicitly aware, is the source, nevertheless, of the objectivity of haptic perception because it is a precondition for the sympathetic presentation of the tangible object that constitutively shapes that haptic experience.




% section the_principle_of_haptic_presentation (end)

% chapter sympathy (end)

