%!TEX root = /Users/markelikalderon/Documents/Git/sympathy/perception.tex
\chapter{Sympathy} % (fold)
\label{cha:sympathy}

\section{The Metaphysics of Haptic Presentation} % (fold)
\label{sec:the_metaphysics_of_haptic_presentation}

Tactile metaphors for perception are primordial and persistent. What makes grasping an apt metaphor for perceptual awareness, even for non-tactile modes of awareness such as vision an audition? In order to answer this question, we undertook a phenomenological investigation into the nature of haptic perception. That investigation was phenomenological in that it confined itself to perceptual appearances and not because of any methodology involved. The hope was that if we better understood how grasping or enclosure, understood as a mode of haptic perception, presents itself from within haptic experience, then we would be in a better position to understand what potentially makes grasping an apt metaphor for perception generally. We discussed two claims about the metaphysics of haptic presentation:
\begin{enumerate}
	\item Tangible qualities of the object of haptic exploration are disclosed over time and so presentation in haptic experience has duration.
	\item Haptic experience formally assimilates to, is constitutively shaped by, the tangible qualities presented to the perceiver's haptic perspective, understood as the distinctive way they are handling the object.
\end{enumerate}
Not only does the hand assimilate to the contours of the object grasped, but the haptic experience that this activity gives rise to itself formally assimilates to its object. Moreover the formal assimilation of haptic experience to its object relative to the perceiver's haptic perspective is not merely causal but constitutive. This, I suggested, was the basis of haptic perception's objectivity and part of what makes it an apt metaphor for perception generally. While haptic experience assimilating to its object is a manifestation of the objectivity of haptic perception, it is not its source. The presentation of tangible qualities of objects external to the perceiver’s body is due, at least in part, to the activity of the hand in grasping and the resistance it encounters. The hand, and haptic experience in turn, only assimilate to the tangible aspects of the rigid, solid body thanks to the force of the hand’s activity in conflict with the self-maintaining forces of that constitute the categorical bases of that body’s solidity and rigidity. At least with grasping or enclosure, understood as a mode of haptic perception, perceptual realism is sustained by the force of the hand’s activity in conflict with the self-maintaining forces of the object grasped. In this way, the hand is the active wax of haptic perception.

However this last insight, if it is one, gave rise to the puzzle that arose at the end of the last chapter. That puzzle revealed no genuine incoherence in the Manifest Image of Nature. The puzzlement was not meant to be the basis of skepticism about objective haptic perception so much as the basis of a how-possible question, how is objective haptic perception so much as possible? The puzzle began with haptic perception's dependence upon bodily awareness. For animals like ourselves bodily awareness is a mode of self-presentation even if there is more to our nature than is revealed in bodily awareness. But how can a mode of self-presentation disclose the presence of some other thing? How is it that bodily awareness is leveraged in haptic perception into disclosing the presence and tangible qualities of an external body? Taking on board Kilwardby's transformed insight that the presentation of tangible qualities of an external body is due, at least in part, to the activity of the hand and the felt resistance it encounters, we refined our how-possible question: How does felt resistance to the hand's activity in grasping or enclosure disclose the overall shape and volume of an external body? How does the experienced limitation to the hand's activity allow the perceiver to feel something in something external to the perceiver's body and in conformity with it?

Reflection on this puzzle or \emph{aporia} shall be the basis for further substantive claims about the metaphysics of haptic presentation. The present chapter thus proceeds dialectically. Chief among the substantive claims to be made on this basis is the perhaps surprising claim that haptic presentation is governed by the principle of sympathy---that feeling something in another thing and in conformity with it is explicable in terms of the operation of sympathy. 

% section the_metaphysics_of_haptic_presentation (end)

\section{The Dependence upon Bodily Awareness} % (fold)
\label{sec:the_dependence_of_haptic_perception_upon_bodily_awareness}

Since our puzzle begins with the dependence of haptic perception upon bodily awareness, perhaps getting clearer on the nature of that dependence will help with its solution. 

In a chapter devoted to discussing the nature of this dependence, \citet[chapter 4.6]{Fulkerson:2014ek} draws the distinction between implicit and explicit experiences:
\begin{quote}
	An implicit bodily experience is one that is the background or recessive. ``Background'' here can be understood as an experiential content that is not consciously attended, in the minimal sense that it does not allow its objects to be open for epistemic appraisal. Such unattended contents or experiences do not incur an additional attentional load on our conscious experiences (we can only actively attend to a limited number of items at any one time, but implicit experiences do not add to this threshold). However, they are in consciousness nonetheless, primed for attention. \citep[90]{Fulkerson:2014ek}
\end{quote}
Explicit experiences, in contrast, involve attending to, or actively thinking about, the object of that experience. 

With the distinction drawn between implicit and explicit experience, we may ask whether haptic perception involved in grasping or enclosure depends upon an explicit bodily experience of the hand's configuration and force, or whether the presentation of the object's overall shape and volume in haptic experience merely depends upon an implicit experience of the hand's configuration and force? If the bodily experience upon which haptic perception depends is explicit, then the perceiver consciously attends to the state and activity of the body and haptic perception of the tangible qualities of an external body depends upon this explicit bodily experience. Fulkerson calls this Strong Experiential Dependence. On the hypothesis of Strong Experiential Dependence, the haptic perception involved in grasping or enclosure, a conscious experience, depends upon another conscious experience, specifically, of the hand's configuration and force.

\citet[chapter 4.8]{Fulkerson:2014ek} argues, instead, that the dependence is best understood in terms of what he calls Informational Bodily Dependence. Though information from processes that underly proprioception and kinesthesis are integrated with afferent information, such as the information provided by cutaneous activation, these give rise to a single conscious experience. The idea is that the sensitivity exhibited by haptic perception, such as grasping or enclosure, depends upon the tactile system drawing upon functionally distinct streams of information involved in bodily awareness. Nevertheless, the percept that is thereby determined is a single conscious experience, in the case of grasping or enclosure, our feeling of the overall shape and volume of the object grasped. This contrasts with Strong Experiential Dependence where conscious haptic experience is understood to depend upon a distinct conscious experience of the body's configuration and motion. On the alternative, conscious haptic experience depends upon, not an explicit, but an implicit experience of the hand's configuration and force. \citet[91]{Fulkerson:2014ek} cites with approval \citet[137]{Gallagher:2005ag} in this regard: ``Our pre-reflexive, kinesthetic-proprioceptive experience thus plays a role in the organization of perception, but in a way that does not require the body itself to be a perceptual object.'' If we understand the perceptual object as something that is actively attended to, then haptic experience merely depends upon an implicit experience of the hand's configuration and force (see also \citealt{Bower:2013aa}).



% Put another way, our capacity for haptic perception draws upon our distinct capacities for proprioception, kinesthesis, motor activity, and our sense of agency in its exercise but its exercise is an experience that affords the perceiver awareness of the presence and tangible qualities of an object external to the perceiver's body.

Campbell in his contribution to \citet[54]{Campbell:2014aa} cites Huang's and Pashler's \citeyearpar{Huang:2007jk} distinction, in visual attention, between selecting something out from its background and characterizing or accessing its features: 
\begin{quote}
	So a property may be used to \emph{select} the object or region. Or the property may be \emph{accessed} as a property of that object or region. Selection is what makes the object or region visible in the first place; selection is what makes it possible for the subject to focus on that objector region in order to ascertain its various properties. Access is a matter of the subject making it explicit, in one way or another, just which manifold properties the object or region has.
\end{quote}
Tactile perception, like visual and auditory perception, involves grouping, segmentation, and recognition. Suppose, then, that this distinction can be drawn, not only within visual attention, but also within tactile and, specifically, haptic attention. So a property may be used to select an object or region for active attention in haptic exploration or a property may be accessed in conscious haptic experience as a property of that object or region. With Huang's and Pashler's distinction in mind, and supposing it may legitimately apply to haptic attention as well, Fulkerson's notion of an explicit experience is characterized in terms of our accessing its object---it is  consciously attended to and open for epistemic appraisal. Now suppose Campbell is right in thinking that a property may be used to select an object in visual attention but not be accessed in consciously attending to it \citep[chapter 3.2]{Campbell:2014aa}. And suppose, further, that this possibility is a consequence of the distinction Huang and Pashler introduced, so that, if it holds, as well, for haptic attention, then there should be cases of selecting an object or region for haptic attention without consciously attending to the tangible quality on the basis of which that object or region was selected. Since explicit experiences are a matter of accessing their objects, then our haptic experience of a tangible quality that selected the body or region but was not consciously attended to would be an implicit experience of that quality.

This is the basis of a worry for a further claim Fulkerson makes about implicit experiences. There is a sense in which, for \citet[91]{Fulkerson:2014ek}, implicit experiences are no experiences at all. The content of an implicit experience is merely the content of a potential, that is to say, non-actual, experience \citep[95]{Fulkerson:2014ek}. And there is an associated tendency in Fulkerson's discussion to identify conscious experience with what is attended to and accessed, with explicit experience. But if the presence of a tangible quality is the basis for the selection of an object or region in haptic exploration, then surely it contributes to the phenomenological character of the haptic experience even if it is not consciously attended to. If that same object or region were selected on the basis of a different tangible quality, the subsequent experience would differ in phenomenological character. For \citet[95]{Fulkerson:2014ek}, the objects of implicit awareness are there for ``potential directedness''. But if we can voluntarily selectively attend to something about which we are implicitly aware, it must be present already in our experience, however recessively and in the background, if it can be thus selected.

The possibility raised by Huang's and Pashler's distinction between selection and access concerns the implicit experience of tangible qualities of external bodies. Our present, focus, however, is not on implicit experiences of external bodies but on implicit experiences of the perceiver's body. But here too it seems implausible that my awareness of my hand's configuration and force in grasping or enclosure, understood as a mode of haptic perception, while implicit, is merely potential and, thereby, non-actual. The information drawn upon from proprioception, kinesthesis, motor activity, and our sense of agency in haptic perception makes a contribution to the phenomenological character of that experience, even if there is, as Fulkerson urges, only one conscious experience (the haptic experience) in play and not two (the haptic experience and a distinct experience of the body's state and activity). The information from bodily awareness drawn upon in the exercise of our haptic capacities specifically makes a difference to the way the object of haptic awareness is presented. As I argued in chapter~\ref{sec:active_wax}, distinct exploratory activities, distinct ways of handling the object of haptic exploration, constitute distinct haptic perspectives on that object, and this perspectival relativity is manifest in the different haptic appearances presented by the constant object of haptic exploration. It is one thing to claim that bodily awareness makes no explicit contribution to haptic experience. In grasping or enclosure, understood as a mode of haptic perception, we attend only to the object grasped and its manifest tangible qualities. But it is a further, contestable claim, that bodily awareness, however implicit, contributes nothing to the phenomenological character of the haptic experience it partly gives rise to. Bodily awareness, however implicit, contributes to the variable haptic appearances in the exercise of constant haptic perception. If the phenomenological character of haptic experience were exhausted by the constant tangible qualities attended to, then no room would be left for the contribution of flux to our haptic experience. But an adequate account of perceptual constancy must determine not only the constant object of perception but its variable appearances as well. In grasping or enclosure, understood as a mode of haptic perception, haptic experience is the joint upshot of the force of the hand's activity and the self-maintaining forces of the object grasped. Constant tangible aspects are presented in haptic experience as the forces that constitute their categorical bases come into conflict with force of the grasping hand. And the variable appearances of these constant tangible aspects are a phenomenological reflection of the variable activity of the hand in haptic exploration.

% section the_dependence_upon_bodily_awareness (end)

\section{Sympathy} % (fold)
\label{sec:sympathy_and_haptic_perception}

When our hominid ancestor reaches out and picks up a rough-hewn stone, perhaps in preparation to skirmish with a competing group of hominids, they feel the overall shape and volume of the stone in their grasp. It is not the hand's shape, the configuration of the hand in grasping or enclosure, that they haptically perceive though they may be aware of it, however implicitly. It is the stone's shape that is disclosed in their grasp. They feel the overall shape and volume in the stone, and its overall shape and volume are tangible qualities of the stone that their hand is felt to conform to. I shall make a suggestion that will be the basis for an answer to our refined how-possible question. Specifically, feeling tangible qualities in something external to the perceiver's body and feeling in conformity with them can fruitfully be understood as due to the operation of sympathy. 

One obstacle to appreciating this concerns out present understanding of sympathy, where sympathy is a kind of emotional response to others, a kind of fellow-feeling, akin to compassion or pity. The notion of sympathy that is being invoked as the principle governing haptic presentation is closer to the notion at work in Stoic physics, if more abstract and not at all reliant on on their vitalistic metaphysics. The present approach thus contrasts with Whitehead's \citeyearpar{Whitehead:1978zr}. Whitehead both explains perceptual prehension partly in terms of sympathy and embraces the association with emotion:
\begin{quote}
	The primitive form of physical experience is emotional---blind emotion---received as felt elsewhere in another occasion and conformally appropriated as a subjective passion. In the language appropriate to the higher stages of experience, the primitive element is sympathy, that is, feeling the feeling in another and feeling conformally with another. \ldots\ The separation of the emotional experience from the presentational intuition is a high abstraction of thought. Thus the primitive experience is emotional feeling, felt in its relevance to a world beyond. The feeling is blind and the relevance is vague. \citep[162-3]{Whitehead:1978zr}
\end{quote}
Whitehead's retention of the emotional associations of sympathy lead him to paradoxically portray perceptual prehension as an outgrowth of blind emotion. However, as we shall see, the principle of sympathy can be understood with sufficient generality so that it may be at work both in haptic presentation and fellow-feeling, without reducing perceptual presentation to blind emotion. Perception may not reduce to blind emotion, but that is consistent with certain natural affective responses being made possible and, indeed, partly constituted by the operation of sympathy in haptic presentation. It would have to be, if, as \citet[chapter 4]{Derrida:2005aa} insists, an adequate philosophy of touch must leave room for both blows and caresses.

Felt resistance to touch, insofar as it is the presentation of an object external to the perceiver's body, is a sympathetic response to the force that resists the hand's activity. Recall our refined version of our how-possible question was this: How is it possible for felt resistance to the hand's activity in grasping or enclosure to disclose a rigid, solid body's overall shape and volume? If feeling tangible qualities in something external to the perceiver's body and in conformity with them is due to the operation of sympathy then we have a basis for an answer. It is when the limit to hand's activity is experienced as a sympathetic response to a countervailing force, as the hand's force encountering an alien force resisting it, one force in conflict with another, like it yet distinct from it, that the self-maintaining forces of the body disclose that body's presence and tangible qualities to haptic awareness.

Earlier, the initial statement of the puzzle was motivated by considering the analogy of felt temperature. We contrasted two cases. In both cases you feel warm, and you feel warm to the same degree. But in the first case, you feel warm because of a fever, and in the second case, you feel warm because because of the ambient heat. There is also, importantly, a phenomenological difference between these cases. In the second case, not only do you feel warm, but you feel, as well, the warmth in the ambient air. Indeed, the warmth you feel is in conformity with the warmth felt in the ambient air. What explains the phenomenological difference is that in the second case, but not in the first, the felt warmth is a sympathetic response to the ambient heat, to the thermal properties of something external to the perceiver's body. In sympathetically responding to ambient heat, the warmth you feel becomes a way of feeling the warmth in something located outside of your body. Moreover, in sympathetically responding to ambient heat, the warmth you feel is in conformity with the warmth felt in the air.

The proposal is that presentation in haptic perception is governed by the principle of sympathy. There are two ways to understand this. The fist proceeds synthetically. That is, beginning with elements and principles understood independently of haptic perception, one constructs the notion of the presentation of tangible qualities of external bodies in haptic experience on their basis. So, for example, one might begin with bodily sensation and ``extend its reach'', so to speak, via the operation of sympathy to construct a notion of the presentation of tangible qualities of external bodies. So understood, haptic presentation would be the coordination of bodily sensations with the tangible qualities of external bodies via the operation of sympathy. The second way proceeds analytically. That is, beginning with the notion of the presentation of tangible qualities of external bodies in haptic experience, one analyses or decomposes that notion into constituent elements that must be present and principles that must be operative if haptic perception is so much as possible. 

The synthetic approach naturally, perhaps inexorably, motivates indirect realism about tactile perception. So consider again our toy model where we begin with bodily sensation and extend its reach through the operation of sympathy. Bodily sensation does not involve the presentation of tangible qualities of external bodies. It is, instead, a mode of self-presentation. Thanks to the operation of sympathy, in being presented with an aspect of our corporeal nature, we are mediately presented with the tangible quality of an external body. But haptic perception is not indirect in this way. When our hominid ancestor grasps a rough-hewn stone they feel its overall shape and volume in the stone. Moreover, the presentation of these tangible qualities in their haptic experience is not apparently mediated. Our hominid ancestor need not attend to their bodily sensations as a means of attending to the tangible qualities of external bodies, rather these are directly disclosed in haptic perception. Indeed, attending to the body and its activity draws attentive resources away from the object of tactile perception. Focus too much on the warmth you feel and you cease to feel the warmth in the air. It is because the tangible qualities of an external body are directly disclosed in haptic perception that grasping becomes, in the cosmology of the Giants, a touchstone for reality. Grasping could not play this rhetorical role if it were apparently mediated.

The problem with the synthetic approach, at least as so far developed, is twofold. First, it posits two experience---the haptic experience and the experience of the perceiver's body---when plausibly there is only one, and the awareness of the perceiver's body is explicit rather than implicit. Moreover both of these features were directly involved in the subsequent indirect realism. On the alternative, analytic approach, indirect realism is simply not a possibility. One begins with an irreducible unity, the presentation of the tangible qualities of external bodies in haptic experience, and then discern what intelligible structure it must display if it is so much as possible. Thus the presentation of tangible qualities of external bodies in haptic experience could not be a construction from elements and principles understood independently of haptic perception, the way they would be if indirect realism were true.

To get a general sense of the analytic approach, consider the following plausible, if contentious, example (\citealt{Johnston:2007qy}, for one, seems to deny it). Arguably at least, any notion of sensory presentation essentially involves a subject--object distinction. If an object is present in perceptual experience then not only is there the object of perception---what is present in that experience--but there is also a perceiver that undergoes that experience---the subject to whom the object is presented. If we allow for modes of self-presentation where the subject and object are the same entity, then the subject--object distinction arguably required by the presupposed unity is merely hyperintensional. So compare Plotinus' view, in the \emph{Fifth Ennead}, that intellection, the presentation of intelligible objects, the highest form of unity short of that displayed by the hyperontic One, requires the distinction between the act of intellect and its object. Nevertheless, the Intellect apprehends only itself insofar as it is an image of the One. So the subject--object distinction required for intelligible presentation is consistent with its being a mode of self-presentation and so hyperintensional. 

If presentation may be self-presentation, and the intelligible distinction between subject and object may be hyperintensional, then I am genuinely uncertain about Johnston's denial of the claim that presentation intelligibly requires a subject. \citet{Johnston:2007qy} invites us to to think of ourselves as Samplers of Presence, where we access objective modes of presentations that are part of a larger reality, both accessible and inaccessible, but where our access, relative to our perspective, though ours, does not involve a subject over and above the accessed objective modes of presentations. But if the subject to whom the object is presented can be one and the same thing, then there being no subject over and above the object is not yet proof that they cannot be intelligibly distinguished. Even if there is no subject over and above the objective mode of presentation accessed from our perspective, the denial that there is no subject which accesses the objective mode of presentation is a further claim. One and the same thing, the objective mode of presentation, may be playing two roles. Just as in self-hate, where, tragically, one thing both hates and is hated, perhaps, in perception, one thing both accesses and is accessed. The present point is not to criticize Johnston, nor to defend neutral monism, but to emphasize how little may be involved in the subject--object distinction.

Intelligible presentation may be a mode of self-presentation, but Plotinus claims that the subject and object of perception must be more than hyperintensionally distinguished, they must be two things. This is a reflection of the fact that the unity presupposed in sensory presentation is a lesser unity than the unity presupposed in intelligible presentation. However, once one adopts a more naturalistic approach to embodiment than Plotinus, it is plausible to allow for forms of sensory self-presentation. Since having a fever is a condition of the body, and we are fundamentally embodied, then feeling a fever is itself a mode of self-presentation, even if there is more to one's nature than the fever one is currently suffering. (For discussion of this example and the puzzlement that results from not allowing modes of sensory self-presentation see \citealt{Yrjonsuuri:2008aa}.) If sensory presentation is partial, and primates like ourselves are fundamentally embodied, then the sensory presentation of aspects of our corporeal nature is a kind of self-presentation even if there is more to our nature than is present in bodily awareness. 

There may, however, be a sense in which Plotinus was right. The unity presupposed in sensory presentation, being partial, is a lesser unity than the unity presupposed in intelligible presentation. When the Inchoate Intellect turns, and looks, and sees only itself insofar as it is the image of the hyperontic One, thus becoming the Intellect in full actuality, this intelligibly differentiated image is wholly present in the act of intellection. An intelligible object is wholly present in the act of intellection in the way that a sensible object never is in perception since sensory presentation is invariably relative to the perceiver's partial perspective.

% The contention that the unity of haptic perception presupposes that there be an intelligible distinction, potentially hyperintensional, between the object presented in haptic experience and the subject to whom it is presented is plausible, if contentious. It is contentious because it has been contended. \citet{Johnston:2007qy}, for one, denies it.

Notice that in proceeding analytically, the subject--object distinction is not something to overcome (a characteristically modern anxiety dramatized by Cartesian skepticism). Instead we are presupposing their unity in an episode or process of sensory presentation. There is no need to bridge the gap between subject and object since we began with their unity in haptic perception and merely discern that their distinction, potentially hyperintensional, is intelligibly required. The need to bridge the gap between the subject and object constituted by their distinction only arises if their unity is not in this way presupposed. Thus bridging the gap between subject an object by having bodily sensation be coordinated with tangible qualities of external bodies via the operation of sympathy and its attendant indirect realism only arises if their unity in perceptual presentation is not presupposed but something to be constructed from elements and principles antecedently understood. 

The analytic approach to sensory presentation is comparable to Frege's approach towards thought, at least at certain stages of his career, on certain interpretations (see, for example, \citealt[essays 7 and 9]{Travis:2011qd}). Frege begins with a unity, a truth-evaluable thought, and discerns what intelligible structure it must display. Beginning with the thought, Frege analyzes or decomposes that thought into constituent elements that must be present and principles that must be operative if that thought is to be so much as truth-evaluable (which is not say that there is a unique such decomposition). The problem of the unity of the proposition simply does not arise for Frege, since he does not begin with independently understood elements and principles and tries to construct thoughts on their basis. Rather the unity of thought is explanatorily prior to the intelligible structure it must display if it is to be so much as truth-evaluable. Frege's position thus contrasts with recent discussions of the problem of the unity of the proposition (compare \citealt{King:2007ad}, \citealt{Soames:2010qq}, and \citealt{King:2014ls}). Similarly, on the analytic approach, the unity of sensory presentation is explanatorily prior to the intelligible structure it must display if it is so much as possible.

In grasping or enclosure the overall shape and volume of the object is directly disclosed in a perceiver's haptic encounter with it. Since I believe that perception quite generally involves an irreducible presentational element, I do not believe that the haptic presentation of the tangible qualities of external bodies could be constructed out of elements and principles understood independently of haptic perception. So I am debarred from the synthetic approach. Thus I proceed analytically. Presupposing the unity of haptic presentation, I try to determine the intelligible structure it must display if it is so much as possible. The claim that the presentation of tangible qualities of external bodies in haptic experience involves the operation of sympathy should be understood in this light. It is not the claim that one thing, the tangible qualities of external bodies, is mediately presented by another thing, the presentation of aspects of the subject's corporeal nature in bodily sensation. Rather, it is the claim that the presentation of tangible qualities of external bodies in haptic experience is an irreducible unity that is governed by the principle of sympathy. Feeling a tangible quality in an external body and in conformity with it just is the presentation of that quality in tactile experience and can be analytically explicated in terms of the operation of sympathy.

I observed earlier that our present conception of sympathy can be an obstacle to appreciating this. To overcome this limitation, as well as to introduce some claims about the operative notion of sympathy, it will be useful to consider briefly a select history. Specifically, I want to consider sympathy as a principle of action at a distance in Stoic physics, Plotinus' use of the Stoic notion in explaining vision, and Hume's use of sympathy in drawing the self--other distinction in social intercourse. 

\subsection{The Stoics} % (fold)
\label{sec:the_stoics}

It is easy to be impressed, as ancient medical opinion was, with how affecting a part of an animal's body may affect another part of their body without affecting the parts between. The Stoics believed that such medical phenomena were subject to a materialist explanation. And since they conceived of the cosmos as a whole as an organism, then the principle involved in that explanation, sympathy, was elevated to the status of a cosmic principle.

According to the Stoics, the soul that pervades and animates a living body is composed of \emph{pneuma}, a kind of rarified mixture of air and fire (\emph{Stoicorum Veterum Fragmenta} 2 773--89). The soul, while material, pervades the body. It does so not by filling interstitial spaces within the body, like waster absorbed by a sponge. Rather, active \emph{pneuma} occupies the same space as the passive matter of the body it animates, the way warmth may pervade a sun-baked stone. The \emph{pneuma} in a living body is in a state of tension. This tension in the \emph{pneuma} gives rise to a continuous wave-like motion (\emph{Stoicorum Veterum Fragmenta} 2 448, 450-7). Since the \emph{pneuma} in a living body is in a state of tensional motion, affecting some part of the body will affect the living body as a whole. Thus when a part of a living body is affected, a similar or different change may be transmitted via the tensional motion of the \emph{pneuma} to another part of the body without affecting the parts between, depending upon the disposition of its parts.

The operation of sympathy was not confined to ordinary living bodies. The cosmos, itself, was conceived to be a living being as well, though perhaps an extraordinary one, at least by our lights. The cosmos was conceived to possess the same kind of unity as living organisms. Like ordinary living beings, the cosmos is united by an all pervading \emph{pneuma} in a state of tensional motion. Thus sympathy was transformed, in Stoic thought, into a cosmic principle of action at a distance. While perhaps Posidonius is the most famous proponent of cosmic sympathy, the doctrine goes back at least as far as Chrysippus and, arguably, has roots in Plato's \emph{Timaeus} (on Stoic sympathy see \citealt{Meyer:2009xp,Brouwer:2015ee}; on the \emph{Timaeus} and sympathy see \citealt{Emilsson:2015wf}). Sympathy, as a principle of action at a distance, it was used to explain a variety of natural phenomena, such as the influence of the moon on the tides.

% subsection the_stoics (end)

\subsection{Plotinus} % (fold)
\label{sec:plotinus}

% subsection plotinus (end)

\subsection{Hume} % (fold)
\label{sec:hume}

% subsection hume (end)

% section sympathy (end)

\section{The Principle of Haptic Presentation} % (fold)
\label{sec:sympathy_as_the_principle_of_haptic_presentation}

% section the_principle_of_haptic_presentation (end)


% chapter sympathy (end)

