%!TEX root = /Users/markelikalderon/Documents/Git/sympathy/perception.tex
\chapter{Sympathy} % (fold)
\label{cha:sympathy}

\section{Sympathy and Haptic Perception} % (fold)
\label{sec:sympathy_and_haptic_perception}

When our hominid ancestor reaches out and picks up a rough-hewn stone, perhaps in preparation to skirmish with a competing group of hominids, they feel the overall shape and volume of the stone in their grasp. It is not the hand's shape, the configuration of the hand in grasping or enclosure, that they haptically perceive though they may be aware of it. It is the stone's shape that is disclosed in their grasp. They feel the overall shape and volume in the stone, and its overall shape and volume are tangible qualities of the stone that their hand is felt to conform to. I shall make a suggestion that will be the basis for an answer to our refined how-possible question. Specifically, feeling tangible qualities in something external to the perceiver's body and feeling in conformity with them can fruitfully be understood as due to the operation of sympathy. One obstacle to appreciating this concerns out present understanding of sympathy, where sympathy is a kind of emotional response to others, a kind of fellow-feeling akin to compassion or pity. The notion of sympathy that is being invoked as the principle governing haptic presentation is closer to the notion at work in Stoic physics, if more abstract and not at all reliant on on their vitalistic metaphysics. Felt resistance to touch, insofar as it is the presentation of an object external to the perceiver's body, is a sympathetic response to the force that resists the hand's activity. Recall our refined version of our how-possible question was this: How is it possible for felt resistance to the hand's activity in grasping or enclosure to disclose a rigid body's overall shape and volume? If feeling tangible qualities in something external to the perceiver's body and in conformity with them is due to the operation of sympathy then we have a basis for an answer. It is when the limit to hand's activity is a sympathetic response to as a countervailing force, as the hand's force encountering an alien force resisting it, one force in conflict with another, like it yet distinct from it, that the self maintaining forces of the body disclose that body's presence and tangible qualities to haptic awareness.

Earlier, the initial statement of the puzzle was motivated by considering the analogy of felt temperature. We contrasted two cases. In both cases you feel warm, and you feel warm to the same degree. But in the first case, you feel warm because of a fever, and in the second case, you feel warm because because of the ambient heat. There is also, importantly, a phenomenological difference between these cases. In the second case, not only do you feel warm, but you feel, as well, the warmth in the ambient air. Indeed, the warmth you feel is in conformity with the warmth felt in the ambient air. What explains the phenomenological difference is that in the second case, but not in the first, the felt warmth is a sympathetic response to the ambient heat, to the thermal properties of something external to the perceiver's body. In sympathetically responding to ambient heat, the warmth you feel becomes a way of feeling the warmth in something located outside of your body. Moreover, in sympathetically responding to ambient heat, the warmth you feel is in conformity with the warmth felt in the air.

The proposal is that presentation in haptic perception is governed by the principle of sympathy. There are two ways to understand this. The fist proceeds synthetically. That is, beginning with elements and principles understood independently of haptic perception, one constructs the notion of the presentation of tangible qualities of external bodies in haptic experience on their basis. So, for example, one might begin with bodily sensation and ``extend its reach'', so to speak, via the operation of sympathy to construct a notion of the presentation of tangible qualities of external bodies. So understood, haptic presentation would be the coordination of bodily sensations with the tangible qualities of external bodies via the operation of sympathy. The second way proceeds analytically. That is, beginning with the notion of the presentation of tangible qualities of external bodies in haptic experience, one analyses or decomposes that notion into constituent elements that must be present and principles that must be operative if the presentation of the tangible qualities of external bodies is so much as possible. 

The synthetic approach naturally, if not inexorably, motivates indirect realism about tactile perception. So consider again our toy model where we begin with bodily sensation and extend its reach through the operation of sympathy. Bodily sensation does not involve the presentation of tangible qualities of external bodies. It is, instead, a mode of self-presentation. Thanks to the operation of sympathy, in being presented with an aspect of our corporeal nature, we are mediately presented with the tangible quality of an external body. But haptic perception is not indirect in this way. When our hominid ancestor grasps a rough-hewn stone they feel its overall shape and volume in the stone. Moreover, the presentation of these tangible qualities in their haptic experience is not apparently mediated. Our hominid ancestor need not attend to their bodily sensations as a means of attending to the tangible qualities of external bodies, rather these are directly disclosed in haptic perception. Indeed, attending to the body and its activity draws attentive resources away from the object of tactile perception. Focus too much on the warmth you feel and you cease to feel the warmth in the air. It is because the tangible qualities of an external body is directly disclosed in haptic perception that grasping becomes, in the cosmology of the Giants, a touchstone for reality. Grasping could not play this rhetorical role if it were apparently mediated.

On the alternative, analytic approach, indirect realism is simply not a possibility. One begins with an irreducible unity, the presentation of the tangible qualities of external bodies in haptic experience, and then discern what intelligible structure it must display if it is so much as possible. Thus the presentation of tangible qualities of external bodies in haptic experience could not be a construction from elements and principles understood independently of haptic perception, the way they would be if indirect realism were true.

To get a general sense of the analytic approach, consider the following plausible, if contentious, example (\citealt{Johnston:2007qy}, for one, seems to deny it). Arguably at least, any notion of sensory presentation essentially involves a subject--object distinction. If an object is present in perceptual experience then not only is there the object of perception---what is present in that experience--but there is also a perceiver that undergoes that experience---the subject to whom the object is presented. If we allow for modes of self-presentation where the subject and object are the same entity, then the subject--object distinction arguably required by the presupposed unity is merely hyperintensional. So compare Plotinus' view, in the \emph{Fifth Ennead}, that intellection, the presentation of intelligible objects, the highest form of unity short of that displayed by the hyperontic One, requires the distinction between the act of intellect and its object. Nevertheless, the Intellect apprehends only itself insofar as it is an image of the One. So the subject--object distinction required for intelligible presentation is consistent with its being a mode of self-presentation and so hyperintensional. 

If presentation may be self-presentation, and the intelligible distinction between subject and object may be hyperintensional, then I am genuinely uncertain about Johnston's denial of the claim that presentation intelligibly requires a subject. \citet{Johnston:2007qy} invites us to to think of ourselves as Samplers of Presence, where we access objective modes of presentations that are part of a larger reality, both accessible and inaccessible, but where our access, relative to our perspective, though ours, does not involve a subject over and above the accessed objective modes of presentations. But if the subject to whom the object is presented can be one and the same thing, then there being no subject over and above the object is not yet proof that they cannot be intelligibly distinguished. Even if there is no subject over and above the objective mode of presentation accessed from our perspective, the denial that there is no subject which accesses the objective mode of presentation is a further claim. One and the same thing, the objective mode of presentation, may be playing two roles. Just as in self-hate, where, tragically, one thing both hates and is hated, perhaps, in perception, one thing both accesses and is accessed. The present point is less to criticize Johnston, than to emphasize how little may be involved in the subject--object distinction.

Intelligible presentation may be a mode of self-presentation, but Plotinus claims that the subject and object of perception must be more than hyperintensionally distinguished, they must be two things. This is a reflection of the fact that the unity presupposed in sensory presentation is a lesser unity than the unity presupposed in intelligible presentation. However, once one adopts a more naturalistic approach to embodiment than Plotinus, it is plausible to allow for forms of sensory self-presentation. Since having a fever is a condition of the body, and we are fundamentally embodied, then feeling a fever is itself a mode of self-presentation, even if there is more to one's corporeal nature than the fever one is currently suffering. (For discussion of this example and the puzzlement that results from not allowing modes of sensory self-presentation see \citealt{Yrjonsuuri:2008aa}.) If sensory presentation is partial, and primates like ourselves are fundamentally embodied, then the sensory presentation of aspects of our corporeal nature is a kind of self-presentation even if there is more to our nature than is present in bodily awareness. 

There may, however, be a sense in which Plotinus was right, at least to this extent---the unity presupposed in sensory presentation, being partial, is a lesser unity than the unity presupposed in intelligible presentation. When the Inchoate Intellect turns, and looks, and sees only itself insofar as it is the image of the hyperontic One, thus becoming the Intellect in full actuality, this intelligibly differentiated image is wholly present in the act of intellection. An intelligible object is wholly present in the act of intellection in the way that a sensible object never is in perception since sensory presentation is invariably relative to the perceiver's partial perspective.

% The contention that the unity of haptic perception presupposes that there be an intelligible distinction, potentially hyperintensional, between the object presented in haptic experience and the subject to whom it is presented is plausible, if contentious. It is contentious because it has been contended. \citet{Johnston:2007qy}, for one, denies it.

Notice that in proceeding analytically, the subject--object distinction is not something to overcome. Instead we are presupposing their unity in an episode or process of sensory presentation. There is no need to bridge the gap between subject and object since we began with their unity in haptic perception and merely discern that their distinction, potentially hyperintensional, is intelligibly required. The need to bridge the gap between the subject and object constituted by their distinction only arises if their unity is not presupposed. Thus bridging the gap between subject an object by having bodily sensation be coordinated with tangible qualities of external bodies via the operation of sympathy and its attendant indirect realism only arises if their unity in perceptual presentation is not presupposed but something to be constructed from elements and principles antecedently understood. 

The analytic approach to sensory presentation is comparable to Frege's approach towards thought, at least at certain stages of his career, on certain interpretations. Frege begins with a unity, a truth-evaluable thought, and discerns what intelligible structure it must display. Beginning with the thought, Frege analyzes or decomposes that thought into constituent elements that must be present and principles that must be operative if that thought is to be so much as truth-evaluable. The problem of the unity of the proposition simply does not arise for Frege, since he does not begin with independently understood elements and principles and tries to construct thoughts on their basis. Rather the unity of thought is explanatorily prior to the intelligible structure it must display if it is to be so much as truth-evaluable. Similarly, on the analytic approach, the unity of sensory presentation is explanatorily prior to the intelligible structure it must display if it is so much as possible.

In grasping or enclosure the overall shape and volume of the object is directly disclosed in a perceiver's haptic encounter with it. Since I believe that perception quite generally involves an irreducible presentational element, I do not believe that the haptic presentation of the tangible qualities of external bodies could be constructed out of elements and principles understood independently of haptic perception. So I am debarred from the synthetic approach. Thus I proceed analytically. Presupposing the unity of haptic presentation, I try to determine the intelligible structure it must display if it is so much as possible. The claim that the presentation of tangible qualities of external bodies in haptic experience involves the operation of sympathy should be understood in this light. It is not the claim that one thing, the tangible qualities of external bodies, is mediately presented by another thing, the presentation of aspects of the subject's corporeal nature in bodily sensation. Rather, it is the claim that the presentation of tangible qualities of external bodies in haptic experience is an irreducible unity that is governed by the principle of sympathy. Feeling a tangible quality in an external body and in conformity with it just is the presentation of that quality in tactile experience and can be analytically explicated in terms of the operation of sympathy.

I observed earlier that our present conception of sympathy can be an obstacle to appreciating this. To overcome this limitation, as well as to introduce some claims about the operative notion of sympathy, it will be useful to consider briefly a select history. Specifically, I want to consider sympathy as a principle of action at a distance in Stoic physics, Plotinus' use of the Stoic notion in explaining vision, and Hume's use of sympathy in drawing the self--other distinction in social intercourse. 

% section sympathy_and_haptic_perception (end)

\section{The Stoics} % (fold)
\label{sec:the_stoics}

% section the_stoics (end)

\section{Plotinus} % (fold)
\label{sec:plotinus}

% section plotinus (end)

\section{Hume} % (fold)
\label{sec:hume}

% section hume (end)

\section{Sympathy as the Principle of Haptic Presentation} % (fold)
\label{sec:sympathy_as_the_principle_of_haptic_presentation}

% section sympathy_as_the_principle_of_haptic_presentation (end)

% chapter sympathy (end)

