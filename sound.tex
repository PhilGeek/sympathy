%!TEX root = /Users/markelikalderon/Documents/Git/sympathy/perception.tex
\chapter{Sound} % (fold)
\label{cha:sound}

\section{Moving Forward} % (fold)
\label{sec:moving_forward}

Tactile metaphors for perception, even for non-tactile modes of awareness such as vision and audition, are primordial and persistent. In trying to understand what, if anything, makes these tactile metaphors for perceptual awareness apt, we undertook a phenomenological investigation of grasping or enclosure, understood as a mode of haptic perception. So far we have identified at least on feature of haptic presentation that might be generalized to other forms of sensory presentation. Specifically, if a tangible quality is present in haptic experience, the conscious character of that experience is constitutively shaped by the tangible quality presented in it, at least relative to the perceiver's haptic experience. The proposed general thesis, then, is that the conscious character of a perceptual experience is constitutively shaped by its object, at least relative to its presentation to the perceiver's partial perspective. More would have to be done to fully defend this general thesis. Among other things, that there is an analogue of visual perspective in each of the sensory modalities would have to be justified. (Can we really have a perspective on an odor, say?) In this chapter and the next, I will say more about the applicability of this idea to vision and audition at least. But what of the other important claim that was made about the metaphysics of haptic perception, that haptic presentation is governed by the principle of sympathy? Does sympathy operate in other modes of sensory presentation? Does the intelligible structure of the unity displayed by sensory presentation require the operation of sympathy quite generally? If so, how are we to understand this?

It was natural to appeal to sympathy to explain how felt resistance to the hand's activity in grasping or enclosure discloses the overall shape and volume of the object grasped, since we began by thinking of haptic perception in terms of the Secret Doctrine that Socrates attributes to Protagoras in the \emph{Theaetetus}. Just as on the Protagorean model, perception is the joint upshot of forces in conflict, grasping or enclosure, understood as a mode of haptic perception, is itself naturally understood as the joint upshot of forces in conflict. On the one hand, there is the force of the activity of the grasping hand. On the other hand, there are the self-maintaining forces of the rigid, solid body. In making an effort to more precisely mold the hand to the body's contours and the resistance encountered by the self-maintaining forces that determine the rigid, solid body's shape together give rise to an experience of that body's overall shape and volume. In trying to determine whether sympathy operates in non-haptic modes of sensory presentation, it will be useful to begin by determining whether this Protagorean model can be extended to other sensory modalities.

Smith's \citeyearpar{Smith:2002sa} discussion of \emph{Anstoss} suggests a way one might generalize from the haptic case. Haptic perception arises from the conflict between the grasping hand and the self-maintaining forces of the rigid, solid body. Reaching out and grasping something is a clear example of voluntary intentional action. Moreover, at least in the case of haptic perception, the hand is, among other things, a sensory organ. Putting these ideas together, it is the voluntary intentional movement of sensory organs that are the activities whose force comes into conflict with the perceptual object. In visual case, then, it is the movement of the eyes in their sockets, and not saccadic movement which is relevant, since the latter is involuntary and non-intentional. Smith faces some difficulties with this proposal. For example, unlike other animals, humans cannot cock their ears, though we may turn toward a sound to hear it better. This is not, however, the only way to generalize from the haptic case. 

Reaching out and grasping something may be a voluntary, intentional movement of a sensory organ, but insofar as it is a mode of perception, it is psychological as well. Consider Cook Wilson's claim (\emph{Correspondence with Stout, 1904}, \citeyear{Cook-Wilson:1926sf}) that in order to feel something in an object, a rough texture say, one must feel that object, and in order to weigh something, one must weigh it. If grasping is understood analogously with feeling and weighing, then this suggestions an alternative generalization. On this alternative, in order to hear something, one must listen. And in order to see, one must look. Grasping, feeling, weighing, listening, and looking, while they may involve the intentional movement of sensory organs, are not themselves reducible to such movements. They are, perhaps, more aptly described as a kind of perceptual stance, where the perceiver opens themselves up, in a directed manner, to experiencing different aspects of the natural environment. In engaging in such activities, in directing perceptual awareness in this way, the perceiver contributes to making different aspects of the natural environment perceptually available.

% section moving_forward (end)

\section{The Berkeley--Heidegger Continuum} % (fold)
\label{sec:the_berkeley_heidegger_continuum}

% section the_berkeley_heidegger_continuum (end)

\section{The Wave Theory} % (fold)
\label{sec:the_wave_theory}

% section the_wave_theory (end)

\section{Phenomenological Objections} % (fold)
\label{sec:phenomenological_objections}

% section phenomenological_objections (end)

\section{Sympathy and Audition} % (fold)
\label{sec:sympathy_and_audition}

% section sympathy_and_audition (end)

% chapter sound (end)