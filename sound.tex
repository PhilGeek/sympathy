%!TEX root = /Users/markelikalderon/Documents/Git/sympathy/perception.tex
\chapter{Sound} % (fold)
\label{cha:sound}

\section{Moving Forward} % (fold)
\label{sec:moving_forward}

Tactile metaphors for perception, even for non-tactile modes of awareness such as vision and audition, are primordial and persistent. In trying to understand what, if anything, makes these tactile metaphors for perceptual awareness apt, we undertook a phenomenological investigation of grasping or enclosure, understood as a mode of haptic perception. So far we have identified at least one feature of haptic presentation that might be generalized to other forms of sensory presentation. Specifically, if a tangible quality is present in haptic experience, the conscious character of that experience is constitutively shaped by the tangible quality presented in it, at least relative to the perceiver's haptic experience. The proposed general thesis, then, is that the conscious character of a perceptual experience is constitutively shaped by its object, at least relative to its presentation to the perceiver's partial perspective. More would have to be done to fully defend this general thesis. Among other things, that there is an analogue of visual perspective in each of the sensory modalities would have to be justified. (Can we really have a perspective on an odor, say?) In this chapter and the next, I will say more about the applicability of this idea to vision and audition at least. But what of the other important claim that was made about the metaphysics of haptic perception, that haptic presentation is governed by the principle of sympathy? Does sympathy operate in other modes of sensory presentation? Does sensory presentation require the operation of sympathy quite generally? If so, how are we to understand this?

It was natural to appeal to sympathy to explain how felt resistance to the hand's activity in grasping or enclosure discloses the overall shape and volume of the object grasped, since we began by thinking of haptic perception in terms of the Secret Doctrine that Socrates attributes to Protagoras in the \emph{Theaetetus}. Just as on the Protagorean model, perception is the joint upshot of forces in conflict, grasping or enclosure, understood as a mode of haptic perception, is itself naturally understood as the joint upshot of forces in conflict. On the one hand, there is the force of the activity of the grasping hand. On the other hand, there are the self-maintaining forces of the rigid, solid body. Making an effort to more precisely mold the hand to the body's contours and the resistance of the self-maintaining forces that determine that body's rigidity and solidity together give rise to an experience of that body's overall shape and volume. In trying to determine whether sympathy operates in non-haptic modes of sensory presentation, we shall begin by determining whether this Protagorean model can be extended to other sensory modalities.

Smith's \citeyearpar{Smith:2002sa} discussion of \emph{Anstoss} suggests one way one might generalize from the haptic case. Haptic perception arises from the conflict between the grasping hand and the self-maintaining forces of the rigid, solid body. Reaching out and grasping something is a clear example of voluntary intentional action. Moreover, at least in the case of haptic perception, the hand is, among other things, a sensory organ. Putting these ideas together, it is the voluntary intentional movement of sensory organs that are the activities whose force comes into conflict with the perceptual object. In visual case, then, it is the movement of the eyes in their sockets, and not saccadic movement which is relevant, since the latter is involuntary and non-intentional. Smith faces some difficulties, not necessarily insuperable, with this proposal. For example, unlike other animals, humans cannot cock their ears, though we may turn toward a sound to better hear it. This is not, however, the only way to generalize from the haptic case. 

Reaching out and grasping something may be a voluntary, intentional movement of a sensory organ, but insofar as it is a mode of perception, it is a psychological activity as well. Consider Cook Wilson's claim (\emph{Correspondence with Stout, 1904}, \citeyear{Cook-Wilson:1926sf}) that in order to feel something in an object, a rough texture say, one must feel that object, and in order to weigh something, one must weigh it. If grasping is understood analogously with feeling and weighing, then this suggests an alternative generalization. On this alternative, in order to hear something, one must listen. And in order to see, one must look. Grasping, feeling, weighing, listening, and looking, while they may or may not involve the intentional movement of sensory organs, they are not themselves reducible to such movements when they do. They are, perhaps, more aptly described as a kind of perceptual stance, sustained by a characteristic activity, where the perceiver opens themselves up, in a directed manner, to experiencing different aspects of the natural environment. In engaging in such activities, in directing perceptual awareness in this way, the perceiver contributes to making different aspects of the natural environment perceptually available.

% section moving_forward (end)

\section{The Berkeley--Heidegger Continuum} % (fold)
\label{sec:the_berkeley_heidegger_continuum}

``In order to hear well,'' Maine de Biran observes, ``it is necessary to \emph{listen}.'' How does listening, the activity of listening out for something, come into conflict with the objects of audition such that these may be sympathetically presented in auditory experience? We can make progress with this question by first getting clearer on the objects of audition, on what there is to listen out for.

We hear sounds. Do we hear, as well, their sources? Philosophers divide on this question. And even those philosophers who maintain that we hear both sounds and their sources divide as to how we do so. Philosopher's views on these matters can be useful represented on a continuum that ranges from Berkeley on the one extreme and Heidegger on the other (see \citealt{Leddington:2014aa} for a similar suggestion). 

\citet{Berkeley:1734fk} follows Aristotle in taking sounds to be the primary objects of audition. For something to be the primary object of a given sensory modality it must be perceptible in itself and perceptible to that sensory modality alone. That a sensory modality having a primary object does not preclude it from having other objects as well. Thus we can see motion and feel motion. Berkeley thus extends the Peripatetic account in claiming, in addition, that sounds are the sole objects of audition. We hear no other thing. In a way, this is a return to an earlier, Platonic view. Plato, in the \emph{Theaetetus}, maintained that a sensory capacity just is the capacity to present its primary objects. Our auditory capacity, so conceived, just is the capacity to present its primary object, sound. So on Berkeley's view, strictly speaking, we hear sounds and not their sources. In part, Berkeley argues for this by distinguishing sounds from their sources by an application of Leibniz's Law. Sounds have acoustical properties that their sources lack, and insofar as sources lack acoustical properties they are inaudible. 

% (\citealt{Smith:2002sa} is an example of a contemporary philosophers who espouses the Berkelean view.) 

The neo-Berkelean accepts that sound is the primary object of audition. They accept as well that the acoustical properties of sounds distinguish them from their sources. But they deny that sound is the only object of audition. The sources of sound that can be perceived by other sensory modalities, such as sight, are also the objects of audition, but only derivatively---we hear the source of the sound by hearing the sound. According to the neo-Berkelean, Berkeley goes too far in denying that we hear the sources of sound. Berkeley mistook sound's being the direct or immediate object of audition for sound's being the sole object of audition. If we allow sources to be the indirect or mediate objects of audition---in the sense that we hear sources by hearing sounds---, then the objects of audition include not only primary sensibles but common sensibles as well.

However, in ``The Origin of the Work of Art'' Heidegger presents an opposing view:
\begin{quote}
    We never really first perceive a throng of sensations, e.g., tones and noises, in the appearance of things \ldots; rather we hear the storm whistling in the chimney, we hear the three-motored plane, we hear the Mercedes in immediate distinction from the Volkswagen. Much closer to us than all sensations are the things themselves. We hear the door shut in the house and never hear acoustical sensations or even mere sounds. \citep[151--152]{Heidegger:1935uq}
\end{quote}
Nothing hangs on Heidegger's apparent acceptance of the empiricist identification of sound with acoustic sensation. What is important is Heidegger's denial of the central neo-Berkelean claim, that we hear the source of sound by hearing the sound. Rather, we hear the source of sound directly.

In undergoing an auditory experience, the source of a sound is directly or immediately present in that experience. When we attend to our auditory experience, as Heidegger invites us to, we attend to the sources of sounds and rarely, if at all, to the sounds in distinction from their sources. That is consistent with maintaining that hearing a source necessarily involves accoustical sensation. And yet Heidegger is clearly denying the neo-Berkelean claim that he hear the source of a sound by hearing the sound. That is a negative result about how to characterize acoustical indirection: There is more to hearing a source by hearing its sound than the necessary accompaniment of the former by the latter.

Heidegger exaggerates when he claims that we never hear acoustical sensations or mere sounds. For he goes on to maintain that we can manage to hear sounds in distinction from their sources only by adopting the aural equivalent of the painterly attitude:
\begin{quote}
    In order to hear a bare sound we have to listen away from things, divert our ears from them, i.e., listen abstractly. \citep[152]{Heidegger:1935uq}
\end{quote}
We can get a sense of how difficult it is to adopt this attitude by considering Pierre Schaeffer's piece ``Étude aux chemins de fer'' (1948). Whereas traditional composition begins with an abstraction, the score, which is made concrete in playing it, \emph{musique concrète} begins with concrete sounds and abstracts them into a composition through tape looping and sound collage. Yet despite these distancing techniques, the material sources never completely fade from the perceived soundscape. We get a sense of the train's speed, its size, the space surrounding the tracks as well as the space of the interior given the character of the resonance. Working in Schaeffer's studio, Stockhausen addressed this problem in the method of tape composition deployed in ``Étude'' (1952). He recorded prepared low piano strings struck with an iron bar and sliced off the heads of the recorded sounds, thus eliminating information about the attack and other material features of the source. These short headless segments were further repeated to form the basic tones of the piece. The effect is uncanny. However, the very uncanniness is itself partly a product of the limitation that beset Schaeffer's earlier piece. The tones are uncanny in that there are at once unfamiliar, indeterminate, and yet familiar, though placing them proves elusive. Indeed, at the end of his career, Schaeffer pronounced \emph{musique concrète} a failure, claiming, perhaps ironically, to have wasted his life. Heidegger's observation was the principle obstacle---it is very difficult to hear bare sounds, to hear sounds without also hearing their sources. And so there are limits to the degree of abstraction that can be achieved with \emph{musique concrète}. It is telling, in this regard, that Stockhausen abandons tape composition for the generation of tones with sine-wave generators as he continued to explore electronic composition.

The Berkelean alternative raises an explanatory challenge to the neo-Berkelean---\-to explain how we can experience a source by experiencing its sound. The Heideggerian alternative is a challenge to the very possibility of such an explanation. At the very least, in undergoing an auditory experience, we do not attend to sources by attending to sounds---according to Heidegger, in normal cases, there is no sound that we are attending to. A neo-Berkelean cannot afford to be as sanguine about the Heideggerean alternative as they may be tempted to be about the Berkelean alternative. A promissory note is worth nothing in the face of an inability to repay.

In this chapter, I propose to simply set the extreme Berkelean alternative to one side and accept that we hear, in addition to the sounds, their sources as well. 

% section the_berkeley_heidegger_continuum (end)

\section{The Wave Theory} % (fold)
\label{sec:the_wave_theory}

An ancient tradition identifies sound with motion. Plato and Aristotle claim that sound is a motion in the air (Plato, \emph{Timaeus} 67 b; Aristotle, \emph{De Anima} 2 8 420\( ^{a} \)8--11, 420\( ^{b} \)11, \emph{De Sensu} 447\( ^{a} \)1--2, though see \citealt[60--1]{OCallaghan:2007xy} for an alternative interpretation).

% section the_wave_theory (end)

\section{Phenomenological Objections} % (fold)
\label{sec:phenomenological_objections}

% section phenomenological_objections (end)

\section{Sympathy and Audition} % (fold)
\label{sec:sympathy_and_audition}

% section sympathy_and_audition (end)

% chapter sound (end)