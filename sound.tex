%!TEX root = /Users/markelikalderon/Documents/Git/sympathy/perception.tex
\chapter{Sound} % (fold)
\label{cha:sound}

\section{Moving Forward} % (fold)
\label{sec:moving_forward}

Tactile metaphors for perception, even for non-tactile modes of awareness such as vision and audition, are primordial and persistent. In trying to understand what, if anything, makes these tactile metaphors for perceptual awareness apt, we undertook a phenomenological investigation of grasping or enclosure, understood as a mode of haptic perception. So far we have identified at least one feature of haptic presentation that might be generalized to other forms of sensory presentation. Specifically, if a tangible quality is present in haptic experience, the conscious character of that experience is constitutively shaped by the tangible quality presented in it, at least relative to the perceiver's haptic experience. The proposed general thesis, then, is that the conscious character of a perceptual experience formally assimilates to its object, understood as a mode of constitutive shaping, at least relative to its presentation to the perceiver's partial perspective. More would have to be done to fully defend this general thesis. Among other things, that there is an analogue of visual perspective in each of the sensory modalities would have to be justified. (Can we really have a perspective on an odor, say?) In this chapter, I will say more about the applicability of this idea to audition at least. But what of the other important claim that was made about the metaphysics of haptic perception, that haptic presentation is governed by the principle of sympathy? Does sympathy operate in other modes of sensory presentation? Does the sensory presentation of the extrasomatic require the operation of sympathy quite generally? If so, how are we to understand this?

It was natural to appeal to sympathy to explain how felt resistance to the hand's activity in grasping or enclosure discloses the overall shape and volume of the object grasped, since we began by thinking of haptic perception in terms of the Secret Doctrine that Socrates attributes to Protagoras in the \emph{Theaetetus}. Just as on the Protagorean model, perception is the joint upshot of forces in conflict, grasping or enclosure, understood as a mode of haptic perception, is itself naturally understood as the joint upshot of forces in conflict. On the one hand, there is the force of the activity of the grasping hand. On the other hand, there are the self-maintaining forces of the rigid, solid body. Making an effort to more precisely mold the hand to the body's contours and the resistance of the self-maintaining forces that determine that body's rigidity and solidity together give rise to an experience of that body's overall shape and volume. In trying to determine whether sympathy operates in non-haptic modes of sensory presentation, we shall begin by determining whether this Protagorean model can be extended to other sensory modalities. Kilwardby, for one, thought it did: ``Two motions come together as if from opposite parts in sensing'' (\emph{De Spiritu Fantastico} 112, \citealt[]{Broadie:1993dz}). Though, of course, the Protagorean model finds its expression in the reconciliation of Peripatetic and Augustinian metaphysics it offers Kilwardby:
\begin{quote}
	One motion proceeds from a sensible thing which causes an alteration, and through the medium this enters to the sense organ and its innermost part where it is united with the sensory soul. The other motion proceeds from the sensory soul to meet the affect which is produced in the sense organ. In the meeting of these motions, an image of a sensible thing is formed in the sensory soul by the action of the sensory soul which attends to its sense organ, and by means of this image a thing is sensed. (Kilwardby, \emph{De Spiritu Fantastico} 112, \citealt[]{Broadie:1993dz})
\end{quote}

Smith's \citeyearpar{Smith:2002sa} discussion of \emph{Anstoss} suggests one way one might generalize from the haptic case. Haptic perception arises from the conflict between the grasping hand and the self-maintaining forces of the rigid, solid body. Reaching out and grasping something is a clear example of voluntary intentional action. Moreover, at least in the case of haptic perception, the hand is, among other things, a sensory organ (though see \citealt{Paterson:2007aa} for the claim that touch lacks a sensory organ). Putting these ideas together, it is the voluntary intentional movement of sensory organs that are the activities whose force comes into conflict with the perceptual object. In the visual case, then, it is the deliberate movement of the eyes in their sockets, and not saccadic movement which is relevant, since the latter is involuntary and non-intentional. Smith faces some difficulties, not necessarily insuperable, with this proposal. For example, unlike other animals, humans cannot cock their ears, though we may turn toward a sound to better hear it. This is not, however, the only way to generalize from the haptic case. 

Reaching out and grasping something may be a voluntary, intentional movement of a sensory organ, but insofar as it is a mode of perception, it is a psychological activity as well. Consider Cook Wilson's claim (\emph{Correspondence with Stout, 1904}, \citeyear{Cook-Wilson:1926sf}) that in order to feel something in an object, a rough texture say, one must feel that object, and in order to weigh something, one must weigh it. If grasping is understood analogously with feeling and weighing, then this suggests an alternative generalization. On this alternative, in order to hear something, one must listen. And in order to see, one must look. Grasping, feeling, weighing, listening, and looking, while they may or may not involve the intentional movement of sensory organs, are not themselves reducible to such movements when they do. They are, perhaps, more aptly described as a kind of perceptual stance, sustained by a characteristic activity, where the perceiver opens themselves up, in a directed manner, to experiencing different aspects of the natural environment. In engaging in such activities, in directing perceptual awareness in this way, the perceiver contributes to making different aspects of the natural environment perceptually available.

``In order to hear well,'' Maine de Biran observes, ``it is necessary to \emph{listen}'' (\emph{Influence de l'habitude sur la faculté de penser}; \citealt[63--4]{Boehm:1929aa}). How does listening, the activity of listening out for something, come into conflict with the objects of audition such that these may be sympathetically presented in auditory experience? We can make progress with this question by first getting clearer on the objects of audition, on what there is to listen out for. That task will occupy us for this chapter and the next.

% section moving_forward (end)

\section{The Berkeley--Heidegger Continuum} % (fold)
\label{sec:the_berkeley_heidegger_continuum}

From the hill in Greenwich Park where the Royal Observatory is located, one can see the towers of the City of London across the Thames. I once witnessed the Ballardian spectacle of a flock of feral parrots flying across this scene. These formerly domesticated tropical birds, having escaped or been released, have gone feral and their population is increasing throughout London. Bright green set against mirrored skyscrapers, the parrots were excited and were calling loudly. I heard the sound of a calling parrot. Did I hear, as well, the parrot's call?

We hear sounds. Do we hear, as well, their sources? Philosophers divide on this question. And even those philosophers who maintain that we hear both sounds and their sources divide as to how we do so. Philosopher's views on these matters can be useful represented on a continuum that ranges from Berkeley on the one extreme to Heidegger on the other (see \citealt{Leddington:2014aa} for a similar suggestion). 

\nocite{Berkeley:1734fk} Berkeley, in \emph{Three Dialogues between Hylas and Philonous}, follows Aristotle in taking sounds to be the proper objects of audition. For something to be the proper object of a given sensory modality it must be perceptible in itself and perceptible to that sensory modality alone. That a sensory modality has a proper object does not preclude it from having other objects as well. Thus we can see motion and feel motion. Berkeley thus extends the Peripatetic account in claiming, in addition, that sounds are the sole objects of audition. We hear no other thing. In a way, this is a return to an earlier, Platonic view. Plato, in the \emph{Theaetetus} (184 e 8--185 a 3), maintained that the perception of a given sense just is the presentation of an object available through the exercise of that capacity alone (compare as well \emph{Republic} 5 477--478). Our auditory capacity, so conceived, just is the capacity to present its proper object, sound. So on Berkeley's view, strictly speaking, we hear sounds and not their sources. In part, Berkeley argues for this by distinguishing sounds from their sources by an application of Leibniz's Law. Sounds have auditory qualities that their sources lack, and insofar as sources lack auditory qualities they are inaudible. 

The neo-Berkelean accepts that sound is the proper object of audition. They accept, as well, that the sounds are distinguished from their sources. But they deny that sound is the only object of audition. Sources of sound that can be perceived by other sensory modalities, such as sight, and are thus common sensibles, are also the objects of audition, but only derivatively---we hear the source of a sound by hearing the sound it generates. According to the neo-Berkelean, Berkeley goes too far in denying that we hear the sources of sound. Berkeley mistook sound's being the direct or immediate object of audition for sound's being the sole object of audition. If we allow sources to be the indirect or mediate objects of audition, then the objects of audition include not only proper sensibles but common sensibles as well.

So according to the neo-Berkelean, perceivers are immediately presented with the proper object of audition, sound, and thereby mediately presented with the source of the sound, the audible activity of a body, say. Sounds are audible. Indeed they are audible in themselves, in the sense that sounds contain within themselves the power of their own audibility. \emph{Pace} Berkeley, sources too are audible. However, the audible sources of sound are not audible in themselves, but are only audible by hearing other objects that are audible in themselves, the sounds that they generate. An explicit experiences of a sound is, according to the neo-Berkelean, the means by which we experience its source.  Auditory experience affords the perceiver with an explicit awareness of a sound that mediates the perceiver's awareness of its source. The explicit experience of a sound and the experience of its source that it gives rise to are, so conceived, distinct experiences, even if the former is a part or constituent of the latter.

In ``The Origin of the Work of Art'' Heidegger presents an opposing view:
\begin{quote}
    We never really first perceive a throng of sensations, e.g., tones and noises, in the appearance of things \ldots\ ; rather we hear the storm whistling in the chimney, we hear the three-motored plane, we hear the Mercedes in immediate distinction from the Volkswagen. Much closer to us than all sensations are the things themselves. We hear the door shut in the house and never hear acoustical sensations or even mere sounds. \citep[151--152]{Heidegger:1935uq}
\end{quote}
Nothing hangs on Heidegger's apparent acceptance of the empiricist identification of sound with acoustic sensation. What is important is Heidegger's denial of the central neo-Berkelean claim, that we hear the source of sound by hearing the sound. Rather, we hear the source of sound directly.

In undergoing an auditory experience, the source of a sound is directly or immediately present in that experience. When we attend to our auditory experience, as Heidegger invites us to, we attend to the sources of sounds and rarely, if at all, to the sounds in distinction from their sources. In hearing the storm whistling in the chimney, the three motored plane, the Mercedes in immediate distinction from the Volkswagen, there is no explicit experience of their sound distinct from hearing these sources. That is consistent with maintaining that hearing a source necessarily involves accoustical sensation. And yet Heidegger is clearly denying the neo-Berkelean claim that he hear the source of a sound by hearing the sound. There is one experience, hearing the storm whistling in the chimney, and no distinct explicit experience of its sound. That is a negative result about how to characterize aural indirection, the presentative function of sound: There is more to hearing a source by hearing its sound, in the sense required by the neo-Berkelean, than the necessary accompaniment of the former by the latter.

Heidegger exaggerates when he claims that we never hear acoustical sensations or mere sounds. For he goes on to maintain that we can manage to hear sounds in distinction from their sources only by adopting the aural equivalent of the painterly attitude:
\begin{quote}
    In order to hear a bare sound we have to listen away from things, divert our ears from them, i.e., listen abstractly. \citep[152]{Heidegger:1935uq}
\end{quote}
We can get a sense of how difficult it is to adopt this attitude by considering Pierre Schaeffer's piece ``Étude aux chemins de fer'' (1948). Whereas traditional composition begins with an abstraction, the score, which is made concrete in playing it, \emph{musique concrète} begins with concrete sounds and abstracts them into a composition through tape looping and sound collage. Yet despite these distancing techniques, the material sources never completely fade from the perceived soundscape. We get a sense of the train's speed, its size, the space surrounding the tracks as well as the space of the interior given the character of the resonance. Working in Schaeffer's studio, Karlheinz Stockhausen addressed these problems in the method of tape composition deployed in ``Étude'' (1952). He recorded prepared low piano strings struck with an iron bar and sliced off the heads of the recorded sounds, thus eliminating information about the attack and other material features of the source. These short headless segments were further repeated to form the basic tones of the piece. The effect is uncanny. However, the very uncanniness is itself partly a product of the limitation, or at least a variant of it, that beset Schaeffer's earlier piece. The tones are uncanny in that there are at once unfamiliar, indeterminate, and yet familiar, though, enigmatically, placing them proves elusive. Indeed, at the end of his career, Schaeffer pronounced \emph{musique concrète} a failure, claiming, perhaps ironically, to have wasted his life. Heidegger's observation was the principle obstacle---it is very difficult to listen away from things and hear bare sounds, to hear sounds without also hearing their sources. And so there are limits to the degree of abstraction that can be achieved with \emph{musique concrète}. It is telling, in this regard, that Stockhausen abandons tape composition for the generation of tones with sine-wave generators as he continued to explore electronic composition.

The Berkelean alternative raises an explanatory challenge to the neo-Berkelean---\-to explain how we can experience a source by experiencing its sound. How is the immediate presentation of sound in auditory experience, the mediate presentation of its source? The aural indirection, as the neo-Berkelean conceives of it, the presentative function of sound, is unlike ordinary cases of perceiving one thing by perceiving another. One might see where the Shogun's army is encamped by seeing the smoke and steam of their cooking rice. But the Shogun's army is directly perceptible---and presents a suitably terrifying aspect---in the way that the sources of sounds could not be, at least by the neo-Berkelean's lights. What is needed is an explanation of how one can hear a source by hearing a sound. What is needed is an explanation of the presentative function of sounds, how the immediate presentation of sound in audition constitutes the mediate presentation of its source (analogous, in many ways, to the presentative function of sense data, at least according to many sense-datum theories, see, for example, \citealt{Price:1932fk}). The Heideggerian alternative is a challenge to the very possibility of such an explanation. At the very least, in undergoing an auditory experience, we do not attend to sources by attending to sounds---according to Heidegger, in normal cases, there is no sound that we are attending to. Any account of the presentative function of heard sound would involve the explicit experience of that sound, but, according to Heidegger, there is no such experience. There is just the auditory experience of the storm whistling in the chimney, of the three motored plane, of the Mercedes in immediate distinction from the Volkswagon. A neo-Berkelean cannot afford to be as sanguine about the Heideggerean alternative as they may be tempted to be about the Berkelean alternative. A promissory note is worth nothing in the face of an inability to repay.

In this chapter and the next, though some continue to accept the Berkelean view that sounds are the sole object of audition (see, for example, \citealt{Smith:2002sa}), I propose to simply set the extreme Berkelean alternative to one side and accept that we hear, in addition to the sounds, their sources as well. 

% section the_berkeley_heidegger_continuum (end)

\section{Sounds and Their Sources} % (fold)
\label{sec:sounds_and_their_sources}

The objects of perception are particulars. Perception is a primitive kind of conscious encounter, and one can only encounter particulars. Entities of diverse ontological categories count as particulars. Thus not only are ordinary material substances particulars, but so are property instances, events, and processes.

Colors are spatially extended, at least in the sense of being instanced only by spatially extended things. We can imagine smaller and smaller things being colored, but we cannot conceive of a thing without extension exhibiting color. Similarly sounds are temporally extended. We can imagine hearing briefer and briefer sounds, but we cannot conceive of a sound without duration. 

The temporal dimension of sound, however, is not exhausted by their having a beginning and an end. In this regard they are no different from mortal animals. But unlike natural substances such as animals and other objects, as well as entities of distinct ontological categories such as states, sounds have a distinctive way of being in time. Like events, at least as the three-dimensionalist conceives of them, sounds unfold in time (see \citealt{Fine:2006fk}; though for criticism see \citealt{Sider:1997fk,Hawthorne:2008uq}). Unlike states which are wholly present whenever they obtain, sounds are not wholly present at every moment of their sounding. They are spread over the interval of time through which they unfold. So sounds have a temporal mode of being that events have. Perhaps some sounds, such as the sound of the wind, or the roar of a waterfall, are more like processes than events \citep[4]{Broad:1952kx}. However, that distinction is not presently relevant, and at any rate, processes are no more wholly present at each moment of their occurrence than events. 

That sounds are not wholly present at every moment of their sounding precludes them from being wholly present in auditory experience at every moment of their hearing. If we further assume that perceptual experience only presents what could be present at any given moment, then a puzzle about the very possibility of audition arises, as Prichard observes: 
\begin{quote}
	We should ordinarily be said to \emph{hear} certain noises, e.g. the sound a bell or the note of a bird. But any sound has duration, however short. If so, how can it ever be true that we apprehend by way of hearing---or more generally perceiving---can only exist at the moment of hearing, and \emph{ex hypothesi}, at least part of the sound said to be heard is over at the moment of hearing, and strictly speaking it is \emph{all} over. And the difficulty seems a double one. For since a sound has duration, it cannot exist \emph{at} the moment of hearing, and therefore we cannot hear a \emph{present} sound---for there is no such thing. And if it is over and so not existing at the moment when we are said to hear it, it cannot be \emph{heard}. Therefore, it seems, it is impossible hear a sound. \citep[47]{Prichard:1950ly}
\end{quote}
The most straightforward way to deal with this puzzle is to abandon the principle that generates it---that perceptual experience only presents what could be present at any given moment. After all, as we have seen (chapter~\ref{sec:assimilation}), this is the principle that was driving the Grand Illusion hypothesis. If we abandon this principle, then we may conclude that since sounds are spread over time, their sensory presentation must also be. Auditory experience unfolds with its object. So auditory presentation, due to the distinctive temporal nature of sound, has duration. Auditory presentation is the disclosure of a sound unfolding through its temporal interval. It discloses its object, then, over time, just like haptic presentation. However, whereas haptic perception may disclose events, it discloses, as well, relatively static features such as texture and temperature. Sounds, by contrast, are essentially dynamic entities, not wholly present at any moment of their existence but unfolding in time. 

Sounds may be particular events or processes, and so have a mode of being that suffices to distinguish them from entities belonging to other ontological categories such as bodies and states, but what of other \emph{audibilia}? Must all audible objects unfold through time? Or is this just a feature of, in Peripatetic vocabulary, the proper objects of audition? 

According to \citet[4]{Broad:1952kx}, we ordinarily speak of hearing bodies. So when Big Ben strikes the time, and is in earshot, we can say that we can hear Big Ben. However, Broad concedes little in acknowledging this point of usage since he also observes that it takes but a little pressure to convince ``the plainest of plain men'' that ``hearing Big Ben'' is shorthand for ``hearing Big Ben striking''. If we accept Broad's suggestion, then we only hear Big Ben insofar as it is a participant in a sound-generating event. And when we do, what we strictly speaking hear is Big Ben striking and not Big Ben, that is, not the body, but an event the body participates in that is the cause of the propagation of the patterned disturbance. It is not clear that Broad thinks that even Big Ben striking is an object of audition. ``Hearing Big Ben'' is meant to be equivalent to ``hearing such and such a noise and taking it to be coming from Big Ben''. But taking the sound that one hears to be generated in an event in which Big Ben participates may be a cognitive, rather than a perceptual, activity or stance. Let us set aside any doubts that Broad may have entertained, and accept, with Heidegger, that we hear not only the sound of Big Ben striking but we hear, as well, Big Ben striking. The view we will have arrived at is one according to which we hear sounds and their sources. Sounds are events or processes and their sources that we hear are the events and processes that generate those sounds. Such a view would be a step closer to vindicating the general claim that \emph{audibilia}, and not just sounds, have the distinctive temporal mode of being of events or processes. Full vindication would further require assurance that sounds and their sources are all that we, strictly speaking, hear.

Allow me to elaborate on sources and their hearing and engage in speculation about a hypothetical sense in which we may be said to hear bodies consistent with the principle, if true, that audition only presents objects with the distinctive temporal mode of being of events or processes.

First, the elaboration. It concerns the sources of sound. In the discussion above, for convenience, I have silently substituted a philosophically motivated precisification for the ordinary notion. Specifically, sources were claimed to be sound-generating events or processes. While it is true that the ordinary notion of a source is a causal notion, we also speak of objects or bodies being the sources of sound. We do so presumably because these bodies possess the causal power to engage in an activity which is a sound-generating event or process. Thus \citet{Casati:2013ca} speak of event sources and thing sources. In effect the precisification identifies sources with the body's activity that generates a sound. The \emph{prima facie} plausibility of this is abated in a philosophical \emph{milieu} where a broadly Humean metaphysics, with its focus on regularities among events, remains widely influential. For the broadly Humean framework encourage the conclusion that sources are events from the recognition that sources are causal. However, the precisification of the ordinary notion was not motivated by a Humean metaphysics. I believe that we should accept the Eleatic Visitor's teaching and acknowledge the being of capacity. (After all, it would be impious to deny the existence of virtue.) But once we do, we can see how sources may be, at once, bodies and causal. Bodies may be the sources of sound by possessing the causal power to sound, to engage in a sound-generating activity. The precification was not motivated by an adherence to a broadly Humean metaphysics but rather had a phenomenological motivation. Specifically, we are presently interested in the sources that we can be said to hear. The sources that we can be said to hear may be a narrower class than what may ordinarily be described as a source. Big Ben is a source of sound. But we don't hear Big Ben, at least not strictly speaking, we hear Big Ben striking. What we hear, strictly speaking, is not the body, but the body's sound-generating activity. 

Both sounds and the sources that we hear are like events or processes in that they are not wholly present at every moment of their occurrence. The speculation, intimated above, is that perhaps this is a general feature of \emph{audibilia}. Perhaps for something to be present in auditory experience it must have a particular temporal mode of being, it must unfold through time. This would preclude, by their very nature, entities such as bodies from being present in auditory experience since they would lack the requisite temporal mode of being. Earlier we noted Broad's helpful suggestion that perhaps ``hearing Big Ben'' is elliptical for ``hearing Big Ben striking''. 

As plausible as this may be, a worry may still persist. One of the uses to which audition may be put is to track a body's progress through the environment. We can listen to an animal's approach, say. And it might be thought that we are attending to the animal in audition in so listening out for them. Moreover, it might seem insufficient for the body to be attended to, that an event in which that body participates is present in auditory experience. Not every part of a visible body is seen, so why assume that every participant of an audible event is heard? How can we listen out for bodies, even though they are precluded, by their temporal mode of being, from being present in auditory experience? 

Bodies may not be present in auditory experience, but perhaps they figure in auditory experience in another way, if not as the intentional object of experience, then something very much like it. Bodies are, on the speculative hypothesis that we are entertaining, not present in auditory experience. Thus bodies are absent in auditory experience. And yet we can attend to bodies in audition. How could this be? 

Aristotle uses this kind of puzzle or \emph{aporia} about presence in absence to argue for, as we might put it, the intentional character of memory (\emph{De Memoria et Reminiscentia} 450\( ^{a} \)25--451\( ^{a} \)1). The Peripatetic response to the puzzle is to straightforwardly accept the claim of absence and reinterpret what purported to be a presentation instead as a kind of re-presentation. When one remembers Corsicus in his absence one contemplates a \emph{phantasma} caused by a previous perception of Corsicus and one conceives of the \emph{phantasma} as a likeness and reminder of Corsicus as he was perceived. How might the Peripatetic response, so abstractly described, be applied to the perceptual case of attending to bodies in audition? 

One obstacle to straightforwardly applying the Peripatetic response to the perceptual case of attending to bodies in audition is this: Memory and imagination are plausibly the primitive intentional capacities in our cognitive economy in the way that perception could not be, \emph{pace} \citet{Burge:2010uq}, if perception essentially involves an irreducible presentational element. And if our perceptual capacities are not intentional, but a necessary precondition for the possession of intentional capacities, then how would the Peripatetic response apply to the perceptual case of attending to bodies in audition?

Perhaps what is present in auditory experience may, nevertheless, constitute a natural image of what is absent. That is, perhaps we can understand hearing the body's sound-generating activity as providing the listener with a dynamic aural image of the body otherwise absent in audition. It is an image, indeed, as I have suggested, a natural image, like a fossil or a footprint \citep[for a recent general discussion of images see][]{Kulvicki:2014hb}. But unlike paradigmatic images it is not a visual image but an aural image (for the denial that there so much as could be such a thing see \citealt{Martin:2012af}). And while visual images are static, aural images, if such there be, would be dynamic as befitting their aural character. Hearing Big Ben striking, then, while not the presentation of Big Ben in auditory experience, would nevertheless provide the listener with a dynamic aural image of Big Ben. In order for this to be so the auditory presentation of a sound-generating event must involve at least the partial disclosure of the event's participants. Audition partially discloses an event's participant by presenting it as a participant of the audible event. It is the body's participation in the event, and not the body \emph{per se}, that is part of the event's audible structure. The disclosure of such audible structure is partial. Only those aspects of the body that are manifest in its participation in the audible event are disclosed. Furthermore, there is no guarantee that if a perceiver hears an event, they hear each of its participants, if any. But that is consistent with audition, in certain circumstances of perception, partially disclosing at least some of the participants in the unfolding audible event. It is only if we can hear Big Ben's participation in its striking that we can use that hearing to attend to Big Ben. It is only if we can hear Big Ben's participation, can that hearing provide us with a dynamic aural image of Big Ben and its activities that we exploit in attending to Big Ben in audition.
% section sounds_and_their_sources (end)

\section{The Wave Theory} % (fold)
\label{sec:the_wave_theory}

An ancient tradition identifies sound with motion. Plato and Aristotle claimed that sound is a motion in a medium. In the cosmology of the \emph{Timaeus} (67 a--c), sound is percussion in the air and the hearing of that sound is the movement it causes through the ears of the perceiver. For Aristotle, sound is motion in a medium, be it air or water (\emph{De Anima} 2 8 420\( ^{a} \)8--11, 420\( ^{b} \)11, \emph{De Sensu} 447\( ^{a} \)1--2; though see \citealt[60--1]{OCallaghan:2007xy} for an alternative interpretation; see also \citealt{Johnstone:2013la}). But the hearing of the sound, while it may involve the sound's acting upon the ears, the organs of audition, is no mere alteration but the exercise of a capacity (\emph{De Anima} 2 5). Though sounds involve the motion of a medium, Aristotle does not conceive of sound as propagating through the medium. When a solid, smooth object, such as a piece of bronze, is struck, it causes the medium, the air, say, to move in a single, continuous mass (\emph{De Anima} 2 8 419\( ^{b} \)33--420\( ^{a} \)2). The medium is a unity that communicates the movement of the distal body to the ear of the perceiver. Think of the way movement may be communicated through a single, continuous mass such as a stick. One may poke with a stick, without the poke propagating through the stick. Aristotle derives this conception of a medium as a continuous unity from Plato's account of perception in the \emph{Timaeus} (see \citealt[chapter 1]{Lindberg:1977aa}; the Stoic stick analogy, reported by Alexander of Aphrodisias, \emph{De Anima} 130 14, also plausibly traces to this source). 

Aristotle's Platonically inspired conception of a medium as a continuous unity shows that conceiving of sound as motion in a medium is not yet to conceive of successive motion through a medium in the way suggested by talk of propagation. Motion, \emph{kinēsis}, is Aristotle's general term for change and need not mean locomotion more generally. Consider one reason for thinking that the pattern disturbance propagates through the medium rather than the medium acting as a single continuous mass. Aristotle's conception of sound as a continuous unity could not explain why two perceivers located at different distances from a source hear the sound it generates at different times. Since it is a continuous unity, the mass of air is acting all at once, like the motion of a rigid stick. And since it does not involve successive movements through the medium, the different distances of the perceivers from the source should not make for the temporal difference of their perceptions. 

At any rate, this conception of a medium as a continuous unity did not long persist. Conceptions of sound as motion in a medium were common in the Middle Ages, if variously developed \citep{Pasnau:2000aa}. For present purposes, we shall understand The Wave Theory, more specifically, as identifying sound with a certain kind of event, the propagation, in all directions, of a patterned disturbance---longitudinal pressure waves that vary in amplitude and frequency---through a dense and elastic medium such as air or water (for contemporary defenses of the wave theory, though this is not their primary aim, see \citealt{OShaughnessy:2009aa} and \citealt{Sorensen:2009aa}). Notice, on The Wave Theory, herein understood, the sound event is not the patterned disturbance in a dense and elastic medium so much as it is the propagation of a patterned disturbance through that medium. 

Among events, sounds have a distinctive temporal character. According to \citet{OShaughnessy:2009aa}, sounds have a ``double duration'', the way other events, such as the alteration of a body's color, do not. When I hear the call of a feral parrot, my hearing of the sound has a certain duration. Suppose I heard the parrot's call from its onset, so that I heard the whole of the call. But notice, on The Wave Theory, the sound does not cease to exist at that moment. At a later moment, as the patterned disturbance continues to propagate in the dense and elastic medium, another perceiver, situated further from the parrot than me, may subsequently hear that same parrot's call. The first duration is determined by the length of the patterned disturbance and the speed at which it is traveling. It is the duration of a potential hearing of the sound. The second duration is determined by how long the patterned disturbance propagates before completely eroding due to the resistance offered by the dense and elastic medium (as well as other potential obstructors).

The Wave Theory, so understood, is subtly but crucially different from the view that \citet{OCallaghan:2007xy} dubbed The Event View. On both views, sounds are particulars, indeed, particular events. But whereas on The Wave Theory, the event is the propagation of the patterned disturbance through a dense and elastic medium, on The Event View, sounds are the events that cause a patterned disturbance to propagate through a medium. On The Wave Theory, the sound event, in a perfectly elastic medium, and ignoring its density, may be envisioned as an ever expanding sphere, the patterned disturbance propagating in every direction from the source (\citealt{Sorensen:2009aa}; compare also Lucretius \emph{De Rerum Natura} \textsc{iv} 603, \citealt{Smith:2001aa}: ``Moreover, a single utterance distributes itself in all directions.''). It is like an expanding ripple caused by a drop in an otherwise calm body of water, except the sound event occurs in three dimensions, not two, and so takes the form of a sphere rather than a circle. On The Event View, the sound event exhibits no such structure. Rather, it is the striking, bowing, grinding, vibrating, resonating, \ldots\ whatever kind of event involving the material source sufficient to propagate a patterned disturbance through a dense and elastic medium, should there be one. This last qualification reveals a further important difference. Whereas on The Wave Theory, the existence of sound depends upon a medium in which the event transpires, on The Event View, sound is existentially independent of a medium. An event involving a material source may be sufficient to cause the propagation of a patterned disturbance through a dense and elastic medium and may yet occur in the absence of such a medium. The existential independence of sound from a medium on The Event View thus allows for sound in a vacuum in the way The Wave Theory could not \citep{OCallaghan:2007xy,OCallaghan:2009aa}.

The Wave Theory, on the present understanding, is an idealized refinement of a traditional view. It represents a metaphysical genus, or class of views, insofar as it admits of further refinements. Are sound events, as The Wave Theory conceives of them, plausibly the objects of audition?

Traditionally, the phenomenology of auditory experience was thought to support The Wave Theory, or at least some version of it. (Though there are, of course, contemporary dissenters. Some of their concerns are addressed in the subsequent section~\ref{sec:phenomenological_objections}) After all, our auditory experience seems to present an emanative phenomenology. Within auditory experience, sounds appear to emanate from their sources. Sounds are heard to come from their sources. And, at least in the context of The Wave Theory, it is natural to understand this as the phenomenological reflection, in auditory experience, of the direction of the propagation of the patterned disturbance. If it is, then an emanative phenomenology potentially contributes to the fitness of the animal, at least with its capacity to hear ecological sound, since the direction of the propagation of the patterned disturbance carries important information about the location of its source. Hearing the approach of another can be of vital concern be it predator or prey.

In ``Elementary reflections on sense-perception'', \citet{Broad:1952kx} provides a careful description of the emanative phenomenology of audition, by contrasting the hearing of sounds with the seeing of colors. Colors are seen to inhere in the surfaces of bodies in a spatiotemporal region located at a distance from the perceiver. In the rare case of a colored event such as a flash or an explosion, the color of the flash, say, is seen confined to the remote spatiotemporal region of its occurrence. Hearing sounds are crucially different, in this regard, from seeing colors:
\begin{quote}
	But the noise is not literally heard as the occurrence of a certain sound-quality within a limited region remote from the percipient's body. It certainly is not heard as having any shape or size. It seems to be heard as coming to one from a certain direction, and it seems to be thought of as pervading with various degrees of intensity the whole of an indefinitely large region surrounding the centre from which it emanates. \citep[5]{Broad:1952kx}
\end{quote}
In this passage, Broad makes clear not only the sense in which a sound is heard to emanate from its source, but he also connects this aspect of auditory phenomenology with a thesis in the metaphysics of sound. For suppose that this emanative phenomenology of auditory experience were determined by an aspect of what it presents, then the sounds that we hear would involve a propagation, in every direction, from the source, of a patterned disturbance that can vary as it travels through a dense and imperfectly elastic medium. That is to say, Broad is explicitly linking the emanative phenomenology of auditory experience, if veridical, with The Wave Theory. \citet{Broad:1952kx}, however, should not be read as necessarily endorsing The Wave Theory here. The description of the emanative phenomenology of auditory experience is part of a larger task of specifying the phenomenological differences between vision, audition, and touch, phenomenological differences that are ultimately belied by the common causal mechanisms that underly all of our sensory modalities.

The Wave Theory not only coheres with, and would explain well, the emanative phenomenology of auditory experience, if veridical, but it would explain, as well, ordinary practices of identifying and re-identifying sound. Ordinarily, we allow that two perceivers located at different distances from a material source may hear the same sound, though at different times, and though their experience of that sound may differ. The sound may be louder for the perceiver located nearer the source, for example. And so the experience of the sound for the perceiver located near and far may differ, and yet it is the same sound that they hear. When invited to envision the sound event, as The Wave Theory conceives of it, as an ever expanding sphere, we were invited, as well, to make certain idealizations, that the medium through which the patterned disturbance propagates is perfectly elastic and that its density made no difference the propagation of the patterned disturbance. Of course, the air and water through which we normally hear sounds are dense and imperfectly elastic. And that is presently relevant. For that means that the patterned disturbance will erode as it propagates through the imperfectly elastic medium. As it loses energy, it will become, not only less loud, but fine detail of the top end will be lost early on and perhaps only the bass will persist the furthest. That the two perceivers, located at different distances from the source, hear the sound at different times is due to the different distances the patterned disturbance had to propagate from the source to reach them. And that the auditory experience of the two perceivers differ in character is due, in part, to the erosion of the patterned disturbance as it propagated through a dense and imperfectly elastic medium. Nevertheless, they can be said to hear the same sound since sound, on The Wave Theory, is not identified with a patterned disturbance but with the propagation of a patterned disturbance through an elastic medium. If sound were identified with a patterned disturbance, then since the patterned disturbance differed in the auditory stimulation of the two perceivers, they would be hearing different sounds. But if sound were, instead, identified with a propagation of a patterned disturbance through an elastic medium, the two perceivers may be said to hear the same sound even if they are hearing it at different stages of its career.

% section the_wave_theory (end)

\section{Auditory Perspective} % (fold)
\label{sec:auditory_perspective}

The auditory experiences of two perceivers hearing the same sound at different stages of its career differ. Is this a matter of their having different auditory perspectives on the same sound? Or consider the related case. Suppose that the perceiver is in the presence of a continual sound, the roar of a waterfall, say. Does the perceiver gain a new perspective on that sound by approaching its source? \citet[135]{Smith:2002sa} denies that this is a difference in perspective if that involves potentially disclosing previously hidden aspects of the sensible object. We saw this feature at work in haptic perspective (chapter \ref{sec:assimilation}). Specifically, the haptic activities, the distinctive ways the perceiver is handling the object, occurring in an ego-centrically and teleologically structured peripersonal space, can disclose previously hidden corporeal aspects of the object of haptic investigation and are, to that extent, partial perspectives on that object. 

Should we accept Smith's denial that auditory perception potentially discloses previously hidden aspects of sound? At least part of the difference between hearing the waterfall from far away and hearing it nearby is due to the erosion of the patterned disturbance, continually generated by the waterfall, as it propagates through a dense and imperfectly elastic medium. There are at least two ways to use this observation to undermine Smith's denial. The first way couples that observation with the claim that since the patterned disturbance carries material information about its source, some of that information, at least, is lost as the patterned disturbance erodes. Suppose the perceiver initially hears the sound but at such a distance that they are unable to recognize it as the sound of a waterfall. As they approach the sound, at some point, if circumstances are propitious, they can recognize the material source of the sound. The difficulty with the first way is that it is not inconsistent with Smith's denial. All that has been claimed is that auditory perception may disclose previously hidden aspects of the material source of the sound, but Smith only denies that auditory perception may disclose previously hidden aspects, not of the material source of the sound, but of the sound itself. The second way of developing the observation avoids this difficulty. With the erosion of the patterned disturbance in a dense and imperfectly elastic medium, not only is information about the material source lost, but so are audible features of the sound itself, or at least audible features of the sound possessed at a certain stage of its career. At a certain distance one may no longer hear the fine play of overtones in a sound, say. As we shall see, Smith himself provides an example of hearing a previously hidden aspect of a sound, though he does not, himself, recognize it as such.

Smith denies that hearing a sound at different distances from its source affords the perceiver with distinct auditory perspectives on that sound if a perspective potentially discloses a previously hidden aspect of the sound. Smith, however, does not himself accept the antecedent of that conditional. Following Husserl and Merleau-Ponty, Smith suggests, instead, that it is sufficient for the notion of perspective to get a grip that there are better or worse perspectives on the given object. And Smith accepts that there are better or worse perspectives in hearing a sound:
\begin{quote}
	We can discover how loud a distant sound really is, or how hot a fire really is, by moving closer to them. If we want to hear the ticking of a pocket-watch ``properly,'' we put it close to our ear; we behave very differently when it is a matter of hearing a cannon fire. \citep[135]{Smith:2002sa}
\end{quote}
While I agree with everything claimed in this passage, I fail to see how the contrast between a conception of a perspective as potentially disclosing a previously hidden aspect and a conception of a perspective as affording a better or worse perspective can be coherently maintained, at least as Smith apparently understands that contrast. Consider Smith's first example, discovering how loud a distant sound really is. Approaching a waterfall, one eventually reaches a position from which one can hear just how loud that waterfall really is. That is to say, it is plausible that what makes hearing the sound of the waterfall from that position a better perspective is precisely that it discloses a previously hidden aspect of the sound, the relative intensity of its loudness. Similarly, it is plausible that what makes feeling the radiant heat of a fire from a certain position a better perspective than a position located further from the fire is that it discloses just how hot the fire really is. And while I agree that placing a pocket-watch close to the ear is the ``proper'' way to listen to its ticking, I suspect that this is because the perceiver is in a position to hear the workings of the watch's mechanism, in which case what is disclosed in the ``proper'' perspective is the material source of the sound. One only hears the watch ticking, understood as the sound of the watch, if one hears the watch ticking, understood as the workings of the watch's mechanism.

Complicating matters, better and worse are said of in many ways. Specifically, whether a position from which a perceiver may hear a sound affords the perceiver with a better or worse perspective on that sound depends upon what is practically at stake in describing the perspective as better or worse. That is to say, it may be an occasion-sensitive matter in Travis' \citeyearpar{Travis:2008la} sense. I own an otherwise fine recording of an Anthony Braxton solo performance marred only by the ill-judged positioning of the microphone. The microphone picked up the clacking of the keys while Braxton played his instrument thus partially obscuring the sound of that playing. One moral might be that one shouldn't stand close enough to the saxophone to hear the clacking of its keys. Sound aesthetic advice. But suppose one is moved by non-aesthetic concerns. A student of Braxton's playing might gain insight into Braxton's technique by hearing the clacking of the keys. So whether a given position counts as affording the perceiver with a better or worse perspective on the audible events unfolding in the perceiver's environment depends upon the practical point and interest in evaluating that perspective.

The position from which a perceiver may hear the sound of a distant source may provide a better or worse perspective, where better and worse is said of in many ways. Sometimes, for certain practical purposes, what makes a perspective better is that it potentially discloses previously hidden aspects of a sound, be it the delicate play of overtones or just how loud that sound really is. Sometimes what makes a perspective better is that it potentially discloses a previously hidden aspect of the source, as when the watch is close enough to hear the workings of its mechanism. Audition provides the perceiver with a partial perspective on the audible events unfolding in the natural environment. Like visual and haptic perspective, auditory perspective is not only partial but occurs in an ego-centrically structured space. Sometimes it is difficult to make out the direction of a sound. Sometimes hearing a sound provides us with only a general sense of its direction. Still, it is possible for us to hear a sound from behind, or to the left. Like vision, and unlike haptic touch, audible events are heard to transpire in an ego-centrically structured extrapersonal space. Some of the distal events that we hear lie far beyond the limits of peripersonal space, the space within which we may immediately act with our limbs. However, unlike vision, audition affords the perceiver 360 degree awareness of extrapersonal space. 

% section auditory_perspective (end)

\section{Phenomenological Objections} % (fold)
\label{sec:phenomenological_objections}

According to \citet{Pasnau:1999ss}, if The Wave Theory were true, then auditory experience would be illusory. Pasnau claims that we do not hear sounds pervading a volume, at least not normally, rather we hear sounds as located at their sources:
\begin{quote}
	We do not hear sounds as being in the air; we hear them as being at the place where they are generated. Listening to the birds outside your window, the students outside your door, the cars going down your street, in the vast majority of cases you will perceive those sounds as being located at the place where they originate. At least, you will hear those sounds as being located somewhere in the distance, in a certain general direction. But if sounds are in the air, as the standard view holds, then the cries of birds and of students are all around you. This is not how it seems (except perhaps in special cases \ldots). \citep[311]{Pasnau:1999ss}
\end{quote}
Other recent writers who have made similar claims about the distal character of experienced sound include \citet{Casati:1994aa} and \citet{OCallaghan:2007xy}.

Auditory experience, so conceived, lacks the emanative phenomenology that \citet{Broad:1952kx} contrasts with the phenomenology of color vision. Rather, sounds are heard to be confined to the remote spatiotemporal region of their origin. Indeed,  \citet{Pasnau:1999ss} understands the distal senses of vision and audition, at least, as being on a par. And since Pasnau follows Locke in treating sounds as sensible qualities (though see \citealt{Pasnau:2009ys}), he is led to conceive of auditory experience as affording the perceiver with awareness of auditory qualities confined to the remote spatiotemporal region of their source. In this way is the analogy of audition with vision, \emph{pace} Broad, completely reinstated. 

Allow me to make a brief digression to highlight an important point of disagreement. Despite O'Callaghan's \citeyearpar{OCallaghan:2009aa} emphasis on Pasnau's \citeyearpar{Pasnau:1999ss} commitment to a Lockean metaphysics of sound, it is incidental to the aim of that paper which is concerned with whether sound qualities inhere in the medium or in the distal source. That question, or a version of it, can be posed without assuming the Lockean metaphysics: Is sound located in the medium or at or near its source? Though incidental to the aim of the paper, the Lockean metaphysics of sound was not unmotivated. Rather, Pasnau is moved by the idea that sensible objects belong to a common metaphysical genus. This is a monism of the sensible. Specifically, Pasnau seems attracted to a monism of at least the objects of the distal senses. And since colors are conceived to be qualities, sounds must also be. Later, \citet{Pasnau:2009ys} abandons the Lockean metaphysics of sound, coming to conceive of sounds as particular events. However, given the monism of the sensible, and the dynamic aspects of the physics of color generation, Pasnau suggests that colors might themselves be events, the event of color. Allow me to register a disagreement, though without offering a reason, it is perhaps merely the expression of a difference in intellectual temperament. The disagreement concerns less Pasnau's (unjustly neglected) Heraclitean metaphysics of color, than the role the monism of the sensible plays in motivating it. Rather than thinking of sensible objects as belonging to a common metaphysical genus, I am impressed by the heterogeneity of the sensible. Far from adhering to the monism of the sensible, on Austinian grounds, I am attracted to a pluralism of the sensible. Just consider the diversity of \emph{visibilia} alone. We see opaque natural bodies such as Price's \citeyearpar{Price:1932fk} red tomato, but we also see transluscent volumes, flashes, reflections, mirror images, rainbows, mirages, shadows, holes. Perhaps as \citet{Sorensen:2004jk,Sorensen:2008kx,Sorensen:2009aa} suggests, we can see darkness and hear silence. I raise the issue without pursuing it. The important point is whether there is unity or diversity in the metaphysics of sensible objects would be relevant to the kind of explanatory role they could play. Compare the way the apparent diversity of the tangible initially seems puzzling given the explanatory framework of \emph{De Anima} 2. The tangible comprises a diverse range of qualitative contrasts---hot and cold, wet and dry, smooth and rough. Given the Platonic strategy of explaining perceptual capacities in terms of the presentation of their proper objects, why are these contrasts perceived by touch, a unitary perceptual capacity, rather than there being separate perceptual capacities for temperature, moistness, and texture? How could touch, a unitary sense, be explained as the capacity to present a diverse range of tangible contrasts? Aristotle's discussion of touch is a historically salient example of how diversity among sensible objects potentially limits their explanatory role.

Pasnau's argument that sounds are heard to be at or near their sources raises a couple of questions. The first question concerns the metaphysical commitments of The Wave Theory. If according to The Wave Theory, sounds have locations, where are the sounds, so conceived, located? After all, it is only if sounds, as The Wave Theory conceives of them, could not be located at their sources is there an alleged conflict, according to Pasnau, with the phenomenology of auditory experience.  The second question concerns the phenomenology of auditory experience. In cases where perceivers genuinely hear something in a distance are what they hear sounds or some other audible object? 

Begin with the second question, about the phenomenology of auditory experience, first. (Discussion of the location of sound according to The Wave Theory will be postponed until we discuss O'Callaghan's \citeyear{OCallaghan:2007xy,OCallaghan:2009aa} objection to the purported emanative phenomenology of auditory experience.) When one listens to the birds outside one's window, the students outside one's door, and the cars going down one's street, what is it that one is listening to? A flat-footed answer would be: birds, students, and cars, or at least their audible activities. But birds, students, and cars, while audible, are not themselves sounds but their sources, at least potentially. But the claim that the source of a sound is heard to be confined to a spatiotemporal region remote from the perceiver is not inconsistent with the sound it generates pervading the surrounding medium. Pasnau moves too quickly from cases involving hearing a distal source to concluding that the sound itself is heard to be remote from the perceiver. Once we allow that we hear not only sounds but their sources, a question naturally arises whether the audibly distal object that we hear is the sound or merely its source \citep[see][123, for a development of this worry]{OShaughnessy:2009aa}.

A similar issue affects Pasnau's discussion of the precedence effect:. 
\begin{quote}
	Even when there is a significant reverberation in a room, we do not hear it as such, as long as the reverberation comes to the ear between 1 and 35 milliseconds after the initial wave enters the ear. In such cases, we hear the sound as being located at its initial source. Although the reverberation affects the perceived loudness and quality of the sound, it does not enter into our perception of its location. (If the reverberation arrived more slowly than 35 milliseconds later, we would hear an echo. If it were faster than 1 millisecond we would hear the sound as centered between the source and the point of reverberation.) This is known as the \emph{precedence effect}. On the standard view, this effect has to be described as a defect in the system. For if the object of hearing is sound, and if sound is a quality belonging to the surrounding air rather than to its source, then the precedence effect would serve to filter out information about sound. The precedence effect, in other words, would stand in the way of accurate detection of sound. Yet this seems absurd, which points to another reason for giving up the standard view of sound. \citep[312--313]{Pasnau:1999ss}
\end{quote}
Once we allow that we hear not only sounds but their sources, then Pasnau's reasoning is undone from the beginning. If the object of hearing is the source of the sound, and the function of the auditory system is to afford the perceiver with auditory awareness of distal sources (see chapter~\ref{sec:the_function_of_audition} for further discussion), then there is nothing particularly mysterious about the precedence effect.

\citet[section 6]{Pasnau:1999ss} claims that The Wave Theory invites us to envision sounds as filling the air around us. But if all sounds fill the air around us, then we should hear them pervading the dense and imperfectly elastic medium through which they propagate. But in fact it is quite rare to experience sound as pervading a volume: ``Perhaps this is how we experience loud music in a disco, or a jack-hammer in a narrow street'' \citep[312]{Pasnau:1999ss}. But these are exceptional cases.

Does The Wave Theory have the consequence that sounds fill the air around us in a sense that is at odds with our auditory experience? Pasnau claims that most sounds do not audibly fill the medium. So filling the medium must be something audibly accessible. Consider a brief sound, a single call of a feral parrot, say, as opposed to the continuous sound of a waterfall. According to The Wave Theory, the sound of the parrot is the propagation, in all directions, of a patterned disturbance through a medium, in the present instance, the dense and imperfectly elastic air. In one clear sense, at any given moment, the only audible aspect of this complex event is the patterned disturbance as it is at that moment. The outer boundary of the sphere, the narrow band which is the patterned disturbance, is audible in the sense of being a potential proximal cause of the auditory experience of the sound. So while the complex event may be envisioned as a growing sphere, since the sound is brief, the only audible aspect of the sound is at the moving boundary of the sphere, the narrow band which is the patterned disturbance. After all, if a perceiver is placed within the sphere between the source of the sound and the narrow band at the sphere's outer boundary, they are no longer in a position to hear the call of the feral parrot. In one clear sense that may be so, but there are other, relevant senses of audibility. So, if circumstances are propitious, in hearing the feral parrot's call, we can hear the direction of the sound's propagation. We may even have a sense of how far off the source is. So aspects of the complex sound event are in another relevant sense audible and in this sense are not merely confined to the patterned disturbance at the outer boundary of the sphere. The Wave Theory, as herein described, is only committed to sounds being heard to fill the air in this latter sense. In this sense, something is audible if it is heard in hearing a sound. Of course, even on the first sense of audible, understood as a potential proximal cause of the perception of the sound, a continuous sound, such as the roar of a waterfall, will audibly fill the air---the continuously produced patterned disturbances will pervade the space between the perceiver and the waterfall. But as Pasnau observes, and The Wave Theory predicts, these are exceptional cases, like loud music in a disco, or the sound of a jack-hammer in a narrow street.

\citet[chapter 3.4]{OCallaghan:2007xy} criticizes The Wave Theory by attempting to undermine its phenomenological motivations. The Wave Theory is motivated, in no small part, by the purported emanative phenomenology of auditory experience. Thus sounds are heard to come from their sources. \citet{OCallaghan:2007xy,OCallaghan:2009aa} argues that, at least on a certain understanding of what hearing a sound coming from its source could be, sounds are not heard to come from their sources, and thus that auditory experience lacks the emanative phenomenology that would motivate The Wave Theory, if veridical.

How are we to understand hearing a sound coming from its source? O'Callaghan writes:
\begin{quote}
	It might be that sounds are heard to come from a particular place by being heard first to be at that place, and then to be at successively closer intermediate locations. But this is not the case with ordinary hearing. Sounds are not heard to travel through the air as scientists have taught us that waves do. \citep[34]{OCallaghan:2007xy}
\end{quote}
And O'Callaghan likens hearing a sound as coming from its source to hearing a sonic missile. Audible emanation or propagation of a sound from its source is being modeled on a specific kind of change, the locomotion of a body. Locomotion is a change in location over time. So locomotion is a species of change that pertains only to those entities, paradigmatically bodies, that possess location. 

I concede that on this understanding of what it is to hear a sound coming from its source, ordinary auditory experience lacks an emanative phenomenology. Hearing a sound coming from its source is not analogous with the audible locomotion of a body. However, that is not the only available understanding of hearing a sound coming from its source. Perhaps the audible emanation or propagation of sound is better modeled on a different kind of change. 

Prichard denies that waves and sounds, being what they are, are subject to locomotion. Only bodies move, and waves and sounds are not bodies:
\begin{quotation}
	But \dots\ I also made the same remark (viz.\ that only a body could move) to a mathematician here. What was in my mind was that it is mere inaccuracy to say that a wave could move, and that where people talk of a wave as moving, say with the velocity of a foot, or a mile, or 150,000 miles, a second, the real movement consisted of the oscillations of certain particles, each of which took place a little later than a neighboring oscillation.
	
	He scoffed for quite a different reason. He said that you could illustrate a movement by a noise---that, for example, if an explosion occurred in the middle of Oxford the noise would spread outwards, being heard at different times by people at varying distances from the centre, so that at one moment the noise was at one place and that a little later it was somewhere else, and in the interval it had moved from one place to the other.
	
	Now, of course, it was not in dispute that in the process imagined people in different places each heard a noise at a rather different time. The only question was, `Was the succession of noises a movement?', and I think that on considering the matter you will have to allow that it was not, and that what happened was that he, being certain of the noises, and wanting to limit the term `movement' to something he was certain of, used the term `movement' to designate the succession of noises, implying that this was the real thing of which we were both talking. But if this is what happened, then he was using the term `movement' in a sense of his own, and in saying that in the imagined case he was certain of a movement, he was being certain of something other than the opposite of what I was certain of. \citep[99]{Prichard:1950kx}
\end{quotation}
(\emph{Caveat Lector}: \citealt[430 n. 29; appendix,]{Burnyeat:1995fk} lampoons Aristotle for making similar claims by citing Prichard echoing them. I argue, that at least in this instance, Burnyeat is hoisted by his own petard, \citealt[chapter 3.2]{Kalderon:2015fr}.)

Prichard's point about wave movement can be put this way. Consider a wave propagating through a liquid mass. At any given moment, the liquid mass has a certain spatial configuration, and the wave form is instanced in a certain part of the liquid mass. At a later moment, the liquid mass will have a different spatial configuration, and the wave form will be instanced in a different part of the liquid mass. Prichard's point is that it is not the wave form that is moving in coming to be instanced in differently located parts of the liquid mass. Rather, the liquid mass is moving, or at least its parts, ``the oscillation of certain particles'', with the effect that the wave form is progressively instantiated. A change of state and travel are different (\emph{De Sensu} 6 446\( ^{b} \)28)

% Even if one accepted Pritchard's claims about wave movement, it might seem that Pritchard's corresponding claims about sound movement could be resisted, at least given some background assumptions. Suppose, first, that sounds are events. Moreover, suppose events are conceived as spatiotemporal particulars. Wouldn't events, so conceived, be capable of locomotion? After all, a fight can break out in a bar and spill out into the street. Fights are events. Didn't the fight change location?

% As will become clear, I believe the reasoning above is too quick as it involves implicit assumptions about locations appropriate only to bodies, indeed, only to certain kinds of bodies. Moreover, even granting that events are spatiotemporal particulars, 

I want to take up Prichard's suggestion that the propagation of patterned disturbance through an elastic medium should not be understood on the model of locomotion. At any rate, The Wave Theory, as herein described, naturally suggests an alternative model based not on locomotion, but on growth. After all, on The Wave Theory, the sound event was envisioned as an ever expanding sphere. Growth, like locomotion, has direction. The emanative phenomenology of auditory experience, our hearing sounds as coming from their sources, is the partial disclose, in audition, of the direction of the growth of the sound event.

In cases of growth, the parts of the whole may be in motion, without growth reducing to such motion, but that does not mean that the whole is in motion, at least not in the specific sense of locomotion, a change in location over time. It is not in general true that motion in the parts of the whole involves motion, understood as locomotion, of the whole. So consider a perfect sphere rotating on a central axis. Since it is rotating, its parts are in motion. Indeed they are in motion in the specific sense that the parts of the sphere are changing their location over time. However, the sphere, while moving in some sense---it is, after all, rotating---is not moving in the specific sense of locomotion. If the location of the sphere is the bounded spatial region occupied by that body, then though its parts are moving in rotating, it is rotating in place, and so not changing its location over time. Similarly, while growth may involve the motion of the parts of a whole, without reducing to such motion, there is a sense in which a whole may grow without changing its location. In which case growth and travel are distinct.

In the 1966 film, \emph{Fantastic Voyage}, a submarine, \emph{The Proteus}, and its crew, consisting of a surgical team, the skipper, and a security agent are miniaturized and injected into the blood stream of a defecting Russian scientist who has suffered a blood clot in the brain, an injury sustained in his escape. Their mission is to destroy the blood clot, inoperable by conventional means. Eventually, the surviving crew emerge from the tear duct of the patient and return to their normal size in the medical laboratory. The surviving crew, in returning to their normal size, grow. Wearily, and dramatically, they are standing in place. While their boundaries may be moving in returning to normal size---their boundaries are changing location over time throughout this process----the crew themselves are not engaged in locomotion. They are standing in place.

As should be evident from the Prichard passage cited earlier, a qualification is needed. Specifically, it is not the claim that the crew are standing in place that needs qualification, but that their boundaries are moving. Initially this might seem unproblematic since their boundaries are located and their locations are changing. However, as \citet[103--4]{Derrida:2005aa} reminds us, boundaries are abstract, on some understanding of that notion. They are at least immaterial. And as \citet{Prichard:1950kx} reminds us, the only material things that are moving are the parts of the bodies of the crew members. I believe that there is a way to retain talk of the movement of boundaries consistent with Prichard's insight. Aristotle distinguishes two ways in which something may move: 
\begin{quote}
	There are two senses in which anything may be moved either indirectly, owing to something other than itself, or directly, owing to itself. Things are indirectly moved which are moved as being contained in something which is moved, e.g. sailors, for they are moved in a different sense from that in which the ship is moved; the ship is directly moved, they are indirectly moved, because they are in a moving vessel. (Aristotle, \emph{De Anima} \textsc{i} 3 406\( ^{a} \)3--8; Smith in \citealt[9]{Barnes:1984uq})
\end{quote} 
Perhaps in cases of growth, without growth reducing to motion, what directly moves, as Prichard insists, are parts of bodies, and what indirectly moves are their boundaries. The change in the location of their boundaries, an indirect motion, is consequent upon the direct motion of the parts of the bodies. Whereas an appropriate body may be said to contain within itself the power of locomotion, a boundary---an abstraction from the body---does not contain within itself the power to change its location over time. Its motion is at best indirect, consequent upon the direct motion of other things. Let a boundary be said to bind the body whose boundary it is. Then, echoing \citet[174]{Witt:1995kx}, we may say: ``Here the relationship is not one of parts to wholes, or contents to containers, but rather'' one of binding of bodies.

Typically, at least for rigid bodies, at least some of the time, their location can be understood as the spatial region encompassed by their boundaries. However, what the example of the surviving crew of \emph{The Proteus} reveals is that this principle fails of bodies generally. If the locations of the surviving crew members are the spatial regions encompassed by their boundaries, then since their boundaries are moving, at least indirectly, so must the crew. But the crew is standing in place. This last judgment must involve a different understanding of what it is for a person to be located where they are.

Being a rigid, solid body may be sufficient, in certain practical circumstances, to locate that body within the spatial region encompassed by its stable and determinate boundaries. But not all bodies possess stable and determinate boundaries. ``Where and what exactly is the surface of a cat?'' asks \citet[lecture 9]{Austin:1962lr}. Even so, in cases where an entity possesses location but lacks stable and determinate boundaries, its location must be understood in terms other than the space encompassed by its stable and determinate boundaries, for it lacks such boundaries.

So far we have been discussing the location of bodies, but what of events? At least some events have locations. Battles are named after the locations where they transpired, or at least significant sites nearby. Duke William \textsc{ii}'s victory over Harold Godwinson in 1066 took place northwest of Hastings. And, sometimes at least, events and processes can change their location. The fight erupted in the bar and spilled out into the street. The conga line began in the dining room and wound its way into the living room.

According to the \citet{Lemmon:1967aa} criterion, events are individuated by the spatiotemporal regions of their occurrence. Suppose, for the sake of argument, that events always involve the activities of bodies that are their participants. (I doubt very much that this principle is true on Nietzschean, and, ultimately, Heraclitean, grounds---there is no lightning that flashes, just the activity, the flashing, \emph{Zur Genealogie der Moral} 1 13.) Finite bodies are generated and destroyed, and while they exist, they occupy space, so we can envision their careers as spacetime worms. Now consider the segment of a spacetime worm bounded by the beginning and end of an event of which it is a participant. By the Lemmon criterion, the event itself is individuated by the mereological sum of the segments of the spacetime worms of its participants. If accepted, it would follow that events are located, indeed in the spatial region of their occurrence understood as the total space occupied by their participants at any given moment of the event's occurrence. However, as Davidson observes, one can accept that events are spatiotemporal particulars, without accepting the Lemmon criterion. ``An explosion is an event to which we find no difficulty in assigning a location, although again we may be baffled by a request to describe the total area'' \citep[304]{Davidson:1969da}. Even if we accept that events are located where they occur, the location of an event may be said of in many ways. It may be an occasion sensitive matter what counts as the location of an event. 

Where is the sound event? An answer may depend upon what is practically at stake in asking the question. On one natural understanding of the location of a sound, sounds are where we hear them. On that understanding, sounds are located at the intersection of the hearer and the propagation of the patterned disturbance. That understanding emphasizes the actualization of sound in hearing (compare \emph{De Anima} 3 2 426\( ^{a} \)2--426\( ^{a} \)26). On many occasions, locating a sound where it is heard is both natural and serviceable.  On other occasions, governed by different practical concerns, the location of a sound may be understood in a different way. 

On occasions where a perceiver-dependent location of a sound would be inappropriate, and given that the sound event, as conceived by The Wave Theory, lacks stable boundaries (they are in constant indirect motion), we might locate the sound event at its epicenter, the point from which the patterned disturbance is propagating in every direction, at its source. That the boundaries of the sound event are in constant indirect motion would suffice for bafflement at a request to describe its total area. And given the neat symmetry of the event, its boundaries are moving in every direction from its source, it is natural to assign its location at the point of origin (compare Sorenson's \citeyear[138--9]{Sorensen:2009aa} discussion of the location of earthquakes). If we locate the sound event at its source, being the epicenter of audible activity, listening to the birds outside your window, the students outside your door, the cars going down your street, the sounds you hear would, on that understanding, be located at the place where they originate. Sounds being located at their sources, at least on this understanding, is, in this way, \emph{pace} \citet{Pasnau:1999ss}, consistent with The Wave Theory \citep[see][123, for a partial anticipation of this point]{OShaughnessy:2009aa}.

According to The Wave Theory, as developed herein, the propagation of a patterned disturbance, in all directions, through a dense and elastic medium is the progressive instantiation of a wave form, a kind of dynamic in-formation, realized by the motion of the local parts of the medium, ``the oscillation of certain particles.'' Though the sound event may be said to have location, the propagation of the patterned disturbance through a dense and elastic medium is not best modeled on the locomotion of a body, like a sonic missile. As \citet{OCallaghan:2007xy,OCallaghan:2009aa} observes, that is not how auditory experience presents sound as coming from its source.  Since the patterned disturbance at the boundary of the sound event is indirectly moving in every direction, thus determining, under certain idealizations, an ever expanding sphere, the propagation of a patterned disturbance is better modeled on growth rather than locomotion. Sounds are heard to come from their sources in the sense that the direction of the propagation of the patterned disturbance in the growth of the sound event is disclosed in auditory experience. On that model, there are certain natural alternative understandings of the location of a sound event. Locating the sound event in the space encompassed by stable and determinate boundaries is not possible since these are in constant indirect motion. On certain occasions, for certain practical purposes, sounds may be said to be where we hear them. On other occasions, for other purposes, sounds may be said to be located at their epicenter, at or near their sources. And each alternative is consistent with the sound event being the propagation, in every direction, of a patterned disturbance through a dense and imperfectly elastic medium understood as the progressive instantiation of a wave form realized by the motion of the local parts of the medium. 

As a dynamic in-formation, the sound event has a kind of unity irreducible to the motion of the local parts of the in-formed medium. Conceiving of the propagation of sound on the model of the locomotion of a body---a sonic missile---mistakes the unity of the sound event for the unity of a body. Sound events may lack the unity of a body. After all, events and bodies have different modes of being. But sound events nevertheless possess sufficient unity to distinguish them from the in-formed medium that they existentially depend upon. It is a dynamic unity as befitting the double duration and spatial mutation of sound. Though the motion involved in the dynamic unity is at best indirect, being the progressive instantiation of a wave form, it is the force with which it propagates in every direction that entitles us to speak of the growth of the sound event.  While the sound event may be realized by the motion of the local parts of the medium, ``the oscillation of certain particles'', it is the force of its propagation, communicated from one part to the next, that determines the dynamic in-formation. The sound event is realized by the motion of the local parts of the medium without reducing to such motion because of its dynamic unity, the force with which it grows in the dense and imperfectly elastic medium. And it is the direction of this force that is disclosed, more or less clearly, in the emanative phenomenology of auditory experience. What Prichard's mathematician was certain about was the unity of sound. In misconceiving the unity of a sound as the unity of a body, he was misled into thinking that sounds travel like missiles. But in conceiving of the unity of the sound, or at least its principle, as the force of the dynamic in-formation, sounds do not so much as travel as they grow.

% Since the progressive instantiation of the wave form is occurring in every direction from its source, the dynamic in-formation determines, under certain idealizations, the growth of a sphere.

We hear sounds. We also hear sources. What is the relation between the sounds that we hear and their audible sources? Is our awareness of the sources of sound mediated in the way that the neo-Berkelean suggests? Or is Heidegger right in insisting that we hear the sources of sound directly? In the next chapter I will argue for the Heideggerian alternative. In hearing the call of the feral parrot I am explicitly aware of the parrot's call and only implicitly aware of its sound. We hear the sources of sound through, or in, the sounds they generate. And, as we shall see, the principle of sympathy explains how this may be so.

% 

% If the unity of the sound event reduced to the local motions of the medium, then the progressive instantiation of the wave form would not constitute the growth of the sound event that exists over and above the local motions of the medium that it existentially depends upon.


% section phenomenological_objections (end)

% chapter sound (end)