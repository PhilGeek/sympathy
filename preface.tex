%!TEX root = /Users/markelikalderon/Documents/Git/sympathy/perception.tex
\chapter*{Preface} % (fold)
\markboth{\MakeUppercase{Preface}}{}
\addcontentsline{toc}{chapter}{Preface}
\label{cha:preface}

The present essay is an unabashed exercise in historically informed, speculative metaphysics. Its aim is to gain insight into the nature of sensory presentation. Allow me to explain why it should be historically informed and in what sense the metaphysics developed herein is speculative.

One of the fundamental issues dividing contemporary philosophers of perception is whether perception is representational or presentational in character. To claim that perception is presentational in character is to claim that it has a presentational element irreducible to whatever intentional or representational content it may have. Representationalists deny that perception has such an irreducible presentational element, claiming that the content of perception is exhaustively specified by its intentional or representational content. If there is indeed a presentational element to perception, then, according to the representationalist, this is because sensory presentation is either reducible to the exercise of a representational capacity or otherwise essentially involves the exercise of such a capacity. There are two aspects of this debate. On the one hand, there are arguments on one side or the other urging that perception must be conceived in one way or another. One the other hand, there is a more positive constructive aspect, where, taking for granted one's preferred side of the debate, one goes on to develop detailed theoretical accounts of perceptual experience. 

Representationalists have been more active in this latter task. And unsurprisingly so. For suppose one took sensory presentation to be an indispensable aspect of perceptual experience and further held, in a Butlerian spirit, that it was reducible to no other thing. What positive account could one give of sensory presentation, so conceived? Since it is irreducible, no positive account could take the form of a reduction. So no causal or counterfactual conditions on sensory representations, understood independently of sensory presentation, could be jointly necessary and sufficient for the presentation, in sensory experience, of its object. One might specify its relational features. But not much insight into the nature of sensory presentation is thereby gained. The tools of contemporary analytic metaphysics does not leave one much to work with, at least in the present instance. So it can seem that if one maintains that perceptual experience involves an irreducible presentational element all that one can do is press the negative point that sensory presentation, an indispensable element of perceptual experience, is reducible to no other thing.

I believe that experience has an irreducible presentational element. And yet I hoped to learn something positive about the metaphysics of sensory presentation. If there was, in fact, anything further to be learned, I could not limit myself to the tools of contemporary analytic metaphysics. The present metaphysics is historically informed, at least in part, as a result of looking for tools more adequate to the task at hand. There is a real question about how such borrowings should be understood, if they are not simply an invitation to roll back philosophical thinking about perception to some earlier period. Before we are in a position to address that question, let us first address two additional motives to look to historical material in thinking about the nature of sensory presentation.

Putnam has described the metaphysical orthodoxy in philosophy of mind as Cartesianism-cum-Materialism. While it is easy to find dissenters to either the Cartesian or Materialist elements of that orthodoxy, it is equally easy to appreciate the way in which Putnam's description is apt. That it is apt, shows that, despite its technical sophistication and being informed by twenty-first century psychology, contemporary philosophy of mind is still working within a seventeenth century paradigm. As I continued to work on color and color perception, it became increasingly clear that I was defending an anti-modern conception of color and perception. The conception of color defended was anti-modern in that the colors were in no way secondary, but mind-independent qualities that inhere in material bodies. The conception of color perception was anti-modern in that it was not conceived as a conscious alteration of a perceiving subject but rather as the presentation of instances of mind-independent color qualities located at a distance in the particular material bodies in which they inhere.

The anti-modern metaphysics provided an additional motive to look to historical, and in particular, pre-modern sources. Doing so was a means of self-consciously disrupting habits of mind inculcated by the modern paradigm that has reigned for four centuries. 

There is a third additional motive for the turn to historical sources, one flowing from the methodology pursued in the present essay. Given our presupposition that sensory presentation is irreducible, and leaving to one side what form a positive account of sensory presentation could take if it is not in the form of a reduction of some sort, how are we to proceed? How can one gain insight into the nature of the irreducible presentational element of perceptual experience? My thought, not at all original, was to proceed dialectically, by considering puzzles about the nature of sensory presentation. As it happens, there are a number of historically salient such puzzles that are useful for a metaphysician proceeding dialectically to consider. Moreover, many of these puzzles are pre-modern and have been obscured by the prevailing modern paradigm.

It can often happen, in the course of dialectical argument, that the insights of one's predecessors are not only preserved but transformed. It can happen that a respected predecessor was right to hold a certain opinion but only on an understanding as of yet unavailable to them. That is one way, at least, in which the insights of our predecessors may be transformed in the course of dialectical argument. This bears on the question of how such historical borrowing are to be understood. There is no real possibility of rolling back philosophical thinking to the fifth century \textsc{bc}, say. Just as there is no real possibility of being a Samurai in our present historical circumstances, as Williams reminds us. In deploying ancient or scholastic concepts in a contemporary metaphysical inquiry new sense is accrued, and such borrowings become a kind of concept formation. Compare Bergson's retrofitting the concepts of Stoic physics in the development of his philosophical psychology. If we are to take it seriously, it can only be understood as a method of concept formation. Moreover, novel concepts are what are needed if one hopes to effect a Kuhnian revolution against the prevailing modern paradigm.






% chapter preface (end)