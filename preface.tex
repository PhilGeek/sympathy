%!TEX root = /Users/markelikalderon/Documents/Git/sympathy/perception.tex
\chapter*{Preface} % (fold)
\markboth{\MakeUppercase{Preface}}{}
\addcontentsline{toc}{chapter}{Preface}
\label{cha:preface}

The present essay is an unabashed exercise in historically informed, speculative metaphysics. Its aim is to gain insight into the nature of sensory presentation. Allow me to explain why it should be historically informed and in what sense the metaphysics developed herein is speculative.

One of the fundamental issues dividing contemporary philosophers of perception is whether perception is presentational or representational in character \citep[see, for example, the recent collection devoted to this topic][]{Brogaard:2014af}. To claim that perception is presentational in character is to claim that it has a presentational element irreducible to whatever intentional or representational content it may have. So conceived, the object of perception is present in the awareness afforded by the perceptual experience and is thus a constituent of that experience. Representationalists deny that perception has such an irreducible presentational element, claiming, instead, that the object of perception is exhaustively specified by its intentional or representational content. If there is indeed a presentational element to perception, then, according to the representationalist, this is because sensory presentation is either reducible to the exercise of an intentional or representational capacity or otherwise essentially involves the exercise of such a capacity (see, for example, \citealt{Chalmers:2006kx,McDowell:2008fk,Searle:2015fu}). There are two aspects of this debate. On the one hand, there are arguments on one side or the other urging that perception must be conceived in presentational or representational terms. One the other hand, there is a more positive, constructive aspect, where, taking for granted one's preferred conception, one goes on to develop detailed theoretical accounts of perceptual experience. 

Representationalists have been more active in this latter task. And unsurprisingly so. For suppose one took sensory presentation to be an indispensable aspect of perceptual experience and further held, in a Butlerian spirit, that it was reducible to no other thing. What positive account could one give of sensory presentation, so conceived? Since it is irreducible, no positive account could take the form of a reduction. So no causal or counterfactual conditions on sensory representations, understood independently of perception, could be jointly necessary and sufficient for the presentation, in sensory experience, of its object. One might specify the relational features of presentation in sensory experience, but not much insight into the nature of sensory presentation is thereby gained. The tools of contemporary analytic metaphysics would seem not to leave one much to work with, at least in the present instance. So it can seem that if one maintains that perceptual experience involves an irreducible presentational element, all that one can do is press the negative point that sensory presentation, an indispensable element of perceptual experience, is reducible to no other thing.

I believe that perception has an irreducible presentational element. And yet I hoped to learn something positive about the metaphysics of sensory presentation. If there was, in fact, anything further to be learned, I could not limit myself to the tools of contemporary analytic metaphysics. The present metaphysics is historically informed, at least in part, as a result of looking for tools more adequate to the task at hand. There is a real question about how such borrowings should be understood, if they are not simply an invitation to roll back philosophical thinking about perception to some earlier period. Before we are in a position to address that question, let us first address two additional motives to look to historical material in thinking about the nature of sensory presentation.

\citet{Putnam:1993kx,Putnam:1994kx,Putnam:1999eu} has described the present metaphysical orthodoxy in the philosophy of mind as ``Cartesianism \emph{cum} materialism'' (compare Merleau-Ponty's \citeyear{Merleau-Ponty:1967fj} related charge of ``psuedo-Cartesianism''). While it is easy to find dissenters to either the Cartesian or materialist elements of that orthodoxy, it is equally easy to appreciate the way in which Putnam's description is apt. That it is apt, shows that, despite its technical sophistication and being informed by twenty-first century psychology, contemporary philosophy of mind is still working within a seventeenth century paradigm. After an initial collaboration \citep{Hilbert:2000on}, as I continued to work on color and color perception \citep{Kalderon:2006tg,Kalderon:2008fk,Kalderon:2007mr,Kalderon:2010fj,Kalderon:2011fk}, it became increasingly clear that I was defending an anti-modern conception of color and perception. The conception of color defended was anti-modern in that the colors were in no way secondary, but mind-independent qualities that inhere in material bodies. The conception of color perception was anti-modern in that it was not conceived as a conscious alteration of a perceiving subject but rather as the presentation of instances of mind-independent color qualities located at a distance from the perceiver. The anti-modern metaphysics provided an additional motive to look to historical, and in particular, pre-modern sources. Doing so was a means of self-consciously disrupting habits of mind inculcated by the modern paradigm that has reigned for four centuries. 

There is a third additional motive for the turn to historical sources, one flowing from the methodology pursued in the present essay. Given our presupposition that sensory presentation is irreducible, and leaving to one side what form a positive account of sensory presentation could take if it is not, indeed, a reduction of some sort, how are we to proceed? How can one gain insight into the nature of the irreducible presentational element of perceptual experience? My thought, not at all original, was to proceed dialectically, by considering puzzles about the nature of sensory presentation. As it happens, there are a number of historically salient such puzzles that are useful for a metaphysician proceeding dialectically to consider (for a detailed historical discussion of at least one of these see \citealt{Kalderon:2015fr}). Moreover, many of these puzzles are pre-modern though have been obscured by the prevailing modern paradigm.

It can often happen, in the course of dialectical argument, that the insights of one's predecessors are not only preserved but transformed. Thus, it can happen that a respected predecessor was right to hold a certain opinion but only on an understanding as of yet unavailable to them. That is one way, at least, in which the insights of our predecessors may be transformed even as they are preserved in the course of dialectical argument. This bears on the question of how such historical borrowing are to be understood. There is no real possibility of rolling back philosophical thinking to the fifth century \textsc{bc}, say, just as there is no real possibility of living ``the life of a Bronze Age Chief, or a Medieval Samurai,'' in our present historical circumstances, as \citet[140]{Williams:1981rt} reminds us. In deploying ancient or Scholastic concepts in a contemporary metaphysical inquiry new sense is accrued, and such borrowings become a kind of concept formation \citep[587--8]{Moore:2012aa}. New sense is accrued when an ancient or Scholastic concept is applied to novel problems that arise in a theoretical and historical context distinct from the one in which the concept was originally formed. Compare Bergson's \citeyearpar{Bergson:1912pi} retrofitting the concepts of Stoic physics in the development of his philosophical psychology. If we are to take it at all seriously, it can only be understood as a method of concept formation. Moreover, novel concepts are what are needed if one hopes to contribute to, if not indeed effect, a Kuhnian revolution against the prevailing modern paradigm.

That the present metaphysical inquiry proceeds dialectically bears on its speculative character. In proceeding dialectically, in taking puzzles about the nature of sensory presentation as a guide to uncovering its nature, the present essay is aporetic and exploratory. Its conclusions necessarily fall short of apodeictic proof. This, at any rate, should be obvious since the conclusion of dialectical argument hardly constitutes an a priori demonstration, drawing, as it may, upon the testimony of the many and the wise, as well as any empirical evidence as may be relevant. 

Self-proclaimed naturalistic metaphysicians sometimes lampoon their opponents as engaging in a priori reasoning from the armchair. But eschewing reductionism about sensory presentation while pursuing insight into its nature by proceeding dialectically, no a priori demonstration is offered. Nor indeed could there be if the ambition is to contribute to, if not indeed effect, a Kuhnian revolution. Demonstrations are only possible at the stage of normal science. Demonstrations require a stable conceptual framework, about which there is widespread and non-collusive agreement, in which to take place. Part of the present task is to disrupt just such a framework.

The present task is to disrupt such a framework, and not supplant new dogma for old. Like Wittgenstein, I would be content, if possible to stimulate someone to thoughts of their own.

A more specific task provides a fourth motivation for why the present metaphysical inquiry should be historically informed. I have long been puzzled by the primordial and persistent tactile metaphors for sensory awareness, even for non-tactile modes of sensory awareness such as vision and audition. Such imagery persists even among those who would eschew any explanation of perception in terms of, or on analogy with, tactile perception. Thus, in a remarkable passage, O'Shaughnessy, a careful, independent thinker, warns against taking such tactile metaphors too literally but cannot restrain himself from deploying such a metaphor in describing the contrasting conception:
\begin{quote}
	I think there is a tendency to conceive of attentive \emph{contact} [my emphasis], which is to say of perceptual awareness, as a kind of palpable or concrete contact of the mind with its object. And in one sense of these terms, this belief is surely correct. \ldots\ However, there is a tendency---or perhaps an imagery of a kind that may be at work in one's mind---to overinterpret this ``concreteness,'' to think of it as in some way akin to, as a mental analogue of, something drawn from the realm of \emph{things}---a palpable connection of some kind, rather as if the gaze literally reach out and touched its object. \citep[183]{OShaughnessy:2003eu}
\end{quote}
And Mike Martin has observed that ``content'' is a metaphor of assimilation---to have a content is to be, in a way, its container, containment being itself a mode of assimilation, as is grasping. Moreover, Martin also notes the way in which this imagery is in tension with the theoretical role content plays in representationalist theories of perception. For surely what is contained within a perception is its object, but the content of that perception is not the object of perception. Rather, the object of that perception is what is represented by its content \citep{Martin:1998nx}.

I wanted to understand why contemporary philosophers apply tactile metaphors for sensory awareness unselfconsciously, indeed, unconsciously---even when such imagery ultimately fails to cohere with their espoused doctrine. One explanation, to be pursued throughout this essay, is that without reducing perception generally to sensation by contact, there is, nonetheless, a way in which tactile metaphors for sensory presentation are apt. Moreover, if tactile metaphors for perception generally are apt in the way that I shall suggest they are, then the resulting conception of perception is anti-modern, or so shall I argue. But if it is, then the unconscious tendency to apply tactile metaphors for sensory awareness, even if it is in tension with one's stated doctrine, is subject to a psychoanalytic explanation, hence rendering the present essay a psychoanalytic narrative. It is the return of the repressed. Or more specifically, the return of what has been repressed by the modern paradigm. Our unconscious use of tactile metaphors for sensory awareness is the vestigial remnant of a vivid sense of the Manifest Image of Nature and our perceptual relation to it not utterly extinguished by four centuries of modernity.

Grasping is at the center of a semantic field of tactile metaphors for sensory awareness loosely organized as modes of assimilation (chapter~\ref{sec:grasping_and_the_dawn_of_understanding}). I attempt to understand what, if anything, makes grasping an apt metaphor for sensory awareness more generally by undertaking a phenomenological investigation into grasping or enclosure understood as a mode of haptic perception. The idea is that if we better appreciate how grasping presents itself from within haptic experience, we will be in a better position to understand what, if anything, makes grasping an apt metaphor for perception generally. Moreover, in undertaking this phenomenological investigation we shall freely draw upon empirical and historical sources. Empirical psychology has a lot to teach us about the phenomenology of haptic experience. But so does the testimony of our respected predecessors and the puzzles that arise both within and without the \emph{endoxa}.

Moreover, there is reason why a phenomenological investigation into haptic experience whose ultimate aim is to uncover the aptness of tactile metaphors for perception generally should take the form of a conceptual genealogy. In looking at earlier occurrences of such metaphors, when they were more strongly etched in light and shadow, one can get a better sense of what made them live for these earlier thinkers and, by extension, a better sense of the power they continue to exercise over us. At any rate, it is almost impossible to get anywhere merely by examining the unselfconscious metaphors deployed by contemporary philosophers---they are lifeless in their hands. Much better to examine earlier occurrences of these metaphors, when they were more strongly and vividly felt, to get a sense of their power and persistent aptness.

Thinking our way to the future by thinking our way through the past may strike some as hopelessly anachronistic. In my defense I only say that, here, I am following \citet[xvii]{Ricoeur:2004ax}, in exercising ``the right of every reader, before whom all the books are open simultaneously.''

The results of the present inquiry may strike analytically inclined philosophers to be more in line with continental metaphysics. And while the present essay is self-consciously a departure from the prevailing orthodoxy of analytic metaphysics, it remains true to, and is a staunch defense of, what has been a central tenet of analytic metaphysics from its inception, namely, realism. And while it is true that recent continental thinkers have recovered for themselves a form of realism, the present perceptual realism is more in line with \citet{Cook-Wilson:1926sf} than \citet{Meillassoux:2008ve}. Moreover, continental philosophers will quickly recognize that the present essay defends, in Heideggerian terminology, a metaphysics of presence. The present conception of sensory presentation is thus fundamentally at odds with conceptions of perception developed within the phenomenological tradition. To be honest, I care little for such categories. And in what follows I have drawn freely from a variety of sources.








% chapter preface (end)