%!TEX root = /Users/markelikalderon/Documents/Git/sympathy/perception.tex
\chapter{Vision} % (fold)
\label{cha:vision}

\section{The Biranian Principle} % (fold)
\label{sec:the_birnaian_principle}

So far we have discussed grasping and listening. We turn now to looking. Our guiding idea, echoing Maine de Biran, is that in order to see well, one must look. Our task is to describe a conception of looking that could plausibly make this principle true. 

Such a conception must satisfy two conditions. 

First, looking must be something the perceiver does. Only in that way is the analogy with grasping, enshrined in the Protagorean model, sustained. However, like the case of audition, this may be understood, on the Peripatetic model, in terms of the capacity to act. Looking, like listening, while not a passive power, may be less than fully active. In the traditional, post-Aristotelian vocabulary it may be a first actuality if a second potentiality. Looking may be a perceptual stance sustained, at a minimum, by the potential to act in visually relevant ways, on some appropriate understanding of that potentiality. While looking and listening may fall short of the exemplar, grasping, since haptic perception requires the second actuality of the hand's activity, still, they are not something done to the perceiver but something that the perceiver does, even if this doing, in certain circumstances, consists in nothing further than a preparedness to act in perceptually relevant ways. At a minimum, then, looking merely requires vigilance (perhaps fortuitously, ``vigilance'' derives from the Latin \emph{vigilare} meaning to watch). But the perceiver's perceptual vigilance over the distal environment, their being prepared to act in visually relevant ways to bring aspects of the distal environment into view, remains a stance actively sustained by the perceiver.

Second, looking is an activity of the perceiver whose end is to bring distal aspects of the natural environment into view. Looking makes aspects of the distal environment perceptually accessible. For the perceiver to act in visually relevant ways is for them to alter their visual perspective on the natural environment so as to increase the acuity with which distal aspects of that environment are seen. A conception of looking that stands a chance of making true the Biranian principle---that in order to see well, one must look---must at once be something that the perceiver does and makes the distal environment perceptually accessible. 

A conception of looking answering to the truth of the Biranian principle---that in order to see well, one must look---will most likely exceed the conception of looking enshrined in ordinary usage---though, perhaps, in the manner of a conservative extension. This might count against describing such a conception as an instance of ``looking.'' However, other alternatives fare less well. ``Gaze'' is, by now, too ethically fraught (see \citealt{Jay:1994aa}). Olivi's \emph{aspectus}, while a historically important antecedent, is too technical sounding and is bound up with Olivi's Augustinian dualism (thus, for example, Olivi distinguishes the physical \emph{aspectus} of the sense organ, the eye pointed in a certain direction, say, from the spiritual \emph{aspectus} of the soul; on Olivi on perception see \citealt{Pasnau:1997aa,Silva:2010zh,Toivanen:2013ul}). In the absence of an adequate alternative, we shall persist with talk of looking, mindful of the ways that the demands of making true the Biranian principle might exceed the conception of looking enshrined in ordinary usage.

Not only shall I defend the Biranian principle, but I shall offer an explanation for it in terms of the operation of sympathy. Looking makes aspects of the distal environment perceptually accessibly by making possible their sympathetic presentation in visual experience.

% section the_birnaian_principle (end)

\section{The Persistence of Extramission} % (fold)
\label{sec:the_persistence_of_extramission}

\citet[48]{Piaget:1929dp} observes the tendency for children to understand vision in terms of an active outward influence of the eyes. This tendency was manifest in reports of looks mixing and in ``a confusion between vision and light''. Concerning the latter Piaget reports:
\begin{quotation}

	Pat (10) stated that a box makes a shadow ``\emph{because the clouds} (Pat believes it to be the clouds which give light when there is no sun) \emph{can't pass through it}'' (\emph{i.e.} because the light cannot pass through the box).
	
	But immediately after Pat said of a portfolio that it made a shadow ``\emph{because the clouds can't see that side}.---Are to see and to give light the same thing?---\emph{Yes}.---Tell me the things which give light?---\emph{The sun, the moon, the stars, the clouds and God}.---Can you give light?---\emph{No} \ldots\ \emph{Yes}.---How?---\emph{With the eyes}.---Why?---\emph{Because if you hadn't eyes you wouldn't see properly}.''
	
	Duc (6 1/2) also stated that the light cannot see through a hand, alike confusing ``seeing'' with ``giving light.''
	
	Sci (6) said that dreams come ``\emph{with the light}.''---``How?---\emph{You are in the street. The lights} (street-lamps) \emph{can see there} \ldots\ \emph{they see on the ground}.'' ``Tell me some things that give light.---\emph{Lights, candles, matches, thunder, fire, cigarettes}.---Do eyes give light or not?---\emph{Yes, they give light}.---Do they give light at night?---\emph{No}?---Why not?---\emph{Because they are shut}.---When they are open do they give light?---\emph{Yes}.---Do they give light like lamps?---\emph{Yes, a little bit}.''
\end{quotation}

And \citet[48--49]{Piaget:1929dp} goes on to compare these reports with Empedocles' lantern analogy:
\begin{verse}
	As when someone planning a journey prepared a lamp,\index{lantern}\\
	the gleam of blazing fire\index{fire} through the wintry night\index{night},\\
	and fastened linen\index{linen} screens\index{screen} against all kinds of breezes,\index{wind}\\
	which scatter the wind of the blowing breezes\\
	But the light\index{light} leapt outwards, as much of it as was finer,\\
	and shone with its tireless beams across the threshold;\\
	in this way [Aphrodite]\index{Aphrodite} gave birth to the rounded pupil\index{eye!pupil},\\
	primeval fire crowded in the membranes and in the fine linens\index{linen}.\\
	And they covered over the depths of the circumfluent water\index{water!external}\\
	and sent forth fire, as much of it as was finer.\\
	(Empedocles, \textsc{dk} 31\textsc{b}84; \citealt[103 259]{Inwood:2001ve})
\end{verse}
Just as there is fire in the interior of a screened lamp, there is a primeval fire in the interior of the eye, or perhaps the pupil. And just as the screen of linen or shaved horn surrounds the fire in the lamp's interior, there is a membrane that surrounds the fire in the eye's interior. Moreover, the membrane plays a similar role to the screen. Just as the screen protects the interior fire from the wind which would extinguish it, the primeval fire is protected from the depth of the surrounding water by the membrane of the eye. Finally, just as light passes through the screen, the primeval fire can pass through passages in the membrane of the eye. (On the lantern analogy, see \citealt[240--243]{Wright:1981zr}; on Empedocles's theory of vision see \citealt{Sedley:1992uq}, \citealt{Ierodiakonou:2005fk}, \citealt[chapter 1]{Kalderon:2015fr}.) Thus according to Empedocles' lantern analogy, vision involves an active outward influence of the eyes that is akin to light---just as Piaget's children report.

\citet[138]{Winer:1996as}, prompted by Piaget's observations, were ``sur\-pri\-sed---indeed shocked'' by the degree and resilience of belief in extramissive perception. Not only do children hold extramission beliefs but so do adults, though such beliefs tend to decline during adulthood. Moreover, these extramission beliefs ``are highly resistant to experimental intervention designed to alter them'' \citep[138]{Winer:1996as}. \citet{Winer:1996as} hypothesize that both the tendency for extramission beliefs to persist into adulthood and their resistance to experimental intervention is partly explained by a phenomenological truth enshrined in extramission models:
\begin{quote}
	We assume that core aspects of the phenomenology of vision underlie extramission interpretations. Consider one phenomenologically salient aspect of vision, namely, its orientational or outer-directed quality. When people see, they are generally oriented toward an external visual referent, that is, they direct their eyes and attention to an object in order to see it. In fact, this quality of vision is reflected in language. People talk about ``looking at'' things, and English has expressions such as ``looking out of a window'' and ``looking out of binoculars.'' Even notions such as ``piercing glances'' and ``cutting looks'' suggest and outer directionality \ldots\ \citep[140]{Winer:1996as}
\end{quote}

% section the_persistence_of_extramission (end)

\section{The Truth in Extramission} % (fold)
\label{sec:the_truth_in_extramission}


% Plato's \emph{Timaeus} contains a close echo of the lantern analogy:
% \begin{quote}
% 	And of the organs they first contrived the eyes\index{eye} to give light\index{light}, and the principle according to which they were inserted was as follows. So much of fire\index{fire} as would not burn, but gave a gentle light, they formed into a substance akin to the light of everyday life, and the pure fire which is within us and related thereto they made flow through the eyes in a stream smooth and dense, compressing the whole eye and especially the center part, so that it kept out everything of a coarser nature and allowed to pass only this pure element. (Plato, \emph{Timaeus} 45\( ^{b-c} \), Jowett in \citealt{Hamilton:1989fk})
% \end{quote}

Merleau-Ponty provides a description of the active, outer-directed phenomenology of vision that would make talk of extramission apt:
\begin{quote}
	If I adhere to what immediate consciousness tells me, the desk which I see in front of me and on which I am writing, the room in which I am and whose walls enclose me beyond the sensible field, the garden, the street, the city and, finally, the whole of my spatial horizon do not appear to me to be causes of the perception which I have of them, causes which would impress their mark on me and produce an image of themselves by a transitive action. It seems to me rather that my perception is like a beam of light which reveals the objects there where they are and manifests their presence, latent until then. Whether I myself perceive or consider another subject perceiving, it seems to me that the gaze ``is posed'' on objects and reaches them from a distance---as is well expressed by the use of the Latin \emph{lumina} for designating the gaze. \citep[185]{Merleau-Ponty:1967fj}
\end{quote}
Merleau-Ponty is not endorsing the extramission theory as a causal model of perception. He is not denying that the object of perception is the ultimate efficient cause of that perception. Rather, in seeing the desk before him, Merleau-Ponty claims only that his experience does not present itself as the exercise of a passive power, a sensory impression caused in him by the mediate causal action of the distal object. There may be an active element to outwardly attending, in vision, to distal aspects of the natural environment, and this may be phenomenologically vivid, but that is consistent with the object of visual perception being among its causal antecedents. ``My present experience of this desk is not complete, \ldots\ it shows me only some of its aspects'' \citep[186]{Merleau-Ponty:1967fj}. A visual experience may be undergone, but seeing is not something done to the perceiver, but something the perceiver does.

Nor is Merleau-Ponty claiming that it appears from within that seeing involves the emission of a fiery effluence akin to light. Rather Merleau-Ponty is pressing an analogy. He is describing what visual experience, from within, is like. Mereleau-Ponty is describing that aspect of our visual phenomenology that \citet{Winer:1996as} claim to underlie extramission beliefs. More explicitly, the awareness afforded by visual experience is like a beam of light that manifests the latent presence of its object. Vision, like illumination, has direction. Light is emitted outward from its source upon the scene that it illuminates.  Vision too is outer directed. In seeing, the perceiver looks out upon the scene before them. Moreover, just as illumination manifests the latent visibility of an object, seeing an illuminated object manifests its latent presence to the perceiver revealing it to be where it is. The explicit awareness of the natural environment afforded by visual experience is akin to light not only in its directionality and its power to manifest latent presence, but in the manner in which it discloses distal aspects of that environment. Just as beam of illumination may ``pose'' on an object that it illuminates and that it reaches from a distance, the perceiver's gaze may ``pose'' on the object that it presents and that it reaches from a distance. The illumination alights upon the object it illuminates at a distance from its source, the perceiver's gave alights upon the object of perception at a distance from the perceiver. The imagery here not only emphasizes that vision is a kind of perception at a distance but invokes an active outward extension, as in Kilwardby's wax actively pushing against the seal (chapter ~\ref{sec:active_wax}).

Accepting the aptness of the analogy is not tantamount to accepting the extramission theory. Consider a similar analogy of Olivi's:
\begin{quote}
	an object, to the extent that the gaze (\emph{aspectus}) and the act of a power are terminated at it, co-operates in their specific production [\ldots] Namely, the cognitive act---and the gaze---is fixed (\emph{figitur}) to the object and it absorbs the object intentionally to itself. This is why a cognitive act is called the apprehension of, and the apprehensive extension to, the object. In this extension and absorption the act becomes intimately conformed and assimilated into the object. The object presents itself or appears as being present to the cognitive gaze, and the object is a kind of representation of itself by an act which is assimilated to it. As an actual illumination of a spherical or quadrangular vase becomes spherical or quadrangular only because the light source generates the illumination in conformity with the figure of the object which receives and confines it; so also, because a cognitive force generates a cognitive act with a certain formative absorption of the act towards the object, and with a certain signet-like and inward (\emph{sigillari et viscerali}) extension of the object, therefore---because it is generated thus---the act becomes a similitude and signet-like expression of the object. (Peter John Olivi, \emph{Questiones in secundum librum Sententiarum} q. 72 35--36; \citealt[146--147]{Toivanen:2013ul})
\end{quote}
Despite his play with neo-Platonic imagery, no doubt an Augustinian heritage, Olivi is not endorsing an extramission theory of perception. Perceptual apprehension may be a form of apprehensive extension to its object, but this apprehensive extension is not corporeal. Though likened to illumination directed upon the object it illuminates, the cognitive act, the apprehensive extension by which that act assimilates to its object, does not consist in, or otherwise involve, the activity of a fiery substance, no matter how rarified. Likening the seeing of an object to light directed upon an object that it illuminates, by itself, carries with it no commitment to the metaphysics of extramission. Rather, Olivi, like Merleau-Ponty after him, is emphasizing the active, outer-directed nature of vision.

Moreover, like Merleau-Ponty, Olivi is presenting a conception of perception that contrasts with a mere passive reception of sensible form. Olivi, working in the same broadly Augustinian metaphysical framework as Kilwardby, is less concessive than Kilwardby to Peripatetic accounts of perception. 

Recall, according to the Peripatetic account, at least as understood by the late Scholastics, the perceived object acts upon the transparent medium such that its sensible form, its species exists, in some sense, in it, and that the medium, in turn, affects the sense organ such that the species comes to, in some sense, exist in it as well (\emph{De Spiritu Fantastico} 69, 97). So understood, the eye's reception of a color species, while not a literal coloration, is the exercise of a passive power, like the power to be heated. The distal object mediately acting upon the perceiver's sense organ posited by the Peripatetic account was understood, by Kilwardby, as necessary if insufficient for perception. In order for perception to occur, the perceptive soul must assimilate the species, but this requires the soul's activity. 

According to Olivi, however, this is not even a necessary condition for perception. The object of perception is not an efficient cause of that perception, no matter how mediate. The powers of the soul, even perceptual powers, are not the passive recipients of external stimuli but are active. Like Merleau-Ponty, Olivi thinks that this is phenomenologically evident \citep[143]{Toivanen:2013ul}. While Olivi does not deny that perception presupposes the presence of its object, he does deny that it is, or even among, the efficient causes of perception. In Olivi's technical vocabulary, the object of perception is a terminative cause. It is controversial how to understand Olivi's terminative causes. Are terminative causes a species of final cause, as \citet[192--195]{Kent:1984zm} and \citet{Pasnau:1999kn} maintain? Or are they kind of cause not classified by the traditional Peripatetic four causes, as \citet[chapter 6]{Toivanen:2013ul} maintains? While it is difficult to form a clear, positive conception of terminative causes, the negative contrast with efficient causes is clear. The actualization of a perceptual power may require the presence of its object but that object acting upon the power is not required for its actualization. The efficient cause of the perceptual act is the power and not the object of perception. The presence of that object merely cooperates by being the \emph{terminus} of the perceptual act, that which it is directed upon, like light directed upon a spherical vase.

So Olivi maintains that the active, outer-directed phenomenology of vision is inconsistent with the object seen being the efficient cause of that perception. Merleau-Ponty, by contrast, merely claims that the object of perception acting upon the perceiver, however mediately, is not manifest in our experience, not that it is inconsistent with it. Perhaps this more cautious attitude is, in the end, warranted. 

To bring this out, first consider an element of the Olivi passage that goes beyond what Merleau-Ponty explicitly claims. Olivi, like Merleau-Ponty, uses the neo-Platonic imagery of illumination to emphasize the active outward extension involved in the visual apprehension of the distal environment, and where this active outward extension is no kind of extramission. Olivi goes further than Merleau-Ponty, however, in coupling the active outward extension of the illumination with being shaped by its \emph{terminus}, the circular or quadrangular vase, say. In illuminating a circular vase, the area illuminated is itself circular. The shape of the area illuminated is constituted by the shape of the object illuminated. The illumination is ``in conformity with the figure of the object which receives it and confines it.'' This is meant to be an analogy for how the perceptual act formally assimilates to its object. Extension and absorption are linked. Like Kilwardby before him, Olivi thinks that the perceptual act only assimilates to its object thanks to the activity of the perceptual soul. Indeed, the passage ends with Olivi echoing Kilwardby's figure of the active wax pressing against the seal. (It is unclear whether Olivi read Kilwardby. Perhaps similar paths were laid out for them by their shared Augustinian heritage. For a comparison of Kilwardby and Olivi see \citealt{Silva:2014cl}.)

However, the neo-Platonic analogy does not support Olivi's extreme position. The object of illumination, the illuminated circular vase, say, receives and confines that illumination. In receiving and confining the illumination the illuminated area takes on the shape that it does. In receiving and confining the illumination the circular vase resists that illumination. It obstructs that illumination and so casts a shadow. It is hard to understand how the spherical vase may confine, resist, and obstruct the activity of the illuminant without being a cause, or, at least, a countervailing force. Of course it is the source of the illuminant that generates the illumination, but the illuminated area takes on the shape that it does because the illuminated object resists the activity of the illumination insofar as it can. Kilwardby's account of how the activities of the soul are limited by the passivities of matter of the object used by the soul was meant to address this kind of difficulty. It is unclear whether Olivi has the resources to give a fully adequate response.

Even if Olivi is wrong to deny that the object of perception plays a causal role in its perception, he may be right in claiming that extension and absorption are linked. The grasping hand only conforms to rigid, solid body by grasping it. The grasping hand extends its grip until it can no more and so conforms to the body's contours. It is only thanks to the activity of the hand that the perceiver's haptic experience formally assimilates to the tangible qualities of the object grasped. In this way is the hand the active wax of haptic perception. If extension and absorption are linked, if the wax only takes on the form of the seal by actively pressing against it, then the active extensive element in Merleau-Ponty's description is the basis for the consequent absorption. The light is posed on the circular vase and is fixed there, and so the illuminated area is shaped by that vase. Merleau-Ponty's gaze is posed on his desk and is fixed there, and so his visual experience is shaped by that desk.

% section the_truth_in_extramission (end)

\section{Looking} % (fold)
\label{sec:looking}

We have been discussing the active outward extensive character of visual phenomenology that underlies persistent belief in extramission in some children and adults and is plausibly the font of classical extramission theories. We have done so in aid of honing in on a conception of looking that stands a chance of make true the Biranian principle, in order to see well, one must look. Such a conception must involve the active outward-directed extension of visual awareness where this involves the emission of nothing, no matter how rarified and akin to light. 

I turn, and look, and see an ancient chestnut tree. It is one of the ancient chestnut trees planted in Greenwich Park in the 1660s by Charles \textsc{ii}. An organism of impressive size and age presents itself. The majority of its burrs remain on the tree and are brighter green than the surrounding foliage. It is early evening, and the light is long and golden. The light both articulates the fine texture of the bark and sets off the overall flow of the trunk in dramatic relief. Despite its manifest strength and solidity, the twisted trunk appears to be flowing in a wave like form. I come to realize that I am witnessing an organic process, the growth of the trunk, occurring so slowly as to appear, from within my limited temporal perspective, to be frozen, static. The difference in the scale of our lives is striking. For a moment, it induces in me a kind a kind of temporal vertigo.  Just as a radical difference in spatial scale can be vertiginous---think of how small one can feel when viewing the Milky Way---a radical difference in temporal scale can be vertiginous as well. The scale of its life and the strength manifest in centuries of growth make the sweet chestnut tree a fit object of awe. I find myself musing that in a different cultural context, perhaps one more prone to animism, the tree might reasonably be reckoned a god. 

In looking at the ancient chestnut tree from across the park, I peer through the intervening space. My gaze perceptually penetrates the intervening air until it encounters the surface of the ancient tree. The tree's surface is the site of visual resistance. Perceptually impenetrable, it determines a visual boundary through which nothing further may appear. The tree is opaque to a significant degree. The air, by contrast, being transparent, is perceptually penetrable. One can see objects through it and in it. (Compare the phenomenological interpretation I give of the bounded and unbounded in \emph{De Sensu}, \citealt[chapter 3.3]{Kalderon:2015fr}).

\citet{Broad:1952kx} describes vision as prehensive and saltatory. It is prehensive insofar as vision involves the presentation of its object in the awareness afforded by visual experience. It is saltatory insofar as vision seems to leap the spatial gap between the perceiver and the object. There are two separable elements to Broad's conception of saltitoriness. The first is simply the frank admission that vision is a kind of perception at a distance, that the objects of visual awareness are located at a distance from the perceiver. That much is unexceptional. The second is a phenomenological claim, that vision seems to leap the spatial gap between the perceiver and the object of perception. For vision to leap the spatial gap would be for the objects of visual awareness to be confined to a remote location. However, I am visually aware not only of the coloring of the ancient chestnut tree but of the intervening space as well. We not only see the colors of distant particulars, but we do so by seeing through intervening illuminated media. Two years after the appearance of “Some elementary reflections on sense-perception”, \citet[518]{Jonas:1954aa} will deny that vision is saltatory in Broad’s sense, and it is the second element of Broad's conception that he takes exception to: “in sight the object faces me across the intervening distance, which in all its potential ‘steps’ is included in the perception”. Broad is right to emphasize the distal character of the objects of vision, but his description of vision as saltatory is inapt since it fails to heed the perceptual penetrability of the intervening medium. Vision would leap the gap between the perceiver and the distal color if the object of visual awareness were confined to the remote spatial region where that color is instantiated. However, vision is not so confined and so does not leap the gap between the perceiver and distal color. Rather, by means of it, the perceiver may peer through the intervening medium, in all its potential steps, and encounter objects facing them across the intervening distance, if the medium is transparent at least to some degree. In the course of an otherwise astute and insightful description of sensory experience, Broad is misled, at this point, by overlooking the active, outer-directed extensive activity involved in visual apprehension. Broad, in effect, overlooks the truth in extramission.

Looking, like the case of audition, may be understood, on the Peripatetic model, in terms of the capacity to act. Looking, like listening, while not a passive power, may be less than fully active. In the traditional, post-Aristotelian vocabulary, it may be a first actuality if a second potentiality. Looking may be a perceptual stance sustained, at a minimum, by the potential to act in visually relevant ways, to alter my visual perspective on the natural environment to better bring into view distal aspects of that environment, but only on a particular understanding of that potentiality. While looking and listening may fall short of the exemplar, grasping, haptic perception requires the second actuality of the hand's activity, still, they are not something done to the perceiver but something the perceiver does. However, what the perceiver does in looking may, in certain circumstances, consist in nothing further than a preparedness to act in perceptually relevant ways. Perhaps to get better sense of the trunk's flowing pattern, I must follow that pattern along with my gaze, at least to a certain degree. Perhaps, I need to move closer, or perhaps further away. Looking at the ancient chestnut tree may involve, at a minimum, a preparedness to act in such visually relevant ways, but such preparedness requires vigilance. In looking at the ancient chestnut tree, at a minimum, I maintain vigilance over the tree and its visually manifest aspects. Being thus vigilant, being prepared to act in visually relevant ways, remains a stance that I must actively sustain.

% section looking (end)

\section{Sympathy and Visual Presentation} % (fold)
\label{sec:sympathy_and_visual_presentation}



% section sympathy_and_visual_presentation (end)



% chapter vision (end)
