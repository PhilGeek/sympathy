%!TEX root = /Users/markelikalderon/Documents/Git/sympathy/perception.tex
\chapter{Vision} % (fold)
\label{cha:vision}

\section{Looking} % (fold)
\label{sec:looking}

So far we have discussed grasping and listening, we turn now to looking. Our guiding idea, echoing Maine de Biran, is that in order to see well, one must look. Our task is to describe a conception of looking that could plausibly make this principle true. Such a conception must satisfy two conditions. First, looking must be something the perceiver does. Only in that way is the analogy with grasping, enshrined in the Protagorean model, sustained. However, like the case of audition, this may be understood, on the Peripatetic model, in terms of the capacity to act. Vision may be a perceptual stance sustained, at a minimum, by the potential to look, on some appropriate understanding of the potentiality. Second, looking is an activity of the perceiver whose end is to bring distal aspects of the natural environment into view. Looking makes aspects of the distal environment perceptually accessible. A conception of looking that stands a chance of making true the Biranian principle---that in order to see well, one must look---must at once be something that the perceiver does and makes the distal environment perceptually accessible. Not only do I defend the Biranian principle, but I offer an explanation for it in terms of the operation of sympathy. Looking makes aspects of the distal environment perceptually accessibly by making possible their sympathetic presentation in visual experience.




% section looking (end)

\section{The Truth in Extramission} % (fold)
\label{sec:the_truth_in_extramission}

\begin{quote}
	If I adhere to what immediate consciousness tells me, the desk which I see in front of me and on which I am writing, the room in which I am and whose walls enclose me beyond the sensible field, the garden, the street, the city and, finally, the whole of my spatial horizon do not appear to me to be causes of the perception which I have of them, causes which would impress their mark on me and produce an image of themselves by a transitive action. It seems to me rather that my perception is like a beam of light which reveals the objects there where they are and manifests their presence, latent until then. Whether I myself perceive or consider another subject perceiving, it seems to me that the gaze ``is posed'' on objects and reaches them from a distance---as is well expressed by the use of the Latin \emph{lumina} for designating the gaze. \citep[185]{Merleau-Ponty:1967fj}
\end{quote}


\begin{quote}
	We assume that core aspects of the phenomenology of vision underlie extramission interpretations. Consider one phenomenologically salient aspect of vision, namely, its orientational or outer-directed quality. When people see, they are generally oriented toward an external visual referent, that is, they direct their eyes and attention to an object in order to see it. In fact, this quality of vision is reflected in language. People talk about ``looking at'' things, and English has expressions such as ``looking out of a window'' and ``looking out of binoculars.'' Even notions such as ``piercing glances'' and ``cutting looks'' suggest and outer directionality \ldots\ \citep[140]{Winer:1996as}
\end{quote}

% section the_truth_in_extramission (end)

% chapter vision (end)