%!TEX root = /Users/markelikalderon/Documents/Git/sympathy/perception.tex
\chapter{Vision} % (fold)
\label{cha:vision}

\section{Looking} % (fold)
\label{sec:looking}

So far we have discussed grasping and listening. We turn now to looking. Our guiding idea, echoing Maine de Biran, is that in order to see well, one must look. Our task is to describe a conception of looking that could plausibly make this principle true. 

Such a conception must satisfy two conditions. 

First, looking must be something the perceiver does. Only in that way is the analogy with grasping, enshrined in the Protagorean model, sustained. However, like the case of audition, this may be understood, on the Peripatetic model, in terms of the capacity to act. Looking, like listening, while not a passive power, may be less than fully active. In the traditional, post-Aristotelian vocabulary it may be a first actuality if a second potentiality. Looking may be a perceptual stance sustained, at a minimum, by the potential to act in visually relevant ways, on some appropriate understanding of that potentiality. While looking and listening may fall short of the exemplar, grasping, since haptic perception requires the second actuality of the hand's activity, still, they are not something done to the perceiver but something that the perceiver does, even if this doing, in certain circumstances, consists in nothing further than a preparedness to act in perceptually relevant ways. At a minimum, then, looking merely requires vigilance (perhaps fortuitously, ``vigilance'' derives from the Latin \emph{vigilare} meaning to watch). But the perceiver's perceptual vigilance over the distal environment, their being prepared to act in visually relevant ways to bring aspects of the distal environment into view, remains a stance actively sustained by the perceiver.

Second, looking is an activity of the perceiver whose end is to bring distal aspects of the natural environment into view. Looking makes aspects of the distal environment perceptually accessible. For the perceiver to act in visually relevant ways is for them to alter their visual perspective on the natural environment so as to increase the acuity with which distal aspects of that environment are seen. A conception of looking that stands a chance of making true the Biranian principle---that in order to see well, one must look---must at once be something that the perceiver does and makes the distal environment perceptually accessible. 

A conception of looking answering to the truth of the Biranian principle---that in order to see well, one must look---will most likely exceed the conception of looking enshrined in ordinary usage---though, perhaps, in the manner of a conservative extension. This might count against describing such a conception as an instance of ``looking.'' However, other alternatives fare less well. ``Gaze'' is, by now, too ethically fraught (see \citealt{Jay:1994aa}). Olivi's \emph{aspectus}, while a historically important antecedent, is too technical sounding and is bound up with Olivi's Augustinian dualism (thus, for example, Olivi distinguishes the physical \emph{aspectus} of the sense organ, the eye pointed in a certain direction, say, from the spiritual \emph{aspectus} of the soul; on Olivi on perception see \citealt{Pasnau:1997aa,Silva:2010zh,Toivanen:2013ul}). In the absence of an adequate alternative, we shall persist with talk of looking, mindful of the ways that the demands of making true the Biranian principle might exceed the conception of looking enshrined in ordinary usage.

Not only shall I defend the Biranian principle, but I shall offer an explanation for it in terms of the operation of sympathy. Looking makes aspects of the distal environment perceptually accessibly by making possible their sympathetic presentation in visual experience.

% section looking (end)

\section{The Truth in Extramission} % (fold)
\label{sec:the_truth_in_extramission}

\citet[48]{Piaget:1929dp} observes the tendency for children to understand vision in terms of an active outward influence of the eyes. This tendency was manifest in reports of looks mixing and in ``a confusion between vision and light''. Concerning the latter Piaget reports:
\begin{quotation}

	Pat (10) stated that a box makes a shadow ``\emph{because the clouds} (Pat believes it to be the clouds which give light when there is no sun) \emph{can't pass through it}'' (\emph{i.e.} because the light cannot pass through the box).
	
	But immediately after Pat said of a portfolio that it made a shadow ``\emph{because the clouds can't see that side}.---Are to see and to give light the same thing?---\emph{Yes}.---Tell me the things which give light?---\emph{The sun, the moon, the stars, the clouds and God}.---Can you give light?---\emph{No} \ldots\ \emph{Yes}.---How?---\emph{With the eyes}.---Why?---\emph{Because if you hadn't eyes you wouldn't see properly}.''
	
	Duc (6 1/2) also stated that the light cannot see through a hand, alike confusing ``seeing'' with ``giving light.''
	
	Sci (6) said that dreams come ``\emph{with the light}.''---``How?---\emph{You are in the street. The lights} (street-lamps) \emph{can see there} \ldots\ \emph{they see on the ground}.'' ``Tell me some things that give light.---\emph{Lights, candles, matches, thunder, fire, cigarettes}.---Do eyes give light or not?---\emph{Yes, they give light}.---Do they give light at night?---\emph{No}?---Why not?---\emph{Because they are shut}.---When they are open do they give light?---\emph{Yes}.---Do they give light like lamps?---\emph{Yes, a little bit}.''
\end{quotation}

And \citet[48--49]{Piaget:1929dp} goes on to compare these reports with Empedocles' lantern analogy:
\begin{verse}
	As when someone planning a journey prepared a lamp,\index{lantern}\\
	the gleam of blazing fire\index{fire} through the wintry night\index{night},\\
	and fastened linen\index{linen} screens\index{screen} against all kinds of breezes,\index{wind}\\
	which scatter the wind of the blowing breezes\\
	But the light\index{light} leapt outwards, as much of it as was finer,\\
	and shone with its tireless beams across the threshold;\\
	in this way [Aphrodite]\index{Aphrodite} gave birth to the rounded pupil\index{eye!pupil},\\
	primeval fire crowded in the membranes and in the fine linens\index{linen}.\\
	And they covered over the depths of the circumfluent water\index{water!external}\\
	and sent forth fire, as much of it as was finer.\\
	(Empedocles, \textsc{dk} 31\textsc{b}84; \citealt[103 259]{Inwood:2001ve})
\end{verse}
Just as there is fire in the interior of a screened lamp, there is a primeval fire in the interior of the eye, or perhaps the pupil. And just as the screen surrounds the fire in the lamp's interior, there is a membrane that surrounds the fire in the eye's interior. Moreover, the membrane plays a similar role to the screen. Just as the screen protects the interior fire from the wind which would extinguish it, the primeval fire is protected from the depth of the surrounding water by the membrane of the eye. Finally, just as light passes through the screen, the primeval fire can pass through passages in the membrane of the eye. (On the lantern analogy, see \citealt[240--243]{Wright:1981zr}; on Empedocles's theory of vision see \citealt{Sedley:1992uq}, \citealt{Ierodiakonou:2005fk}, \citealt[chapter 1]{Kalderon:2015fr}.) Thus according to Empedocles' lantern analogy, vision involves an active outward influence of the eyes that is akin to light---just as Piaget's children report.

\citet[138]{Winer:1996as}, prompted by Piaget's observations, were ``sur\-pri\-sed---indeed shocked'' by the degree and resilience of belief in extramissive perception. Not only do children hold extramission beliefs but so do adults, though such beliefs tend to decline during adulthood. Moreover, these extramission beliefs ``are highly resistant to experimental intervention designed to alter them'' \citep[138]{Winer:1996as}. \citet{Winer:1996as} hypothesize that both the tendency for extramission beliefs to persist into adulthood and their resistance to experimental intervention is partly explained by a phenomenological truth enshrined in extramission models:
\begin{quote}
	We assume that core aspects of the phenomenology of vision underlie extramission interpretations. Consider one phenomenologically salient aspect of vision, namely, its orientational or outer-directed quality. When people see, they are generally oriented toward an external visual referent, that is, they direct their eyes and attention to an object in order to see it. In fact, this quality of vision is reflected in language. People talk about ``looking at'' things, and English has expressions such as ``looking out of a window'' and ``looking out of binoculars.'' Even notions such as ``piercing glances'' and ``cutting looks'' suggest and outer directionality \ldots\ \citep[140]{Winer:1996as}
\end{quote}

% Plato's \emph{Timaeus} contains a close echo of the lantern analogy:
% \begin{quote}
% 	And of the organs they first contrived the eyes\index{eye} to give light\index{light}, and the principle according to which they were inserted was as follows. So much of fire\index{fire} as would not burn, but gave a gentle light, they formed into a substance akin to the light of everyday life, and the pure fire which is within us and related thereto they made flow through the eyes in a stream smooth and dense, compressing the whole eye and especially the center part, so that it kept out everything of a coarser nature and allowed to pass only this pure element. (Plato, \emph{Timaeus} 45\( ^{b-c} \), Jowett in \citealt{Hamilton:1989fk})
% \end{quote}

Merleau-Ponty provides a description of the active, outer-directed phenomenology of vision that would make talk of extramission apt:
\begin{quote}
	If I adhere to what immediate consciousness tells me, the desk which I see in front of me and on which I am writing, the room in which I am and whose walls enclose me beyond the sensible field, the garden, the street, the city and, finally, the whole of my spatial horizon do not appear to me to be causes of the perception which I have of them, causes which would impress their mark on me and produce an image of themselves by a transitive action. It seems to me rather that my perception is like a beam of light which reveals the objects there where they are and manifests their presence, latent until then. Whether I myself perceive or consider another subject perceiving, it seems to me that the gaze ``is posed'' on objects and reaches them from a distance---as is well expressed by the use of the Latin \emph{lumina} for designating the gaze. \citep[185]{Merleau-Ponty:1967fj}
\end{quote}
Merleau-Ponty is not endorsing the extramission theory as a causal model of perception. He is not denying that the object of perception is the ultimate efficient cause of that perception. Rather, in seeing the desk before him, Merleau-Ponty claims only that his experience does not present itself as the exercise of a passive power, a sensory impression caused in him by the mediate causal action of the distal object. There may be an active element to outwardly attending, in vision, to distal aspects of the natural environment, and this may be phenomenologically vivid, but that is consistent with the object of visual perception being among its causal antecedents. A visual experience may be undergone, but seeing is not something done to the perceiver, but something the perceiver does.

Nor is Merleau-Ponty claiming that it appears from within that seeing involves the emission of a fiery effluence akin to light. Rather Merleau-Ponty is pressing an analogy. He is describing what visual experience, from within, is like. More explicitly, the awareness afforded by visual experience is like a beam of light that manifests the latent presence of its object. Vision, like illumination, has direction. Vision is outer directed. In seeing, the perceiver looks upon the scene before them. Moreover, just as illumination manifests the latent visibility of an object, seeing an illuminated object manifests its latent presence to the perceiver revealing it to be where it is. The explicit awareness of a the natural environment afforded by visual experience is akin to light not only in its directionality and its power to manifest latent presence, but in the manner in which it discloses distal aspects of that environment. Just as beam of illumination may ``pose'' on an object that it illuminates and that it reaches from a distance, the perceiver's gaze may ``pose'' on the object that it presents and that it reaches from a distance. The illumination alights upon the object it illuminates at a distance from its source, the perceiver's gave alights upon the object of perception at a distance from the perceiver. The imagery suggests active outward extension, as in Kilwardby's wax actively pushing against the seal.

Accepting the aptness of the analogy is not tantamount to accepting the extramission theory. Consider a similar analogy of Olivi's:
\begin{quote}
	an object, to the extent that the gaze (\emph{aspectus}) and the act of a power are terminated at it, co-operates in their specific production [\ldots] Namely, the cognitive act---and the gaze---is fixed (\emph{figitur}) to the object and it absorbs the object intentionally to itself. This is why a cognitive act is called the apprehension of, and the apprehensive extension to, the object. In this extension and absorption the act becomes intimately conformed and assimilated into the object. The object presents itself or appears as being present to the cognitive gaze, and the object is a kind of representation of itself by an act which is assimilated to it. As an actual illumination of a spherical or quadrangular vase becomes spherical or quadrangular only because the light source generates the illumination in conformity with the figure of the object which receives and confines it; so also, because a cognitive force generates a cognitive act with a certain formative absorption of the act towards the object, and with a certain signet-like and inward (\emph{sigillari et viscerali}) extension of the object, therefore---because it is generated thus---the act becomes a similitude and signet-like expression of the object. (Peter John Olivi, \emph{Questiones in secundum librum Sententiarum} q. 72 35--36; \citealt[146--147]{Toivanen:2013ul})
\end{quote}
Despite his play with neo-Platonic imagery, no doubt an Augustinian heritage, Olivi is not endorsing an extramission theory of perception. Perceptual apprehension may be a form of apprehensive extension to its object, but this apprehensive extension is not corporeal. Though likened to illumination directed upon the object it illuminates, the cognitive act, the apprehensive extension by which that act assimilates to its object, does not consist in, or otherwise involve, the activity of a fiery substance, no matter how rarified. Likening the seeing of an object to light directed upon an object that it illuminates, by itself, carries with it no commitment to the metaphysics of extramission. Rather, Olivi, like Merleau-Ponty after him, is emphasizing the active, outer-directed nature of vision.

Moreover, like Merleau-Ponty, Olivi is presenting a conception of perception that contrasts with a mere passive reception of sensible form. Olivi, working in the same broadly Augustinian metaphysical framework as Kilwardby, is less concessive than Kilwardby to Peripatetic accounts of perception. 

Recall, according to the Peripatetic account, at least as understood by the late Scholastics, the perceived object acts upon the transparent medium such that its sensible form, its species exists, in some sense, in it, and that the medium, in turn, affects the sense organ such that the species comes to, in some sense, exist in it as well (\emph{De Spiritu Fantastico} 69, 97). So understood, the eye's reception of a color species, while not a literal coloration, is the exercise of a passive power, like the power to be heated. The distal object mediately acting upon the perceiver's sense organ posited by the Peripatetic account was understood, by Kilwardby, as necessary if insufficient for perception. In order for perception to occur, the perceptive sole must assimilate the species, but this requires the soul's activity. 




% section the_truth_in_extramission (end)

% chapter vision (end)