%!TEX root = /Users/markelikalderon/Documents/Git/sympathy/perception.tex
\chapter{Vision} % (fold)
\label{cha:vision}

\section{The Biranian Principle} % (fold)
\label{sec:the_birnaian_principle}

So far we have discussed grasping and listening and how they make sympathetic presentation possible in haptic and auditory experience respectively. We turn now to looking. I shall argue that looking makes possible sympathetic presentation in visual experience. Our guiding idea, echoing Maine de Biran, is that in order to see well, one must look. Our task is to describe a conception of looking that could plausibly make this principle true. 

Such a conception must satisfy two conditions. A conception of looking that stands a chance of making true the Biranian principle must at once be something that the perceiver does and that makes the distal environment perceptually accessible.

First, looking must be something the perceiver does. Only in this way is the analogy with grasping, enshrined in the Protagorean model, sustained. However, like the case of audition, this psychological stance may be sustained, in the Peripatetic fashion, by a capacity to act. Looking, like listening, while not a passive power, may be less than fully active. In the traditional, post-Aristotelian vocabulary, that stance may be sustained by a first actuality if a second potentiality. Looking may be a psychological stance sustained, at a minimum, by the potential to act in visually relevant ways, on some appropriate understanding of that potentiality. While looking and listening may fall short of the exemplar, grasping, since haptic perception requires the second actuality of the hand's activity to sustain it, still, they are not something done to the perceiver but something that the perceiver does. What the perceiver does in looking may be sustained, in certain circumstances, by nothing further than a preparedness to act in perceptually relevant ways. At a minimum, then, looking merely requires vigilance (perhaps fortuitously, ``vigilance'' derives from the Latin \emph{vigilare} meaning to watch). But the perceiver's perceptual vigilance over the distal environment, their being prepared to act in visually relevant ways to bring aspects of the distal environment into view or at least increase the acuity with which they are seen, remains a stance actively sustained by the perceiver. Moreover, the stance sustained is itself a kind of activity. Looking is something the perceiver does. The perceiver, in maintaining vigilance, looks outward. Looking is a kind of outer-directed opening up to the visible. Looking is something the perceiver does, even if in seeing what they do in looking they undergo an experience caused in them, at least in part, by what they perceive.

Second, looking is an activity of the perceiver whose end is to bring distal aspects of the natural environment into view. Looking makes aspects of the distal environment perceptually accessible. For the perceiver to act in visually relevant ways is for them to alter their visual perspective on the natural environment so as to present distal aspects of that environment, or, at least, increase the acuity with which those aspects are seen. 

A conception of looking answering to the truth of the Biranian principle---that in order to see well, one must look---would most likely exceed the conception of looking enshrined in ordinary usage, though, perhaps, in the manner of a conservative extension. This might count against describing such a conception as an instance of ``looking.'' However, other alternatives fare less well. ``Gaze'' is, by now, perhaps too ethically fraught (especially after \citealt{Mulvey:1975aa}, see also \citealt{Jay:1994aa}). Olivi's \emph{aspectus}, while a historically important antecedent, is too technical sounding and is bound up with Olivi's Augustinian dualism. Thus, for example, Olivi distinguishes the physical \emph{aspectus} of the sense organ, the eye pointed in a certain direction, say, from the spiritual \emph{aspectus} of the soul. (Though Olivi's notion of \emph{aspectus} may owe as much to Alhazen's \emph{De aspectibus} as to Augustine, see \citealt[41 especially n. 43]{Tachau:1988aa}; on Olivi on perception see \citealt[3--26, 39--54]{Tachau:1988aa}, \citealt[215--224]{Spruit:1994qq}, \citealt[121--124, 130--134, 168--181]{Pasnau:1997aa}, \citealt[part 1]{Toivanen:2009zf}, \citealt{Silva:2010zh}, \citealt[part 2]{Toivanen:2013ul}). In the absence of an adequate alternative, we shall persist with talk of looking, mindful of the ways that the demands of making true the Biranian principle might exceed the conception of looking enshrined in ordinary usage.

Not only shall I defend the Biranian principle, but I shall offer an explanation for it in terms of the operation of sympathy. Looking makes aspects of the distal environment perceptually accessibly by making possible their sympathetic presentation in visual experience. We shall begin our search for a conception of looking that makes true the Biranian principle in an unlikely place, in what historians describe as extramission theories of perception. Extramission theories provide a false causal model of distal perception, where a part of the perceiver extends through space so that it is in contact with the perceived object such that the perceiver, or at least a part of them, is substantially located where the perceived object is. Thus Nemesius Bishop of Emesa attributes such a view to Hipparchus, a second century \textsc{bce} astronomer: 
\begin{quote}
	Hipparchus says that rays extend from the eyes and with their extremities lay hold on external bodies like the touch of hands \ldots\ (\emph{De natura hominis} 7, \citealt[104]{Sharples:2008aa})
\end{quote}
This conception of perception spontaneously arises for many, outside of explicitly theoretical contexts, and is surprisingly resilient to empirical counter-evidence. As we shall see this is because there is a phenomenological insight enshrined in extramission theories, a phenomenological insight that may be preserved even should we abandon the false causal model that it provides of distal perception. Developing that phenomenological insight will result in the advertised conception of looking that makes true the Biranian principle.

% Section the_birnaian_principle (end)

\section{The Persistence of Extramission} % (fold)
\label{sec:the_persistence_of_extramission}

\citet[48]{Piaget:1929dp} observed that there is a the tendency for children to understand vision in terms of an active, outward influence of the eyes. This tendency was manifest in reports of looks mixing and in ``a confusion between vision and light''. Concerning the latter Piaget reports:
\begin{quotation}

	Pat (10) stated that a box makes a shadow ``\emph{because the clouds} (Pat believes it to be the clouds which give light when there is no sun) \emph{can't pass through it}'' (\emph{i.e.} because the light cannot pass through the box).
	
	But immediately after Pat said of a portfolio that it made a shadow ``\emph{because the clouds can't see that side}.---Are to see and to give light the same thing?---\emph{Yes}.---Tell me the things which give light?---\emph{The sun, the moon, the stars, the clouds and God}.---Can you give light?---\emph{No} \ldots\ \emph{Yes}.---How?---\emph{With the eyes}.---Why?---\emph{Because if you hadn't eyes you wouldn't see properly}.''
	
	Duc (6 1/2) also stated that the light cannot see through a hand, alike confusing ``seeing'' with ``giving light.''
	
	Sci (6) said that dreams come ``\emph{with the light}.''---``How?---\emph{You are in the street. The lights} (street-lamps) \emph{can see there} \ldots\ \emph{they see on the ground}.'' ``Tell me some things that give light.---\emph{Lights, candles, matches, thunder, fire, cigarettes}.---Do eyes give light or not?---\emph{Yes, they give light}.---Do they give light at night?---\emph{No}?---Why not?---\emph{Because they are shut}.---When they are open do they give light?---\emph{Yes}.---Do they give light like lamps?---\emph{Yes, a little bit}.'' \citep[48]{Piaget:1929dp}
\end{quotation}
And \citet[48--49]{Piaget:1929dp} goes on to compare these reports with Empedocles' lantern analogy. (On the lantern analogy, see \citealt[240--243]{Wright:1981zr}; on Empedocles' theory of vision see \citealt{Sedley:1992uq}, \citealt{Ierodiakonou:2005fk}, \citealt[Chapter 1]{Kalderon:2015fr}.)
% \begin{verse}
% 	As when someone planning a journey prepared a lamp,\index{lantern}\\
% 	the gleam of blazing fire\index{fire} through the wintry night\index{night},\\
% 	and fastened linen\index{linen} screens\index{screen} against all kinds of breezes,\index{wind}\\
% 	which scatter the wind of the blowing breezes\\
% 	But the light\index{light} leapt outwards, as much of it as was finer,\\
% 	and shone with its tireless beams across the threshold;\\
% 	in this way [Aphrodite]\index{Aphrodite} gave birth to the rounded pupil\index{eye!pupil},\\
% 	primeval fire crowded in the membranes and in the fine linens\index{linen}.\\
% 	And they covered over the depths of the circumfluent water\index{water!external}\\
% 	and sent forth fire, as much of it as was finer.\\
% 	(Empedocles, \textsc{dk} 31\textsc{b}84; \citealt[103 259]{Inwood:2001ve})
% \end{verse}
% Just as there is fire in the interior of a screened lamp, there is a primeval fire in the interior of the eye, or perhaps the pupil. And just as the screen surrounds the fire in the lamp's interior, there is a membrane that surrounds the fire in the eye's interior. Moreover, the membrane plays a similar role to the screen. Just as the screen protects the interior fire from the wind which would extinguish it, the primeval fire is protected from the depth of the surrounding water by the membrane of the eye. Finally, just as light passes through the screen, the primeval fire can pass through passages in the membrane of the eye.  Thus according to Empedocles' lantern analogy, vision involves an active, outward influence of the eyes that is akin to light---just as Piaget's children report. Perhaps the occurence of extramission beliefs early on in our cognitive development explains, in part, the occurence of extramission theories early on in our philosophical history (though unpopular in the Latin West, extramission theories do not completely disappear until the thirteenth century; on extramission theories see \citealt[3--67]{Lindberg:1977aa}).

\citet[138]{Winer:1996as}, prompted by Piaget's observations, were ``sur\-pri\-sed---indeed shocked'' by the degree and resilience of belief in extramissive perception. Not only do children hold extramission beliefs but so do adults, though such beliefs tend to decline during adulthood. To the simple question that required a ``yes'' or ``no'' response:
\begin{quote}
	When we look at someone or something, does anything such as rays, waves, or energy go out of our eyes?
\end{quote}
49\% of the first graders, 70\% of the third graders, 51\% of the fifth graders, and 33\% of the college students affirmed extramission. Moreover, these extramission beliefs proved ``highly resistant to experimental intervention designed to alter them'' \citep[138]{Winer:1996as}. 

\citet{Winer:1996as} augmented their use of verbal questions with graphic displays:
\begin{quote}
	The computer graphics portrayed various interpretations of the process of vision by displaying one or more renditions of a person looking at a rectangle, with visual input and output depicted by lines that appeared to move between the person's eye and the rectangle. Thus, in one graphic, lines, presumably representing rays, appeared to move inward from the rectangle to the eye of the figure on the screen, demonstrating the process of intromission. In another graphic, lines appeared to move outward from the eye toward the rectangle, demonstrating pure extramission. \citep[139]{Winer:1996as}
\end{quote}
They did so for two reasons. First, the graphic displays were used to filter out any misinterpretations that might have been suggested by the verbal questions. Second, they predicted that, given a hypothesized source of extramission belief, exposure to the graphic displays would increase the affirmation of extramission.

What is this hypothesized source? \citet{Winer:1996as} hypothesize that both the tendency for extramission beliefs to persist into adulthood and their resistance to experimental intervention is partly explained by a phenomenological truth enshrined in extramission models:
\begin{quote}
	We assume that core aspects of the phenomenology of vision underlie extramission interpretations. Consider one phenomenologically salient aspect of vision, namely, its orientational or outer-directed quality. When people see, they are generally oriented toward an external visual referent, that is, they direct their eyes and attention to an object in order to see it. In fact, this quality of vision is reflected in language. People talk about ``looking at'' things, and English has expressions such as ``looking out of a window'' and ``looking out of binoculars.'' Even notions such as ``piercing glances'' and ``cutting looks'' suggest and outer directionality \ldots\ \citep[140]{Winer:1996as}
\end{quote}
On this basis, they predicted an increase in the affirmation of extramission because of the way that the graphics ``present representations that are suggestive of the orientational aspects of vision.'' And subsequent studies confirmed this.

It is unclear, at least to me, what to make of this increased affirmation of extramission in response to the use of graphic displays. The displays do not unambiguously represent the intended interpretations of the process of vision. Specifically, they do not unambiguously represent lines of causal influence. As \citet{Winer:1996as} observe, they are at least suggestive of the orientational aspects of vision. But given the iconic nature of the pictorial representation, moving lines might represent lines of causal influence, but they might just as easily represent lines of sight. Perhaps, the increased affirmation of extramission is less an expression of belief in extramission than an expression of the active, outer-directed phenomenology of vision. Perhaps, the affirmation of extramission involved belief in, not a scientific misconception, but a phenomenological truth misleading expressed (see \citealt{Robbin:2003xy} for a similar worry). \citet{Winer:1996as} claim to control for this, but whether they did so successfully is difficult to independently assess.

Whatever the genuine extent of extramission belief, it is the phenomenological diagnosis for it that we shall focus upon. In the next Section, we shall examine the active, outer-directed phenomenology of vision that \citet{Winer:1996as} take to underly belief in extramission.

% Section the_persistence_of_extramission (end)

\section{The Truth in Extramission} % (fold)
\label{sec:the_truth_in_extramission}


% Plato's \emph{Timaeus} contains a close echo of the lantern analogy:
% \begin{quote}
% 	And of the organs they first contrived the eyes\index{eye} to give light\index{light}, and the principle according to which they were inserted was as follows. So much of fire\index{fire} as would not burn, but gave a gentle light, they formed into a substance akin to the light of everyday life, and the pure fire which is within us and related thereto they made flow through the eyes in a stream smooth and dense, compressing the whole eye and especially the center part, so that it kept out everything of a coarser nature and allowed to pass only this pure element. (Plato, \emph{Timaeus} 45\( ^{b-c} \), Jowett in \citealt{Hamilton:1989fk})
% \end{quote}

Merleau-Ponty provides a description of the active, outer-directed phenomenology of vision that would make talk of extramission apt:
\begin{quote}
	If I adhere to what immediate consciousness tells me, the desk which I see in front of me and on which I am writing, the room in which I am and whose walls enclose me beyond the sensible field, the garden, the street, the city and, finally, the whole of my spatial horizon do not appear to me to be causes of the perception which I have of them, causes which would impress their mark on me and produce an image of themselves by a transitive action. It seems to me rather that my perception is like a beam of light which reveals the objects there where they are and manifests their presence, latent until then. Whether I myself perceive or consider another subject perceiving, it seems to me that the gaze ``is posed'' on objects and reaches them from a distance---as is well expressed by the use of the Latin \emph{lumina} for designating the gaze. \citep[185]{Merleau-Ponty:1967fj}
\end{quote}
Merleau-Ponty is not endorsing the extramission theory as a causal model of perception. He is not denying that the object of perception is the ultimate efficient cause of that perception. Rather, in seeing the desk before him, Merleau-Ponty claims only that his experience does not present itself as the exercise of a passive power, a sensory impression caused in him by the mediate causal action of the distal object. There may be an active element to outwardly attending, in vision, to distal aspects of the natural environment, and this may be phenomenologically vivid, but that is consistent with the object of visual perception being among its causal antecedents. ``My present experience of this desk is not complete, \ldots\ it shows me only some of its aspects'' \citep[186]{Merleau-Ponty:1967fj}. Merleau-Ponty's experience may be incomplete in that it reveals only some aspects of its object, but once we allow that perception is partial in this way, it is at least open that experience is incomplete, as well, in that it only manifests some aspects of its nature. The active, outer-directed nature of vision may be phenomenologically vivid, but vision may still require that the distal object mediately act upon the perceiver. A visual experience may be undergone, but seeing is not something done to the perceiver, but something the perceiver does.

Nor is Merleau-Ponty claiming that it appears from within that seeing involves the emission of a fiery effluence akin to light. Rather Merleau-Ponty is pressing an analogy. He is describing what visual experience, from within, is like. And not only from within but from without as well. The analogy holds not only when Merleau-Ponty considers his own experience but also when he considers the experience of another perceiving subject. Consider another's piercing glance or cutting look \citep[140]{Winer:1996as}. Piaget's reports of looks mixing are cases where the analogy would hold, as well, of other perceiving subjects: 
\begin{quotation}
	From a boy of 5 years old: ``\emph{Papa, why don't our looks mix when they meet}.''
	
	From one of our collaborators: ``\emph{When I was a little girl I used to wonder how it was that when two looks met they did not somewhere hit one another. I used to imagine the point to be half-way between the two people. I used also to wonder why it was one did not feel someone else's look, on the cheek for instance if they were looking at one's cheek.}'' \citep[48]{Piaget:1929dp}
\end{quotation}
Merleau-Ponty, then, is describing that aspect of our visual phenomenology, considered from within and without, that \citet{Winer:1996as} claim to underlie extramission beliefs. 

More explicitly, the awareness afforded by visual experience is like a beam of light that manifests the latent presence of its object. Vision, like illumination, has direction. Light is emitted outward from its source upon the scene that it illuminates.  Vision too is outer-directed. In seeing, the perceiver looks out upon the scene before them. Not only do vision and light have direction but they are both rectilinear as well. Moreover, just as illumination manifests the latent visibility of an object, seeing an illuminated object manifests its latent presence to the perceiver revealing it to be where it is. The explicit awareness of the natural environment afforded by visual experience is akin to light not only in its rectilinear directionality and its power to manifest latent presence, but in the manner in which it discloses distal aspects of that environment. Just as a beam of light may ``pose'' on an object that it illuminates and that it reaches from a distance, the perceiver's gaze may ``pose'' on the object that it presents and that it reaches from a distance. The illumination alights upon the object it illuminates at a distance from its source, the perceiver's gaze alights upon the object of perception at a distance from the perceiver. The imagery here not only emphasizes that vision is a kind of perception at a distance but invokes an active outward extension, as in Kilwardby's wax actively pushing against the seal (Chapter ~\ref{sec:active_wax}).

Accepting the aptness of the analogy is not tantamount to accepting the extramission theory. Consider a similar analogy of Olivi's:
\begin{quote}
	an object, to the extent that the gaze (\emph{aspectus}) and the act of a power are terminated at it, co-operates in their specific production [\ldots] Namely, the cognitive act---and the gaze---is fixed (\emph{figitur}) to the object and it absorbs the object intentionally to itself. This is why a cognitive act is called the apprehension of, and the apprehensive extension to, the object. In this extension and absorption the act becomes intimately conformed and assimilated into the object. The object presents itself or appears as being present to the cognitive gaze, and the object is a kind of representation of itself by an act which is assimilated to it. As an actual illumination of a spherical or quadrangular vase becomes spherical or quadrangular only because the light source generates the illumination in conformity with the figure of the object which receives and confines it; so also, because a cognitive force generates a cognitive act with a certain formative absorption of the act towards the object, and with a certain signet-like and inward (\emph{sigillari et viscerali}) extension of the object, therefore---because it is generated thus---the act becomes a similitude and signet-like expression of the object. (Peter John Olivi, \emph{Questiones in secundum librum Sententiarum} q. 72 35--36; \citealt[146--147]{Toivanen:2013ul})
\end{quote}
The passage is complex and is replete with suggestive detail. But to begin with, focus on the analogy with illumination.

Despite his play with neo-Platonic imagery, no doubt an Augustinian heritage \citep[198]{Kent:1984zm}, Olivi is not endorsing an extramission theory of perception. Olivi explicitly denies that extension involves any real emission (\emph{Questiones in secundum librum Sententiarum} q. 58 ad 14.8). Perceptual apprehension may be a form of apprehensive extension to its object, but this apprehensive extension is not corporeal. Though likened to illumination directed upon the object it illuminates, the perceptual act, the apprehensive extension by which that act assimilates to its object, does not consist in, or otherwise involve, the emission of a fiery substance, no matter how rarified. Nor does the apprehensive extension involve the emission of any spiritual matter. Perception is a simple, spiritual act that takes place in the soul, and the soul can only be in the body, at least when alive. So no part of the soul is substantially located where the perceived object is as would be the case if extramission were true. Likening the seeing of an object to light directed upon an object that it illuminates, by itself, carries with it no commitment to the metaphysics of extramission. Rather, Olivi, like Merleau-Ponty after him, is emphasizing the active, outer-directed nature of vision.

Moreover, like Merleau-Ponty, Olivi is presenting a conception of perception that contrasts with a mere passive reception of sensible form. Olivi, however, working in the same broadly Augustinian metaphysical framework as Kilwardby, is less concessive to Peripatetic accounts of perception. 

Recall (Chapter~\ref{sec:active_wax}), according to the Peripatetic account, at least as understood by the late Scholastics, the perceived object acts upon the transparent medium such that its sensible form, its species exists, in some sense, in it, and that the medium, in turn, affects the sense organ such that the species comes to, in some sense, exist in it as well (\emph{De spiritu fantastico} 69, 97). So understood, the eye's reception of a color species, while not a literal coloration, is the exercise of a passive power, like the power to be heated. The distal object mediately acting upon the perceiver's sense organ posited by the Peripatetic account was understood, by Kilwardby, as a necessary if insufficient condition for perception. In order for perception to occur, the perceptive soul must assimilate the species, but this requires the soul's activity. 

According to Olivi, however, the affection of the sense organ by a species originating from the distal object is not even a necessary condition for its perception. Echoing Plotinus (Plotinus, \emph{On Difficulties about the Soul \textsc{iii}, or On Sight}, \emph{Ennead} 4 5 2 50–55, Chapter~\ref{sec:plotinus}), Olivi maintains that if perception were mediated by a corporeal species, the species would be the object of perception thus screening off the distal object in the natural environment (on Plotinus' argument, see \citealt[Chapter 3]{Emilsson:1988uq}, on Olivi's argument see \citealt[43--45, especially n. 53]{Tachau:1988aa} and \citealt[Chapter 7.3]{Pasnau:1997aa}). The object of perception is not an efficient cause of that perception, no matter how mediate. 

Like Kilwardby, Olivi is moved, in part, by the Augustinian doctrine of the ontological superiority of the soul over the body, though perhaps Olivi interprets that doctrine more stringently than Kilwardby. The way in which the soul and its powers and acts are superior to the body is inconsistent with a body ever acting upon the soul. So an extended corporeal species could not activate the perceptual power, the power and its act being simple and spiritual (\emph{Questiones in secundum librum Sententiarum} q. 73 83--4). \citet[263]{Silva:2010zh} observe how Olivi anticipates, here, the Cartesian distinction between \emph{res extensa} and \emph{res cogitans} \citep[see also][46]{Tachau:1988aa}. The powers of the soul, even perceptual powers, are not the passive recipients of external stimuli but are active. Like Merleau-Ponty, Olivi thinks that the active character of our perceptual powers is phenomenologically evident (\citealt[3--26, 39---54]{Tachau:1988aa}, \citealt[236--47]{Pasnau:1997aa}, \citealt[143]{Toivanen:2013ul}). 

While Olivi does not deny that perception presupposes the presence of its object in the natural environment, he does deny that it is, or even among, the efficient causes of perception. In Olivi's technical vocabulary, the object of perception is a terminative cause. It is controversial how to understand Olivi's terminative causes. Are terminative causes a species of final cause, as \citet[192--195]{Kent:1984zm} and \citet{Pasnau:1999kn} maintain? Or are they a kind of cause not classified by the traditional Peripatetic four causes (\emph{Physica} 2 3, \emph{Metaphysica} E 2), as \citet[Chapter 6]{Toivanen:2013ul} maintains? While it is difficult to form a clear, positive conception of terminative causes, the negative contrast with efficient causes is clear. The actualization of a perceptual power may require the presence of its object in the natural environment but that object acting upon the power is not required for its actualization. The efficient cause of the perceptual act is the power and not the object of perception. The presence of that object merely cooperates by being the \emph{terminus} of the perceptual act, that which it is directed upon, like light directed upon a spherical vase.

So Olivi maintains that the active, outer-directed phenomenology of vision is inconsistent with the object seen being the efficient cause of that perception. Merleau-Ponty, by contrast, merely claims that the object of perception acting upon the perceiver, however mediately, is not manifest in our experience, not that it is inconsistent with it. Perhaps this more cautious attitude is, in the end, warranted. I do not recommend this more cautious attitude merely as a beneficiary of optical knowledge unavailable to Olivi but on philosophical grounds as well.

To bring this out, first consider an element of the Olivi passage that goes beyond what Merleau-Ponty explicitly describes. Olivi, like Merleau-Ponty, uses the neo-Platonic imagery of illumination to emphasize the active outward extension involved in the visual apprehension of the distal environment, and where this active, outward extension is no kind of extramission. Olivi goes further than Merleau-Ponty, however, in coupling the active, outward extension of the illumination with being shaped by its \emph{terminus}, the circular or quadrangular vase, say. In illuminating a circular vase, the area illuminated is itself circular. The shape of the area illuminated is constituted by the shape of the object illuminated. The illumination is ``in conformity with the figure of the object which receives it and confines it.'' This is meant to be an analogy for how the perceptual act formally assimilates to its object. Extension and absorption are linked. In neo-Platonic vocabulary, extension and absorption are a kind of procession and return. Like Kilwardby before him, Olivi thinks that the perceptual act only assimilates to its object thanks to the activity of the perceptual soul. Indeed, the passage ends with Olivi echoing Kilwardby's figure of the active wax pressing against the seal. (It is unclear whether Olivi read Kilwardby. Perhaps similar paths were laid out for them by their shared Augustinian heritage. For a comparison of Kilwardby and Olivi see \citealt{Silva:2010zh}.)

Extension and absorption, a kind of procession and return, is important, so it is perhaps worth a brief digression on a detail of the passage that we have so far glossed over. The perceiver's gaze, in being fixed on its object, a circular vase, say, absorbs the object intentionally to itself. It is only in intentionally absorbing the object of perception that the perceptual act assimilates to its object. Moderns should resist the temptation to understand the qualifier ``intentionally'' in terms of the notion of intentionality derived from \citet{Brentano:1874aa} (on the historical development of the concept of intentionality see \citealt{Sorabji:2003fk}; on Olivi's role in the development of intentionality in late Scholasticism see \citealt[Chapter 2]{Pasnau:1997aa}). A sensible form inhering in a body, the whiteness inhering in a circular vase, say, has natural existence in that body. Part of the point of the qualifier is to deny that the perceived sensible form has natural existence in the perceptual act. In part, then, the point of the qualifier is to rule out a position like the one Crathorn will later endorse where perception becomes colored in seeing a colored object and so avoid Theophrastus' \emph{aporia} (Chapter~\ref{sec:assimilation}). However, not only does Olivi deny natural existence to the sensible form in the perceptual act, he denies, as well, its real existence. This prompts \citet[67]{Pasnau:1997aa} to remark that with Olivi, there is ``movement toward making intentionality mysterious.''

Moreover, according to Olivi, the intentional existence of the object in the perceptual act---in virtue of which it assimilates to that object and so becomes like it, if not naturally like, in the manner of Crathorn---involves the perceptual power's virtual presence to that object (on the meaning of \emph{virtualis} in late Latin and Olivi specifically see \citealt[172--173]{Pasnau:1997aa}). Specifically, it is because the power is virtually present to its object that that object comes to exist intentionally in the actualization of that power: 
\begin{quote}
	A power can be present to something either essentially or virtually. This is to say that it can be present to something in such a way that its essence really is beside that thing, or in such a way that the gaze (\emph{aspectus}) of its power is so efficaciously directed to the thing that it, as it were, really touches the thing. If the power is not present to its object or recipient (\emph{patienti}) in this second way, it cannot act, even if it were present to it by its essence or according to the first way. The visual power is present to a thing that is seen from a distance in this [second] way. \dots\ This [kind of] presence suffices for an act of seeing. (Olivi, \emph{Quaestiones in secundum librum Sententiarum}, q. 58 486--487; \citealt[151--152]{Toivanen:2013ul})
\end{quote}
In speaking of a power's presence to its object as opposed to the object's presence to the power, Olivi is emphasizing the active, outer-directed nature of that power. If a power is essentially present to an object, then the power and the object are contiguous, ``its essence really is beside that thing,'' and there is a real connection between them akin to the perception by contact involved in touch. In contrast, if a power is virtually present to an object, then the object and the power are not contiguous but are at a distance from one another. Moreover, there is no real connection between the object and the power whose act contains it. Virtual presence is a necessary condition for visual perception. It is only by the visual power being virtually present to an object that seeing that object may formally assimilate to it. Virtual presence is also a sufficient condition. The virtual presence of the visual power to an object suffices for its extensive apprehension. 

The virtual presence of a power to its object precludes the need for any real connection between them. A visible object need not be palpable to vision the way in which a corporeal body must be palpable to touch if it is to be felt (though contrast the account of vision that Socrates attributes to Empedocles in the \emph{Meno} 76 a--d; see \citealt[Chapter 1.2]{Kalderon:2015fr} for discussion). There need be no contact between sight and its object in order for the latter to be seen, not even mediate contact. And, at least by Olivi's lights, contact is required for a real connection. Olivi's notion of a terminative cause is meant to explain how a sensory power may be the total efficient cause of its act and yet its content be determined by an object in the distal environment to which that power is merely virtually present.

Like intentional existence, the virtual presence of a power to an object contrasts with, not only the natural existence of that object in that power's act, but its real existence as well. Moreover, while the presence of the object in the natural environment may be required for its perception, it is not among the efficient causes of perception. But if what is intentionally absorbed by the perceptual act lacks both natural and real existence, and the object of perception in no way acts upon the perceiver, one may well wonder how, exactly, it may shape that act such that the perceptual act formally assimilates to its object.

Contrast Olivi's position with the neo-Platonically inspired account of perception developed herein. Recall, sympathy played two roles in Plotinus' account of vision (Chapter~\ref{sec:plotinus}). First, it was meant to explain the action at a distance involved in visual perception. Specifically, sympathy was the principle by which the distal object may affect the sense organ without affecting anything in between. For Plotinus, at least, this was a real connection. Plotinus denies that a real connection requires contact. There is action at a distance, and sympathy is its principle. Second, sympathy was meant to explain how the distal object, and not sensible aspects of the medium, is present in the perceiver's visual experience of it. It is this second suggestion that we have taken up and generalized. In taking the visual power to be merely virtually present to its object, Olivi overlooks the possibility of sympathetic presentation. 

Linked to this is contrasting attitudes to the location of the perceptual act. Though Olivi may have inherited the neo-Platonic imagery from Augustine, one thing that he does not inherit is the neo-Platonic tendency to locate the perceptual act in its object. Thus in the \emph{Sermones} 277 10 Augustine writes ``to have opened the eye is to have arrived'' at the object seen \citep[82]{ODaly:1987fq}. We have seen an example of this already in a passage of Plotinus cited earlier  (Chapter~\ref{sec:sympathy_as_the_principle_of_haptic_presentation}), though it passed by uncommented: 
\begin{quote}
	It is clear in presumably every case that when we have a perception of anything through the sense of sight, we look where it is and direct our gaze where the visible object is situated in a straight line from us; \emph{obviously it is there that the apprehension takes place} [my emphasis] and the soul looks outwards. (Plotinus, \emph{On Sense-Perception and Memory}, \emph{Ennead} 4 6 1 14--18; \citealt[321]{Armstrong:1984aa})
\end{quote}
Toward the end of a passage emphasizing the active nature of visual perception, Plotinus makes, at least to our post-Cartesian ears, a startling pronouncement: That the apprehension of the visible object takes place in the object seen. Olivi, by contrast, denies that the perceptual act takes place in its object (at least if this is understood non-metaphorically \emph{Quaestiones in secundum librum Sententiarum} q. 37 obj. 13, ad. 13). Rather, it is a simple, spiritual act of the immaterial soul, and the soul is located where the body it animates is, at least when alive (for a comparison of Olivi's conception of perception with the neo-Platonic conception see \citealt[151]{Toivanen:2013ul}). Olivi, in making this denial, overlooks the possibility of sympathetic presentation. When I look where the ancient chestnut tree is and direct my gaze at that tree situated in a straight line from me, sympathy places me in the very heart of things, and it is there, where the tree grows too slowly to be perceptible, that my visual apprehension of it takes place.

There is nothing virtual about the sympathetic presence of the ancient chestnut tree in my perception of it. Even allowing that presence may be said of in many ways, virtual presence is no presence at all. If I were merely virtually present to the tree in seeing it, it is hard to understand how my visual experience could be shaped by that tree. And if my visual experience is not shaped by that tree, then it is not present in my experience. (Similar remarks apply to Noë's \citeyear{Noe:2012aa} more recent account of perception in terms of virtual presence. My criticism of Olivi may thus be read as an indirect criticism of Noë.)

To bring this out, consider the way the neo-Platonic analogy fails to support Olivi's extreme position. Indeed, attending to its details, reveals a striking \emph{aporia}. The object of illumination, the illuminated circular vase, say, receives and confines that illumination. In receiving and confining the illumination the illuminated area takes on the shape that it does. In receiving and confining the illumination the circular vase resists that illumination. It obstructs that illumination and so casts a shadow. It is hard to understand how the spherical vase may confine, resist, and obstruct the activity of the illuminant without being a cause, or, at least, a countervailing force. Of course, it is the source of the illuminant that generates the illumination, but the illuminated area takes on the shape that it does because the illuminated object resists the activity of the illumination insofar as it can. Kilwardby's doctrine that the soul's use of a body is limited by the passivities of matter (\emph{De spiritu fantastico} 99–100) was meant to address this kind of difficulty. However, the invocation of the neo-Platonic analogy just is Olivi's response. Olivi is drawing our attention to the fact that it is the source that generates the illumination and not the object illuminated. But that does not suffice to make the analogy consistent with taking the object of perception to be a terminative cause with all that that entails. Visual consciousness may extend to its object, but it must somehow come into conflict with it, as on the Protagorean model, if the subsequent absorption is to be so much as possible. 

How is the Peripatetic analogy, the ancient figure of the wax and seal, meant to be understood by Olivi's lights? It occurs at the point where Olivi spells out the consequences of the neo-Platonic analogy for perception. One curious feature of Olivi's treatment is the way that way that extension and absorption are transposed at this point. Whereas earlier in the passage Olivi speaks of the act's extension to its object, he now speaks of the ``formative absorption of the act towards the object''. And whereas earlier in the passage Olivi speaks of the act's absorption of the object by which the act assimilates to it, he now speaks of ``a certain signet-like and inward extension of the object''. I am uncertain of the significance of this transposition, if it is not, indeed, a slip on Olivi's part. If intentional, perhaps it is meant to emphasize the unity of extension and absorption. Extension is at once a formative absorption to the object, just as absorption is at once an inward extension of the object. Notice, on this hypothesis, the unity of extension and absorption only holds for extensive apprehension, the kind of extensive activity characteristic of perception, as opposed to a non-perceptual visual experience, such as a hallucination. In cases of hallucination, there is nothing to absorb. And so while such experiences may involve extensive activity, there is no subsequent absorption, merely the illusion of such.

The \emph{aporia} involved in Olivi's use of the neo-Platonic imagery affects his treatment of the ancient figure of the wax and seal. Even if, in line with the neo-Platonic analogy, the visual power generates the perceptual act in conformity with the figure of the object which receives it and confines it, how are we to understand this reception and confinement? ``Because it is generated thus the act becomes a similitude and signet-like expression of the object.'' Perception formally assimilates to its object because it is generated thus. It only conforms with its object by being received and confined. But reception and confinement is naturally understood as arising in the face of a countervailing force, the upshot of a conflict between the perceptual act and its object that resists it insofar as it can. It is hard to understand how the presence in the natural environment of an object which is the \emph{terminus} of the perceptual act could determine the content of that act, even if the act is directed upon it, like a beam of light, without somehow coming into conflict with it, as on the Protagorean model. Somehow the \emph{terminus} must determine the content of the perceptual act without itself being a determinant. But how could that be?

The present worry is anticipated by Duns Scotus. Scotus at least presses a parallel point about the intellect in his \emph{Ordinatio} and on the same general grounds. Though Scotus does not name names, Olivi is clearly a target as he reproduces a number of arguments from Olivi's \emph{Sentences} commentary \citep[148]{Pasnau:1997aa}. Scotus concedes to Olivi that the object of the intellect could not be the complete cause of the intellectual act. However, Scotus insists that the object must play some causal role if the act of intellect is to be a likeness of it (\emph{Ordinatio} 1 3 3 4 n. 486). Generalizing, Scotus' idea is that the demands of formally assimilating to the object require that the object play an explanatory role inconsistent with being a terminative cause. And it is the application of this general idea to the case of perception that constitutes the present worry (on Scotus on Olivi see \citealt[Chapter 4.4]{Pasnau:1997aa}, on Scotus on cognitive powers, both sensory and intellectual, see \citealt[Chapter 3]{Tachau:1988aa}, \citealt[257--266]{Spruit:1994qq}, and \citealt{Cross:2014aa}, for a related worry see \citealt[174--175]{Pasnau:1997aa}).

The worry reveals the way in which Olivi's view is a step along the way toward adverbialism \citep[see][]{Ducasse:1942oq}. Moreover, this is due, in part, to proto-Cartesian aspects of Olivi's view, an effect of their shared Augustinian heritage, such as Olivi's anticipation of the Cartesian distinction between \emph{res extensa} and \emph{res cogitans}, manifest, for example, in his denial that an extended, corporeal species may actualize the soul's perceptual power (on Descartes's Augustinianism see \citealt{Menn:1998nr}, not to mention Malebranche's testimony in \emph{Recherche de la Vérité}; for a non-Cartesian development of adverbialism see \citealt{Chirimuuta:2015aa}). The perceptual power is the total efficient cause of the perceptual act. Though the act is directed upon its \emph{terminus}, the object is not among the efficient causes of the perception. The perceptual power is merely virtually present to the object and so has no real connection with it. Though the presence of its object in the natural environment may occasion it, perception is a simple, spiritual act of the immaterial soul. To the extent to which the object present in the natural environment is a terminative cause, and so no determinant of the simple, spiritual act, that act is independent of its object in a way that anticipates more modern adverbialist theories. According to adverbialism, seeing blue is not a matter of being presented with an instance of blue in sight but rather seeing bluely. On adverbialist theories, then, the perceptual act is not constitutively shaped by its object but has its conscious character independently of that object. Olivi, of course, is no modern adverbialist. The simple, spiritual act may be determined independently of its object, but it is meant to be an intentional absorption of and assimilation to that object. The problem, of course, is to understand how Olivi could coherently maintain this.

Even if Olivi is wrong to deny that an object plays a causal role in its perception, he may be right in claiming that extension and absorption are linked. If extension and absorption are linked, if the wax only takes on the form of the seal by actively pressing against it, then the active extensive element in Merleau-Ponty's description is the basis for a subsequent absorption. The light is posed on the circular vase and is fixed there, and so the illuminated area is shaped by that vase. Merleau-Ponty's gaze is posed on his desk and is fixed there, and so his visual experience is shaped by that desk. Indeed, Olivi was criticized precisely by holding fast to the link between extension and absorption, a kind of procession and return, and drawing out what that entails, namely, that the active, outward extension's coming into conflict with the object is what explains, in part, that object's subsequent absorption. The grasping hand only conforms to rigid, solid body by grasping it. The grasping hand extends its grip until it can no more, consistent with its ends, and so conforms to the body's contours. It is only thanks to the activity of the hand and the resistance that it encounters that the perceiver's haptic experience formally assimilates to the tangible qualities of the object grasped. In this way is the hand the active wax of haptic perception. It is the force of the hand's activity coming into conflict with the self-maintaining forces of the object grasped that makes possible the sympathetic presentation of that object in haptic experience and its formal assimilation to that object, understood as a mode of constitutive shaping. The grasped object plays an explanatory role, inconsistent with being a mere terminative cause, in the conflict with the hand's grasp that discloses it. If perception's formal assimilation to its object, understood as a mode of constitutive shaping, is the basis of its objectivity, that is only so because of the explanatory priority of its object, an explanatory priority inconsistent with being a terminative cause. There is a connection, then, between perceptual objectivity and explanatory priority (Chapter~\ref{sec:haptic_perception}).

% Section the_truth_in_extramission (end)

\section{Looking} % (fold)
\label{sec:looking}

We have been discussing the active, outward, extensive character of visual phenomenology that underlies persistent belief in extramission in some children and adults and is plausibly the font of classical extramission theories. We have done so in aid of honing in on a conception of looking that stands a chance of make true the Biranian principle---in order to see well, one must look. Such a conception must involve the active, outer-directed extension of visual awareness where this involves the emission of nothing, no matter how rarified and akin to light. 

Like grasping and listening, there are three distinguishable moments in looking. The first moment corresponds to the preparatory reach in grasping, and might be performed for the end of grasping some particular object or for the end of grasping what there is to be grasped. Just as someone may reach out for something, and listen out for something (Chapter~\ref{sec:listening}), they may look out for something as well. A perceiver may look out with the end of seeing some particular object or with the end of seeing what there is to be seen. The second moment corresponds to the enclosure of the object grasped. Just as the hand, in reaching out, may come to conform to the contours of the object grasped, in listening out the perceiver may come to audibly attend to something and so listen to it. Similarly, in looking out for something the perceiver may come to look at something and so see it. In looking out an object is sighted. The third moment corresponds to the sustaining of enclosure. If the activity of the hand relaxes, the object will slip from its grasp. So not only is activity required to enclose the object in the hand’s grasp, but it is also required to sustain that grasp. Moreover, listening is required to sustain audible attention. Should the perceiver listen away, attending to some other audible event in a sonically complex environment, they would cease to listen to what they were initially listening to. Similarly, looking is required to sustain the explicit awareness afforded by visual experience. Should the perceiver look away, attending to some other visible aspect of the natural environment, they would cease to look at, and so visually attend to, the object of perception.

I turn, and look, and see an ancient chestnut tree. It is one of the ancient chestnut trees replanted in Greenwich Park when Charles \textsc{ii} had the park redesigned in the 1660s. An organism of impressive size and age presents itself. The majority of its burrs remain on the tree and are brighter green than the surrounding foliage. It is early evening, and the light is long and golden. The light both articulates the fine texture of the bark and sets off the overall flow of the trunk in dramatic relief. Despite its manifest strength and solidity, the twisted trunk appears to be flowing in a wave-like form. I come to realize that I am witnessing an organic process, the growth of the trunk, occurring so slowly as to appear, from within my limited temporal perspective, to be frozen, static. The difference in the scale of our lives is striking. For a moment, it induces in me a kind of temporal vertigo.  Just as a radical difference in spatial scale can be vertiginous---think of how small one can feel when viewing the Milky Way---a radical difference in temporal scale can be vertiginous as well. The scale of its life and the strength manifest in centuries of growth make the sweet chestnut tree a fit object of awe. I find myself musing that in a different cultural context, perhaps one more prone to animism, the tree might reasonably be reckoned a god. 

In looking at the ancient chestnut tree, I do so from across the park. I look at the tree by peering through the intervening space. My gaze perceptually penetrates that space until it encounters the ancient tree. The tree's surface is the site of visual resistance. Perceptually impenetrable, it determines a visual boundary through which, and in which, nothing further may appear. The tree is opaque to a significant degree. Its opacity consists in its resistance to my gaze. The illuminated air between, by contrast, being transparent, is perceptually penetrable. One can see in it and through it. Thus a scrub brush can appear in the water of a bath, and a cherry tree can appear through a window. Appearing through a medium does not require that the object be embedded in that medium the way appearing in does, though it is consistent with the object being so embedded at least if the perceiver is as well. Thus, it is through the illuminated air that the ancient chestnut tree is disclosed to me in sight. Looking, at least in the potentially extended sense that makes true the Biranian principle, involves the perceiver's gaze coming into conflict with what is perceptually impenetrable. (Compare the phenomenological interpretation I give of the bounded and unbounded in \emph{De sensu}, \citealt[Chapter 3.3]{Kalderon:2015fr}.)

\citet{Broad:1952kx} describes vision as prehensive and saltatory. It is prehensive insofar as vision involves the presentation of its object in the explicit awareness afforded by visual experience. It is saltatory insofar as this awareness seems to leap the spatial gap between the perceiver and the object. There are two separable elements to Broad's conception of saltitoriness. The first is simply the frank admission that vision is a kind of perception at a distance, that the objects of visual awareness are located at a distance from the perceiver. That much is unexceptional. The second is a phenomenological claim, that visual awareness seems to leap the spatial gap between the perceiver and the object of perception. For visual awareness to leap the spatial gap would be for the objects of visual awareness to be confined to a remote location and so not to have traversed the space between. However, I am visually aware not only of the coloring of the ancient chestnut tree and the wave-like form of its trunk, but of the intervening space as well. We not only see the colors of distant particulars and their shapes, but we do so by seeing through intervening illuminated media.

Two years after the appearance of ``Some elementary reflections on sense-per\-cep\-tion'', \citet[518]{Jonas:1954aa} will deny that vision is saltatory in Broad’s sense, and it is the second element of Broad's conception that he takes exception to and not the first: ``in sight the object faces me across the intervening distance, which in all its potential `steps' is included in the perception''. Broad is right to emphasize the distal character of the objects of vision, but his description of vision as saltatory is inapt since it fails to heed the perceptual penetrability of the intervening medium. Vision would leap the gap between the perceiver and the distal color if the object of visual awareness were confined to the remote spatial region where that color is instantiated. Vision, so conceived, would be a kind of ``remote viewing''. However, vision is not so confined and so does not leap the gap between the perceiver and distal color. Rather, by means of it, the perceiver may peer through the intervening medium, in all its potential steps, and encounter objects facing them across the intervening distance, if the medium is transparent at least to some degree. In the course of an otherwise astute and insightful comparative phenomenology of the senses, Broad is misled, at this point, by overlooking the active, outer-directed phenomenology of vision. Broad, in effect, overlooks the truth in extramission.

As in the case of audition, this psychological stance may be sustained, in the Peripatetic fashion, by a capacity to act. Looking, like listening, while not a passive power, may be less than fully active. In the traditional, post-Aristotelian vocabulary, looking, a psychological stance, may be sustained by a first actuality if a second potentiality. Looking may be a psychological stance sustained, at a minimum, by the potential to act in visually relevant ways, to alter one's visual perspective on the natural environment to better bring into view distal aspects of that environment, but only on a particular understanding of that potentiality. While looking and listening may fall short of the exemplar, grasping---haptic perception requires the second actuality of the hand's activity in order to sustain it---still, they are not something done to the perceiver but something the perceiver does. What the perceiver does in looking may may be sustained, in certain circumstances, by nothing further than a preparedness to act in perceptually relevant ways. Perhaps to get better sense of the trunk's flowing pattern, I must follow that pattern along with my gaze, at least to a certain degree, or in a certain way. Perhaps, I need to move closer, or perhaps further away. Looking at the ancient chestnut tree may involve, at a minimum, a preparedness to act in such visually relevant ways, but such preparedness requires vigilance. In looking at the ancient chestnut tree, I maintain vigilance over the tree and its visually manifest aspects. Being thus vigilant, being prepared to act in visually relevant ways, remains a stance that I must actively sustain. So the characteristic activity that sustains the psychological stance may be a first actuality if second potentiality, but the relevant sense of potentiality involves a preparedness to act in a way that itself requires activity to sustain, a kind of perceptual vigilance.

Looking may be a psychological stance, sustained by a characteristic activity, where the perceiver opens themselves up, in a directed manner, to visually experience distal aspects of the natural environment, but that stance is itself an activity. In maintaining perceptual vigilance, I open myself up to the visible. My gaze, that the tree resists insofar as it can, is something I direct at the tree. Looking through a window, or into a fish tank, or across a park is something that the perceiver does. Looking at the tree, gazing upon it, remains something that I do, even if in seeing the tree I undergo an experience caused in me, at least in part, by the tree itself.

Looking, so conceived, may not be a simple, spiritual act of the immaterial soul as Olivi maintains, but its outward, extensive activity remains something that the perceiver does independently of any visible object it may encounter. In opening their eyes, the perceiver opens themselves up to visually experiencing the natural environment, and that is something they do independently of whatever they encounter in so doing. However, accommodating this insight, if it is one, does not require the object of perception to be a terminative cause. In openning themselves up, in a directed manner, to visually experiencing distal aspects of the natural environment, the content of their perception is determined by what they encounter in so looking in a manner inconsistent with the object of perception being a mere terminative cause.

Looking, understood as a psychological stance sustained by characteristic activity, is an outward gaze, a looking into the distance, an outer-directed opening up to the visible. It can sometimes happen, if circumstances are propitious, that in looking outward, aspects of the natural environment, facing us from across the intervening distance, are presented to us in our visual experience. The next Section shall discuss how looking, so conceived, helps make possible the sympathetic presentation of distal objects in the natural environment. If looking, understood as an outer-directed opening up to the visible, makes possible the sympathetic presentation of distal aspects of the natural environment, then looking, so understood, suffices for the truth of the Biranian principle---in order to see well, one must look. A conception of looking that would make true the Biranian principle must at once be something that the perceiver does and that makes the distal environment perceptually accessible. Looking outward is something the perceiver does. And looking outward, in so far as it makes possible the sympathetic presentation of the distal environment in visual experience, makes that environment perceptually accessible.

% Section looking (end)

\section{Sympathy and Visual Presentation} % (fold)
\label{sec:sympathy_and_visual_presentation}

I look where the ancient chestnut tree is and direct my gaze at that tree situated in a straight line from me. My gaze is fixed upon the tree. My gaze reaches it from a distance and is posed on it. The visual awareness afforded me by my perceptual experience is not merely confined to the remote spatial region where the tree is located. I peer through the intervening space, in all its potential steps, and encounter an ancient chestnut tree facing me from across the intervening distance. Being opaque to a significant degree, the tree is a site of visual resistance. The ancient chestnut tree determines a perceptually impenetrable boundary that resists my gaze. In resisting my gaze, the ancient chestnut tree facing me is present in my visual experience. In looking at the ancient chestnut tree in the early evening, my experience assimilates to that tree and that tree shapes my experience of it. In looking, my visual awareness extends to the tree and absorbs it. And it is the resistance that the tree offers to my visual extension that explains, in part, its subsequent absorption and formal assimilation.

In order to see well, one must look. Looking makes aspects of the distal environment perceptually accessible by making possible their sympathetic presentation in visual experience. It is the role that looking plays in making possible the sympathetic presentation of the visible that makes true the Biranian principle.

I turn, and look, and see an ancient chestnut tree. In so doing, I direct my gaze across the park. I look through the illuminated space, a space perceptually penetrated by my gaze, until I can no more. It is the resistance to my looking, my visual encounter with the perceptually impenetrable, that presents opaque objects arrayed in the distal environment. The ancient chestnut tree resists my visual activity. The ancient chestnut tree prevents me from seeing further. I can see nothing in it or through it. However, not all limits to my gaze are external. There are internal limits to how far I may look into the distance. Other perceivers posses the capacity to look further than I can. So how is it possible for an experienced limit to my visual activity to disclose the perceptually impenetrable tree? If the visual presentation of the perceptually impenetrable is due to the operation of sympathy, then we have the basis of an answer. It is only when I experience the tree's limit to my visual activity, its resistance to my gaze, its perceptual impenetrability, as a sympathetic response to a countervailing force, my gaze encountering an alien force that resists it, one force in conflict with another, like it yet distinct from it, that the perceptually impenetrable body discloses itself to visual awareness.

In \emph{De sensu}, Aristotle distinguishes between the limit of the transparent, and the limit of a body. The limit of the transparent is a perceptually impenetrable visual boundary. The limit of a body is its spatial boundaries. These are distinct limits. Whereas the former is qualitative, the latter is quantitative. However, importantly, they can coincide. A bounded body, in being perceptually impenetrable, determines a visual boundary that coincides with the limit of the body. Moreover, Aristotle’s claim that color is the limit of the transparent in a determinately bounded body (\emph{De sensu} 3 439\( ^{b} \)11) gives expression to just this coincidence (or so I argue, \citealt[Chapter 3.3]{Kalderon:2015fr}). Color, that is, surface color, is the limit of the transparent in being the terminal qualitative state in a progression of qualitative states ordered by decreasing perceptual penetrability. A determinately bounded body is one such that, being perceptual impenetrable, determines a visual boundary through which nothing further may appear. This visual boundary is spatially coincident with the limit of the body and is where the body’s surface color is seen to inhere. In experiencing the visual resistance of the colored body as a sympathetic response to a countervailing force that resists the perceiver's gaze, the perceptually impenetrable chromatic body discloses itself in visual awareness.

To get a sense of this, compare David Katz's description of the way that the appearance of surface color contrasts with the appearance of spectral color:
\begin{quote}
	The paper has a surface in which the colour lies. The plane on which the spectral color is extended in space before the observer does not in the same sense possess a surface. One feels that one can penetrate more of less deeply \emph{into} the spectral color, whereas when one looks at the colour of a paper the surface presents a barrier beyond which the eye cannot pass. It is as though the colour of the paper offered resistance to the eye. We have here a phenomenon of visual resistance which in its way contributes to the structure of the perceptual world as something existing in actuality. \citep[8]{Katz:1935qv}
\end{quote}
The phenomenon of visual resistance contributes to the structure of the perceptual world as something existing in actuality. And it does so, or so I claim, by being a necessary precondition for the sympathetic presentation of what resists the perceiver's gaze. Katz's discussion also nicely brings out how, from among the many determinate forms of visual resistance, there is a distinctly chromatic form of visual resistance at work in the contrasting appearances of surface and spectral color.

Despite philosophers' penchant for limiting their visual examples to opaque bodies, such as Moore's \citeyearpar{Moore:1903uo} blue bead or Price's \citeyearpar{Price:1932fk} red tomato, not all \emph{visibilia} are opaque and not all are bodies, as Katz's example of spectral colors illustrates. Can the account of the sympathetic presentation in vision of opaque bodies be extended to, at least, non-opaque things? Is the principle of sympathy operative in the presentation of the visible more generally?

In \emph{De sensu}, Aristotle observes that transparency comes in degrees. By the transparent, Aristotle means what is actually transparent, what is illuminated by the contingent presence and activity of the fiery substance. The transparent offers insufficient visual resistance to determine a perceptually impenetrable boundary. But offering insufficient visual resistance to determine a perceptually impenetrable boundary is consistent with offering visual resistance nonetheless. Something is perfectly transparent if it offers no visual resistance to sight. Something is imperfectly transparent if it offers visual resistance to sight but not sufficient to determine a perceptually impenetrable boundary. From perfect transparency, as we approach the limit of perceptual penetrability, the perceptually impenetrable that determines a visual boundary through which and in which nothing further may be seen, there is a range of states of imperfect transparency ordered by declining degrees of perceptual penetrability. 

The illuminant is a perceptual medium in the way that I claimed sounds to be (Chapter~\ref{sec:sympathy_and_auditory_presentation}). Sounds make the audible activities of distal objects perceptually accessible and are in that sense audible media. We hear the distal source through or in the sound it generates. Similarly, we may see an opaque body through or in the illumination. Whereas physical media answer to the demands of being a causal intermediary, perceptual media answer to the demands of perceptual accessibility. Light does not require physical media in which to propagate in the way that sound waves do. As the Michelson–Morley experiment of 1887 went some way toward showing, there is no Luminiferous aether. But the illuminated air may be a perceptual medium, nonetheless. Moreover, not only are perceptual media themselves perceptible, but they are perceptible in a certain way. Specifically, they are not perceptible in themselves, but owe their perceptibility to other things which are perceptible in themselves, the objects the perceptual media make perceptually accessible. So the illuminant is visible, though not visible in itself, but owes its visibility to the objects that it illuminates. One sees the brightness of a pantry, not in itself, but by seeing the brightly lit objects arranged in it. This is the way in which the perceptually penetrable presents itself to the perceiver's gaze.

The more the perceptually penetrable resists the perceiver's gaze, the more visible in its own right it becomes and so loses, to that degree, the capacity for other things to be perceived in it, or through it. Visual resistance can take many forms. For example, the determinate kind of visual resistance offered by a perceptually penetrable thing, such as a liquid mass, may consist in its possessing a volume color. A volume color pervades the perceptually penetrable mass, and that liquid mass has that color, independently of the colors of the things arrayed in it, or seen through it \citep[though see][]{Mizrahi:2010aa}. If the liquid mass is sufficiently perceptually penetrable, seeing the colors of things arrayed in it may be within the bounds of normal human color constancy. That is, one may see a red bead in a yellow liquid and that bead may be seen to be red, though, of course, looking the way a red thing would when seen through a yellow liquid. The red bead will look to be red, and the same shade of red, when seen through a clear liquid, though, of course, it will look another way. In moving from the yellow liquid to the clear, the red bead's appearance changes but the bead does not appear to change color. There are limits, however, to the normal human color constancy. If the liquid is strongly enough colored, if it offers sufficient visual resistance in that way, this will erode the perceiver's ability to visually recognize the determinate shade of the bead, or even that it is red. Volume color is not the only form of visual resistance offered by otherwise perceptually penetrable media. As Katz observed, spectral color also offers visual resistance. And refractions, reflections, specular highlights, shadows, all contribute, in determinate ways, to the visual resistance of the imperfectly transparent.

The perfectly transparent, insofar as it can be seen at all, is visually presented by the objects seen in it or through it. Its visibility is entirely parasitic on the visibility of the objects it enables. Insofar as the perceptually impenetrable is presented in sight as a sympathetic response to the experienced limit to the perceiver's gaze, and the perfectly transparent medium is thereby presented, the principle of sympathy makes possible the presentation, in vision, of the perfectly transparent. The imperfectly transparent, by contrast, offers visual resistance at least to some degree, but not to a degree sufficient to determine a perceptually impenetrable boundary. To the degree that it manifestly resists perceptual penetration, it is possible to sympathetically present it in visual experience. Think of the way in which the volume color or refraction of an imperfectly transparent medium may present that medium in our visual experience of it. However, the more visible in its own right the imperfectly transparent becomes, the more it erodes the sympathetic presentation of objects arrayed in that medium. The more we hear audible features of the sound had independently of the source that generates it, the less capable we are of hearing that source through or in that sound. The more we see visible features of the illuminated media had independently of the objects that it illuminates, the less capable we are of seeing through it or in it. Illumination may reveal the latent visibility of things, but if it is sufficiently strong, it may blind us to the scene. Perceptual media, in calling attention to themselves, erode the sympathetic presentation of distal objects they otherwise make possible.

We have explained the visual presentation of the perceptually impenetrable in terms of the operation of sympathy. The perceptually impenetrable is presented in sight when the limit to the perceivers gaze is experienced as a sympathetic reaction to a countervailing force that resists that gaze. However, the operation of sympathy is not confined to the presentation, in vision, of the perceptually impenetrable. We see perceptually penetrable things as well. The visual presentation of the perfectly transparent, if that is so much as possible, entirely derives from the sympathetic presentation of objects seen in it. So sympathy would suffice to explain the visual presentation of the perfectly transparent, if it can genuinely be said to be visible at all (whether it can, may depend upon the practical point of so saying in the given circumstances). Sympathy played an additional role in the visual perception of the imperfectly transparent. Insofar as it is perceptually penetrable to some degree, it makes possible the sympathetic presentation of perceptually impenetrable objects seen in it or through it. It is only because the gaze may penetrate to the site of visual resistance, facing it from across a distance, that the perceptually impenetrable is sympathetically presented in visual experience. However, insofar as the imperfectly transparent is visible in its own right, the resistance it offers becomes the means of sympathetically responding to it, and this erodes the sympathetic presentation of distal objects otherwise made possible. 

So we have the following argument by cases. The visible exhaustively divides into the perceptually impenetrable and the perceptually penetrable. The perceptually penetrable is either perfectly perceptually penetrable, offering no visual resistance, or imperfectly penetrable, offering visual resistance to some degree. The operation of sympathy suffices to explain the visual presentation of the perceptually impenetrable. Moreover this explanation suffices, as well, for the visual presentation of the perfectly penetrable, as we have explained. Sympathy explained as well not only the presentation of the imperfectly penetrable insofar as other objects may be sympathetically presented in it, or through it, but also the respects in which it is visible in its own right and the way that this erodes the sympathetic presentation of objects seen in it, or through it. So the operation of sympathy suffices for the presentation of the visible, in sight, quite generally.

We began by explaining the visual presentation of an opaque body in terms of sympathy. Since the objects of sight are not limited to opaque bodies, this raised the question whether sympathy operates in visual presentation quite generally. The following worry might arise about the argument so far: While we have explicitly addressed the visual presentation of non-opaque things, we have failed to explicitly address the visual presentation of non-corporeal things, such as events and processes. However, perceptual impenetrability does not merely pertain to the surfaces of opaque bodies. A flame, should the fire be burning intensely enough, may be perceptually impenetrable, obstructing the view of other \emph{visibilia}. Thus Herbert Mason reports that as he waited to take his iconic photograph of St Paul's on 29 December 1940, ``glares of many fires and sweeping clouds of smoke'' obscured the dome of St Paul's. It is not just the sweeping clouds of smoke, masses of particulate matter, that obscured the dome of St Paul's but the glares of many fires. The general point is that the way in which we have characterized the visible, in terms of degrees of perceptual penetrability, is equally applicable to visible objects of distinct ontological categories. Perceptual penetrability applies equally to corporeal and non-corporeal things and so does not preclude the visual presentation of events and processes.

There may be further doubts about whether the taxonomy of the visible provided by degrees of perceptual penetrability is, in fact, complete. On a clear day, the sky is blue. And the sky, at night, when unobstructed by cloud cover or  light pollution, is black, albeit speckled with points of irradiation that vary chromatically. Is the blue of the day sky, or the black of the night sky, qualities of something perceptually penetrable or perceptually impenetrable? And if we feel uncomfortable answering, doesn't this show that the proposed taxonomy of the visible is incomplete? The puzzlement is resolved, however, once we realize the sense in which each of these responses is at least partly right consistent with one another. And if the puzzlement is resolved in this way, the completeness of the taxonomy of the visible is not thereby challenged.

We have accepted Aristotle's claim that transparency comes in degrees, degrees to which it resists perceptual penetration. Aristotle also claims that the blue appearance of the day sky can be explained in terms of the imperfect transparency of the illuminated air (\emph{De sensu} 3 439\( ^{b} \)1--3). In this way, it is like water. From a cliff overhanging the sea, the sea may appear blue. But, if enticed by the sea, one were to descend to the beach and examine a handful of sea water, it would not be blue at all but transparent. Similarly, looking up at the sky on a clear autumn afternoon, one sees an expanse of blue. But if one were to travel to that region of the sky, by helicopter, say, nothing blue would be found. The visual resistance of an imperfectly transparent medium increases with an increase in volume. The further one sees into a transparent medium, the more resistance that medium offers to sight. In the case of a clear sky, its scattering of light is what offers progressive resistance to our gaze. And its blue appearance is the effect of this resistance. Aristotle is explicit about the effects of such resistance in \emph{Meteorologica}: ``For a weak light shining through a dense medium \ldots\ will cause all kinds of colours to appear, but especially crimson and purple'' (\emph{Meteorologica} \textsc{i} 5 342\( ^{b} \)5--8; Webster in \citealt[8--9]{Barnes:1984uq}).

When I look into the blue of the clear autumn sky, I see as far as I can see. Other people and animals may see further than I do, but the power of sight of all finite creatures is limited in this way. So while the blue appearance of day sky is due to the degree to which it resists perceptual penetration, its scattering of light offering progressive resistance to my gaze, there is a limit to how far I may peer in it or through it. This perceptual limitation is manifest in our experience of the dome of the heavens. In a clear blue sky, in any direction I may look, there is a limit to how far I may see. The finite lines of sight extending in every direction from the perceiver's vantage point determines a sphere. This is what we experience as the dome of the heavens. I confess to recoiling somewhat from the impiety of this expression. The dome of the heavens, construed literally, is not only a reification of a perceptual limitation but is misattributed to the heavens as well. Its impiety consists in giving expression to an anthropomorphic conceit of cosmic proportions.

Our experience of the dome of the heavens is relevant to our initial puzzlement. Recall we wondered whether the blue of the sky inhered in something perceptually penetrable or in something perceptually impenetrable. We are now in a position to see how each response is at least partly right consistent with one another. The blue of the sky inheres in the perceptually penetrable illuminated air, the resistance it offers by the scattering of light resulting in a blue appearance should one peer deeply enough into it. However, there is a limit to how far one may see, and this is reflected in our experience of the blue sky, specifically, in its apparent dome shape. The surface of the dome represents the limits of visibility, and is, to that extent, perceptually impenetrable. But the blue of the sky is not seen to inhere in the dome of the heavens. It is a volume color not a surface color. A blue inhering in the surface of the dome would be a vulgar simulacrum of the voluminous blue of the sky, its appearance more akin to the interior design of a Vegas casino than the clear autumn sky that it apes.

% Similar remarks apply to the color of ganzfelds, induced, say, by covering each eye with half a ping pong ball. The ping pong ball is imperfectly transparent. It is not perceptually impenetrable---one may see light through it. However, it is perceptually penetrable to a very low degree, and the color of the ganzfeld is due the resistance offered, just like the blue of the sea or the blue of the sky, at least when seen from a suitable distance. The indefinite depth we experience a ganzfeld to have is due to its limited perceptual penetrability.

In sympathetically disclosing the ancient chestnut tree, my visual experience absorbs that tree and is constitutively shaped by it. The conscious character of seeing the tree is constituted, in part, by its bright green burs and the wave-like form of its trunk sympathetically presented to my partial perspective on that tree in the given circumstances of perception. What it is like for me to see the tree depends upon and derives from, at least in part, what the tree is like, at least in visible respects. Visual experience formally assimilates to its object, relative to the perceiver's partial perspective, as a consequence of being constitutively shaped by that object as presented to that perspective, a constitutive shaping made possible by the sympathetic presentation of that object in visual experience. Constitutive shaping of visual experience by its object is a ``communion'' with that object---in undergoing that experience the perceiver is united, in a way, with the object of their perception. Moreover, as with Plotinus (Chapter~\ref{sec:sympathy_as_the_principle_of_haptic_presentation}), this unity explains in part, the similarity between the visual experience and its object. The formal assimilation of visual perception to its object, at least relative to the perceiver's partial perspective, is the effect of constitutive shaping, and thus its conscious character depends upon and derives from, at least in part, the visible character of the object seen.

Recall, we are generalizing from Plotinus in taking the unity of visual presentation to be explanatorily prior to the operation of sympathy (Chapter~\ref{sec:sympathy_as_the_principle_of_haptic_presentation}). The visual presentation of distal aspects of the natural environment is not being constructed from elements and principles understood independently of their visual presentation, rather the unity of the perceiver and the distal aspects of the natural environment is presupposed, and sympathy merely analytically explicates the intelligible structure of this presupposed unity. Not only does sympathy only operate within a unity, but that unity is reducible to no other thing.

Visual presentation is an irreducible unity. If sensory presentation is a distinctive kind of unity, a ``communion'' with its object, then visual presentation is more distinctive still. Insofar as visual presentation, like haptic and auditory presentation, is governed by the principle of sympathy, it is a mode of being with. Turning, and looking, and seeing the ancient chestnut tree is a way of being with that tree. Sartre's overly aggressive conception of the look, in \emph{L'Être et le néant}, blinds him to this possibility. Sartre fails to see how the look's coming into conflict with its object may be the means of the latter's sympathetic presentation to the former. (See \citealt[Chapter 5, especially 287]{Jay:1994aa} where he remarks that Heidegger's conception of \emph{mitsein} was, perhaps, too irenic for Sartre.) Like auditory presentation, and unlike haptic presentation, visual presentation is incompletely corporeal. Haptic presentation involves a conscious animate body, the perceiver, being with another corporeal body. It is a way for one body to be with another body. Auditory presentation, by contrast, is incompletely corporeal since it involves a conscious animate body, the perceiver, being with an event or process, even events or processes that do not have bodies as participants. Events and processes may be seen as well as heard and so visual presentation is to that extent incorporeal as well. Visual presentation may, at least in certain circumstances, be a disclosure with duration but the objects disclosed are not essentially dynamic as are the \emph{substrata} of audible qualities. Like haptic perception, vision may disclose relatively static features of the distal environment. But even seeing relatively static features of body, such as their color, may only be disclosed over time. A color is wholly present in a body at every moment of its instantiation. Nevertheless, the unchanging color of a body may only be disclosed in the distinctive manner it interacts with changes to its relations to the perceiver, the illuminant, and the circumstances of perception \citep{Broackes:1997pa,Noe:2004fk,Matthen:2005md}. And that is compatible with, if circumstances are propitious, with the perceiver being able to recognize at a glance the color of a thing.

The unity presupposed by sensory presentation generally, being partial, is a lesser unity than the unity presupposed by intelligible presentation. The intelligibly differentiated image of the hyperontic One is wholly present to the Intellect. An intelligible object is wholly present in the act of intellection in the way that a sensible object never is in perception since sensory presentation is invariably relative to the perceiver's partial perspective. Though a lesser unity, being partial, it is a kind of unity nonetheless. Being the kind of unity it is, a mode of being with whose principle is sympathy, there is a sense in which, sensory presentation, despite its partial character, places the perceiver in the object perceived. As we observed earlier (Chapter~\ref{sec:the_truth_in_extramission}), this is a neo-Platonic heritage. 

In Chapter~\ref{sec:assimilation}, we stopped just short of embracing that heritage. We considered, instead, a related but weaker claim about haptic experience. Beginning with the \emph{prima facie} absurdity of supposing that haptic experience is in the perceiver's head (an absurdity mitigated, somewhat, in a philosophical \emph{milieu} in which ``Cartesianism \emph{cum} Materialism'' is the reigning metaphysical orthodoxy, \citealt{Putnam:1993kx,Putnam:1994kx,Putnam:1999eu}), we claimed, instead, that it is more natural to suppose, at least initially, that haptic experience is closer to where its object is at, in our handling of that object. The Plotinian claim, if made on behalf of haptic presentation, is stronger still. It would be the claim that haptic experience places us within the object of haptic experience. In grasping or enclosure, the haptic experience is in the perceived overall shape and volume of the object that the perceiver is handling. The earlier, weaker claim hedged at the boundary between the apparent body, the region wherein bodily sensation is potentially felt \citep{Martin:1992aa}, and extrapersonal space. However, if haptic perception involves a mode of sympathetic presentation, then the haptic variant of the Plotinian claim, that haptic perception places us in the object of haptic investigation, must be true, at least on a certain interpretation of that claim. 

The next chapter will explore whether good sense can be made of this neo-Platonic heritage. I shall argue that the neo-Platonic heritage is best understood as articulating an aspect of the phenomenology of explicit awareness made possible by sympathetic presentation. The overall aim of the next chapter is to explicate the conception of perceptual objectivity that sympathetic presentation affords us. It will turn out that this conception of objectivity is the basis of a strong form of perceptual realism, a form of realism on which the distinction between the phenomenal and the noumenal collapses. Things in themselves are perceptible, albeit partially and imperfectly. That perception, via the operation of sympathy, places us into the very heart of things, explains how this may be so.


% Section sympathy_and_visual_presentation (end)



% Chapter vision (end)
