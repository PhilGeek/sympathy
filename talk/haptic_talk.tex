%!TEX TS-program = xelatex 
%!TEX TS-options = -synctex=1 -output-driver="xdvipdfmx -q -E"
%!TEX encoding = UTF-8 Unicode
%
%  haptic_talk
%
%  Created by Mark Eli Kalderon on 2014-12-16.
%  Copyright (c) 2014. All rights reserved.
%

\documentclass[12pt]{article} 

% Definitions
\newcommand\mykeywords{haptic, perception}
\newcommand\myauthor{Mark Eli Kalderon}

% Packages
\usepackage{geometry} \geometry{a4paper} 
\usepackage{url}
% \usepackage{txfonts}
\usepackage{color}
% \usepackage{enumerate}
\definecolor{gray}{rgb}{0.459,0.438,0.471}
\usepackage{setspace}
% \doublespace % Uncomment for doublespacing if necessary
% \usepackage{epigraph} % optional

% XeTeX
\usepackage[cm-default]{fontspec}
\usepackage{xltxtra,xunicode}
\defaultfontfeatures{Scale=MatchLowercase,Mapping=tex-text}
\setmainfont{Hoefler Text}

% Bibliography
% \usepackage[round]{natbib}

% Title Information
\title{Haptic Presence}
\author{\myauthor} 
\date{} % Leave blank for no date, comment out for most recent date

% PDF Stuff
\usepackage[plainpages=false, pdfpagelabels, bookmarksnumbered, backref, pdftitle={Form Without Matter}, pagebackref, pdfauthor={\myauthor}, pdfkeywords={\mykeywords}, xetex, colorlinks=true, citecolor=gray, linkcolor=gray, urlcolor=gray]{hyperref} 

%%% BEGIN DOCUMENT
\begin{document}

% Title Page
\maketitle

% Layout Settings
\setlength{\parindent}{1em}

% Main Content

In a justly famous scene from \emph{2001: A Space Odyssey}, set to Richard Strauss' \emph{Also Sprach Zarathustra}, a hominid ancestor, squatting among the skeletal remains of a boar, reaches out and tentatively grasps a femur. It is telling that this is how Stanley Kubrick chose to dramatize the initial transformation, induced by an alien obelisk, of our hominid ancestors, that eventually gives rise to space-exploring humanity in the twenty-first century. Not only does our hominid ancestor grasp the femur, but they grasp as well an important application. Squatting among the skeletal remains, femur in hand, our hominid ancestor taps the bones in exploratory manner. Each strike of the femur grows in force until finally, in a crescendo of activity, they smash the boar's skull to pieces. Our hominid ancestor has reached a crucial insight, that an implement, such as the femur, might transform boar into prey. Moreover, the application generalizes. The femur might also be used as a weapon against competing groups of hominids. The acquired technology thus has political consequences. What is presently important, however, is the connection between grasping and cognition. Kubrick dramatizes the connection between grasping and cognition by having our hominid ancestor's grasping the femur among the boar's skeletal remains be the primal scene of a dawning understanding.

We have \emph{grasped} a situation when we have understood it. We have a \emph{grip} on it. If the understanding in question is practical, we might say that we have matters \emph{in hand}. And we \emph{touch upon} subjects of discussion. Nor are tactile metaphors confined to forms of higher cognition and their expression in rational discourse. They persist, as well, in our description of perceptual awareness. Not only do we speak of recognizing an object that we see as \emph{grasping} the object present in our perceptual experience, but the presentation in experience is itself a kind of grasping. Perception puts us in \emph{contact} with its object. In perceiving an object we \emph{apprehend} it. The tactile metaphors for perceptual awareness tend to be modes of assimilation, and \emph{ingestion} is a natural variant. Our hominid ancestor, looking up from the boar's remains, \emph{takes in} the scene before them. Indeed this metaphor is inscribed into the history of the English language---``perception'' derives from the Latin \emph{perceptio} meaning to \emph{take in} or \emph{assimilate}. If in looking up from the boar's remains, they see the obelisk, then, in a manner of speaking common among contemporary philosophers, the obelisk is the \emph{content} of our hominid ancestor's perception. But if the obelisk is the content of their perception then their perception of it is its \emph{container}. To bring something into view so that it figures in the content of perception would be to contain it within that perception. But containment itself is a mode of assimilation. 

What makes tactile metaphors for perception apt? Tactile metaphors for perceptual awareness, even for non-tactile modes of awareness such as vision and audition, are primordial and persistent. Most contemporary philosophers of perception apply them unselfconsciously, indeed, unconsciously. That they do is a testament to the power of such metaphors. Understanding the power they have over us, understanding what makes them so compelling, we may gain insight into the object of these metaphors. In understanding what makes grasping an apt metaphor for perception generally, if it is indeed one, we may gain insight into the nature of sensory presentation.

Grasping may be an exemplar of sensory presentation, but it does not follow that all perception is a form of touch. One may grant that tactile metaphors for perceptual awareness are in some sense apt while eschewing any such reductive explanatory ambition. Such ambitions were rife in Greek antiquity. Thus Lindberg observes that in the ancient world ``the analogy of perception by contact in the sense of touch seemed to establish to nearly everybody’s satisfaction that contact was tantamount to sensation, and it was not apparent that further explanation was required.'' Conceiving of non-tactile modes of perceptual awareness on the model of touch will only seem explanatory insofar as touch is antecedently understood to be an unproblematic mode of perception. However, Aristotle's belaboring and not always completely resolving the \emph{aporiai} concerning touch in \emph{De Anima} 2 11 undermines that assumption. And if further explanation is required, then we can no longer simply assume that contact is tantamount to sensation. Nevertheless, Aristotle accepts the aptness of the metaphor. Aristotle defines perception as the assimilation of sensible form without the matter of the perceived particular. So grasping may be apt metaphor for perception without perception being reduced to a form of touch.

Grasping is a form of haptic touch. Haptic touch involves active exploration of the tangible object. This can involve a range of different stereotypical exploratory activities often combined in sequence. The different stereotypical exploratory activities are suited to presenting different ranges of tangible qualities. Holding a stone in its hand, our hominid ancestor may feel the roughness of the stone by rubbing their thumb across its surface. And its hardness may be felt by applying pressure to it. According to the taxonomy of Susan Lederman and Roberta Klatzky, grasping is a distinctive exploratory activity that they describe as ``enclosure''. Grasping an object allows the perceiver to discern a different range of tangible qualities. If texture is perceived by lateral motion and hardness by applying pressure, grasping or enclose makes volume and global shape available in tactile experience. Other stereotypical exploratory activities include: ``static contact''---passively resting one's hand on an externally supported object, without an effort to mold to its contours, to determine its temperature, ``unsupported holding''---holding the object without external support, and without molding, to determine the object's heft or weight often involving a ``weighing'' motion, ``contour following''---a smooth, nonrepetative tracing of the contours of the object, ``part motion test''---moving a part of the object independently of the whole, and ``specific function test''---moving the object in such a way as to perform various functions. Though these stereotypical exploratory activities are optimized for determining a specific range of tangible qualities, they can also determine other tangible qualities, though perhaps less well, with less tactual acuity. Thus while grasping may present the overall shape of the object, to determine its exact shape the perceiver must use contour following. Grasping, however, like contour following, is relatively general in the range of tangible qualities it can present. Not only are these stereotypical exploratory activities optimized to determine a specific range of tangible qualities that vary in generality, but they can also be chained together to provide the perceiver with a more complete profile of the corporeal aspects of the object under investigation.

With enclosure, Lederman and Klatzky write:
\begin{quote}
	the hand maintains simultaneous contact with as much of the envelope of the object as possible. Often one can see an effort to mold the hand more precisely to object contours. Periods of static enclosure may alternate with shifts of the object in the hand(s).
\end{quote}
The quoted passage brings out several important features of grasping, understood as a mode of haptic perception. 

First, grasping a rigid, solid body involves the hand's maintaining simultaneous contact with as much of its overall surface as possible. Grasping is thus a kind of incorporation. Recall, what unites the various tactile metaphors for perception is that they tend to be modes of assimilation, and grasping exemplifies this pattern. It may not be as complete an incorporation as the variant, ingestion, but it remains a clear mode of assimilation nonetheless. As we shall see, that the grasping hand assimilates to the object grasped is a manifestation of the objectivity of that haptic perception. 

Second, not only does the grasping hand assimilate to the overall shape and volume of the object grasped, but, as Lederman and Klatzky observe, effort is typically exerted to mold the hand more precisely to the object's contours. So grasping or enclosure involves not only the hand's configuration in maintaining simultaneous contact with the overall surface of the object, but the force of the hand's activity as well. Not only is this force exerted in achieving the end of molding the hand more perfectly to contours of the object grasped, but it is exerted as well in the end's achievement---maintaining simultaneous contact with the overall surface of the object requires continued effort to sustain. This is physiologically and phenomenologically significant. It is physiologically significant in that the activation of different sets of receptors are coordinated in haptic perception. Grasping or enclosure will involve not only cutaneous activation but also the distinct sets of activations involved in kinesthesis, motor control, and our sense of agency. Moreover, this is reflected in our phenomenology. We feel the force with which we grip the object as well as the object's overall shape and volume.

Third, there is tendency, in grasping or enclosure, to shift the object periodically in one's hands. What explains this? Begin with Lederman's and Klatzky's observation that there is a tendency for perceivers to exert effort to mold their hand more precisely to the contours of the object grasped. Consider grasping a solid, rigid body, such as a stone. Since the stone is solid, it resists penetration. Since it is rigid, it maintains its overall shape and volume even when in the hominid's grasp. Of course, hands are unevenly shaped and imperfectly elastic. This means that an effort to mold one's hand to a rigid body thus disclosing its overall shape and volume will most likely be imperfectly realized. There may be some areas of the object's surface that the grasping hand does not conform to. The tendency to shift the grasped object in our hands compensates for this partial and imperfect disclosure. In shifting the object in one's hand, an area that the hand did not previously conform to may become accessible to touch. Successive grips and the manner in which the object moves in one's hands as one shifts between them may provide a better overall sense of the shape and volume of the rigid body. 

Allow me to make two further observations about this passage, though now about issues that are merely implicit.

First, in periodically shifting the object in their hands to compensate for the partial and imperfect disclosure of the object grasped, the perceiver's haptic experience exhibits perceptual constancy. What the perceiver feels in moving the object between successive grips changes throughout this process, but the object disclosed by this haptic exploration is not Protean in character. If the object were changing its overall shape and volume in the process of the perceiver's handling it, then shifting the object could be no compensation for the partial and imperfect disclosure of the object grasped. If the object were Protean, and the perceiver shifted it in their hands, then its overall shape an volume would change, and the opportunity to feel what was unfelt would be forever lost. In grasping, understood as a mode of haptic perception, the perceiver attends only to the constant tangible qualities it presents, in the case of a rigid, solid body, the perceiver attends to its constant overall shape and volume. Though there may be a felt difference in changing patterns of intensive sensation in handling the object, haptic experience presents the constant overall shape and volume of the object. Of course, different aspects of the overall shape and volume may be present at different times, given the different ways the body is being handled. But these presented aspects of the overall shape of a rigid, solid body are experienced as stable aspects of a body that retains its shape, despite the perceiver's handling, because of the self-maintaining forces at work in its constitution.

Second, that the grasping hand assimilates to the overall shape and volume of the object grasped is potentially epistemically significant. A rigid, solid body has a certain overall shape and volume prior to being grasped. Moreover, it is sufficiently rigid and solid to maintain that overall shape and volume even when grasped. In making an effort to more precisely mold the hand to the contours of the rigid object, the hand thus takes on a certain configuration determined by the hand's anatomy, the activity of the hand, and the overall shape of the object grasped. Moreover, the hand, so configured, encompasses a region of a certain volume itself determined by the hand and the volume of the object grasped. And the volume of the region that the hand encompasses approximates the volume of the object grasped. That is the point of making an effort to more precisely mold the hand to contours of the rigid object. In engaging in such haptic activity, in molding one's hand more precisely to the contours of the object, the overall shape and volume of the object had prior to being grasped, and maintained in being grasped, explains, in part, the hand's configuration in grasping the object and the force that needs to be exerted to maintain that configuration. If the object is absent, there is nothing for the hand to assimilate to. Perhaps the objectivity of grasping, understood as a mode of haptic perception, consists in the grasping hand's assimilating to the tangible qualities of the object had prior to grasping.

In the \emph{Theaetetus} 156 a--c, Socrates elaborates the Secret Doctrine of Protagoras by providing an account of perception as the contingent outcome of active and passive forces in conflict. Grasping as a mode of haptic perception can seem to approximate to that account. At the very least, the felt shape and volume of the object grasped is determined by conflicting forces. On the one hand, there is the force exerted in molding the hand more precisely to the contours of the rigid body. On the other hand, there are the self-maintaining forces of the rigid body itself. A rigid, solid body, such as a stone picked up by a hominid ancestor, is no mere sum of matter. It has a form or material structure determined by forces that are the categorical bases for its rigidity and solidity. Haptic perception is the joint upshot of the force exerted by the grasping hand and the self-maintaining forces of the object grasped. There remains a crucial difference, however, from the account elaborated by Socrates. The overall shape and volume of the object and our haptic perception of them are not ``twin births'' as Protagoras maintains. The forces that determine the object's rigidity and solidity are sufficient to maintain the object's overall shape and volume within the hand's grasp. So the perceived tangible qualities of the external body inhere in that body prior to being perceived, whereas in the account attributed to Protagoras, the perceived object comes into being with the perceiver's perception of it. At least with respect to grasping or enclosure, understood as a mode of haptic perception, this perceptual realism is sustained by the force of the hand's activity in conflict with the self-maintaining forces of the object grasped. 

The hand's assimilating to the overall shape and volume of the object grasped is a manifestation, if not the source, of that haptic perception's objectivity. This, I suggested, is part of what makes grasping or enclosure an apt metaphor for sensory presentation more generally. It is important to get clearer about what this assimilation amounts to, and how it may be generalized, if assimilation is genuinely part of what makes grasping an apt metaphor for sensory presentation.

Grasping is, like the variant meta\-phor, ingestion, a kind of incorporation. This can suggest that the mode of assimilation is material---that it is a taking in, or incorporation, of a material body. But while some forms of sensory perception involve material assimilation such as tasting, not all do. Vision and audition involve the material assimilation of no thing. So if the assimilation at work in grasping or enclosure is part of what makes it an apt metaphor for sensory presentation generally, it must be understood in some other way.

Perhaps, the assimilation at work in grasping or enclosure is not merely material but formal. In grasping or enclosure, the hand assimilates to the contours of the object grasped. The shape of the interior of the hand is similar to the overall shape of the object, and the volume of the region it encloses is similar to the volume of that object. Perhaps, in this way, the hand assimilates the tangible form of the object grasped, by becoming similar to it. However, while our hand may be warmed when feeling the warmth of an object, our eyes do not become red when viewing a traditional English phone booth (though such a view has been attributed, incorrectly to my mind, to Aristotle). So it can seem that formal assimilation is no better off than material assimilation in this regard.

However, this latter problem for assimilation understood formally, if not materially, may be avoided by means of a small generalization. In grasping an object, where is the overall shape and volume that you feel? If grasping is a mode of haptic perception, then surely they are in the object that you grasp. Now, where is your haptic experience of that object? In your head? That answer seems so implausible on its face that only a philosopher could believe it. If anywhere, it seems more reasonable to suppose, at least initially, that it is closer to where the overall shape and volume are felt, in your handling of the object. Perhaps in trying to come to an understanding of formal assimilation at work in grasping or enclosure that may be generalized to other sensory modalities, we focussed too closely on the shape of the interior of the hand and the volume it encloses. If our haptic experience is where we handle the object grasped, perhaps the similarity obtains not only between the hand and certain tangible qualities of the object, but between the haptic experience that the hand's activity gives rise to and the tangible qualities presented in it. And, arguably at least, this feature is generalizable to other sensory modalities as well.

Earlier we noted that haptic perception, like perception generally, is partial. The partial character of grasping, understood as a mode of haptic perception, explained the tendency, observed by Lederman and Klatzky, for the perceiver to shift the object of haptic exploration periodically in their hands. Such behavior compensates for the partial and imperfect disclosure of the overall shape and volume of the object grasped. Successive grips and the manner in which the object moves in one's hands provide a more complete profile of the corporeal aspects of the object under investigation. If the successive grips disclose different aspects of the object's overall shape and volume, then they provide something like different haptic perspectives on the object grasped. 

While talk of ``perspective'' derives from the case of vision, a clear analogue of that notion finds application in the haptic case. To the extent that it does, then talk of ``haptic perspective'', while in a sense visuocentric, is not pejoratively so. Suppose the rigid, solid body is irregularly shaped, then it potentially feels different in successive grips. And in the case of contour following, different paths may be followed, and at different rates, giving rise to different progressions of intensive sensation, themselves constituting different haptic perspectives on the constant contour of the object of haptic investigation. And we may pinch, squeeze, and pull on the object of haptic investigation and these distinct activities provide us with distinct haptic perspectives on that object. 

This perspectival relativity bears on our understanding of the formal assimilation at work in grasping understood as a mode of haptic perception. In haptic perception, the tangible qualities of the object are presented to the perceiver's haptic perspective on that object---the distinctive way they are handling that object in the given circumstances---and this is reflected in the conscious character of their haptic experience. So with respect to grasping or enclosure understood as a mode of haptic perception, the doctrine of formal assimilation should be understood as the claim that the phenomenological character of haptic experience formally assimilates to the tangible qualities presented to the perceiver's haptic perspective. 

It might be objected that haptic experience formally assimilating to the tangible qualities presented in it is absurd on its face. Perhaps in grasping a cube, my hand will approximate to a cube shape, but is it really the case that my experience is cube shaped? The claim that in seeing an English phone booth my visual experience becomes red seems even worse than the view literalists attribute to Aristotle, that in seeing the phone booth my eye becomes red. What does it even mean for an experience to be cubical or red? It is important in this regard to recognize that the posited similarity need not be exact. It is only on that assumption that formal assimilation involves the sharing of qualities. But if we abandon that assumption, then there is a clear sense in which, in seeing the phone booth, the qualitative character of my color experience depends upon and derives from the qualitative character of the color presented in that experience. And similarly we might say that in haptic perception, the conscious character of haptic experience depends upon and derives from the tangible qualities present in that experience. 

Earlier I claimed that the partiality of haptic perception only explained the tendency, observed by Lederman and Klatzky, for the perceiver to periodically shift the object in their hands if the haptic experience this behavior gives rise to exhibits perceptual constancy. One of the philosophical challenges posed by perceptual constancy is to adequately describe and explain the phenomenology of stability and flux. In cases of perceptual constancy, a constant unaltered object of perception is presented though its appearance varies. In explaining perceptual constancy, it is not enough to determine the constant object of perception. That object continues to present itself unchanged even though its appearance may change with a change in the perceiver's perspective or circumstances of perception. In determining only the constant object of perception, one explains the phenomenology of stability at the expense of the contribution to our phenomenology of flux. Even if, in the case of grasping or enclosure, understood as a mode of haptic perception, we attend only to the constant overall shape and volume of the object grasped, these feel differently in different successive grips. Accommodating the contribution of flux to our phenomenology of grasping or enclosure requires acknowledging that haptic presentation, like sensory presentation more generally, is perspective relative.

We now turn to another important distinction. Consider Lederman's and Klatzky's claim that that grasping or enclosure involves molding one's hand to the contours of the object grasped. Molding is a kind of shaping. And there are causal and constitutive senses of shaping that can be distinguished. So consider the way that the Nazi air campaign shaped the London skyline. The destructive impact of the bombing caused the London skyline to be shaped in a certain way. This contrasts sharply with the way that St Paul's shapes the London skyline, as Herbert Mason's iconic photograph dramatically demonstrates. St Paul's defiantly shapes the London skyline by being part of it despite the devastating impact of the bombing campaign. Whereas Nazi bombing shaped the London skyline in a merely causal sense, St Paul's constitutively shapes that skyline by being a part or contour of it.

The causal--constitutive distinction plays out, I believe, in the use that Aristotle makes of Plato's wax analogy from the \emph{Theaetetus}. Plato, in the \emph{Theaetetus}, appeals to an impression made on wax as an analogy for the operation of memory in the context of explaining how error in judgment is possible. In \emph{De Anima}, Aristotle uses the wax analogy, not for memory and knowledge as Plato does, but for explaining his definition of perception as the assimilation of the sensible form without the matter of the perceived particular. There is a further, and for present purposes, more important way in which Aristotle departs from Plato's use of the wax analogy. There is a sense in which Aristotle takes seriously, in a way that Plato does not, the distinctive discursive role of signet rings as opposed to a stylus, say. Moreover, this makes a difference to how the shaping of the wax by the ring is to be understood. Plato’s explanation of the reliability of memory crucially relies on causal features of the situation. An object’s impression is the effect it has on the mind’s wax. But whereas Plato has in mind a causal notion of shaping, Aristotle has in mind the constitutive notion. 

If we reflect on the distinctive discursive role of a signet ring over a stylus, say, this can motivate the constitutive understanding of shaping. Notice that the impression of a signet ring plays a similar role to a signature. Just as a signature is linked to the particular person whose signature it is, the impression sealed upon the wax by a signet ring is linked to the legitimate possessor of that ring. Of course, signatures can be forged, as can signet rings, which can also be stolen, but these practices gain there point precisely by the link between a signature and sealed impression, on the one hand, and their legitimate possessors, on the other. Signet rings and styli thus have distinctive discursive roles. The impression made by a stylus is not linked to its legitimate possessor the way an impression sealed by a signet ring is.

Taking this feature of the analogy seriously has an important consequence for how sensory impressions are individuated. Just as a forged signature is not my signature, an impression sealed by a forged ring, or by a stolen ring, is not the seal of the ring’s legitimate possessor. If this feature of the analogy carries over, then perceptions, conceived on the model of sealed impressions, are individuated by their objects which are their source. A perception of Castor and a perception of Pollux are different perceptions, no matter how closely the twins may resemble one another. Just as a forged seal is not my seal, a perception of Castor is not a perception of Pollux. A forged seal may be a perfect duplicate of a genuine seal but it is not the seal of the ring’s legitimate possessor. Castor may be a perfect duplicate of Pollux, but my tactile impression of Castor is not an impression of Pollux.

What taking seriously the distinctive discursive role of the signet ring in the wax analogy brings out is that the formal assimilation at work in haptic perception and, arguably at least, in perception more generally, might be understood, not on the model of causal shaping, but rather on the model of constitutive shaping. If sensory impressions are individuated by their objects, perhaps these objects shape sensory consciousness not causally, or at least not merely. Perhaps in being individuated by their objects, these objects constitutively shape our sensory impressions of them. On the causal model, a haptic experience is a sensory impression caused in a perceiver with an appropriate sensibility by the object of haptic investigation. Moreover, if the causal structure of the world cooperates, then the conscious qualitative character of the haptic experience may be like, if not exactly like, the qualitative character of the tangible object. (Locke thinks something like this about primary quality perception.) On the constitutive model, haptic experience formally assimilates to its tangible object as well. However, that object does not merely cause the perceiver to undergo a haptic experience with a certain conscious qualitative character. Rather, corporeal aspects of the object constitutively shape the perceiver's haptic experience of it. If something feels metallic, and this is a case of tactile perception, then not only is this because of its metallic feel, but something's feeling metallic is also constituted, in part, by that metallic feel. The metallic feel of the thing is felt in it and in conformity with it. That is just what it is for something to be present in tactile experience.

If this feature of grasping  generalizes to other modes of perception, then it is easy to see its epistemic significance. One can only perceptually assimilate what is there to be assimilated. If perceptual experience is a formal mode of assimilation understood on the model of constitutive shaping, then one could not undergo such an experience consistent with a Cartesian demon eliminating the object of that experience. If there is no external object, then there is nothing to which the perceiver, or perhaps their experience, can assimilate to. If the phenomenological character of perception is constitutively shaped by the object presented to the perceiver's partial perspective, then it is the grounds for an epistemic warrant for the range of propositions whose truth turns on what is presented in that perceptual experience.

Robert Kilwardby provides a vitalist twist on the Peripatetic analogy that potentially sheds light on the epistemic significance of the force of the hand's activity:
\begin{quote}
	You will have some kind of simile for understanding this if you assume that there is a seal in front of the wax so that it touches it and that the wax has a life by which it turns itself towards the seal, and by pressing itself against it, makes itself like it. (Kilwardby, \emph{De Spiritu Fantastico} 103)
\end{quote}
Kilwardby transform's the Peripatetic analogy by imagining life to inhere in the wax so that it is actively pressing against the seal and so taking its sensible form upon itself.

Kilwardby's account is motivated, in no small part, by his conviction, grounded in his reading of Augustine, that the soul cannot be acted upon by the body (\emph{De Spiritu Fantastico} 47--54). Kilwardby tentatively accepts a Peripatetic model where, in vision, say, the perceived object acts upon the transparent medium such that its image exists, in some sense, in it, and that the medium, in turn, affects the sense organ such that the image comes to, in some sense, exist in it as well (\emph{De Spiritu Fantastico} 69, 97). But how does the sensory soul receive the image that informs the sense organ, if the sense organ is precluded, by its corporeal nature, from acting upon the soul? The vitalist twist on the Peripatetic analogy is meant to address this problem. The sensory soul pervades the sense organ, and animates it, and in so doing makes itself like the external body. So it is the sensory soul that is the efficient cause of the likeness of the body occurring in it. The sensory soul makes itself like the external body by pressing against the sense organ that it animates itself impressed with the image of the object.

What does the metaphor of the sensory soul pressing against the impressed sense organ mean? According to Kilwardby, the soul's use of a body is limited by the passivities of matter (\emph{De Spiritu Fantastico} 99--100). So a feather striking a boar's skull will not break it, but a femur will, even if it is the same hominid striking the skull with equivalent force in each instance. The difference is due to the way in which the activity of the agent is limited by the passivities of matter inhering in the body that is being used. A sensible species inhering in a sense organ is among the passivities of matter exhibited by that corporeal body. And Kilwardby explains the soul's assimilation of sensible form of the perceived object in terms of how the sensible species inhering in the sense organ limits the sensory soul's use of it.

It is not clear whether the subsequent account constitutes a genuine reconciliation of Augustinian and Peripatetic metaphysics. Regardless of Kilwardby's intent, however, and dropping his Augustinian dualism, the hand, the mobile and elastic instrument of haptic exploration, is the active wax in grasping or enclosure. It is the hand that is actively molding itself to the object in grasping or enclosure. And it is the hand that is thereby taking upon itself a configuration and enclosing a certain volume determined by the overall shape and volume of the object grasped. In making an effort to mold more precisely to the contours of the rigid, solid body, not only does the hand assimilate to the contours of the object grasped, but the perceiver's haptic experience---there where the perceiver is handling the object---assimilates to the overall shape and volume of the object presented in it. Further, I take it that it is at least part of Kikwardby's suggestion that it is the activity of the wax and the resistance it encounters in pressing against the seal that discloses the shape of the seal had prior to perception. So if the hand is the active wax in grasping, understood as a mode of haptic perception, then it is the force of the hand's activity and the resistance it encounters in maintaining simultaneous contact with a non-insignificant portion of the object's overall surface that discloses the tangible qualities of the object had prior to that haptic encounter. Kilwardby's suggestion, then,---if released from the confines of Augustinian metaphysics, if, in turn, narrowly confined to haptic presentation---is that the presentation of tangible qualities of objects external to the perceiver's body is due, at least in part, to the activity of the hand in grasping and the felt resistance it encounters. The hand, and haptic experience in turn, only assimilate to the tangible aspects of the rigid, solid body thanks to the force of the hand's activity in conflict with the self-maintaining forces that constitute the categorical bases of that body's solidity and rigidity.

In discussing the objectivity of grasping, we supposed that it is our hand's configuration in grasping and the force that needs to be exerted in maintaining that configuration that discloses the overall shape and volume of the object grasped. I believe this supposition to be both plausible and true, but once it is clearly stated, a puzzle immediately arises. 

An animal's awareness of its body is a mode of self-presentation. There may be more to an animal than is revealed in bodily awareness, but bodily awareness nevertheless presents corporeal aspects of the animal whose awareness it is. Bodily awareness remains a mode of self-presentation even if its disclosure of the animal whose awareness it is is partial in this way. Let bodily awareness be understood broadly enough to comprise both proprioception and kinesthesis and potentially more besides. So bodily awareness affords the perceiver with, among other things, awareness of the configuration of their limbs as well as awareness of their motion. So understood, awareness of the hand's configuration in grasping and awareness of the force that needs to be exerted in maintaining that grasp are both modes of bodily awareness. And since bodily awareness is a kind of self-presentation, so are awareness of the hand's configuration and awareness of the force exerted in maintaining it. 

Our puzzle now is this. How can a mode of self-presentation disclose the presence of some other thing? After all, perceivers, in being aware of their body, in presenting only themselves, present no other thing. What alchemy transmutes bodily sensation into tactile perception? The puzzle is not meant to underwrite skepticism about haptic perception or tactile perception more generally.so much as to underwrite a ``how-possible'' question. How is it that the configuration of the hand and the force exerted in maintaining that configuration disclose the overall shape and volume of the object grasped?

There is an aspect of grasping or enclosure that has so far remained implicit in our discussion of the puzzle but is crucial for refining our how-possible question in such a way as to point toward an adequate solution. The perceiver, in exerting effort in more precisely molding their hand to the contours of the object grasped, encounters felt resistance to their efforts. Maybe it is the hand's encounter with felt resistance---the activity of the wax limited by the passivities of matter---that discloses the tangible qualities of an external body. The suggestion, here, is not merely that the puzzle overlooked the contribution of cutaneous activation to haptic awareness, but rather with how cutaneous activation interacts with kinesthesis and bodily awareness more generally in giving rise to the experience of an external limit to the body's activity.

There is a long history connecting objectivity to felt resistance to touch. In the \emph{Sophist}, Plato recasts the Gigantomachy, the struggle for supremacy over the cosmos between the Olympian Gods and the Giants, as a metaphysical dispute. The Gods, or Friends of the Forms, insist that only imperceptible Forms are most real. Against them, the Giants, the offspring of Gaia, insist that only material bodies exist:
\begin{quote}
	One party is trying to drag everything down to earth out of heaven and the unseen, literally grasping rocks and trees in their hands, for they lay hold upon every stock and stone and strenuously affirm that real existence belongs only to that which can be handled and offers resistance to the touch. (Plato, \emph{Sophist} 246a)
\end{quote}
For the Giants, felt resistance to touch has become a touchstone for reality. Only that which can be handled and offers resistance to touch is real. Even if one rejects the materialist metaphysics of the Giants, one can accept that the experience that grounds their materialist conviction is phenomenologically compelling. Grasping something which offers resistance to touch is a phenomenologically vivid and primitively compelling experience of what is external to us. 

The phenomenologically vivid and primitively compelling experience of felt resistance to touch will underwrite the dramatic episode involving Dr Johnson outside of a church in Harwich:
\begin{quote}
	After we came out of the church, we stood talking for some time together of Bishop Berkeley’s ingenious sophistry to prove the non-existence of matter, and that every thing in the universe is merely ideal. I observed, that though we are satisfied his doctrine is not true, it is impossible to refute it. I never shall forget the alacrity with which Johnson answered, striking his foot with mighty force against a large stone, ’till he rebounded from it, ``I refute it thus.'' This was a stout exemplification of the first truths of Pere Buffier, or the original principles of Reid and Beattie; without admitting which, we can no more argue in metaphysicks, than we can argue in mathematicks without axioms. To me it is inconceivable how Berkeley can be answered by pure reasoning \ldots\ 
\end{quote}
The reality of external matter was demonstrated in the resistance it offered to Dr Johnson’s foot, which rebounded despite its mighty force. It was a demonstration not in the sense of proof, since it is inconceivable how Berkeley can be answered in pure reasoning. Moreover, what was stoutly exemplified was metaphysically axiomatic, but proof proceeds from axioms, it does not establish them. Rather Dr Johnson’s performance was a demonstration of first truths by showing or exhibiting them. Dr Johnson's demonstration, like the Giants' before him, draws its dramatic power from the phenomenologically vivid and primitively compelling experience of felt resistance to touch.

How does felt resistance to touch disclose tangible qualities inhering in external bodies prior to perception? After all, not all passivities of matter that limit the hand's activity are external to the perceiver's body. There are internal limitations to the body's activity as well. We encounter an internal limitation to the body's activity due to fatigue or in an inability to touch one's toes. So not every experience of a limitation to the body's activity is due to the tangible qualities inhering in an external body prior to perception. So how is it that in grasping, or enclosure, the limitation to the hand's activity in molding more precisely to the contours of the object grasped and the consequent felt resistance to touch disclose that object's overall shape and volume? 

This, then, is the refined version of our how-possible question: How is it possible for felt resistance to the hand's activity in grasping disclose a rigid body's overall shape and volume? 

Since our puzzle begins with the dependence of haptic perception upon bodily awareness, perhaps getting clearer on the nature of that dependence will help with its solution. In a chapter devoted to discussing the nature of this dependence, Fulkerson draws the distinction between implicit and explicit experiences:
\begin{quote}
	An implicit bodily experience is one that is the background or recessive. ``Background'' here can be understood as an experiential content that is not consciously attended, in the minimal sense that it does not allow its objects to be open for epistemic appraisal. 
\end{quote}
Explicit experiences, in contrast, involve attending to, or actively thinking about, the object of that experience. 

With the distinction drawn, we may ask whether haptic perception depends upon an explicit bodily experience of the hand's configuration and force, or whether the presentation of the object's overall shape and volume in haptic experience merely depends upon an implicit experience of the hand's configuration and force? If the bodily experience upon which haptic perception depends is explicit, then the perceiver consciously attends to the state and activity of the body, and haptic perception of the tangible qualities of an external body depends upon this explicit bodily experience. Fulkerson calls this Strong Experiential Dependence. On the hypothesis of Strong Experiential Dependence, the haptic perception involved in grasping or enclosure, a conscious experience, depends upon another conscious experience, specifically, of the hand's configuration and force.

Fulkerson argues, instead, that the dependence is best understood in terms of what he calls Informational Bodily Dependence. Though information from processes that underly proprioception and kinesthesis are integrated with afferent information, such as the information provided by cutaneous activation, these give rise to a single conscious experience. The idea is that the sensitivity exhibited by haptic perception, such as grasping or enclosure, depends upon the tactile system drawing upon functionally distinct streams of information involved in bodily awareness. Nevertheless, the percept that is thereby determined is a single conscious experience, in the case of grasping or enclosure, our feeling of the overall shape and volume of the object grasped. On this alternative, conscious haptic experience depends upon, not an explicit, but an implicit experience of the hand's configuration and force in a way that does not require the hand to be the object of active attention.

Fulkerson, however, makes a further claim about implicit experiences that I find implausible. There is a sense in which, for Fulkerson, implicit experiences are no experiences at all. The content of an implicit experience is merely the content of a potential, that is to say, non-actual, experience. But it seems implausible that my awareness of my hand's configuration and force, while implicit, is merely potential and, thereby, non-actual. The information drawn upon from proprioception, kinesthesis, motor activity, and our sense of agency in haptic perception makes a contribution to the phenomenological character of that experience, even if there is, as Fulkerson urges, only one conscious experience in play and not two. The information from bodily awareness drawn upon in the exercise of our haptic capacities specifically makes a difference to the way the object of haptic awareness is presented. As I argued, distinct exploratory activities constitute distinct haptic perspectives on that object, and this perspectival relativity is manifest in the different haptic appearances presented by the constant object of haptic exploration. It is one thing to claim that bodily awareness makes no explicit contribution to haptic experience. But it is a further, contestable claim, that bodily awareness, however implicit, contributes nothing to the phenomenological character of the haptic experience it partly gives rise to. Bodily awareness, however implicit, contributes to the variable haptic appearances in the exercise of constant haptic perception. If the phenomenological character of haptic experience were exhausted by the constant tangible qualities attended to, then no room would be left for the contribution of flux to our haptic experience. But an adequate account of perceptual constancy must determine not only the constant object of perception but its variable appearances as well. 

% Constant tangible aspects are presented in haptic experience as the forces that constitute their categorical bases come into conflict with force of the grasping hand. And the variable appearances of these constant tangible aspects are a phenomenological reflection of the variable activity of the hand in haptic exploration.

I shall make a suggestion that is a basis for an answer to our refined how-possible question. Specifically, feeling tangible qualities in something external to the perceiver's body and feeling in conformity with them can fruitfully be understood as due to the operation of sympathy. One obstacle to appreciating this concerns out present understanding of sympathy, where sympathy is a kind of emotional response to others, a kind of fellow-feeling, akin to compassion or pity. The notion of sympathy that is being invoked as the principle governing haptic presentation is closer to the notion at work in Stoic physics, if more abstract and not at all reliant on on their vitalistic metaphysics. 

Felt resistance to touch, insofar as it is the presentation of an object external to the perceiver's body, is a sympathetic response to the force that resists the hand's activity. Recall our refined version of our how-possible question was this: How is it possible for felt resistance to the hand's activity in grasping or enclosure to disclose a rigid, solid body's overall shape and volume? If feeling tangible qualities in something external to the perceiver's body and in conformity with them is due to the operation of sympathy then we have a basis for an answer. It is when the limit to hand's activity is experienced as a sympathetic response to a countervailing force, as the hand's force encountering an alien force resisting it, one force in conflict with another, like it yet distinct from it, that the self-maintaining forces of the body disclose that body's presence and tangible qualities to haptic awareness.

The proposal is that presentation in haptic perception is governed by the principle of sympathy. There are two ways to understand this. The fist proceeds synthetically. That is, beginning with elements and principles understood independently of haptic perception, one constructs the notion of the presentation of tangible qualities of external bodies in haptic experience on their basis. So, for example, one might begin with bodily sensation and ``extend its reach'', so to speak, via the operation of sympathy to construct a notion of the presentation of tangible qualities of external bodies. So understood, haptic presentation would be the coordination of bodily sensations with the tangible qualities of external bodies via the operation of sympathy. The second way proceeds analytically. That is, beginning with the notion of the presentation of tangible qualities of external bodies in haptic experience, one analyses or decomposes that notion into constituent elements that must be present and principles that must be operative if haptic perception is so much as possible. 

The analytic approach to sensory presentation is comparable to Frege's approach towards thought, at least at certain stages of his career, on certain interpretations. Frege begins with a unity, a truth-evaluable thought, and discerns what intelligible structure it must display. Beginning with the thought, Frege analyzes or decomposes that thought into constituent elements that must be present and principles that must be operative if that thought is to be so much as truth-evaluable (which is not say that there is a unique such decomposition). The problem of the unity of the proposition simply does not arise for Frege, since he does not begin with independently understood elements and principles and tries to construct thoughts on their basis. Rather the unity of thought is explanatorily prior to the intelligible structure it must display if it is to be so much as truth-evaluable. Similarly, on the analytic approach, the unity of sensory presentation is explanatorily prior to the intelligible structure it must display if it is so much as possible.

The synthetic approach naturally, perhaps inexorably, motivates indirect realism about tactile perception. So consider again our toy model where we begin with bodily sensation and extend its reach through the operation of sympathy. Bodily sensation does not involve the presentation of tangible qualities of external bodies. It is, instead, a mode of self-presentation. Thanks to the operation of sympathy, in being presented with an aspect of our corporeal nature, we are mediately presented with the tangible quality of an external body. But haptic perception is not indirect in this way. When our hominid ancestor grasps a rough-hewn stone they feel its overall shape and volume in the stone. Moreover, the presentation of these tangible qualities in their haptic experience is not apparently mediated. Our hominid ancestor need not attend to their bodily sensations as a means of attending to the tangible qualities of external bodies, rather these are directly disclosed in haptic perception. Indeed, attending to the body and its activity draws attentive resources away from the object of tactile perception. It is because the tangible qualities of an external body are directly disclosed in haptic perception that grasping becomes, in the cosmology of the Giants, a touchstone for reality.

The problem with the synthetic approach, at least as so far developed, is twofold. First, it posits two experience---the haptic experience and the experience of the perceiver's body---when plausibly there is only one, and the awareness of the perceiver's body is explicit rather than implicit. Moreover both of these features were directly involved in the subsequent indirect realism. On the alternative, analytic approach, indirect realism is simply not a possibility. One begins with an irreducible unity, the presentation of the tangible qualities of external bodies in haptic experience, and then discern what intelligible structure it must display if it is so much as possible. Thus the presentation of tangible qualities of external bodies in haptic experience could not be a construction from elements and principles understood independently of haptic perception, the way they would be if indirect realism were true.

In grasping or enclosure the overall shape and volume of the object is directly disclosed in a perceiver's haptic encounter with it. The claim that the presentation of tangible qualities of external bodies in haptic experience involves the operation of sympathy should be understood in this light. It is not the claim that one thing, the tangible qualities of external bodies, is mediately presented by another thing, the presentation of aspects of the subject's corporeal nature in bodily sensation. Rather, it is the claim that the presentation of tangible qualities of external bodies in haptic experience is an irreducible unity that is governed by the principle of sympathy. Feeling a tangible quality in an external body and in conformity with it just is the presentation of that quality in tactile experience and can be analytically explicated in terms of the operation of sympathy.
 
\end{document}