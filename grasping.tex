%!TEX root = /Users/markelikalderon/Documents/Git/sympathy/perception.tex

\chapter{Grasping} % (fold)
\label{cha:grasping}

\section{The Dawn of Understanding} % (fold)
\label{sec:grasping_and_the_dawn_of_understanding}

In a justly famous scene from \emph{2001: A Space Odyssey}, set to Richard Strauss' \emph{Also Sprach Zarathustra}, a hominid ancestor, squatting among the skeletal remains of a boar, reaches out and tentatively grasps a femur. It is telling that this is how Stanley Kubrick chose to dramatize the initial transformation, induced by an alien obelisk, of our hominid ancestors, that eventually gives rise to space-exploring humanity in the twenty-first century. Not only does our hominid ancestor grasp the femur, but they grasp as well an important application. Squatting among the skeletal remains, femur in hand, our hominid ancestor taps the bones in exploratory manner. Each strike of the femur grows in force until finally, in a crescendo of activity, they smash the boar's skull to pieces. Our hominid ancestor has reached a crucial insight, that an implement, such as the femur, might transform boar into prey. Moreover, the application generalizes. The femur might also be used as a weapon against competing groups of hominids. The acquired technology thus has political consequences. What is presently important, however, is the connection between grasping and cognition. We say we have grasped a situation when we have understood it. And philosophers are prone to speak of thinkers grasping the thoughts they think. Kubrick dramatizes the connection between grasping and cognition by having our hominid ancestor's grasping the femur among the boar's skeletal remains be the primal scene of a dawning understanding.

We have \emph{grasped} a situation when we have understood it. We have a \emph{grip} on it. If the understanding in question is practical, we might say that we have matters \emph{in hand}. Nor are tactile metaphors confined to forms of higher cognition. They persist as well in our description of perceptual awareness. Not only do we speak of recognizing an object that we see as grasping the object present in our perceptual experience, but the presentation in experience is itself a kind of grasping. Perception puts us in \emph{contact} with its object. In perceiving an object we \emph{apprehend} it. The tactile metaphors for perceptual awareness tend to be modes of assimilation, and \emph{ingestion} is a natural variant. Our hominid ancestor, looking up from the boar's remains, \emph{takes in} the scene before them. If they see the obelisk, then, in a manner of speaking common among contemporary philosophers, the obelisk is the \emph{content} of our hominid ancestor's perception. But if the obelisk is the content of their perception then their perception of it is its container. To bring something into view so that it figures in the content of perception would be to contain it within that perception. But containment itself is a mode of assimilation. 

What makes tactile metaphors for perception apt? Tactile metaphors for perceptual awareness, even for non-tactile modes of awareness such as vision and audition, are primordial and persistent. Most contemporary philosophers of perception apply them unselfconsciously. That they do is a testament to the power of such metaphors. Understanding the power they have over us, understanding what makes them so compelling, we gain insight into the object of these metaphors. In understanding what makes grasping an apt metaphor for perception generally, we gain insight into the nature of sensory presentation. Or so I suggest.

We shall begin with a phenomenological investigation into the nature of grasping, a form of haptic touch. The investigation is phenomenological in that it seeks to uncover how grasping, understood as a mode of haptic perception, presents itself from within tactile experience. It is phenomenological because the object of investigation is restricted to perceptual appearances and not because of any methodology deployed in pursuing that investigation. The investigation thus need not involve ``bracketing'', nor need it confine itself to the deliverances of introspection in determining the nature of haptic appearance. In trying to understand how grasping, understood as a mode of haptic perception, presents itself from within tactile experience, we may avail ourselves of empirical and literary resources. Once we have a better understanding of how grasping presents itself from within tactile experience, we will be in a better position to understand why grasping also presents itself as an exemplar of sensory presentation more generally. 

Grasping may be an exemplar of sensory presentation, but it does not follow that all perception is a form of touch. One may grant that tactile metaphors for perceptual awareness are in some sense apt while eschewing any such reductive explanatory ambition. Such ambitions were rife in Greek antiquity. Thus \citet[39]{Lindberg:1977aa} observes that in the ancient world ``the analogy of perception by contact in the sense of touch seemed to establish to nearly everybody’s satisfaction that contact was tantamount to sensation, and it was not apparent that further explanation was required.'' Aristotle criticizes this explanatory strategy. Conceiving of non-tactile modes of perceptual awareness on the model of touch will seem explanatory insofar as touch is antecedently understood to be an unproblematic mode of perception. However, Aristotle's belaboring and not always completely resolving the \emph{aporiai} concerning touch in \emph{De Anima} 2 11 undermines that assumption \citep{Derrida:2005aa,Kalderon:2015fr}. And if further explanation is required, then we can no longer simply assume that contact is tantamount to sensation. Nevertheless, Aristotle accepts the aptness of the metaphor. Perception, for Aristotle, remains a mode of assimilation. The Stagirite defines perception as the assimilation of sensible form without the matter of the perceived particular (\emph{De Anima} 2 12 424\( ^{a} \)18–23, 2 5 418\( ^{a} \)3–6 ). So acceptance of the aptness of the metaphor carries with it no commitment to any such reductive explanatory ambition. Grasping may be apt metaphor for perception, even for non-tactile modes of perceptual awareness, without perception being reduced to a form of touch.

% section grasping_and_the_dawn_of_understanding (end)

\section{Haptic Perception} % (fold)
\label{sec:haptic_perception}

% section haptic_perception (end)

Grasping is a form of haptic touch. Haptic touch involves active exploration of the tangible object. This can involve a range of different stereotypical exploratory activities often combined in sequence. The different stereotypical exploratory activities are suited to presenting different ranges of tangible qualities. Thus to discern the texture of an object the perceiver may deploy lateral movement across its surface. Holding a stone in its hand, our hominid ancestor may feel the roughness of the stone by rubbing their thumb across its surface. And its hardness may be felt by applying pressure to it. According to the taxonomy of \citet{Lederman:1987fr}, grasping is a distinctive exploratory activity that they describe as ``enclosure''. Grasping an object allows the perceiver to discern a different range of tangible qualities. If texture is perceived by lateral motion and hardness by applying pressure, grasping or enclose makes volume and global shape available in tactile experience. Other stereotypical exploratory activities include: ``static contact''---passively resting one's hand on an externally supported object, without an effort to mold to its contours, to determine its temperature, ``unsupported holding''---holding the object without external support, and without molding, to determine the object's heft or weight often involving a ``weighing'' motion, ``contour following''---a smooth, nonrepetative tracing of the contours of the object, ``part motion test''---moving a part of the object independently of the whole, and ``specific function test''---moving the object in such a way as to perform various functions. Though these stereotypical exploratory activities are optimized for determining a specific range of tangible qualities, they can determine other tangible qualities, though perhaps less well, with less tactual acuity. Thus, for example, grasping is itself a way of applying pressure to an object and, hence, a way of perceiving its hardness as well as other of the object's tangible qualities such as temperature, moistness, feeling metallic, and so on. Not only are these stereotypical exploratory activities optimized to determine a specific range of tangible qualities but they can be chained together to provide the perceiver with a more complete profile of the corporeal nature of the object under investigation.

With enclosure, Lederman and Klatzky write:
\begin{quote}
	\ldots the hand maintains simultaneous contact with as much of the envelope of the object as possible. Often one can see an effort to mold the hand more precisely to object contours. Periods of static enclosure may alternate with shifts of the object in the hand(s). \citep[346--7]{Lederman:1987fr}
\end{quote}
The quoted passage brings out several important features of grasping, understood as a mode of haptic perception. 

First, grasping a rigid, solid body involves the hand's maintaining simultaneous contact with as much of its overall surface as possible. Grasping is thus a kind of incorporation. Recall, what unites the various tactile metaphors for perception, even for non-tactile modes of perceptual awareness such as vision and audition, is that they tend to be modes of assimilation, and grasping exemplifies this pattern. It may not be as complete an incorporation as the variant, ingestion, but it remains a clear mode of assimilation nonetheless. In maintaining simultaneous contact with as much of its overall surface as possible, the hand assimilates to the contours of the object. As we shall see, that the grasping hand assimilates to the object grasped is a manifestation of the objectivity of that haptic perception. This is part of what it makes it an apt metaphor for perceptual presentation more generally.

Second, not only does the grasping hand assimilate to the overall shape and volume of the object grasped, but, as Lederman and Klatzky observe, effort is typically exerted to mold the hand more precisely to the object's contours. So grasping or enclosure involves not only the hand's configuration in maintaining simultaneous contact with the overall surface of the object, but the force of the hand's activity as well. Not only is this force exerted in achieving the end of molding the hand more perfectly to contours of the object grasped, but it is exerted as well in the end's achievement---maintaining simultaneous contact with the overall surface of the object requires continued effort to sustain. This is physiologically and phenomenologically significant. It is physiologically significant in that the activation of different sets of receptors are coordinated in haptic perception \citep[see][chapter 3, for discussion]{Fulkerson:2014ek}. Grasping or enclosure will involve not only cutaneous activation but also the distinct sets of activations involved in kinesthesis, motor control, and our sense of agency. Moreover, this is reflected in our phenomenology. We feel the force with which we grip the object as well as the object's overall shape and volume.

Third, there is tendency, in grasping or enclosure, to shift the object periodically in one's hands. What explains this? Begin with Lederman's and Klatzky's observation that there is a tendency for perceivers to exert effort to mold their hand more precisely to the contours of the object grasped. In grasping an object, the grasping hand in this way assimilates to the overall shape and volume of the object grasped. Consider grasping a solid, rigid body, such as a stone. In grasping a stone, our hominid ancestor extends their hand's activity, they tighten their grasp, until they can no more. Since the stone is solid, it resists penetration. Since it is rigid, it maintains its overall shape and volume even when in the hominid's grasp. Contrast the way the overall shape and volume of an elastic body, such a sponge, deforms as it is squeezed. With the stone in its grip, the hand of our hominid ancestor assimilates to the overall shape and volume of the stone. Of course, hands are unevenly shaped and imperfectly elastic. This means that an effort to mold one's hand to a rigid body thus disclosing its overall shape and volume will most likely be imperfectly realized. There may be some areas of the object's surface that the grasping hand does not conform to. Tactile perception is partial in something like Hilbert's \citeyearpar{Hilbert:1987jq} sense. The tendency to shift the grasped object in our hands compensates for this partial and imperfect disclosure. In shifting the object in one's hand, an area that the hand did not previously conform to may become accessible to touch. Successive grips and the manner in which the object moves in one's hands as one shifts between them may provide a better overall sense of the shape and volume of the rigid body. 

Allow me to make three further observations about this passage, though now about issues that are merely implicit.

First, in periodically shifting the object in their hands to compensate for the partial and imperfect disclosure of overall shape and volume in grasping or enclosure, the perceiver's haptic experience exhibits perceptual constancy (on the importance of constancy phenomena to understanding perception see \citealt{Smith:2002sa,Burge:2010uq}). What the perceiver feels in moving the object between successive grips changes throughout this process, but the object disclosed by this haptic exploration is not protean in character. If the object were changing its overall shape and volume in the process of the perceiver's handling it, then shifting the object could be no compensation for the partial and imperfect disclosure of its overall shape and volume in grasping or enclosure. If the object were protean, in shifting it, its overall shape an volume would change. The opportunity to feel what was unfelt would be forever lost. In grasping, understood as a mode of haptic perception, the perceiver attends only to the constant tangible qualities it presents, in the case of a rigid, solid body, the perceiver attends to its constant overall shape and volume (as well as other constant tangible qualities that grasping the body may disclose, such as the thermal or material qualities of the body). Though there may be a felt difference in changing patterns of intensive sensation (including, among others, pressure and thermal sensation) in handling the object, haptic experience presents the constant overall shape and volume of the object. Of course, different aspects of the overall shape and volume may be present at different times, given the different ways the body is being handled. Sensory presentation being partial, the perceiver may now feel this corner and now that. But these presented aspects of the overall shape of a rigid, solid, body are experienced as stable aspects of a body that retains its shape, despite the perceiver's handling, because of the self-maintaining forces at work in its constitution. So the tendency, observed by Lederman and Klatzky, for the perceiver to periodically shift the object in their hands is not only explained by the partiality of haptic perception, but could only be so explained if the haptic experience this behavior gives rise to exhibits perceptual constancy.

Second, grasping is an activity and so is spread over time. It has duration. Not only does our hominid ancestor tentatively reach out and grasp the boar's femur from amongst its skeletal remains---an event with duration---, but its grasp must be actively maintained over a period of time. Maintaining simultaneous contact with the overall surface of a rigid body, or some nonsignificant portion of it, is a state sustained by activity. In this regard, it is like Ryle's \citeyearpar[149]{Ryle:1949qr} example of keeping the enemy at bay. The state thus obtains for the duration of the sustaining activity. Moreover, in coming to perceive its overall shape and volume, the perceiver may shift the object in their hand. The tactile sense of an object's overall shape or volume is disclosed by such activity. And since activity has duration, it is disclosed over time. The presentation of the overall shape and volume of an object in tactile experience is itself spread over time like the activity that discloses it. One potential lesson, then, for the metaphysics of sensory presentation, is that the object of perception may be disclosed over time, that its presentation in perceptual experience may have duration.

Third, that the grasping hand assimilates to the overall shape and volume of the object grasped is potentially epistemically significant. The full case for this will have to wait, but we can begin to get a sense of why this might be so. A rigid, solid body has a certain overall shape and volume prior to being grasped. Moreover, it is sufficiently rigid and solid to maintain that overall shape and volume even when grasped. In making an effort to more precisely mold the hand to the contours of the rigid object, the hand thus takes on, to an approximate degree, the overall shape and volume of the object grasped. That is to say, the hand takes on a certain configuration determined by the hand's anatomy, the activity of the hand, and the overall shape of the object grasped. And with the hand so configured, the shape of its interior approximates the overall shape of the object grasped. Moreover, the hand, so configured, encompasses a region of a certain volume itself determined by the hand and the volume of the object grasped. And the volume of the region that the hand encompasses approximates the volume of the object grasped. That is the point of making an effort to more precisely mold the hand to contours of the rigid object. In engaging in such haptic activity, in molding one's hand more precisely to the contours of the object, the overall shape and volume of the object had prior to being grasped, and maintained in being grasped, explains, in part, the hand's configuration in grasping the object and the force that needs to be exerted to maintain that configuration. Suppose that it is our hand's configuration in grasping and the force that needs to be exerted in maintaining that configuration that discloses the overall shape and volume of the object. If so, at least in the present instance, haptic perception is dependent, in some appropriate sense, upon proprioception, kinesthesis, our capacity for motor activity, and our sense of agency (for relevant discussion see \citealt{OShaughnessy:1989zp,OShaughnessy:1995ty,Martin:1992aa,Fulkerson:2014ek}). Since the object's overall shape and volume explains the hand's configuration and force, if the object eludes the hand's grasp, then that configuration and force would not have occurred. If the object is absent, there is nothing for the hand to assimilate to. Perhaps the objectivity of grasping, understood as a mode of haptic perception, consists in the grasping hand's assimilating to the tangible qualities of the object had prior to grasping.

Against this suggestion, it might be objected that, at least for certain graspings, it is possible for the object to be absent and yet the hand to be in a duplicate configuration. However, a felt difference would remain. Maintaining the hand's configuration in the absence of the object requires different musculature activity since the perceiver can no longer rely on pressing against the rigid body in maintaining that configuration. The different pattern of activation of receptors in muscles and joints will result in a felt difference. Compare leaning against a wall with making as if to lean against a wall. Sustaining that posture in the absence of the supporting wall can be difficult to do. Miming is an acquired skill. As Jacques Tati demonstrates in \emph{Cours du Soir}, it can be taught and learned. So in the case of duplicate configuration, where the hand takes on the configuration it would have had if it were grasping the object, while the hand's configuration has been maintained in the absence of the object, there is a felt difference in the force exerted.

That the grasping hand assimilates to the contours of the object grasped is potentially epistemically significant. It is, if not the source of that haptic perception's objectivity, then its manifestation. In grasping an object, the hand assimilates to the object's contours. If in grasping an object, the hand's configuration and force discloses the object's overall shape and volume, and that configuration and force would not have occurred in the absence of the object grasped, then our tactile experience would not be as it is when we haptically perceive if that object were in fact absent. While not yet proof against a Cartesian demon, one can begin to see the potential epistemic significance of the effort exerted in more precisely molding one's hand against the contours of the object grasped. It is the means by which certain tangible qualities of an external body are disclosed in our grasp. 

In the \emph{Theaetetus} 156 a--c, Socrates elaborates the Secret Doctrine of Protagoras by providing an account of perception as the contingent outcome of active and passive forces in conflict. Grasping as a mode of haptic perception can seem to approximate to that account. At the very least, the felt shape and volume of the object grasped is determined by conflicting forces. On the one hand, there is the force exerted in molding the hand more precisely to the contours of the rigid body. On the other hand, there are the self-maintaining forces of the rigid body itself. A solid, rigid body, such as a stone picked up by a hominid ancestor, is no mere sum of matter. It has a form or material structure determined by forces that are the categorical basis for its rigidity and solidity \citep{Johnston:2006js}. Haptic perception is the joint upshot of the force exerted by the grasping hand and the self-maintaining forces of the object grasped. There remains a crucial difference, however, from the account elaborated by Socrates. The overall shape and volume of the object and our haptic perception of them are not ``twin births'' as Protagoras maintains. The forces that determine the object's rigidity and solidity are sufficient to maintain the object's overall shape and volume within the hand's grasp. So the perceived tangible qualities of the external body inhere in that body prior to being perceived; whereas in the account attributed to Protagoras, the perceived object comes into being with the perceiver's perception of it. One might concede to Protagoras that the presentation of the object's overall shape and volume in tactile experience and the perceiver's feeling its overall shape and volume are, in fact, ``twin births''. It is at least the case that if overall shape and volume are not present in tactile experience then they are not felt, and if they are not felt, they are not present in tactile experience, at least not in that way. But not only is this consistent with perceptual realism, but, arguably, it is only intelligibly sustained against the background of a realist metaphysics. If a tangible quality's presentation in tactile experience is explained, in part, by that quality inhering in the object perceived, then the object must possess this quality prior to perception. There is a connection, then, between explanatory priority and objectivity (this, I argue, is Aristotle's view, \citealt{Kalderon:2015fr}). At least with grasping or enclosure understood as a mode of haptic perception, this perceptual realism is sustained by the force of the hand's activity in conflict with the self-maintaining forces of the object grasped. Explaining how this may be so is the task of this chapter and the next.

\section{Active Wax} % (fold)
\label{sec:active_wax}

So far in our discussion of grasping or enclosure we have established at least one claim about the metaphysics of sensory presentation, that sensory presentation is of such a nature that its objects may be disclosed over time. \citet{Broad:1952kx} took this dynamical aspect of sensory presentation to be confined to haptic perception. This is, at best, an exaggeration. Since the objects of audition, sounds and their sources, are spread over time, then it is at least natural to think that their presentation in auditory experience is itself disclosed over time. Moreover, there is reason to think that the presentation in visual experience of color qualities may itself be spread over time, at least some of the time. Thus as Broackes observes: 
\begin{quote}
	in order to tell what colour an object is, we may try it out in a number of different lighting environments. It is not that we are trying to get it into one single `standard' lighting condition, at which point it will, so to speak, shine in its true colours. Rather, we are looking, in the way it handles a variety of different illuminations (all of which are more or less `normal'), for its constant capacity to modify light. \citep[215]{Broackes:1997pa}
\end{quote}
Notice perceived colors belong to a distinct ontological category than audibilia. Audibilia, sounds and their sources, may be particulars like perceived colors, but whereas perceived colors are quality instances, audibilia are events or processes. So the fact that sensory presentation is spread over time need not be a consequence of the temporal mode of being of its object. Thus our phenomenological investigation into grasping understood as a mode of haptic perception has made vivid at least one claim about the metaphysics of sensory presentation, that the presence of the object of perception may be disclosed over time in perceptual experience, that sensory presentation may have duration. Moreover, this holds not only for the sensory presentation at work in haptic perception but plausibly for the sensory presentation at work in other sensory modalities as well.

Though a small claim about the metaphysics of sensory presentation, it has significant consequences. To take but one example, consider the claim that our ordinary experience of the natural environment that partly constitutes the Manifest Image of Nature is nothing more than a Grand Illusion. When our hominid ancestor turns, and looks, and sees, they are seemingly presented with a richly detailed scene of the alien obelisk rising among the reddish rocks set against a cloudy dawn sky. And this is true of the experience of twenty-first century humanity as well. When we visually perceive something, we are seemingly presented with a richly detailed scene. However, empirical research into change and inattentional blindness has suggested to some philosophers that this aspect of our phenomenology is illusory. Our visual experience may present itself as the presentation of a richly detailed scene, but, in fact, at any given moment, we are at best visually presented with a detail of some fragment of that scene. For at least some cases, the reasoning for the Grand Illusion hypothesis may be resisted. For it seems to presuppose that experience only presents what could be present in experience at any given moment. But if perceptual experience may disclose its object over time, then the claim that visual perception presents a richly detailed scene is consistent with the claim that, at any given moment, visual perception at best presents a fragment of that scene, as long as the richly detailed scene is understood to be disclosed over time and not present at a moment. Some of the arguments, then, if not all of them, for the Grand Illusion hypothesis turn on denying this claim about the metaphysics of sensory presentation---that sensory presentation may be a kind of disclosure with duration.

Our first claim about the metaphysics of sensory presentation involved a literal feature of grasping or enclosure. Grasping is a mode of haptic perception, and the presentation of its object is spread over time. That observation suffices to establish that sensory presentation may be a kind of disclosure with duration. Consider now another feature of grasping or enclosure, that the grasping hand assimilates to the rigid, solid body in its grasp. The hand's assimilating to the overall shape and volume of the object grasped is a manifestation, if not the source, of that haptic perception's objectivity. This, I suggested, is part of what makes grasping or enclosure an apt metaphor for sensory presentation more generally. It is important to get clearer about what this assimilation amounts to, and how it may be generalized, if assimilation is genuinely part of what makes grasping an apt metaphor for sensory presentation.

Grasping, understood as a mode of haptic perception, is, like the variant meta\-phor, ingestion, a kind of incorporation. This can suggest that the mode of assimilation is material---that it is a taking in, or incorporation, of a material body. Thus, for example, in eating an olive, the matter of the olive is taken in and presented to the organ of taste and thereby tasted (the organ of taste, understood as flavor, is not confined to the tongue with its taste buds but arguably includes retronasal receptors as well). But while some forms of sensory perception involve material assimilation such as tasting, not all do. Vision and audition involve the material assimilation of no thing. So if the assimilation at work in grasping or enclosure is part of what makes it an apt metaphor for sensory presentation generally, it must be understood in some other way.

Perhaps, the assimilation at work in grasping or enclosure is not merely material but formal. In grasping or enclosure, the hand assimilates to the contours of the object grasped. The interior of the hand thus approximates to the overall shape of the object, and the volume it encloses approximates to the object's volume. The shape of the interior of the hand is similar to the overall shape of the object, and the volume of the region it encloses is similar to the volume of that object. Perhaps, in this way, the hand assimilates the tangible form of the object grasped, by becoming similar to it. However, while our hand may be warmed when feeling the warmth of an object, our eyes do not become red when viewing a traditional English phone booth (though such a view has been attributed, incorrectly to my mind, to Aristotle, \citealt{Slakey:1961ss, Sorabji:1974fk,Everson:1997ep}). So it can seem that formal assimilation is no better off than material assimilation in this regard.

However, this latter problem for assimilation understood formally, if not materially, may be avoided by means of a small generalization. In grasping an object, where is the overall shape and volume that you feel? If grasping is a mode of haptic perception, then surely they are in the object that you grasp. Now, where is your haptic experience of that object? In your head? That answer seems so implausible on its face that only a philosopher could believe it. If anywhere, it seems more reasonable to suppose, at least initially, that it is closer to where the overall shape and volume are felt, in your handling of the object. Perhaps in trying to come to an understanding of formal assimilation at work in grasping or enclosure that may be generalized to other sensory modalities, we focussed too closely on the shape of the interior of the hand and the volume it encloses. If our haptic experience is where we handle the object grasped, perhaps the similarity obtains not only between the hand and certain tangible qualities of the object, but between the haptic experience and the tangible qualities presented in it. Haptic experience, like perceptual experience more generally has a conscious character. Perhaps, in grasping or enclosure, understood as a mode of haptic perception, the phenomenological character of haptic experience formally assimilates to the tangible qualities presented in it. And, arguably at least, this feature is generalizable to other sensory modalities as well---that in sensory perception quite generally, the phenomenological character of perceptual experience formally assimilates to the object presented in it.

Before considering whether that generalization partly grounds the aptness of grasping or enclosure as a metaphor for sensory presentation, even for non-tactile modes of perceptual awareness such as vision and audition, let us look closer at formal assimilation at work in haptic perception. Earlier we noted that haptic perception, like perception generally, is partial. The partial character of grasping, understood as a mode of haptic perception, explained the tendency, observed by Lederman and Klatzky, for the perceiver to shift the object of haptic exploration periodically in their hands. Such behavior compensates for the partial and imperfect disclosure of the overall shape and volume of the object grasped. Successive grips and the manner in which the object moves in one's hands provide a more complete profile of the corporeal aspects of the object under investigation. If the successive grips disclose different aspects of the object's overall shape and volume, then they provide something like different haptic perspectives on the object grasped. 

While talk of ``perspective'' derives from the case of vision, at the very least a clear analogue of that notion finds application in the haptic case. To the extent it does, then talk of ``haptic perspective'', while in a sense visuocentric, is not pejoratively so (on visuocentrism in philosophy of perception see \citealt{OCallaghan:2007xy}). Suppose the rigid, solid body, is irregularly shaped, then it potentially feels different in successive grips. And in the case of contour following, different paths may be followed giving rise to different progressions of intensive sensation, themselves constituting different haptic perspectives on the constant contour of the object of haptic investigation. In a part motion test on a set of keys, the perceiver may pick up a single key and move it to the left or to the right. They may even lift it straight up and jiggle the keys thus performing a specific function test. And we may pinch, squeeze, and pull on the object of haptic investigation and these distinct activities provide us with distinct haptic perspectives on that object. 

This perspective relativity bears on our understanding of the formal assimilation at work in grasping understood as a mode of haptic perception. In haptic perception, the tangible qualities of the object are presented to the perceiver's perspective on that object---the distinctive way they are handling that object in the given circumstances---and this is reflected in the conscious character of their haptic experience. So with respect to grasping or enclosure understood as a mode of haptic perception, the doctrine of formal assimilation should be understood as the claim that the phenomenological character of haptic experience formally assimilates to the tangible qualities presented to the perceiver's haptic perspective. Na\"{i}ve realists and disjunctivists accept something like this view if not the Peripatetic vocabulary with which I have described it. Thus na\"{i}ve realists and disjunctivists are prone to speak of the phenomenological character of perceptual experience being shaped by its object \citep[see][]{McDowell:1998vn,Martin:2004fj,Fish:2009fk,Kalderon:2011fk}.

It might be objected that haptic experience formally assimilating to the tangible qualities presented in it is absurd on its face. Perhaps in grasping a cube, my hand will approximate to a cube shape, but is it really the case that my experience is cube shaped? The claim that in seeing an English phone booth my visual experience becomes red seems even worse than the view literalists attribute to Aristotle, that in seeing the phone booth my eye becomes red. What does it even mean for an experience to be cubical or red? It is important in this regard to recognize that the posited similarity need not be exact. It is only on that assumption that the similarity involved in formal assimilation involves the sharing of qualities. But if we abandon that assumption, then there is a clear sense in which, in color vision say, in seeing the phone booth the qualitative character of my color experience depends upon and derives from the qualitative character of the color presented in that experience (for defense of this claim see \citealt{Kalderon:2008fk,Kalderon:2007mr,Kalderon:2011fk}). And similarly we might say that in haptic perception, the conscious character of haptic experience depends upon and derives from the tangible qualities present in that experience. 

Consider again the claim that haptic experience only formally assimilates to the tangible object it presents relative to the perceiver's haptic perspective. The perspectival relativity of formal assimilation bears on the inexactness of the similarity between experience and its object. The assimilation is formal in that, not only the shape of the interior of the hand and the region it encloses is similar to the overall shape and volume of the object, but the haptic experience, its conscious qualitative character, is similar to the tangible object at least as it is presented to the perceiver's haptic perspective. However, this does not require that the similarity be exact. The perspective relativity of formal assimilation nicely brings this out. Thus an irregularly-shaped, rigid, solid body, thanks to the self-maintaining forces that constitute the categorical bases of its rigidity and solidity, maintains its overall shape and volume despite progressive handling and the successive grips with which it is held. But that same shape feels different with different grips. If the phenomenological character were wholly determined by the tangible qualities present in haptic experience, then we would be hard pressed to explain why this is so. 

Earlier I claimed that the partiality of haptic perception only explained the tendency, observed by Ledermand and Klatzky, for the perceiver to periodically shift the object in their hands if the haptic experience this behavior gives rise to exhibits perceptual constancy. One of the philosophical challenges posed by perceptual constancy is to adequately describe and explain the phenomenology of stability and flux. In explaining perceptual constancy, it is not enough to determine the constant object of perception, that object continues to present itself unchanged even though its appearance may change with a change in the perceiver's perspective. In determining only the constant object of perception, one explains the phenomenology of stability at the expense of the contribution to our phenomenology of flux (for discussion in the color case, see \citealt{Cohen:2008hc,Hilbert:2007qy}). Even if, in the case of grasping or enclosure understood as a mode of haptic perception, we attend only to the constant overall shape and volume of the object grasped, these feel differently in different successive grips. Accommodating the contribution of flux to our phenomenology of grasping or enclosure requires acknowledging that haptic presentation, like sensory presentation more generally, is perspective relative.

So far we have distinguished material and formal modes of assimilation, and have suggested that while grasping or enclosure, understood as a mode of haptic perception, involves material assimilation---it is a kind of incorporation---, its objectivity is connected with the way in which the hand and  haptic experience more generally formally assimilates to its object. Moreover, we have emphasized the way that the similarity involved in formal assimilation need not be exact so as to involve the sharing of qualities. We now turn to another important distinction. Consider Lederman's and Klatzky's claim that that grasping or enclosure involves molding one's hand to the contours of the object grasped. Molding is a kind of shaping, and there are causal and constitutive senses of shaping that can be distinguished. So consider the way that the Nazi air campaign in the Battle of Britain shaped the London skyline. The destructive impact of the bombing caused the London skyline to be shaped in a certain way. This contrasts sharply with the way that St Paul's shapes the London skyline. This is dramatically demonstrated in Herbert Mason's iconic photograph of St Paul's on 29 December 1940. St Paul's defiantly shapes the London skyline by being part of it despite the devastating impact of the bombing campaign. Whereas Nazi bombing shaped the London skyline in a merely causal sense, St Paul's constitutively shapes that skyline by being a part or contour of it.

The causal--constitutive distinction plays out, I believe, in the use that Aristotle makes of Plato's wax analogy from the \emph{Theaetetus}. Plato, in the \emph{Theaetetus}, appeals to an impression made on wax as an analogy for the operation of memory in the context of explaining how error in judgment is possible:
\begin{quote}
	We may look upon it, then, as a gift of Memory, the mother of the Muses. We make impressions upon this of everything we wish to remember among the things we have seen or heard or thought of ourselves; we hold the wax under our perceptions and thoughts and take a stamp from them, in the way in which we take the imprints of signet rings. Whatever is impressed upon the wax we remember and know so long as the image remains in the wax; whatever is obliterated or cannot be impressed, we forget and do not know. (Plato, \emph{Theaetetus} 191 d--e, Levett and Burnyeat in \citealt[212]{Cooper:1997fk})
\end{quote}

In \emph{De Anima}, Aristotle uses the wax analogy, not for memory and knowledge as Plato does, but for explaining his definition of perception as the assimilation of the sensible form without the matter of the perceived particular:
\begin{quote}
	Generally, about all perception, we can say that a sense is what has the power of receiving into itself the sensible forms of things without the matter, in the way in which a piece of wax takes on the impress of a signet-ring without the iron or gold. (Aristotle, \emph{De Anima} 2 12 424a18–23; Smith in \citealt[42--43]{Barnes:1984uq})
\end{quote}
Part of the point of using Plato's wax analogy, not for memory or knowledge, but for perception is to highlight that Aristotle is assigning to perception functions that Plato assigned only to reason. Consider just one example. On the conception of perception in the \emph{Theaetetus}, the objects of perception are restricted to, in Peripatetic vocabulary, the proper sensibles. Perception just is the the power to present colors and sounds, and so on (\emph{Theaetetus} 184 e 8--185 a 3). According to Plato, while color may be presented in sight through the eyes, and sound in hearing through the ears, it is only reason which distinguishes color from sound (\emph{Theaetetus} 185 a--185 e). However, according to Aristotle, we perceive the difference between color and sound. So part of the point of deploying Plato's wax analogy to perception instead of forms of judgment is to emphasize that Aristotle is assigning to perception some of the functions that Plato assigned to reason. (For further discussion of how far Aristotle departs from Plato in drawing the distinction between perception and cognition see \citealt{Sorabji:1971fr,Sorabji:2003fk,Kalderon:2015fr})

There is a further, and for present purposes, more important way in which Aristotle departs from Plato's use of the wax analogy. There is a sense in which he takes the signet ring in the analogy more seriously than Plato. Or rather, Aristotle takes seriously, in a way that Plato does not, the distinctive discursive role of signet rings as opposed to a stylus, say. Moreover, this makes a difference to how the shaping of the wax by the ring is to be understood. Whereas Plato has in mind a causal notion of shaping, Aristotle has in mind the constitutive notion (or at least, so I argue \citealt[chapter 9]{Kalderon:2015fr}). Plato’s explanation of the reliability of memory crucially relies on causal features of the situation. An object’s impression is the effect it has on the mind’s wax. So the operation of peoples' memories may vary as to how hard or soft their mind's wax is, or how pure or impure it is, since these features causally bear on how clear an impression the object will produce and how long it may persist in the mind's wax. 

If, however, we reflect on the distinctive discursive role of a signet ring over a stylus, say, this can motivate the alternative, constitutive understanding of shaping. Notice that the impression of a signet ring plays a similar role to a signature. Just as a signature is linked to the particular person whose signature it is, the impression sealed upon the wax by a signet ring is linked to the legitimate possessor of that ring. Moreover, signatures, like sealed impressions, carry a certain authority, the authority endowed by their legitimate possessors. Of course, signatures can be forged, as can signet rings, which can also be stolen, but these practices gain there point precisely by the link between a signature and sealed impression, on the one hand, and their legitimate possessors, on the other. Signet rings and styli thus have distinctive discursive roles. The impression made by a stylus is not linked to its legitimate possessor---one scribe may borrow another scribe’s stylus---the way an impression sealed by a signet ring is.

Taking this feature of the analogy seriously has an important consequence for how sensory impressions are individuated. Just as a forged signature is not my signature, an impression sealed by a forged ring, or by a stolen ring, is not the seal of the ring’s legitimate possessor. Impressions are individuated by their legitimate sources. If this feature of the analogy carries over, then perceptions, conceived on the model of sealed impressions, are individuated by their objects which are their source. A perception of Castor and a perception of Pollux are different perceptions, no matter how closely the twins may resemble one another. Just as a forged seal is not my seal, a perception of Castor is not a perception of Pollux. A forged seal may be a perfect duplicate of a genuine seal but it is not the seal of the ring’s legitimate possessor. Castor may be a perfect duplicate of Pollux, but my visual impression of Castor is not an impression of Pollux.

Notice that a causal understanding of sensory impressions, as merely the effects of causal shaping, does not have this consequence. If, as Hume maintained, cause and effect are contingently connected, the same effect, the same impression, could have been produced by a different cause. Sensory impressions, understood as the effects of causal shaping, are not individuated by their causes. If sensory impressions are individuated by their objects which are their sources, they cannot be understood as merely the effects of causal shaping. How else might they be understood?

% If sensory impressions are individuated by their objects, perhaps these objects shape sensory consciousness not causally, or at least not merely. Perhaps in being individuated by their objects, these objects constitutively shape our sensory impressions of them (for contemporary discussion of this suggestion see \citealt{Kalderon:2008fk,Kalderon:2007mr,Kalderon:2011fk}). Our hominid ancestor turns, and looks, and sees the alien obelisk rising among the red rocks set against a cloudy dawn sky. The blackness of the obelisk is a constituent of their visual experience. The blackness of the obelisk is a constituent of their experience insofar as that experience involves the presentation of that blackness in the visual awareness afforded them by their experience of that scene. And since the experience of our hominid ancestor is constitutively linked to the blackness of the alien obelisk---an awful darkness in which stars may appear if, into it, one peers too deeply---the obelisk’s blackness shapes the contours of their visual consciousness by being present in that consciousness. The blackness of the obelisk shapes the contours of their visual experience in the way that St Paul’s defiantly shapes the London skyline, the Shard notwithstanding, simply by being present. The blackness of the obelisk is present in the awareness that sight affords our hominid ancestor of the scene before them. That experience has a certain character. The character of that experience depends upon and derives from, at least in part, the character of the presented blackness. Their experience, in this sense, becomes like the way the obelisk actually is, black. Just as what the London skyline is like depends, in part, upon what St Paul’s is like, since the London skyline involves the presence of St Paul’s as a part, what our hominid ancestor's experience of the obelisk is like depends, in part, upon what the obelisk’s blackness is like, since their experience involves the presentation in sight of that blackness.

What taking seriously the distinctive discursive role of the signet ring in the wax analogy brings out is that the formal assimilation at work in haptic perception and, arguably at least, in perception more generally, might be understood, not on the model of causal shaping, but rather on the model of constitutive shaping. If sensory impressions are individuated by their objects, perhaps these objects shape sensory consciousness not causally, or at least not merely. Perhaps in being individuated by their objects, these objects constitutively shape our sensory impressions of them (for contemporary discussion of this suggestion see \citealt{Kalderon:2008fk,Kalderon:2007mr,Kalderon:2011fk}). Recall that the assimilation at work in grasping or enclosure understood as a mode of haptic perception is formal in that, not only the shape of the interior of the hand and the region it encloses is similar to the overall shape and volume of the object grasped, but that the haptic experience, its conscious qualitative character, is similar to the tangible object at least as it is presented to the perceiver’s haptic perspective. On the causal model, a haptic experience, with its conscious qualitative character, is a sensory impression caused in the perceiver by the object of haptic investigation. Moreover, if the causal structure of the world cooperates and the circumstances of perception are propitious, then the conscious qualitative character of the haptic experience may be like, if not exactly like, the qualitative character of the tangible object. On the constitutive model, haptic experience formally assimilates to its tangible object as well. However, that object does not merely cause the perceiver to undergo a haptic experience with a certain conscious qualitative character. Rather, corporeal aspects of the object constitutively shape the perceiver's haptic experience of it. Not only does the perceiver's haptic experience formally assimilate to its tangible object relative to their haptic perspective, in the sense that the conscious qualitative character of the experience is like, if not exactly like, the qualitative character of the tangible object present in it, but the tangible quality present in their haptic experience constitutively shapes that experience. If something feels metallic, and this is a case of tactile perception, then not only is this because of its metallic feel, but something's feeling metallic is also constituted, in part, by that metallic feel. The metallic feel of the thing is felt in it and in conformity with it. That is what it is for to be present in tactile experience.

In grasping or enclosure, understood as a mode of haptic perception, the hand maintains simultaneous contact with a nonsignificant portion of the overall surface of the object grasped. Grasping is a kind of incorporation and thus a material mode of assimilation. Moreover, in grasping the hand is so configured that it approximates to the contours of the object. Just as the shape of the interior of the hand and the region it encloses is like, if not exactly like, the overall shape and volume of the object grasped, the phenomenological character of the haptic experience, its conscious qualitative character, is like, if not exactly like the overall shape and volume presented to the perceiver's haptic perspective on it. Moreover, the shaping involved, at least in the latter formal assimilation, is not merely causal but constitutive. The conscious qualitative character of the haptic experience is constituted, in part, by the tangible qualities presented in it.

While not all modes of perception involve material modes of assimilation, arguably at least, the formal assimilation of haptic experience to its object relative to the perceiver's haptic perspective generalizes to other modes of perception. The conscious qualitative character of perceptual experience is constituted, in part, by the qualitative character of the object presented to the perceiver's partial perspective. Our hominid ancestor turns, and looks, and sees the alien obelisk rising among the red rocks set against a cloudy dawn sky. The blackness of the obelisk is a constituent of their visual experience. The blackness of the obelisk is a constituent of their experience insofar as that experience involves the presentation of that blackness in the visual awareness afforded them by their experience of that scene. And since the experience of our hominid ancestor is constitutively linked to the blackness of the alien obelisk---an awful darkness in which stars may appear---the obelisk’s blackness shapes the contours of their visual consciousness by being present in that consciousness. The blackness of the obelisk shapes the contours of their visual experience in the way that St Paul’s defiantly shapes the London skyline, the Shard notwithstanding, simply by being present. 
% The blackness of the obelisk is present in the awareness that sight affords our hominid ancestor of the scene before them. That experience has a certain character. The character of that experience depends upon and derives from, at least in part, the character of the presented blackness. Their experience, in this sense, becomes like the way the obelisk actually is, black. Just as what the London skyline is like depends, in part, upon what St Paul’s is like, since the London skyline involves the presence of St Paul’s as a part, what our hominid ancestor's experience of the obelisk is like depends, in part, upon what the obelisk’s blackness is like, since their experience involves the presentation in sight of that blackness.

If this feature of grasping or enclosure, understood as a mode of haptic perception, generalizes to other modes of perception, then it is easy to see its epistemic significance. If perception involves becoming like the perceived object actually is, then it is a genuine mode of awareness. One can only perceptually assimilate what is there to be assimilated. If perceptual experience is a formal mode of assimilation understood as a kind of constitutive shaping, then one could not undergo such an experience consistent with a Cartesian demon eliminating the object of that experience. If there is no external object, then there is nothing to which the perceiver, or perhaps their experience, can assimilate to.

I have claimed that the assimilation at work in grasping or enclosure, understood as a mode of haptic perception, is the manifestation, if not the source, of the objectivity of haptic perception. I have also claimed this is part of what makes grasping an apt metaphor for sensory presentation more generally. We are now in a position to elaborate further. Not only does the grasping hand assimilate to the contours of the object, but the perceiver's haptic experience---there where they are handling the object---assimilates to the overall shape and volume of the object as well, at lest relative to their haptic perspective on it, the specific manner in which they are handling the object. But one can only assimilate to what is there to be assimilated. The objectivity of haptic perception is thereby manifested. And if this formal assimilation, this constitutive shaping, generalizes to other modes of perception, then part of what makes grasping an apt metaphor for perception generally is our consequent understanding of perceptual objectivity. The formal assimilation of haptic experience to its object relative to the perceiver's haptic perspective on it, the constitutive shaping of the phenomenological character of that experience by the presentation of its object to the perceiver's perspective, is the manifestation of the objectivity of that haptic perception. But what is its source? What explains haptic experience assimilating to its object? If we bear in mind that haptic experience is where the perceiver is handling the object, then a plausible thought is that it is the force of the hand's activity, the effort exerted in more precisely molding the hand to the contours of the object, that is the source of the hand, and consequently our haptic experience, assimilating to the object.

While the assimilation of the hand to the contours of the object is the manifestation of the objectivity of that haptic perception, it is the force of the hand's activity that is its source. It is because the hand tightens its grip that it's flexible interior surface may more precisely mold to the object's contours. Robert Kilwardby provides a vitalist twist on the Peripatetic analogy that potentially sheds light on the epistemic significance of the force of the hand's activity in grasping or enclosure, understood as a mode of haptic perception. Kilwardby composed \emph{De Spiritu Fantastico Sive de Recptione Specierum} most likely while in Blackfriars in Oxford in the 1250s prior to being elevated to the Archbishop of Canterbury. In a remarkable passage, Kilwardby writes:
\begin{quote}
	You will have some kind of simile for understanding this if you assume that there is a seal in front of the wax so that it touches it and that the wax has a life by which it turns itself towards the seal, and by pressing itself against it, makes itself like it. (Kilwardby, \emph{De Spiritu Fantastico} 103, \citealt[94]{Broadie:1993dz})
\end{quote}
Kilwardby transform's the Peripatetic analogy by imagining life to inhere in the wax so that it is actively pressing against the seal and so taking its sensible form upon itself. The vitalist twist on the wax analogy accomplishes two things. First, in the active wax taking upon itself the sensible form of the seal, the analogy makes intelligible how perception may be a non-material mode of assimilation, an internalization or mode of taking in. But importantly, what sense it provides to this non-material mode of assimilation is consistent with what is assimilated in this way existing and having its character independently of that perception. Indeed, it is the resistance to the wax's activity that discloses the sensible form of the object had prior to perception.

Kilwardby's account is motivated, in no small part, by his conviction, grounded in his reading of Augustine, that the soul cannot be acted upon by the body (\emph{De Spiritu Fantastico} 47--54). Kilwardby tentatively accepts a Peripatetic model where, in vision, say, the perceived object acts upon the transparent medium such that its image, in some sense, exists in it, and that the medium, in turn, affects the sense organ such that the image comes to, in some sense, exist in it as well (\emph{De Spiritu Fantastico} 69). But how does the sensory soul receive the image that informs the sense organ, if the sense organ is precluded, by its corporal nature, from acting upon the soul? The vitalist twist on the Peripatetic analogy is meant to provide insight here. The sensory soul pervades the sense organ and makes itself like the external body. So it is the sensory soul that is the efficient cause of the likeness of the body occurring in it. It does so by pressing against the sense organ itself impressed with the image of the object. In actively pressing against the impressed sense organ, the soul makes within itself the image of the external body: ``For in this way the sensory soul, by turning itself more attentively to its sense organ which has been informed by a sensible species, makes itself like the species, and by turning its own eye upon itself it sees that it is like the species'' (\emph{De Spiritu Fantastico} 103, \citealt[94]{Broadie:1993dz}).

It is not clear whether the subsequent account constitutes a genuine reconciliation of Augustinian and Peripatetic metaphysics (for discussion of Kilwardby on perception see \citealt{Silva:2008yg}; selections from \emph{De Spiritu Fantastico} are also translated in \citealt{Knuuttila:2014rc}). At any rate, a Humean worry arises. According to Kilwardby, while a body cannot act upon the sensory soul, the activity of the sensory soul may be resisted by a body, specifically by the sense organ being informed by the object's sensible species. But, as Hume plausibly argues in Book \textsc{ii} of the \emph{Treatise}, if reason cannot determine the will, neither can it oppose the passions determination of the will. So how is it that the sensible species of the body informing the sense organ opposes the attentive activity of the sensory soul such that it receives that image within itself if the sense organ in which the sensible species inheres, being a body, cannot act upon the sensory soul? Even supposing that a sensible species inhering in a sense organ is among the passivities of matter, Kilwardby's doctrine that the soul's use of a body is limited by the passivities of matter can seem merely to give an Augustinian description to the Humean problem.

Regardless of Kilwardby's intent, and dropping his Augustinian dualism, the hand, the mobile and elastic instrument of haptic exploration, is the active wax in grasping or enclosure, understood as a mode of haptic perception. It is the hand that is actively molding itself to the object in grasping or enclosure. And it is the hand that is thereby taking upon itself a configuration and enclosing a certain volume determined by the overall shape and volume of the object grasped. In making an effort to mold more precisely to the contours of the rigid, solid body, not only does the hand assimilate to the contours of the object grasped, but the haptic experience---there where the perceiver is handling the object---assimilates to the overall shape and volume of the object presented in it. Further, I take it that it is at least part of Kikwardby's suggestion that it is the activity of the wax and the resistance it encounters in pressing against the seal that discloses the shape of the seal had prior to perception. So if the hand is the active wax in grasping, understood as a mode of haptic perception, then it is the force of the hand's activity and the resistance it encounters in maintaining simultaneous contact with a non-significant portion of the object's overall surface that discloses the tangible qualities of the object had prior to the haptic encounter. Kilwardby's suggestion, then,---if released from the confines of Augustinian metaphysics, if, in turn, narrowly confined to haptic presentation---is that the presentation of tangible qualities of objects external to the perceiver's body is due, at least in part, to the activity of the hand in grasping and the resistance it encounters. The hand, and haptic experience in turn, only assimilate to the tangible aspects of the rigid, solid body thanks to the force of the hand's activity in conflict with the self-maintaining forces of that constitute the categorical bases of that body's solidity and rigidity. At least with grasping or enclosure, understood as a mode of haptic perception, perceptual realism is sustained by the force of the hand’s activity in conflict with the self-maintaining forces of the object grasped.

% section active_wax (end)

\section{A Puzzle} % (fold)
\label{sec:a_puzzle}

% To fully appreciate the epistemic significance of the force of the hand's activity in more precisely molding to the contours of the object grasped, we need an account of how it is that assimilating to an external body discloses the tangible qualities that inhere in that body prior to haptic perception. The question here is a ``how possible'' question \citep{Cassam:2007lq}. How is objective haptic perception so much as possible?

In discussing the objectivity of grasping, understood as a mode of haptic perception, we supposed that it is our hand's configuration in grasping and the force that needs to be exerted in maintaining that configuration that discloses the overall shape and volume of the object grasped. I believe this supposition to be both plausible and true, but once it is clearly stated, a puzzle immediately arises. 

Embodiment is a fundamental feature of animal existence and so a fundamental feature of the existence of primates like ourselves. An animal's awareness of its body is a mode of self-presentation. There may be more to an animal than is revealed in bodily awareness, bodily awareness nevertheless presents corporeal aspects of the animal whose awareness it is. Bodily awareness remains a mode of self-presentation even if its disclosure of the animal whose awareness it is is partial in this way. Let bodily awareness be understood broadly enough to comprise both proprioception and kinesthesis and potentially more besides. So bodily awareness affords the perceiver with, among other things, awareness of the configuration of their limbs as well as awareness of their motion. So understood, awareness of the hand's configuration in grasping and awareness of the force that needs to be exerted in maintaining that grasp are both modes of bodily awareness. And since bodily awareness is a kind of self-presentation so are awareness of the hand's configuration and awareness of the force exerted in maintaining it. 

Our puzzle now is this. How can a mode of self-presentation disclose the presence of some other thing? How can bodily awareness be leveraged into disclosing the presence of something external to the perceiver's body? What alchemy transmutes bodily sensation into tactile perception?

Our puzzle concerns whether grasping so much as could be a mode of haptic perception. Though our interest is presently restricted to grasping as a mode of haptic perception, we can, however, get a better sense of that puzzle by considering an analogous case. So consider felt temperature. Contrast two cases. In both cases you feel warm, and you feel warm to the same degree. But in the first case, you feel warm because of a fever, and in the second case, you feel warm because because of the ambient heat. In both cases, your body is warmed. They differ only in the source of the warmth, with whether the warmth of your body is internally or externally generated. And in both cases, you feel equally warm. Nevertheless, a phenomenological difference remains. In the second case, not only do you feel warm, but you feel, as well, the warmth in the ambient air. Indeed, the warmth you feel is in conformity with the warmth felt in the ambient air. What explains this phenomenological difference? How are tangible qualities felt in something external to the perceiver's body such that perceiver feels in conformity with such qualities?

The puzzle is not meant to underwrite skepticism about haptic perception or tactile perception more generally. We are taking it for granted that in grasping a stone, say, our hominid ancestor feels the overall shape and volume of that stone. We are taking it for granted that grasping is a mode of haptic perception that affords the perceiver awareness of tangible qualities that inhere in the object grasped. Our puzzle is not meant to underwrite skepticism about wether grasping is a genuine mode of haptic perception of the tangible qualities of external bodies so much as to underwrite a ``how-possible'' question \citep{Cassam:2007lq}. How is it that the configuration of the hand and the force exerted in maintaining that configuration disclose the overall shape and volume of the object grasped? How is objective haptic perception so much as possible? The puzzle, then, is at best proof of an explanatory lacuna rather than proof of the impossibility of objective haptic perception.

There is an aspect of grasping or enclosure that has so far remained implicit in our discussion of it but is crucial for refining our how-possible question in such a way as to point to an adequate solution. The perceiver, in exerting effort in more precisely molding their hand to the contours of the object grasped, encounters felt resistance to their efforts. It is because the self-maintaining forces of the body resist the hand's encroachment that the hand can assimilate to the body's contours. The forces that constitute the body's solidity ensure that the force of the grasping hand does not penetrate it. And the forces that constitute the body's rigidity ensure that it maintains its overall shape and volume despite the force of the hand's grasp. Maybe it is the hand's encounter with felt resistance that discloses the tangible qualities of an external body. The suggestion, here, is not merely that the puzzle overlooked the contribution of cutaneous activation to tactile awareness, but rather with how cutaneous activation interacts with kinesthesis and bodily awareness more generally in giving rise to the experience of an external limit to the body's activity.

Smith has appropriated Fichte's term, \emph{Anstoss}, for the way in which the experienced limitation of the body's activity can disclose sensible aspects of an external body:
\begin{quote}
	Atlthough neither touch sensations nor the active / passive distinction suffices for perceptual consciousness, when the two are taken together we \emph{do} find something that suffices \ldots\ Although no mere impact on a sensitive surface as such will give rise to perceptual consciousness, \emph{we} certainly feel objects impacting on us from without. This fact needs to be recognized in any adequate perceptual theory. I shall name the phenomenon that is central here by the term that is at the heart of Fichte's treatment of the ``external world,'' or the ``not-self'': the \emph{Anstoss}. This phenomenon is that of a \emph{check} or \emph{impediment} to our active movement; an experienced obstacle to our animal striving, as when we push or pull against things. \citep[153]{Smith:2002sa} 
\end{quote}
Part of what we shall learn from the refined how-possible question is that \emph{Anstoss}, at least as Smith conceives of it, is itself subject to further explanation (elaborating that explanation is the task of the next chapter).

There is a long history connecting objectivity to felt resistance to touch. In the \emph{Sophist}, Plato recasts the Gigantomachy, the struggle for supremacy over the cosmos between the Olympian Gods and the Giants, as a metaphysical dispute. The Gods, or Friends of the Forms, insist that only imperceptible forms are most real. Against them, the Giants, the offspring of Gaia, insist that only material bodies exist:
\begin{quote}
	One party is trying to drag everything down to earth out of heaven and the unseen, literally grasping rocks and trees in their hands, for they lay hold upon every stock and stone and strenuously affirm that real existence belongs only to that which can be handled and offers resistance to the touch. (Plato, \emph{Sophist} 246a; Cornford in \citealt[990]{Hamilton:1989fk})
\end{quote}
For the Giants, felt resistance to touch has become a touchstone for reality. Only that which can be handled and offers resistance to touch is real. Even if one rejects the materialist metaphysics of the Giants, one can accept that the experience that grounds their materialist conviction is phenomenologically compelling. It would have to be to elicit such cosmic conviction. Grasping something which offers resistance to touch is a phenomenologically vivid and primitively compelling experience of what is external to us. 

The phenomenologically vivid and primitively compelling experience of felt resistance to touch will underwrite the dramatic episode involving Dr Johnson outside of a church in Harwich:
\begin{quote}
	After we came out of the church, we stood talking for some time together of Bishop Berkeley’s ingenious sophistry to prove the non-existence of matter, and that every thing in the universe is merely ideal. I observed, that though we are satisfied his doctrine is not true, it is impossible to refute it. I never shall forget the alacrity with which Johnson answered, striking his foot with mighty force against a large stone, ’till he rebounded from it, ``I refute it thus.'' This was a stout exemplification of the first truths of Pere Buffier, or the original principles of Reid and Beattie; without admitting which, we can no more argue in metaphysicks, than we can argue in mathematicks without axioms. To me it is inconceivable how Berkeley can be answered by pure reasoning \ldots \citep[\textsc{i} 471]{Boswell:1935fk}
\end{quote}
The reality of external matter was demonstrated in the resistance it offered to Dr Johnson’s foot, which rebounded despite its mighty force. It was a demonstration not in the sense of proof, since it is inconceivable how Berkeley can be answered in pure reasoning. Moreover, what was stoutly exemplified was metaphysically axiomatic, a first truth, but proof proceeds from axioms, it does not establish them. Rather Dr Johnson’s performance was a demonstration of first truths by showing or exhibiting them. (On the character of Johnson’s refutation of Berkeley see \citealt{Patey:1986uq}). Dr Johnson's demonstration, like the Giants' before him, draws its dramatic power from the phenomenologically vivid and primitively compelling experience of felt resistance to touch. And this remains true even if the dramatic power of that gesture is all but exhausted in the twentieth century clich\'{e} of the exasperated, table-pounding realist.

Campbell, in his contribution to \citet[71]{Campbell:2014aa}, argues, instead, that Dr Johnson's demonstration was essentially multimodal, depending not only upon the kicking of the stone but upon seeing it as well:
\begin{quote}
	It is important that Johnson's kicking the rock is a multimodal affair. It would not have had the same visceral impact if Johnson had rebounded off the thing while kicking it in the pitch dark. That would merely have established the presence of some force or another. (Campbell in \citealt[71]{Campbell:2014aa})
\end{quote}
There is more, however, to the experience of kicking a stone in the dark than Campbell allows. For example, despite the darkness, Dr Johnson, perhaps through the reverberation of his foot, which rebound despite its mighty force, might discern that it was stone and not a log that he was kicking. The characteristic density of stone as opposed to wood might be felt in this manner. And if it is sufficiently cold, he might feel the coldness of the stone through the leather of his boot. So it is not true that all that kicking the stone in the dark presents is some force or another. It can present as well material and thermal qualities of the object kicked. Campbell underestimates the experience of kicking a stone in the dark in a further way. Not only would that experience establish the presence of some force or another, it would disclose the self-maintaining forces that constitute a rigid, solid body external to Dr Johnson's body. If Dr Johnson's exasperation merely grew with the rebounding of his foot, he might kick it again. But as exasperated as he is in the dark, Dr Johnson's haptic experience presents him with the same stone kicked twice. Each kicking of the stone constitutes distinct haptic perspectives on that object, and Dr Johnson has the capacity to haptically reidentify the stone presented to distinct haptic perspectives, distinct kickings in the dark. Notice that this would not be possible if kicking the stone in the dark merely presented some force or another. Earlier in \citet[26]{Campbell:2014aa}, however, and more plausibly to my mind, Campbell claimed that it was ``the obstinance of the rock, its resistance to the will'' that manifest its mind independence. But surely the obstinance of the rock, its resistance to the will, the effect of the rock's self-maintaining forces which reveals it to be mind-independent matter, was manifest in Dr Johnson's haptic encounter with it independently of being seen.

How does felt resistance to touch disclose tangible qualities inhering in external bodies prior to perception? In exerting effort to mold more precisely the hand to the contours of the object grasped, in assimilating to the object, the perceiver experiences felt resistance to touch, they experience a limit to the hand's activity. How does the limit to the hand's activity in grasping a solid, rigid body disclose its overall shape and volume? After all, not all limitations to the body's activity are due to its interaction with external bodies. There are internal limitations to the body's activity as well. We encounter an internal limitation to the body's activity due to fatigue or in an inability to touch one's toes. And \citet[154]{Smith:2002sa} gives the nice example of separating your index and middle fingers until you can no more. So not every experience of a limitation to the body's activity is due to the tangible qualities inhering in an external body prior to perception. So how is it that in grasping, or enclosure, the limitation to the hand's activity in molding more precisely to the contours of the object grasped and the consequent felt resistance to touch disclose that object's overall shape and volume? How does the experienced limitation to the hand's activity become, in haptic perception, an experience of the tangible qualities of an external body? How is it that by means of an experienced limitation to the hand's activity tangible qualities are felt in something external to the perceiver's body and felt in conformity with those qualities?

This, then, is the refined version of our how-possible question: How is it possible for felt resistance to the hand's activity in grasping or enclosure to disclose a rigid body's overall shape and volume? How does the experience limitation to the hand's activity allow the perceiver to feel something in an external body and in conformity with it? Earlier I claimed that the refinement of our question could point to an adequate solution. Indeed we have all but stated it. Though perhaps that can only be appreciated once the solution is clearly in view.

% section a_puzzle (end)

% chapter grasping (end) 
