%!TEX root = /Users/markelikalderon/Documents/Git/sympathy/perception.tex
\chapter*{Acknowledgements} % (fold)
\markboth{\MakeUppercase{Acknowledgements}}{}
\addcontentsline{toc}{chapter}{Acknowledgements}
\label{cha:acknowledgements}

Fortuitous serendipity has been all too evident in the composition of the present essay. Tempering the humility I feel in recognizing this (there, but for the hand of chance, go I...), is the further recognition of just how much work must go in to make such serendipitous encounters both possible and fortuitous. I owe a debt to many, both for providing occasions for such encounters and for preparing the way for them. Allow me to acknowledge some of them.

For a number of years now, I have taught a course structured around the opening remarks of C.D. Broad's ``Some elementary reflections on sense-perception.'' The first five pages of that essay involves a comparative phenomenology of vision, audition, and touch. The class proceeds by evaluating Broad's comparative phenomenological claims in light of more recent literature about these senses. Sometimes I feel that my students got a raw deal. Not that I was neglectful in my pedagogical duty. Rather, I feel that I learned more from these class discussions than they did. The salutary effects of teaching that class are particular evident in chapters \ref{cha:sound} and \ref{cha:sources_of_sound}. For all that I have learned from them, and all the serendipitous encounters that they helped prepare the way for, I am most grateful.

To Maarten Steenhagen I am grateful for one such serendipitous encounter. Robert Kilwardby provides a vitalist twist on the Peripatetic analogy of wax receiving the impression of a seal and the assimilation of form without matter in perception. Specifically, Kilwardby imagines life to inhere in the wax and to be actively pressing against the seal. Reflection on Kilwardby's vitalist twist on the Peripatetic analogy forms one of the key threads throughout this book. I am very grateful to Steenhagen for bringing my attention to it. I am also grateful for his intellectual companionship. We have discussed these and related issues over the years. Steenhagen also read some preliminary drafts of early chapters which helped me to improve them greatly, for which I am also indebted.

Craig French also read drafts of two chapters. The level-headed clarity of his comments, and more than that, the demand that I too should sometimes display such clarity, prompted considerable improvement, and for that I am most grateful.

Clare Mac Cumhail provided another serendipitous encounter in reminding me of a passage in Hans Jonas that plays a key role in chapter~\ref{cha:vision} for which I am grateful as well.

A not unsympathetic audience at the University of Edinburgh to whom I presented a paper culled from chapters \ref{cha:grasping} and \ref{cha:sympathy} provided much needed feedback and prompted considerable improvement.

Finally, I owe a debt of gratitude to Charles Travis for his friendship, intellectual companionship, and encouragement.


% chapter acknowledgements (end)