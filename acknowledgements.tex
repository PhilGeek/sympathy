%!TEX root = /Users/markelikalderon/Documents/Git/sympathy/perception.tex
\chapter*{Acknowledgements} % (fold)
\markboth{\MakeUppercase{Acknowledgements}}{}
\addcontentsline{toc}{chapter}{Acknowledgements}
\label{cha:acknowledgements}

Fortuitous serendipity has been all too evident in the composition of the present essay. Tempering the humility I feel in recognizing this---there, but for the hand of chance, go I---is the further recognition of just how much work must go in to make such serendipitous encounters both possible and fortuitous. I owe a debt to many, both for providing occasions for such encounters and for preparing the way for them. Allow me to acknowledge some of them.

For a number of years now, I have taught a course structured around the opening remarks of C.D. Broad's \citeyearpar{Broad:1952kx} ``Some elementary reflections on sense-per\-cep\-tion.'' The first five pages of that essay involves a comparative phenomenology of vision, audition, and touch. (Broad's topics are discused in reverse order in the present essay.) The class proceeds by evaluating Broad's comparative phenomenological claims in light of more recent literature about the senses. Sometimes I feel that my students got a raw deal. Not that I was neglectful in my pedagogical duty. Rather, I feel that I learned more from these class discussions than they did. The salutary effects of teaching that class are particular evident in chapters \ref{cha:sound} and \ref{cha:sources_of_sound}. For all that I have learned from them, and all the serendipitous encounters that they have helped prepare the way for, I am most grateful.

Material from the first two chapters was presented in a research seminar at UCL in 2015. I am very grateful to all that participated, especially for the many clarifications they elicited from me that resulted in considerable improvement of the text.

To Maarten Steenhagen I am grateful for one such serendipitous encounter. In \emph{De spiritu fantastico sive de recptione specierum}, Robert Kilwardby provides a vitalist twist on the Peripatetic analogy of perception with wax receiving the impression of a seal. Specifically, Kilwardby imagines life to inhere in the wax and to be actively pressing against the seal. Reflection on Kilwardby's vitalist twist on the Peripatetic analogy forms one of the key threads throughout this book. I am very grateful to Steenhagen for bringing my attention to it. I am also grateful for his intellectual companionship. We have discussed these and related issues over the years. Steenhagen also read some preliminary drafts of early chapters which helped me to improve them greatly, for which I am also indebted.

Craig French also read drafts of two chapters. The level-headed clarity of his comments, and more than that, the demand that I too should sometimes display such clarity, prompted considerable improvement, and for that I am most grateful. I am also very grateful to have had the opportunity to discuss the nature of perception with French over a number of years. I have learned a lot. Though I doubt he wholly approves of how I have applied what I have learned.

I have long wondered whether extramission theories of perception, though false if interpreted as causal models of perception, might, nonetheless, express some phenomenological truth. A serendipitous encounter with Keith Allen introduced me to the research of \citet{Winer:1996as}. Allen also pointed out this research's relevance to a passage in Merleau-Ponty. This provided renewed impetus to think about the phenomenological underpinnings of extramission and chapter~\ref{cha:vision} is the result. I am also grateful to Allen for discussions, over the years, about color and the nature of perception.

Clare Mac Cumhail provided another serendipitous encounter in reminding me of a passage in Hans Jonas that plays a key role in chapter~\ref{cha:vision} for which I am grateful as well. My colleague Sarah Richmond, upon encountering me in the hallway clutching a copy of Maine de Biran's \emph{Influence de l'habitude sur la facult\'{e} de penser}, pointed out to me some relevant passages in Sartre which proved very useful and for which I am most grateful.

A not unsympathetic, if not exactly credulous, audience at the University of Glasgow to whom I presented material culled from chapters \ref{cha:grasping} and \ref{cha:sympathy} in 2014 provided much needed feedback and prompted considerable improvement. I would especially like to thank Fiona MacPherson for her comments on that occasion.

I owe a debt of gratitude to Charles Travis for his friendship, intellectual companionship, and encouragement. His encouragement proffered at an early critical period kept me motivated, and for that, I am especially grateful. I am also indebted to Matt Soteriou who also generously proffered encouragement at a critical period.

Mike Martin has been a friend and colleague since I first arrived at UCL. My discussions with him about the nature of perception have been invaluable. Though, as in the case of French, I doubt he would approve of the application of his insights, which must appear in the text as if reflected through a glass darkly. Sometimes, as I wrote, I fancied that I could hear a Humean growling somewhere. Is it wrong to give thanks when, perhaps, an apology is due?

% Mark Johnston first got me thinking about the nature of perception as a graduate student in Princeton in the 1990s, and his work has been a persistent inspiration. He used to complain, when I was there, that the graduate students weren't adventurous enough. A lesson I perhaps learned only in middle age. Though it is testimony to him as a teacher that I continue to learn from him, albeit slowly. And not only about the boldness with which one should pursue speculative metaphysics, but about the seriousness of that enterprise as well.

Greenwich Park is a ten minute walk from where I live in Blackheath. As I composed the present work, I walked through that park almost daily. In a Peripatetic fashion, much of my thinking was done on these walks. And before I even embarked upon the present work, Plotinus' \emph{Enneads}, an inspiration to much of what follows, were read, for the most part, in the rose garden of Greenwich Park. It is perhaps unsurprising, then, that the park emerges as a minor character in the examples that I give. Let these remarks serve as both an acknowledgement and expression of gratitude.  

Finally, I would like to thank the readers for the press who provided detailed  and insightful comments. I have learned a lot from these, and I am very grateful for the spur they provided. I would also like to thank Hilary Gaskin for her help and encouragement in seeing the present essay into print.

% chapter acknowledgements (end)