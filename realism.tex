%!TEX root = /Users/markelikalderon/Documents/Git/sympathy/perception.tex
\chapter{Realism} % (fold)
\label{cha:realism}

\section{Grasping and the Rhetoric of Objectivity} % (fold)
\label{sec:grasping_and_the_rhetoric_of_objectivity}

Haptic perception plays a privileged role in the rhetoric of objectivity. In chapter~\ref{sec:a_puzzle}, we discussed two historical exemplars of this rhetorical impulse, the Giants shaking trees and boulders at the Friends of the Forms as they affirm their materialism, and Dr Johnson's kicking the stone outside of the church in Harwich as an exasperated affirmation of its material existence independent of our ideas. While by no means dead, this rhetorical trope has, perhaps, lost some of its sheen in giving birth to the late twentieth-century clich\'{e} of the table-pounding realist.

I once attended a lecture, where the speaker literally pounded on the podium at each mention of an objective worldly correlate of our conceptual scheme. Through this performance, the philosopher was expressing the objectivity of the worldly correlate, and, in a rather bullying fashion, demanding our assent to it. As with Dr Johnson's performance, it was a multimodal affair (chapter~\ref{sec:a_puzzle}, \citealt[71]{Campbell:2014aa}). The audience, in sympathetically responding to the philosopher's tactile experience, is meant to vividly experience the tangible resistance of the podium, revealed, in part, in the loud, sharp sound it produces when pounded. It is this resistance to touch that is meant to reveal the podium to be objectively there, independently of the speaker's pounding, just as the worldly correlate is meant to be there, independently of our conceptual scheme.

Haptic perception plays a privileged role in the rhetoric of objectivity. It does so, in part, because the experience of felt resistance to touch is phenomenologically vivid and primitively compelling. Though in no doubt about the presence or solidity of a thing, we may, nevertheless, be drawn to touch it. Thus we must endeavour to teach our children to keep their hands to themselves, and even in maturity, polite notices are required to remind adults to not touch the display cabinet. Haptic perception plays a privileged role in the rhetoric of objectivity, in part, because the experience of felt resistance to touch is phenomenologically vivid and primitively compelling. Moreover, it plays a privileged role, as well, because grasping provides a model for perceptual objectivity quite generally, in the assimilation of the hand, and the haptic experience it gives rise to, to the object of haptic investigation.

% section grasping_and_the_rhetoric_of_objectivity (end)

\section{Perceptual Objectivity} % (fold)
\label{sec:perceptual_objectivity}

That perception assimilates to its object is the manifestation of its objectivity. The assimilation is a formal rather than material. The conscious qualitative character of the perceptual experience becomes like the presented object without materially absorbping it. Moreover, the formal assimilation of perception to its object is not exact in the way that would entail the sharing of qualities. The experience of our hominid ancestor, in seeing the black obelisk, does not itself become black. Perception is not chameleon-like, as Crathorn imagined, and Holcort complained of. 

An ineliminable source of the inexactness of perception's formal assimilation consists in its perspectival relativity. Perception only formally assimilates to its object relative to the perceiver's partial perspective. I have not done enough to defend the general claim that all sensory experience is perspectival. I have not, for example, argued that olfaction is perspectival. I have, however, argued that there are haptic, auditory, and, of course, visual perspectives. Each allows for better or worse perspectives, and each involves the potential disclosure of previously hidden aspects of an object. Haptic, auditory, and visual perspectives are also, importantly, distinguished. While each structures a space, not only may the space differ, be it peripersonal or extrapersonal space, but the manner of its structuring may differ as well.

Perception only provides a partial perspective on the natural environment. The object of perception may not be wholly present to the perceiver's partial perspective of it. To that extent, their perception is imperfect in the sense of being incom\-plete---there are perceptible aspects of the object not disclosed to the perceiver's perspective. As \citet[]{Merleau-Ponty:1964aa} stresses, however, perception is not imperfect in a further, normative sense:
\begin{quote}
	But in immediate consciousness this perspectival character of my knowledge is not conceived as an accident in its regard, as an imperfection relative to the existence of my body and its proper point of view; and knowledge by ``profiles'' is not treated as the degradation of a true knowledge which would grasp the totality of the possible aspects of the object all at once. Perspective does not appear to me to be a subjective deformation of things but, on the contrary, to be one of their properties, perhaps their essential property. It is precisely because of it that the perceived possesses in itself a hidden and inexhaustible richness, that it is a ``thing.'' \ldots\ Far from introducing a coefficient of subjectivity into perception, it provides it on the contrary with the assurance of communicating with a world which is richer than what we know of it, that is, of communicating with a real world. The profiles of my desk are not given to direct knowledge as appearances without value, but as ``manifestations'' of the desk. \citep[186]{Merleau-Ponty:1964ab}
\end{quote}
Far from being an obstacle to perception's objectivity, the perspectival character of perception is what makes possible its objective disclosure of the natural environment. Objectivity and the parochial are linked (for an insightful exploration of this theme, though not within the philosophy of perception, see \citealt{Travis:2011qd}).

Not only does perception formally assimilate to its object, relative to the perceiver's partial perspective, but this formal assimilation is a kind of constitutive shaping. The object present in perceptual experience constitutively shapes that experience, the way that St Paul's constitutively shapes the London skyline. That skyline is like, in certain respects, St Paul's by virtue of St Paul being a part or contour of that skyline. Perceptual experience is like, in certain respects, its object by virtue of that object being a constituent of that experience. Constitutive shaping entails formal assimilation, though formal assimilation need not involve constitutive shaping. Consider Locke on primary quality perception. In perceiving a primary quality, the perceiver's experience resembles its object, but not by having that object as a constituent. The object constitutively shaping the perceiver's perceptual experience of it, is, as \citet{Ardley:1958aa}, stressed, the result of the perceiver's ``communion'' with that object. It is the unity of the perception with its object that ultimately explains the similarity between the conscious qualitative character of perceptual experience and the qualitative character of the object presented to the perceiver's partial perspective.

It is because this feature of grasping or enclosure, understood as a mode of haptic perception, generalizes to other forms of perception, such as vision and audition, that grasping is an apt metaphor for perception, more generally. If this feature carries over to perception generally, if the object of perception constitutively shapes the perceiver's perceptual experience, then it is easy to see its epistemic significance. If perception involves becoming like the perceived object actually is, then it is a genuine mode of awareness. One can only perceptually assimilate what is there to be assimilated. If perceptual experience is a formal mode of assimilation understood as a mode of constitutive shaping, then one could not undergo such an experience consistent with a Cartesian demon eliminating the object of that experience. If there is no external object, then there is nothing to which the perceiver, or perhaps their experience, could assimilate to. If the phenomenological character of perception is constitutively shaped by the object presented to the perceiver’s partial perspective, then we can begin to see the epistemic significance of perceptual phenomenology. If the phenomenological character of perception is constitutively shaped by the object presented to the perceiver’s partial perspective, then it is the grounds for an epistemic warrant for the range of propositions whose truth turns on what is presented in that perceptual experience \citep{Johnston:2006uq,Johnston:2011fk,Kalderon:2011fk}.

The present metaphysics of perception, while inconsistent with a Cartesian demon eliminating the object of experience, is not, by itself, sufficient to refute skepticism. Even conceding the conception of perceptual experience as a kind of formal assimilation understood as a mode of constitutive shaping, that conception is nevertheless consistent with the possibility of ringers. This possibility can arise in two ways. The objects of perception, what we perceive, may have ringers. I may see Castor and shake his hands, but his twin, Pollux, is a dead ringer. Moreover, the perceptual episode may itself admit of ringers. A perception of Pollux is a ringer for a perception of Castor, as is a perfectly matching hallucination of Castor. And a skeptic might try to exploit this latter possibility to undermine the epistemic warrant afforded by perception, a warrant not shared with its ringers. (Though this is not the place to go into it, I doubt very much that these further skeptical maneovers could succeed. For a sense of this, see how these ideas work themselves out in the tradition of Oxford realism as discussed in \citealt{Kalderon:2010fk}.)

Perception is a fundamental form of objectivity in our cognitive economy since it affords us explicit awareness of aspects of the natural environment. It is not a fundamental form of objectivity in our cognitive economy by being a primitive form of objective representation as \cite{Burge:2010uq} contends. Rather it is a sensory assimilation to its object. After all, something can only assimilate to what is there to be assimilated. Perception involves the objective presentation of its object in the explicit awareness afforded by experience. That object, of which the perceiver is explicitly aware, is only subsequently re-presented, if at all, in imagination and memory. \emph{Pace} Burge, I favor, instead, the Peripatetic doctrine that imagination and memory, and not perception, are the basic forms of intentional or representational capacities in our cognitive economy. That is, we are presented aspects of the natural environment in our perceptual experience of it, and these aspects are only subsequently re-presented, if at all, in imagination and memory. Perception is the basis of an epistemic warrant not by making the perceiver aware of a truth, though recognition of what one is perceiving may afford such awareness. Rather perception affords awareness of those aspects of the natural environment upon which the truth of a variety of propositions or thoughts depend.

Moreover, this awareness, the explicit awareness afford by perception of the natural environment, as opposed to awareness of truths about that environment, is epistemically distinctive. Information can go stale. What once passed for knowledge may accrete into dogma if the world changes without a corresponding change in cognitive state. The explicit awareness afforded by perceptual experience, in contrast, keeps the perceiver \emph{au courant} with their environment. In disclosing, partially and imperfectly, that environment, their perceptual experience will change with every change of what is presented in it. If timeliness is important in your practical circumstances, perception offers a distinct advantage over, not only belief, but what passes for knowledge. If you value timeliness, given your practical circumstances, if being \emph{au courant} with some aspect of the natural environment is of particular practical significance, then you should keep an eye on it or, at least, be perceptually vigilant, more generally.

I have been discussing perceptual objectivity and its epistemic significance. Specifically, I have been spelling out the conception of perceptual objectivity that one arrives at once one conceives of perception, in the hylomorphic fashion, in terms of the assimilation of form without matter, and the epistemic significance of the resulting conception of perceptual objectivity. However, much of what was claimed would remain true on any conception of perception that postulates an indispensable presentational element. Thus, for example, \citet{McDowell:2008fk} believes that perception affords the perceiver with a non-propositional mode of awareness that grounds an epistemic warrant. McDowell neither endorses a hylomorphic conception of sensory presentation nor event entertains its neo-Platonic elaboration in terms of sympathy. However, the metaphysics of sensory presentation that I have defended, offers, I claim, a further, distinct possibility.

If perceptual presentation is sympathetic presentation, then perception places us into the heart of things and allows us to experience them from within.  Perhaps sympathy at work in fellow-feeling would provide the most vivid and suggestive example. Fellow-feeling involves feeling along with the object of sympathy. One experiences their plight from within. The sympathetic presentation of an object in perceptual experience involves the perceiver placing themselves in the object, coinciding with it, and so experiencing it from within. The present account of sensory presentation in terms of sympathy naturally belongs to the broader neo-Platonic heritage of thinking of perception as placing the perceiver in its object. There is a way in which the present account, where sensory presentation is governed by the principle of sympathy, can make sense of this neo-Platonic heritage, though perhaps it is not the only way. On the understanding of this neo-Platonic heritage afforded by sympathy, perception presents how things are from within. Sympathy makes possible the presentation of a thing's inner nature, and thus one may perceive how a thing is in itself. Echoing Herbart, we may say ``the world is a world of things-in-themselves [and] the things-in-themselves are perceivable.'' If perceptual presentation is governed by the principle of sympathy, then the distinction between the phenomenal and the noumenal collapses.

% section perceptual_objectivity (end)

% \section{Bergson's Critique of Kant} % (fold)
% \label{sec:bergson_s_critique_of_kant}
%
% \begin{quote}
% 	By intuition is meant the kind of \emph{intellectual sympathy} by which one places oneself within an object in order to coincide with what is unique in it and consequently inexpressible. (Bergson, \emph{Introduction to Metaphysics}, 7)
% \end{quote}
%
% % section bergson_s_critique_of_kant (end)

\section{Perceiving Things in Themselves} % (fold)
\label{sec:perceiving_things_in_themselves}



% section perceiving_things_in_themselves (end)

% chapter realism (end)