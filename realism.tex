%!TEX root = /Users/markelikalderon/Documents/Git/sympathy/perception.tex
\chapter{Realism} % (fold)
\label{cha:realism}

\section{Grasping and the Rhetoric of Objectivity} % (fold)
\label{sec:grasping_and_the_rhetoric_of_objectivity}

Haptic perception plays a privileged role in the rhetoric of objectivity. In chapter~\ref{sec:a_puzzle}, we discussed two historical exemplars of this rhetorical impulse, the Giants shaking trees and boulders at the Friends of the Forms as they affirm their materialism, and Dr Johnson's kicking the stone outside of the church in Harwich as an exasperated affirmation of its material existence independent of our ideas. While by no means dead, this rhetorical trope has, perhaps, lost some of its sheen in giving birth to the late twentieth-century clich\'{e} of the table-pounding realist.

Being a clich\'{e} is no proof against existence. I once attended a lecture, where the speaker pounded on the podium at each mention of an objective worldly correlate of our conceptual scheme. Through this performance, the philosopher was expressing the objectivity of the worldly correlate, and, in a rather bullying fashion, demanding our assent to it. Like Dr Johnson's performance, it was a multimodal affair (chapter~\ref{sec:a_puzzle}, \citealt[71]{Campbell:2014aa}). The audience, in sympathetically responding to the philosopher's tactile experience, is meant to vividly experience the tangible resistance of the podium, revealed, in part, in the loud, sharp sound it produced when pounded. It is this resistance to touch that is meant to disclose the podium to be objectively there, independently of the speaker's pounding, just as the worldly correlate is meant to be there, independently of our conceptual scheme.

Haptic perception plays a privileged role in the rhetoric of objectivity. It does so, in part, because the experience of felt resistance to touch is phenomenologically vivid and primitively compelling. Though in no doubt about the presence or solidity of a thing, we may, nevertheless, be drawn to touch it. Thus we must endeavour to teach children to keep their hands to themselves, and even in maturity, polite notices are required to remind adults to not touch the display cabinet. Aristotle discerns an existential concern in touch. While the distal senses, such as sight and audition, are for the well-being of an animal equipped with locomotion, touch is for that animal's very existence. Perhaps this existential dimension is part of what makes touch so primitively compelling. Haptic perception plays a privileged role in the rhetoric of objectivity, in part, because the experience of felt resistance to touch is phenomenologically vivid and primitively compelling. Moreover, and for our purposes more importantly, it plays a privileged role, as well, because grasping provides a model for perceptual objectivity quite generally, in the assimilation of the hand, and the haptic experience it gives rise to, to the object of haptic investigation.

% section grasping_and_the_rhetoric_of_objectivity (end)

\section{Perceptual Objectivity} % (fold)
\label{sec:perceptual_objectivity}

That perception assimilates to its object is the manifestation of its objectivity. Perceptual assimilation is formal rather than material. The conscious qualitative character of the perceptual experience becomes like, if not exactly like, the presented object without materially absorbing it. Moreover, the formal assimilation of perception to its object is not exact in the way that would entail the sharing of qualities. The experience of our hominid ancestor, in seeing the alien obelisk, does not itself become black. The qualitative character of their visual experience in seeing the obelisk may be like, in some sense, the blackness presented in it. But that blackness enjoys no natural existence in our hominid ancestor's perception of it the way it enjoys natural existence in the alien obelisk. Perception may be a capacity to become like, as Aristotle contends, but it is not chameleon-like, as Crathorn imagined, and Holcot complained of (chapter~\ref{sec:assimilation}). 

An ineliminable source of the inexactness of perception's formal assimilation to its object consists in its perspectival relativity. Perception only formally assimilates to its object relative to the perceiver's partial perspective (chapter~\ref{sec:assimilation}, chapter~\ref{sec:auditory_perspective}). I have not done enough to defend the general claim that all sensory experience is perspectival. I have not, for example, argued that olfaction is perspectival. I have, however, argued that, in addition to visual perspectives, there are, as well, haptic and auditory perspectives. Each allows for better or worse perspectives, and each involves the potential disclosure of previously hidden aspects of a sensible object. Moreover, each does so in an ego-centrically structured space. While there are similarities among them in virtue of which they each count as perspectives, haptic, auditory, and visual perspectives are also, importantly, distinct. While each structures a space, not only may the space differ, be it peripersonal or extrapersonal space, but the manner of its structuring may differ as well. Vision is rectilinear in a way that audition, in providing the perceiver with 360 degree awareness of the natural environment, is not.

Perception only provides a partial perspective on the natural environment. The object of perception may not be wholly present to the perceiver's partial perspective on it. To that extent, their perception is imperfect in the sense of being incom\-plete---there are perceptible aspects of the object not disclosed to the perceiver's perspective. As Merleau-Ponty stresses, however, perception is not imperfect in a further, normative sense:
\begin{quote}
	But in immediate consciousness this perspectival character of my knowledge is not conceived as an accident in its regard, as an imperfection relative to the existence of my body and its proper point of view; and knowledge by ``profiles'' is not treated as the degradation of a true knowledge which would grasp the totality of the possible aspects of the object all at once. Perspective does not appear to me to be a subjective deformation of things but, on the contrary, to be one of their properties, perhaps their essential property. It is precisely because of it that the perceived possesses in itself a hidden and inexhaustible richness, that it is a ``thing.'' \ldots\ Far from introducing a coefficient of subjectivity into perception, it provides it on the contrary with the assurance of communicating with a world which is richer than what we know of it, that is, of communicating with a real world. The profiles of my desk are not given to direct knowledge as appearances without value, but as ``manifestations'' of the desk. \citep[186]{Merleau-Ponty:1964ab}
\end{quote}
In most cases, the object of perception, in all its particularity, exceeds what is disclosed of it in perceptual experience. Touch provides a vivid example of this in what I earlier described as the allure of the tangible (chapter~\ref{sec:haptic_perception}). The allure of the tangible is the sense, or premonition, that, at any given moment, the body exceeds what is disclosed to us by touch. We have the sense, when touching an object, that it is tangibly determined in ways that we have yet to feel. Our tactile sense of a body's ``thingness''---its concrete particularity---consists, in part, in this allure. While, perhaps, particularly vivid in tactile phenomenology, Merleau-Ponty maintains that something like this is true of perceptual phenomenology more generally, that perception's partial disclosure is an objective manifestation of an object that exceeds what is disclosed of it in experience. Far from being an obstacle to perception's objectivity by introducing a coefficient of subjectivity into perception, the perspectival character of perception is what makes possible its objective disclosure of the natural environment. Objectivity and the parochial are linked (for an insightful exploration of this theme, though not within the philosophy of perception, see \citealt{Travis:2011qd}).

To bring out one way in which objectivity and the parochial may be linked in perception consider the limits to normal human color constancy. Human color constancy is imperfect. Not only does human color vision display constancy for only some scenes and some conditions of illumination, but human color vision displays different degrees of constancy in different kinds of scenes in different ranges of illumination. Human color constancy is imperfect in that it displays these various kinds of incompleteness. Hilbert explains how human color constancy is imperfect in a further important sense:
	\begin{quote}
		Many theories of color constancy take the form of explaining how it is that the visual system manages to extract information about the reflectance of the objects in a scene from the color signal from those objects. Since this involves separating the contributions of the reflectance and the illuminant to the color signal these theories are often characterized as ``discounting the illuminant''. Perfect color constancy in these terms would involve accurate recovery of reflectance for any scene under any lighting conditions. The perceived color of objects would be perfectly correlated with their reflecting characteristics and not vary at all with changes in the illuminant of the composition and arrangement of objects in view. This type of perfect color constancy is not possible. \citep[143]{Hilbert:2007qy}
	\end{quote}
Human color constancy is imperfect. As Merleau-Ponty emphasizes, this should not be thought of as a deficit. Suppose there could be a perceiver whose perception displayed perfect color constancy in Hilbert's sense. What would it be like for them to see a field of grass set against a blue summer sky? The field would appear uniformly green and the sky uniformly blue. Moreover, no difference in color appearance would differentiate any portion of the uniformly green field. The experience of the scene would be not unlike a young child's drawing of the scene. The grass would be uniformly green and lack the golden cast that we might observe in viewing the same scene, nor would it be dappled, as we observe the scene to be, by sunlight and shadow. Furthermore, no difference in color appearance would differentiate any portion of the uniformly blue sky. The sky would be uniformly blue and would manifest no deepening azure to the east. Children's drawings also intimate what perfect size constancy might be like---they will draw a car as larger than an adult even if the car is at a great distance from that person. Just as with perfect size constancy we would lose information about distance, so with perfect color constancy we would lose information about the illuminant. So the partial and variable character of human color constancy is no deficit. And not merely because it lacks the garish character of children's crayon drawings, but because we would be insensitive to important aspects of our environment. Our environment is only objectively disclosed to the partial perspective we have on it.

Not only does perception formally assimilate to its object, relative to the perceiver's partial perspective, but this formal assimilation is a kind of constitutive shaping. The object present in perceptual experience constitutively shapes that experience, the way that St Paul's constitutively shapes the London skyline. What that skyline is like is determined, in part, by what St Paul's is like. St Paul's determines what the London skyline is like, at least in part, by virtue of being a part or contour of that skyline. Similarly, what the perceiver's experience of an object is like is determined, in part, by what that object is like. The object of perception determines what the perceiver's experience of it is like, at least in part, by virtue of being a constituent of that experience. Constitutive shaping entails formal assimilation, though formal assimilation need not involve constitutive shaping. Consider Locke on primary quality perception. In perceiving a primary quality, the perceiver's experience resembles its object, but not by having that object as a constituent. The object constitutively shaping the perceiver's perceptual experience of it, is, as \citet{Ardley:1958aa}, stressed, the result of the perceiver's ``communion'' with that object. It is the unity of the perception with its object that ultimately explains the similarity between the conscious qualitative character of perceptual experience and the qualitative character of the object presented to the perceiver's partial perspective. (Recall, according to Plotinus, it is because of the unity provided by the World-Soul that potentially distant parts of the sensible cosmos that are suitably disposed to become like or unlike may sympathetically interact. See chapter~\ref{sec:plotinus}. For the generalization and application of this point to the case of haptic perception see chapter~\ref{sec:sympathy_as_the_principle_of_haptic_presentation}.)

It is because this feature of grasping or enclosure, understood as a mode of haptic perception, generalizes to other forms of perception, such as vision and audition, that grasping is an apt metaphor for perception, more generally. If this feature carries over to perception generally, if the object of perception constitutively shapes the perceiver's perceptual experience, then it is easy to see its epistemic significance. If perception involves becoming like the perceived object actually is, then it is a genuine mode of awareness. One can only perceptually assimilate what is there to be assimilated. If perceptual experience is a formal mode of assimilation understood as a mode of constitutive shaping, then one could not undergo such an experience consistent with a Cartesian demon eliminating the object of that experience. If there is no external object, then there is nothing to which the perceiver, or perhaps their experience, could assimilate to. If the phenomenological character of perception is constitutively shaped by the object presented to the perceiver’s partial perspective, then we can begin to see the epistemic significance of perceptual phenomenology. If the phenomenological character of perception is constitutively shaped by the object presented to the perceiver’s partial perspective, then it is the grounds for an epistemic warrant for the range of propositions whose truth turns on what is presented in that perceptual experience \citep{Johnston:2006uq,Johnston:2011fk,Kalderon:2011fk}.

The warrant, here, should be understood as an entitlement to judge (in the ordinary sense of ``entitlement'' and not in Burge's \citeyear{Burge:2003fk} technical sense of the term; compare \citealt[132n]{McDowell:2009ys}). Entitlements may be possessed without being exercised. In being aware of some aspect of the natural environment, the perceiver may possess an epistemic warrant that entitles them to know various things without the perceiver, in fact, coming to know these things. The perceiver is knowledgeable of the object of perception in the sense that knowledge is \emph{available} to the subject in perceiving the object, whether or not such knowledge is in fact ``activated'' (in Williamson's \citeyear{Williamson:1990uq} terminology). The epistemic warrant grounded in perceptual awareness is not a factor in terms of which knowledge could be analyzed or otherwise explained. Moreover, it is an \emph{epistemic} entitlement: The object of awareness is an epistemic warrant for the range of propositions whose truth turns on what the perceiver is aware of. Perception confers this epistemic entitlement given the alethic connection between the particular that is the object of perceptual awareness and the proposition potentially known. Awareness of the sensible particulars affords the subject with a reason that is in this way akin to proof---it is logically impossible for the particular to exist and the proposition to be false \citep[see][]{Cook-Wilson:1926sf,Kalderon:2010fk,Travis:2005kx}. Because in seeing the bright green burrs of the ancient chestnut tree, I possess a reason that would, in the given circumstance, warrant my coming to know that the burrs are bright green, I am authoritative about the color of the chestnut tree's burrs. My seeing the bright green of the burrs can stand proxy for any inquiry on your part about the color of the burrs. If in coming to know that the burrs are bright green, I express my knowledge by stating it, I extend to you an offer to take it on my authority that the burrs are the color that I see them to be. 

The present metaphysics of perception, while inconsistent with a Cartesian demon eliminating the object of experience, is not, by itself, sufficient to refute skepticism. Even conceding the conception of perceptual experience as a kind of formal assimilation understood as a mode of constitutive shaping, that conception is nevertheless consistent with the possibility of ringers. This possibility can arise in two ways. The objects of perception, what we perceive, may have ringers. I may see Castor and shake his hand, but his twin, Pollux, is a dead ringer. Moreover, not only do the objects of perception, those sensible aspects of the natural environment that we encounter in experience, admit of ringers, but our perceptual episodes, our experiences, may themselves admit of ringers. A perception of Pollux is a ringer for a perception of Castor, as is a perfectly matching hallucination of Castor. And a skeptic might try to exploit this latter possibility to undermine the epistemic warrant afforded by perception, a warrant not shared with its experiential ringers. The mere existence of experiential ringers is, by itself, insufficient for the skeptic's conclusion. The skeptic would need, in addition, the claim that if a perceptual episode affords the perceiver with epistemic warrant, it must not admit of ringers that do not. So conceived, epistemic warrant requires ringerless proof. Though this is not the place to go into it, I doubt very much that these further skeptical maneovers could succeed. They rely on an overdemanding conception of epistemic warrant that is difficult to coherently maintain. (For a sense of this, see how these ideas work themselves out in the tradition of Oxford realism as discussed in \citealt{Kalderon:2010fk}, especially their discussion of ``the accretion''. See also Williamson's \citeyearpar{Williamson:2000lr} discussion of luminosity. For more on Oxford realism, see \citealt{Marion:2000al,Marion:2000bi}.)

Perception is a fundamental form of objectivity in our cognitive economy since it affords us explicit awareness of sensible aspects of the natural environment. It is not a fundamental form of objectivity, however, by being a primitive form of objective representation as \cite{Burge:2010uq} contends. Sensory awareness is a mode of assimilation, and something can only assimilate to what is there to be assimilated. Consider the following analogy. Knowledge is factive, let us suppose. If the perceiver knows something, then there is some fact that they know. If there is no fact that they know, then there is nothing that they know. Similarly we might say that perception is objective. If the perceiver perceives something, then there some object that they perceive (``object'', here, is not the ontological category, but perception's \emph{terminus}, namely, what is perceived). If there is no object that they perceive, then there is nothing that they perceive. Perception involves the objective presentation of its object in the explicit awareness afforded by that experience. That object, of which the perceiver is explicitly aware, is only subsequently re-presented, if at all, in imagination and memory. \emph{Pace} Burge, I favor, instead, the Peripatetic doctrine that imagination and memory, and not perception, are the basic forms of intentional or representational capacities in our cognitive economy. That is, we are presented aspects of the natural environment in our perceptual experience of it, and these aspects are only subsequently re-presented, if at all, in imagination and memory. Perception is the basis of an epistemic warrant not by making the perceiver aware of a truth, though recognition of what one is perceiving may afford such awareness. Rather perception affords awareness of those aspects of the natural environment upon which the truth of a variety of propositions depend.

Moreover, sensory awareness, the explicit awareness afford by perception of the natural environment, as opposed to awareness of truths about that environment, is epistemically distinctive. Information can go stale. What once passed for knowledge may accrete into dogma if the world changes without a corresponding change in cognitive state. The explicit awareness afforded by perceptual experience, in contrast, keeps the perceiver \emph{au courant} with their environment \citep[173--174]{Travis:2013tk}. In disclosing, partially and imperfectly, that environment, their perceptual experience will change with every change of what is presented in it. If timeliness is important in your practical circumstances, perception offers a distinct advantage over, not only belief, but what passes for knowledge. If you value timeliness, given your practical circumstances, if being \emph{au courant} with some aspect of the natural environment is of particular practical significance, then you should keep an eye on it or, at the very least, be perceptually vigilant, more generally.

That perception assimilates to its object, in the sense that it does, is due, in part, to the activity of the perceiver. Perhaps that is why Olivi describes the outward extensive activity of perception as being, at the same time, a formative absorption toward its object.  The haptic experience of our hominid ancestor only assimilates to the stone thanks to the activity of their hand's grasp and the resistance it encounters. It is when the limit to the hand's activity is experienced as a sympathetic response to an alien force, like it yet distinct from it, that the stone is presented in their grasp. Haptic touch discloses the overall shape and volume of the stone by grasping it. Its roughness is disclosed by feeling it. Its heft, by weighing it. Grasping, feeling, weighing, listening, and looking are all outward extensive activities by which the perceiver opens themselves up, in a directed manner, to the sensible, in all its varieties. The activity of the perceiver is a necessary precondition for the objective disclosure of the perceived object. For it is the resistance that such activity encounters in the natural environment that makes possible the sympathetic presentation of the sensible world without.

I have been discussing perceptual objectivity and its epistemic significance. Specifically, I have been spelling out the conception of perceptual objectivity that one arrives at once one conceives of perception, in the hylomorphic fashion, in terms of the assimilation of form without matter, as sustained by the perceiver's activity, and the epistemic significance of the resulting conception of objectivity. However, much of what was claimed would remain true on any conception of perception that merely postulates an indispensable presentational element. Thus, for example, \citet{McDowell:2008fk} believes that perception affords the perceiver with a non-propositional mode of awareness that grounds an epistemic warrant, understood as an epistemic entitlement. McDowell neither endorses a hylomorphic conception of sensory presentation (though he does help himself to the Peripatetic metaphor of shaping, \citealt{McDowell:1998vn}) nor even entertains its neo-Platonic elaboration in terms of sympathy with the natural environment that resists the force of the perceiver's activity. However, the metaphysics of sensory presentation that I have defended offers not only an explanation of the epistemic significance of perceptual phenomenology in the form of an analytic explication of its intelligible structure (chapter~\ref{sec:sympathy_and_haptic_perception}), but it also offers a further, distinct possibility.

If perceptual presentation is sympathetic presentation, then perception places us into the very heart of things, thus allowing us to experience them from within.  Perhaps the sympathy at work in fellow-feeling would provide the most vivid and suggestive example. Fellow-feeling involves feeling along with the object of sympathy. One experiences their plight from within. The sympathetic presentation of an object in perceptual experience involves the perceiver placing themselves in the object, coinciding with it, and so experiencing it from within. The present account of sensory presentation in terms of sympathy naturally belongs to the broader neo-Platonic heritage of thinking of perception as placing the perceiver in its object. There is a way in which the present account, where sensory presentation is governed by the principle of sympathy, can make sense of this neo-Platonic heritage, though perhaps it is not the only way. On the understanding of this neo-Platonic heritage afforded by sympathy, perception presents how things are from within. Sympathy makes possible the presentation of a thing's inner nature, and thus one may perceive how a thing is in itself. Echoing Johann Friedrich Herbart, we may say that the world is a world of things in themselves and things in themselves are perceptible. Things in themselves are what appear in our perceptual experience. They are the objects of sensory awareness. If perceptual presentation is governed by the principle of sympathy, then the distinction between the phenomenal and the noumenal collapses.

% section perceptual_objectivity (end)


\section{Kantian Humility} % (fold)
\label{sec:kantian_humility}

Sympathy allows the perceiver to experience the presented object from within. In sympathetically presenting that object in their experience of it, the perceiver coincides with that object and experiences how that thing is in itself, its inner nature, albeit imperfectly, from a partial perspective. We can begin to make sense of these claims through a critical examination of Langton's \citeyearpar{Langton:1998aa} defense of Kantian Humility.

Consider the following puzzle for Kant's position, first raised by Jacobi. According to Kant, things in themselves exist and are the cause of phenomenal appearances. But, if Kant's critical philosophy is correct, then it would seem that we can have no knowledge of things in themselves. But if we have no knowledge of things in themselves, then how could we know that they exist and are the cause of phenomenal appearances? This puzzle prompts \citet[304]{Jacobi:1815bs} to remark of the first edition of the \emph{Kritik der reinen Vernunft} that without the presupposition of the thing in itself I ``cannot enter into the system, yet \emph{with} this presupposition I cannot remain in it'' (\citealt[335]{Guyer:1987xe}; for discussion of Jacobi's puzzle see \citealt[247--54]{Allison:1983ly}, \citealt[chapter 15]{Guyer:1987xe}, \citealt[chapter 1]{Langton:1998aa}).

Langton's interpretation of Kant provides a straightforward solution by qualifying our ignorance of things in themselves. The qualification proceeds on the back of a metaphysical interpretation of the distinction between phenomena and things in themselves. Things in themselves are substances that have intrinsic properties whereas phenomena are relational properties of substances \citep[20]{Langton:1998aa}. Our ignorance pertains not to the existence of things in themselves, nor to their relational effects, such as their causing in us of phenomenal appearances, but to their inner natures. We cannot know how things are in themselves. We cannot know, specifically, their intrinsic properties \citep[13]{Langton:1998aa}. 

Kantian Humility is the name that Langton bestows upon the doctrine that we cannot know how things are in themselves, that we are irredeemably ignorant of the intrinsic natures of things in themselves. Part of the interest of Langton's book is not just the interpretation of Kant she provides, but her conviction that Kant, so interpreted, might just be right. The case Langton makes for Kantian Humility inspired \citet{Lewis:2009ax} to construct a Ramseyan variant. (For discussion of Ramseyan and Kantian Humility, from a standpoint that similarly takes perception to have an indispensable and irreducible presentational element, see \citealt{Brewer:2011ks}.) What case for Kantian Humility does Langton claim that we can find in Kant's writing?

In the \emph{Bounds of Sense}, Strawson doubts that there is any such case to be found:
\begin{quote}
	Knowledge through perception of things existing independently of perception, as they are in themselves, is impossible. For the only perception which could yield us any knowledge at all of such things must be the outcome of our being affected by those things; and for this reason such knowlege can be knowledge only of those things as they appear---of the appearances of those things---and not of those things as they really are or are in themselves. The above is a fundamental and unargued complex premise of the \emph{Critique}. \citep[250]{Strawson:1966kx}
\end{quote}
Strawson, however, hints at potential grounds for Kantian Humility in the receptivity of human sensibility, its propensity to be affected from without. Indeed it is partly on these grounds that Langton herself sees a case for Kantian Humility.

According to Langton, Kant's case for Kantian Humility rests upon another doctrine of Kant's:
\begin{quote}
	the \emph{receptivity} of our mind, its power of receiving representations in so far as it is in any way affected, is called sensibility \ldots\ Our nature is so constituted that our intuition can never be other than sensible, that it contains only the way in which we are affected by objects. (Kant, \emph{Kritik der reinen Vernunft}, A51/B75; \citealt[93]{Smith:1965jw})
\end{quote}
From this and other passages, Langton attributes to Kant the thesis she describes as Receptivity:
\begin{quote}
	Human knowledge depends on sensibility, and sensibility is receptive: we can have knowledge of an object only in so far as it affects us. \citep[125]{Langton:1998aa}
\end{quote}
Langton's Kant is driven to embrace Kantian Humility, in part, by working out the consequences of Receptivity for human knowledge. Specifically, Langton sees the case for Kantian Humility as resting upon the distinction between the phenomenal and the noumenal (on its metaphysical interpretation), the irreducibility of relational properties to intrinsic properties (an issue she sees as at stake between Leibniz and Kant, \citealt[chapters 4 and 5]{Langton:1998aa}), and Receptivity as formulated above. 

That things in themselves cause in human subjects phenomenal appearances is a phenomenal, that is to say, relational feature of these substances. These phenomenal appearances might yet acquaint human subjects with how things are in themselves if relations somehow reduced to intrinsic properties. But no such reduction is in the offing \citep[chapter 5]{Langton:1998aa}. So it would seem that perception, being essentially receptive, only affords human subjects with knowledge of the phenomenal features of the world, of the relational properties of substances whose intrinsic nature remains forever hidden from us.

I must confess to a lingering Strawsonian worry. Langton's argument only works on the assumption that the object of perception is relational in character. Without that assumption, the argument simply has no grip. To see how there may be a further issue here, look closely at the difference between Langton's official formulation of Receptivity and the passage from A51/B75. According to Receptivity, we can perceive an object only insofar as it affects us. That is a relatively weak claim. Perhaps only Olivi, Malebranche, and Leibniz deny it. However, among our predecessors who accept that claim, many would deny that the content of perception is restricted to the relational properties of substances. Notice, however, how the claim in A51/B75 is stronger. Kant claims that our nature is such that sensible intuition ``contains only the way in which we are affected''. If sensible intuition contains only the way in which we are affected, then the content of a sensible intuition is restricted to the subject being affected from without. The way in which we are affected is a causal, relational feature. So the content of sensible intuition would be relational in the way required. The lingering Strawsonian worry concerns what grounds there could be for this stronger Kantian claim, for without it, Langton's case for Kantian Humility collapses. 

Why assume that the content of perception is restricted to relational properties of substances? It does not follow from the mere fact that perception requires being affected from without. So what grounds this restriction? The lingering Strawsonian worry is that this is a fundamental and unargued assumption of Langton's case for Kantian Humility. Notice Langton could not legitimately reformulate Receptivity in terms of the stronger Kantian language of A51/B75. That would have ``the advantages of theft over honest toil'' \citep[71]{Russell1919Introduction-to}. For suppose she did. Then since the content of perception is restricted to the relational properties of substances, the content of perception would exclude the intrinsic properties of substances. A perception, so conceived, would not be a way of becoming knowledgeable of the intrinsic properties of substances since these do not figure in its content. And given a minimal empiricism, that is tantamount to Kantian Humility. 

Suppose that Kant and Langton in fact provide no further grounds for this assumption. (For what it is worth, I, at least, can find no further grounds in their writing.) Perhaps the claim that the content of perception is restricted to relational properties of substances is grounded, not in an argument, but in an inability to conceive of the alternative. Perhaps in thinking about the passive reception of sensory impressions they could frame for themselves no positive conception of how, being affected thus, perception could present how things are in themselves, the intrinsic properties of substances.

I shall not here speculate on the source of this inability. (Though a more complete anti-Kantian polemic, of a similar scale and ambition as Prichard's \citeyear{Prichard:1909yg}, would provide a diagnosis of this.) Rather, I shall try to provide the wanted positive conception. Interestingly, doing so in the terms argued for in the present essay parallels, in certain respects, an anti-Kantian argument of Bergson's.

% section kantian_humility (end)

\section{Bergson contra Kant} % (fold)
\label{sec:bergson_contra_kant}

In ``Introduction \`{a} la m\'{e}taphysique'' \citet{Bergson:1903nx} marks a distinction between relative and absolute knowledge. Surprisingly, at least to readers of \emph{Mati\`{e}re et M\'{e}moire: essai sur la relation du corps \`{a} l'esprit}, Bergson counts perceptual knowledge as relative knowledge. It is hard to understand how perceptual knowledge being relative could be consistent with the conception of pure perception developed in chapter 1 of \emph{Mati\`{e}re et M\'{e}moire}, for there Bergson rejects indirect realism, arguing, instead, that pure perception, at least, directly acquaints us with its object \citep[though see][for a reconciliationist reading, 39--41]{Moore:1996rt}. That perceptual knowledge is relative is, perhaps, merely a dialectical concession to a Kantian opponent and not a claim that Bergson is himself endorsing. We shall not resolve this exegetical matter here, for our focus is not on relative knowlege, but on absolute knowledge and what, according to Bergson, makes that possible.

What is the distinction between relative and absolute knowledge? Bergson introduces the distinction this way:
\begin{quote}
	philosophers, in spite of their apparent divergencies, agree in distinguishing two profoundly different ways of knowing a thing. The first implies that we move round the object; the second that we enter into it. The first depends on the point of view at which we are placed and on the symbols by which we express ourselves. The second neither depends on a point of view nor relies on any symbol. The first kind of knowledge may be said to stop at the \emph{relative}; the second, in those cases where it is possible, to attain the \emph{absolute}. \citep[1]{Bergson:1912ud}
\end{quote}
Absolute knowledge, whatever else it might be \citep[for discussion see][chapter 6]{Lacey:1989bv}, involves knowledge of things in themselves precluded by Kantian Humility. How is such knowledge obtained? How may we enter into the object of knowledge and so know it absolutely?

It is impossible to obtain absolute knowledge of an object merely by integrating partial perspectives on that object into a harmonious, unified whole. ``Were all the photographs of a town, taken from all possible points of view, to go on indefinitely completing one another, they would never be equivalent to the solid town in which we walk about'' \citep[5]{Bergson:1912ud}. According to Bergson, one may come, instead, to have absolute knowledge by means of the faculty of intuition whose principle is sympathy:
\begin{quote}
	By intuition is meant the kind of \emph{intellectual sympathy} by which one places oneself within an object in order to coincide with what is unique in it and consequently inexpressible. \citep[7]{Bergson:1912ud}
\end{quote}
Intuition, here, is intellectual as opposed to sensible. Intuition involves a kind of intimate unity between the act of intuition and its object (and presumably, it displays a greater degree of unity then that at work in perception which yields only relative knowledge). Sympathy, as the principle of intuition, allows the thinker to enter into or coincide with the object of absolute knowledge. In this passage, Bergson makes the rather strong claim that one places oneself within the object in order to coincide with what is unique in it. And this, Bergson, claims, has the consequence that the content of that intuition is inexpressible. Bergson obviously thinks that what is expressible is a kind of generality. But intuition, in presenting what is unique in its object, lacks the kind of generality that would otherwise make it expressible (compare Aristotle \emph{De Interpretatione} 7 17\( ^{a} \)37--38, \emph{Categoriae} 2 1\( ^{a} \)20--1\( ^{b} \)9, \citealt[4]{Frege:1882uq}, \citealt[44]{Prichard:1909yg}, \citealt[52]{Lewis:1929fk}). That the content of the intuition is inexpressible might seem incompatible with its being intellectual (hence Russell's \citeyear{Russell:1912rt} charge of anti-intellectualism). But notice that there is precedent for this. According to Plotinus, the image of the hyperontic One intuited by the Intellect is a higher form of intelligibility than what can be expressed in discursive rationality. The intuition that apprehends the image is intellectual, but the image in manifesting a higher degree of intelligibility than what can be expressed in discursive rationality is to that extent inexpressible. Similarly, Bergson's thought is that the intuition that yields the absolute knowledge of metaphysics is at once intellectual and its content inexpressible. Notice how Bergson, in this passage, is cleaving to what I earlier described as a neo-Platonic heritage---in intuiting an object one places oneself within that object. This is, perhaps, no accident. Bergson regularly lectured on Plotinus. The important point for us is that Bergson took intellectual sympathy to explain the way intuition places one within its object. Thanks to the operation of sympathy, in intuition one experiences the object from within and so may gain absolute knowledge of it.

So, Bergson maintains, as against Kant, that absolute knowlege, knowledge of how things are in themselves, is possible on the basis of intuition and that sympathy makes this so. Some, admittedly, have been unimpressed. And not only \citet{Russell:1912rt}, who was writing as a polemicist, as was \citet{Stebbing:1914kx}. \citet[202]{Jay:1994aa}, by no means a polemicist working on behalf of an emerging analytic philosophy, for one, pronounces it lame. However, I suspect such judgments are not informed by an appreciation of the role of sympathy in Stoic and neo-Platonic physics. Bergson's thought was so informed, and his work is best appreciated when read in light of these ancient sources. We have endeavoured to understand haptic, auditory, and visual perception in terms of the operation of sympathy. If sensible intuition, though a lesser unity than intellectual intuition, if such there be, operates too by means of sympathy, could it not also disclose how things are in themselves, if partially and imperfectly, despite its perspectival character? Could not perception, so understood, make the perceiver knowledgeable about how things are in themselves?

% section bergson_contra_kant (end)

\section{Perceiving Things in Themselves} % (fold)
\label{sec:perceiving_things_in_themselves}

Throughout this essay I have argued that sensory presentation---at least as it occurs in haptic, auditory, and visual perception---is governed by the principle of sympathy. So sympathy has a broader domain of application than in an intellectual intuition that makes the absolute knowledge of metaphysics available as Bergson contends. Moreover, the operation of sympathy in sensory presentation is perspective relative. An object is only sympathetically presented to the perceiver from their partial perspective on the natural environment. However, the perspectival relativity of sensory presentation is no obstacle to its objectivity. As \citet{Merleau-Ponty:1967fj} stresses, it is, rather, a precondition of perceptual objectivity (chapter~\ref{sec:perceptual_objectivity}). Objectivity and the parochial are linked. 


There may be a higher degree of unity involved in an act of intellectual intuition, if such there be, than in a perceptual act. And this may be reflected in the fact that the content of intellectual intuition is more than just what would be disclosed in the totality of potential perspectives on its object. But that is not yet grounds for maintaining that perception discloses only the relations the perceiver bares to its object. All that really follows from the perspectival relativity of sensory presentation is that it is partial and imperfect, in the sense of being incomplete, if not in a normative sense that implies a kind of deficit. Sensory presentation may disclose the intrinsic features of things, but being partial and imperfect, it may disclose only some of these and with different degrees of acuity in different circumstances of perception.

Just as sympathy, as it operates in fellow-feeling, allows us to experience from within what another undergoes, sympathy, as it operates in perception, allows us to experience from within what something external to us is like. It is this aspect of sympathetic sensory presentation that vindicates what I earlier described as a neo-Platonic heritage, that perception places us in the object perceived. If sensory presentation operates by means of sympathy, then the sensory presentation of an object in perceptual experience is a way of entering into or coinciding with that object, albeit partially and imperfectly. Perception places us into the very heart of things and reveals their inner natures.

Consider again Olivi's and Merleau-Ponty's claim that in looking at a distal object, the perceiver's gaze is posed on that object. The active, outer-directed, opening up to the visible comes to rest on a distal body that resits this activity insofar as it can. It is only in experiencing the body's limit to the perceiver's visual activity, its resistance to the perceiver's gaze, its perceptual impenetrability, as a sympathetic response to a countervailing force, the perceiver's gaze encountering an alien force that resists it, that the perceptually impenetrably body discloses itself to visual awareness. If the visual resistance of the body is the means by which conflicting forces are sympathetically presented in visual experience, then in being sympathetically presented with a distal body, the perceiver is naturally attending to the distal body, the object of visual perception. The perceiver's gaze is posed on the body. This is the effect of the body's sympathetic presentation in visual experience that arises when the perceiver looks to that body. Perception places us in the body. That is where the perceiver's explicit awareness is. And the perceiver's explicit awareness, in alighting upon the distal body, may disclose its intrinsic properties. This is what the neo-Platonic heritage amounts to in the present account: Perception places us in the very heart of things in the sense that the explicit awareness afforded by perceptual experience is directed upon the body in such a way as to disclose its intrinsic properties, its color or shape, say, in that awareness. And it is the principle of sympathy that makes this possible.

That we must be affected in some way by the object of perception is no obstacle to the sympathetic presentation of a thing's intrinsic properties. Rather, as the Protagorean model reveals, at least as herein elaborated, the force of the perceiver's activity coming into conflict with the self-maintaining forces of the object perceived is what makes its sympathetic presentation possible. It is only when the perceiver experiences the limit to their perceptual activity as a sympathetic response to a countervailing force from without that sympathy may disclose what is external to us. What appears to us in perceptual experience are things in themselves, both in their relational and intrinsic aspects. The fallen burr resting upon the grass may be to the left of a foraging squirrel and bright green. Vision discloses such things to us. But if things in themselves are what appear to us in perceptual experience, then the phenomenal--noumenal distinction collapses (on a certain understanding of that distinction if not the metaphysical one that Langton endorses). A thing may at once be a thing in itself, a substance if you like, and appear in perceptual experience. More than that, how that thing is in itself may itself be disclosed, partially and imperfectly, in perceptual experience. What is disclosed is an aspect of the substance's inner nature, how that thing is in itself apart from other things, how it is intrinsically. We confront the burr's greenness in seeing it. Sympathy is what presents the world without the mind in sensory experience and discloses how things are in themselves, at least partially and imperfectly.

% section perceiving_things_in_themselves (end)

% chapter realism (end)
