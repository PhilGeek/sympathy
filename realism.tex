%!TEX root = /Users/markelikalderon/Documents/Git/sympathy/perception.tex
\chapter{Realism} % (fold)
\label{cha:realism}

\section{Grasping and the Rhetoric of Objectivity} % (fold)
\label{sec:grasping_and_the_rhetoric_of_objectivity}

Haptic perception plays a privileged role in the rhetoric of objectivity. In chapter~\ref{sec:a_puzzle}, we discussed two historical exemplars of this rhetorical impulse, the Giants shaking trees and boulders at the Friends of the Forms as they affirm their materialism, and Dr Johnson's kicking the stone outside of the church in Harwich as an exasperated affirmation of its material existence independent of our ideas. While by no means dead, this rhetorical trope has, perhaps, lost some of its sheen in giving birth to the late twentieth-century clich\'{e} of the table-pounding realist.

I once attended a lecture, where the philosopher speaking literally pounded on the podium at each mention of an objective worldly correlate of our conceptual scheme. Through this performance, the philosopher was expressing the objectivity of the worldly correlate, and, in a rather bullying fashion, demanding our assent to it. As with Dr Johnson's performance, it was a multimodal affair. The audience, in sympathetically responding to the philosopher's tactile experience, is meant to vividly experience the tangible resistance of the podium, revealed, in part, in the loud, sharp sound it produces when pounded. It is this resistance to touch that is meant to reveal the podium to be objectively there, independently of the speaker's pounding, just as the worldly correlate is meant to be there, independently of our conceptual scheme.

Haptic perception plays a privileged role in the rhetoric of objectivity. It does so, in part, because the experience of felt resistance to touch is phenomenologically vivid and primitively compelling. Though in no doubt about the presence or solidity of a thing, we may, nevertheless, be drawn to touch it. Thus we must endeavour to teach our children to keep their hands to themselves, and even in maturity, polite notices are required to remind adults to not touch the display cabinet. Haptic perception plays a privileged role in the rhetoric of objectivity, in part, because the experience of felt resistance to touch is phenomenologically vivid and primitively compelling. Moreover, it plays a privileged role, as well, because grasping provides a model for perceptual objectivity quite generally, in the assimilation of the hand, and the haptic experience it gives rise to, to the object of haptic investigation.

% section grasping_and_the_rhetoric_of_objectivity (end)

\section{Perceptual Objectivity} % (fold)
\label{sec:perceptual_objectivity}



% section perceptual_objectivity (end)

\section{Bergson's Critique of Kant} % (fold)
\label{sec:bergson_s_critique_of_kant}

\begin{quote}
	By intuition is meant the kind of \emph{intellectual sympathy} by which one places oneself within an object in order to coincide with what is unique in it and consequently inexpressible. (Bergson, \emph{Introduction to Metaphysics}, 7)
\end{quote}

% section bergson_s_critique_of_kant (end)

\section{Perceiving Things in Themselves} % (fold)
\label{sec:perceiving_things_in_themselves}



% section perceiving_things_in_themselves (end)

% chapter realism (end)