%!TEX TS-program = xelatex 
%!TEX TS-options = -synctex=1 -output-driver="xdvipdfmx -q -E"
%!TEX encoding = UTF-8 Unicode
%
%  location
%
%  Created by Mark Eli Kalderon on 2015-03-29.
%  Copyright (c) 2015. All rights reserved.
%

\documentclass[12pt]{article} 

% Definitions
\newcommand\mykeywords{perception, location}
\newcommand\myauthor{Mark Eli Kalderon}

% Packages
\usepackage{geometry} \geometry{a4paper} 
\usepackage{url}
% \usepackage{txfonts}
\usepackage{color}
% \usepackage{enumerate}
\definecolor{gray}{rgb}{0.459,0.438,0.471}
\usepackage{setspace}
% \doublespace % Uncomment for doublespacing if necessary
% \usepackage{epigraph} % optional

% XeTeX
% \usepackage[cm-default]{fontspec}
\usepackage{xltxtra,xunicode}
\defaultfontfeatures{Scale=MatchLowercase,Mapping=tex-text}
\setmainfont{Hoefler Text}

% Bibliography
\usepackage[round]{natbib}

% Title Information
\title{Where is your haptic experience located?}
\author{\myauthor} 
\date{} % Leave blank for no date, comment out for most recent date

% PDF Stuff
\usepackage[plainpages=false, pdfpagelabels, bookmarksnumbered, backref, pdftitle={Form Without Matter}, pagebackref, pdfauthor={\myauthor}, pdfkeywords={\mykeywords}, xetex, colorlinks=true, citecolor=gray, linkcolor=gray, urlcolor=gray]{hyperref} 

%%% BEGIN DOCUMENT
\begin{document}

% Title Page
\maketitle
% \begin{abstract} % optional
% \noindent
% \end{abstract} 
% \vskip 2em \hrule height 0.4pt \vskip 2em
% \epigraph{\textsc{}{}

% Layout Settings
\setlength{\parindent}{1em}

% Main Content

\section{In the heart of things} % (fold)
\label{sec:in_the_heart_of_things}

Toward the end of a passage emphasizing the active nature of visual perception, Plotinus makes, at least to our post-Cartesian ears, a startling pronouncement: That the apprehension of the visible object takes place in the object seen. We have considered a related but weaker claim about haptic experience. Beginning with the prima facie absurdity of supposing that haptic experience is in the perceiver's head, we claimed that it is more natural to suppose, at least initially, that haptic experience is closer to where its object is at, in our handling of that object. The Plotinian claim, if made on behalf of haptic presentation, is stronger still. It would be the claim that haptic experience places us within the object of haptic experience. In grasping or enclosure, the haptic experience is in the perceived overall shape and volume of the object that the perceiver is handling. The earlier, weaker claim hedged at the boundary between the apparent body, the region wherein bodily sensation is potentially felt \citep{Martin:1992aa}, and extrapersonal space. However, if haptic perception involves a mode of sympathetic presentation, then the haptic variant of the Plotinian claim, that haptic perception places us in the object of haptic perception, must be true, at least on a certain interpretation of that claim. Understanding this potentially sheds light on the perceptual significance of what Thompson Clarke describes as ``the surface inquiry''.

Let us begin, not with the entertained haptic variant, but with Plotinus' claim about vision itself, that the apprehension involved in visual experience takes place in the object seen. But before we can make sense of Plotinus' startling pronouncement, let us first examine the views of some of his predecessors.

% section in_the_heart_of_things (end)

\section{Beyond the Limits} % (fold)
\label{sec:beyond_the_limits}

In grasping or enclosure, the the overall shape and volume of an external body is present in haptic experience thanks to an implicit experience of an external limit to the hand's activity. If an experienced limit to the hand's activity discloses tangible qualities of an external body, then the idea of the experience of a limit, however implicit, must be in good order. But is it really? Within the phenomenological tradition, \citet{Derrida:2005aa} has expressed his doubts.

In a representative passage, Derrida describes the \emph{aporia} involved in the figure of touch:
\begin{quote}
	Above all, nobody, no body, no body proper has ever touched---with a hand or through skin contact---something as abstract as a limit. Inversely, however, and that is the destiny of this figu­rality, all one ever does touch is a limit. To touch is to touch a limit, a surface, a border, an outline. Even if one touches an inside, ``inside'' of any­ thing whatsoever, one does it following the point, the line or surface, the borderline of a spatiality exposed to the outside, offered---precisely---on its running border, offered to contact. \ldots\ This surface, line or point, this limit, therefore, \ldots\ finds itself to be at the same time touchable and untouchable: it is as is every limit, certainly, but also well-nigh at and to the limit, and on the exposed, or exposing, edge of an abyss, a nothing, an ``unfoundable'' unfathomable, seeming still less touchable, still more untouchable, if this were possible, than the limit it­ self of its exposition.
\end{quote}
There is a lot to say about this passage and how the \emph{aporia} it describes may, if at all, \emph{pace} Derrida, be resolved. One thing to get clearer about is the sense in which a surface, understood as a limit, is abstract. On at least one good sense of the abstract--concrete distinction, the surfaces of material bodies count as concrete---they at least exist in space and time. But notice, as well, that the surfaces of material bodies could not themselves be material. They are not themselves material parts of the bodies whose surfaces they are. Surfaces are, in Sellars' \citeyearpar[\textsc{iv} 23]{Sellars:1956xp} apt phrase, bulgy two-dimensional particulars. They are two-dimensional in the sense that they lack thickness. But no material thing lacks thickness. This suggests an alternative understanding of the sense in which such limits are abstract. Whether it is sufficient to underwrite Derrida's \emph{aporia} is another matter. 

Notice, however, that the limit at work in this passage is a spatial boundary, the surface of the object of tactile perception. An external limit to the hand's activity is not a spatial boundary or a surface, though it may disclose these, if it is experienced as their sympathetic presentation. However, if there is a puzzle about how anything as abstract as the limit of a bounded body may be tangible, surely a limit to the hand's activity is even more abstract. After all, the limit to the hand's activity---like the being of capacity, as the Eleatic Visitor instructs the Giants in the \emph{Sophist}---is intangible. Our question is whether anything as abstract as a limit to the hand's activity so much as could be the object of bodily awareness.

What would it take to be aware of a limit to the hand's activity? Such an awareness would have to afford the subject with a contrast between the hand's present configuration and a potential configuration that extends beyond the points at which the hand's force is resisted by the self-maintaining forces of the object grasped. Such an awareness would depend upon a psychological representation of potential motor activity, a sense of how far one's grasp may extend if unimpeded. The representation of potential motor activity need only be apparent. I may have a sense that I could reach the top shelf, but trying may reveal that I was mistaken.

A sense of the contrast between the hand's present configuration and a potential configuration beyond the limit of the grasped object's boundaries may be necessary for awareness of an external limit to the hand's activity but it is not sufficient. There is a crucial additional element involved in being aware of a limit to the hand's activity. Whenever I deliberately hold my hand in a certain configuration that is not completely outstretched, I may have a sense of potential configurations extending beyond the present one, but I do not thereby experience a limit. The relevant sense of limit involves a check or impediment to the will. So not only does an awareness of a limit to the hand's activity involve a kinesthetic representation of potential motion, but it must also draw upon our sense of agency. Not only must one have a sense of how far one's grasp may extend if unimpeded, but one must also have a sense of an impediment to one's grasp. A sense of impediment arises out of a frustration of the will in being unable to extend one's grasp further. The location of the hand's configuration in the space of potential motor activity is only experienced as a limit insofar as it is the frustration and not the fulfillment of the will.



% section beyond_the_limits (end)


% Bibligography
\bibliographystyle{plainnat}
\bibliography{Philosophy}

\end{document}