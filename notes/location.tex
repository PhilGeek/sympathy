%!TEX TS-program = xelatex 
%!TEX TS-options = -synctex=1 -output-driver="xdvipdfmx -q -E"
%!TEX encoding = UTF-8 Unicode
%
%  location
%
%  Created by Mark Eli Kalderon on 2015-03-29.
%  Copyright (c) 2015. All rights reserved.
%

\documentclass[12pt]{article} 

% Definitions
\newcommand\mykeywords{perception, location}
\newcommand\myauthor{Mark Eli Kalderon}

% Packages
\usepackage{geometry} \geometry{a4paper} 
\usepackage{url}
% \usepackage{txfonts}
\usepackage{color}
% \usepackage{enumerate}
\definecolor{gray}{rgb}{0.459,0.438,0.471}
\usepackage{setspace}
% \doublespace % Uncomment for doublespacing if necessary
% \usepackage{epigraph} % optional

% XeTeX
% \usepackage[cm-default]{fontspec}
\usepackage{xltxtra,xunicode}
\defaultfontfeatures{Scale=MatchLowercase,Mapping=tex-text}
\setmainfont{Hoefler Text}

% Bibliography
\usepackage[round]{natbib}

% Title Information
\title{Where is your haptic experience located?}
\author{\myauthor} 
\date{} % Leave blank for no date, comment out for most recent date

% PDF Stuff
\usepackage[plainpages=false, pdfpagelabels, bookmarksnumbered, backref, pdftitle={Form Without Matter}, pagebackref, pdfauthor={\myauthor}, pdfkeywords={\mykeywords}, xetex, colorlinks=true, citecolor=gray, linkcolor=gray, urlcolor=gray]{hyperref} 

%%% BEGIN DOCUMENT
\begin{document}

% Title Page
\maketitle
% \begin{abstract} % optional
% \noindent
% \end{abstract} 
% \vskip 2em \hrule height 0.4pt \vskip 2em
% \epigraph{\textsc{}{}

% Layout Settings
\setlength{\parindent}{1em}

% Main Content


Toward the end of a passage emphasizing the active nature of visual perception, Plotinus makes, at least to our post-Cartesian ears, a startling pronouncement: That the apprehension of the visible object takes place in the object seen. We have considered a related but weaker claim about haptic experience. Beginning with the prima facie absurdity of supposing that haptic experience is in the perceiver's head, we claimed that it is more natural to suppose, at least initially, that haptic experience is closer to where its object is at, in our handling of that object. The Plotinian claim, if made on behalf of haptic presentation, is stronger still. It would be the claim that haptic experience places us within the object of haptic experience. In grasping or enclosure, the haptic experience is in the perceived overall shape and volume of the object that the perceiver is handling. The earlier, weaker claim hedged at the boundary between the apparent body, the region wherein bodily sensation is potentially felt \citep{Martin:1992aa}, and extrapersonal space. However, if haptic perception involves a mode of sympathetic presentation, then the haptic variant of the Plotinian claim, that haptic perception places us in the object of haptic perception, must be true, at least on a certain interpretation of that claim. Understanding this potentially sheds light on the perceptual significance of what Thompson Clarke describes as ``the surface inquiry''.

Let us begin, not with the entertained haptic variant, but with Plotinus' claim about vision itself, that the apprehension involved in visual experience takes place in the object seen. But before we can make sense of Plotinus' startling pronouncement, let us first examine the views of some of his predecessors.



% Bibligography
\bibliographystyle{plainnat}
\bibliography{Philosophy}

\end{document}