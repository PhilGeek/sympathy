%!TEX TS-program = xelatex 
%!TEX TS-options = -synctex=1 -output-driver="xdvipdfmx -q -E"
%!TEX encoding = UTF-8 Unicode
%
%  veridical_hallucination
%
%  Created by Mark Eli Kalderon on 2014-10-14.
%  Copyright (c) 2014. All rights reserved.
%

\documentclass[12pt]{article} 

% Definitions
\newcommand\mykeywords{veridical hallucination, perception, presentation, representation}
\newcommand\myauthor{Mark Eli Kalderon}

% Packages
\usepackage{geometry} \geometry{a4paper} 
\usepackage{url}
% \usepackage{txfonts}
\usepackage{color}
% \usepackage{enumerate}
\definecolor{gray}{rgb}{0.459,0.438,0.471}
\usepackage{setspace}
\doublespace % Uncomment for doublespacing if necessary
% \usepackage{epigraph} % optional

% XeTeX
% \usepackage[cm-default]{fontspec}
\usepackage{xltxtra,xunicode}
\defaultfontfeatures{Scale=MatchLowercase,Mapping=tex-text}
\setmainfont{Hoefler Text}

% Bibliography
\usepackage[round]{natbib}

% Title Information
\title{Veridical Hallucination and Perceptual Presentation}
\author{\myauthor} 
\date{} % Leave blank for no date, comment out for most recent date

% PDF Stuff
\usepackage[plainpages=false, pdfpagelabels, bookmarksnumbered, backref, pdftitle={Form Without Matter}, pagebackref, pdfauthor={\myauthor}, pdfkeywords={\mykeywords}, xetex, colorlinks=true, citecolor=gray, linkcolor=gray, urlcolor=gray]{hyperref} 

%%% BEGIN DOCUMENT
\begin{document}

% Title Page
\maketitle
% \begin{abstract} % optional
% \noindent
% \end{abstract} 
% \vskip 2em \hrule height 0.4pt \vskip 2em
% \epigraph{\textsc{}{}

% Layout Settings
\setlength{\parindent}{1em}

% Main Content

\section{Bergson} % (fold)
\label{sec:bergson}

This essay was conceived in sin. Let my opening remarks thus serve both as an introduction and as a confession.

In \emph{Matter and Memory}, \citet{Bergson:1912pi} criticizes both the realist and the idealist though he seeks to retain insights from each. At first this can sound as if he is rejecting the very distinction in much the same way that \citet{Quine:1951fk} rejects the very distinction between the analytic and the synthetic. However, closer attention to what Bergson means specifically by ``realism'' reveals that realism and idealism do not constitute a partition on available alternatives. According to the idealist, the objects of perception are not independent of perceptual experience. Rather, the objects of perception are constituted by the awareness our perceptual experience affords us of them. With idealism so understood, it might reasonably be thought that realism just is its denial. So understood, realism would merely consist in the claim that the objects of perception exist and have the character that they do independently of the awareness afforded by our perceptual experience of them. Call this minimal realism. However, according to Bergson, there is more to realism, as he conceives of it, than minimal realism. Not only must the objects of perception exist and have their character independently of our perceptual experience of them, but they must also be the cause of that perception and unlike it in character. Bergson's realist is exemplified by Descartes. Matter, for Descartes, is a mode of extension and homogenous, but perception seems to present us with qualitative heterogeneity. Descartes maintained, in effect, that certain homogenous modes of extension cause in us internal sensations that display a qualitative heterogeneity. 

Bergson proposes to argue, in defense of common sense, that the realist is right to insist that the objects of perception exist and have their character independently of our perceptual experience of them, but, also, that Berkeley was right in insisting that the so-called secondary qualities have as much reality as primary qualities. In siding with Berkeley in defense of common sense, Bergson is rejecting a characteristic doctrine of early modern philosophy, that some experiences do not resemble their causes, specifically, our experience of what Aristotle described as the proper sensibles---sensory objects available to one sense alone and about whose presence no error is possible, \emph{De Anima} \textsc{ii} 6 418\( ^{a} \)11–12----do not resemble their causes (on the significance of the transformation of the Aristotelian proper sensibles into secondary qualities in the seventeenth century see \citealt{Smith:1990sm,Winkler:2011aa}).

In chapter one of \emph{Matter and Memory}, Bergson presents a battery of arguments against the realist and the idealist. I want to focus here on a claim he makes towards the end of that chapter:
\begin{quote}
	But for realism as for idealism, perceptions are `veridical hallucinations,' states of the subject, projected outside himself; and the two doctrines differ merely in this: that in the one these states constitute reality, in the other they are sent forth to unite with it. \citep[73]{Bergson:1912pi}
\end{quote}

How is it that for realism as for idealism perceptions are merely veridical hallucinations? In the quotation, Bergson speaks of perceptual experiences as conceived by the realist and the idealist as states of the subject. Earlier on the same page, he elaborates. Both the realist and the idealist conceive of perceptual experience as an internal state of a subject, a conscious modification of the perceiving subject. Bracket, for the moment, why they should be thought to be saddled with such a commitment, though it is plausibly attributed to both Descartes and Berkeley. Focus instead on why Bergson thinks that perceptual experience, so conceived, is at best a mode of veridical hallucination. Bergson's thought, here, is that perception is a mode of awareness. And that in order for experience to afford us an awareness of an object that both exists and has its character independently of that awareness, that experience must, in some sense, place us in that object. However, conceiving of perceptual experience as an internal state precludes this. If we conceive of perceptual experience as ``an internal state, a mere modification of our personality'' then:
\begin{quote}
	our eyes are closed to the primordial and fundamental act of perception, ---the act, constituting pure perception, whereby we place ourselves in the very heart of things. \citep[73]{Bergson:1912pi}
\end{quote}
There is much to say about Bergson's evocative description of the primordial and fundamental act of perception as placing ourselves in the very heart of things. Unfortunately, this is not the place to go into it. (Bergson's doctrine, here, essentially involves his take on a dialectic involving Aristotle, Alexander of Aphrodisias, and Plotinus. Moreover, it involves his claim that the content of our perception is determined by our motor capacities.). Nevertheless, it is easy to see the general form of Bergson's worry. Internal states of a perceiver, conscious modifications of the perceiving subject, do not place the perceiver in objects external to their body. Thus such objects are not present in undergoing such conscious modifications of the perceiving subject. And if they are not present in undergoing such conscious modifications of the perceiving subject then such an experience affords no awareness of them. So even if the realist and the idealist can, at least by their own lights, construct for themselves a sense in which such internal states may be veridical, since such states afford no genuine awareness of the natural environment, they are at best, veridical hallucinations of that environment. That, at the very lest, is the general form of Bergson's worry.


The stage has now been set for my confession. For various reasons, I found Bergson's charge---that for realism and idealism, at least as he conceives of them, perceptions are veridical hallucinations---both striking and apt. I was envious of the rhetorical gesture and sought to appropriate it for my own. My sins are thus pride and avarice and this essay is their fruit. Specifically, in this essay, I want to argue for a variant of Bergson's claim, that perceptions, as conceived by contemporary representationalists, are veridical hallucinations. I will thus be developing, in my own way, a theme from Putnam's Dewey Lectures.

According to a familiar story, analytic philosophy begins with a revolt against idealism, the rebellion beginning principally in Cambridge and subsequently spreading throughout the anglophone world. Cambridge realists, such as \cite{Russell:1912uq}, \citet{Moore:1953nx}, and \citet{Price:1932fk}, maintained that perception affords us with a sensory mode of awareness and that we our knowledgeable of the mind-independent environment by being aware of that environment. They also maintained that \emph{all} sense experience, and not just perception, involves this sensory mode of awareness. Cambridge realists were thus committed to a kind of experiential monism (in Snowdon's \citeyear{Snowdon:2008oz} terminology): All sense experience involves, as part of its nature, a sensory mode of awareness. Even subject to illusion or hallucination, there is something of which one is aware. And with that, they were an application of the argument from illusion, or hallucination, or conflicting appearances away from immaterial sense data and a representative realism that tended, over time, to devolve into a form of phenomenalism. Interestingly, the Aristotelian proper sensibles that had been transformed in the seventeenth century into secondary qualities, are, within the tradition of Cambridge realism, transformed again, sense data being their new avatar. Most contemporary analytic philosophers believe neither in sense data, representative realism, nor phenomenalism. While realistically inclined, these aspects of Cambridge realism have largely been abandoned. Instead of sense data, contemporary philosophers of perception are more likely to speak of accuracy conditions or the representational content of perception \citep[though see][]{Robinson:1994ms}. And while perception is deemed to have a representational content, they will insist that this is insufficient for a representative realism. Rather, one is directly aware of the environment by one's perception accurately representing that environment. Moreover, few remain with phenomenalist sympathies (though see \citealt{Foster:2000ny} and \citealt{Noe:2004fk}). Nevertheless, \citet{Putnam:1994kx} has suggested that this contemporary orthodoxy has perhaps more in common with the sense-data theory, representative realism, and phenomenalism than they care to admit. It struck Putnam, at least at the time of the Dewey Lectures, as insufficiently demanding by the standards of James' natural realism that he admired and, like Cambridge realism, ultimately unsustainable. Taking my cue from Bergson, I shall develop Putnam's worry by reflecting on veridical hallucination.

% section bergson (end)

\section{Representationalism} % (fold)
\label{sec:representationalism}

% One of the fundamental debates in contemporary philosophy of perception concerns whether experience has a representational character. That debate can be seen to turn on the following questions:
% \begin{enumerate}
% 	\item Does perceptual experience involve the presentation of its object?
% 	\item If it does, is perceptual presentation explicable in terms of representational content?
% \end{enumerate}
%
% Does perceptual experience present its object? Perhaps not. Perhaps perceptual experience involves a primitive form of objective representation as \citet{Burge:2010uq} maintains. Or perhaps to suppose that perception presents its object is to fun afoul of the Myth of the Given on some appropriate understanding of Sellar's \citeyearpar{Sellars:1956xp} myth.

Our target is representationalism. The charge is that perception, as conceived by the representationalist, is no better than veridical hallucination. To press this charge, we must first get clearer on its intended target, not least because ``representationalism'' has been used in a variety of ways.

Begin with the object of perception. The object of perception is whatever is perceived in an episode of perception. The representationalist that is our target at the very least maintains that the object of perception is determined by the intentional or representational content of that perceptual experience. Thus if I can truly be said to see the pig happily graving among the canipes at a wedding party, then what I see, the pig, is determined as the object of my perception by the intentional or representational content of my visual experience. There is an explanatory lacuna here, one that is hardly ever addressed let alone discharged. The representationalist, so characterized, explains the fact that our perceptual experiences take an object in terms of the intentional or representational content of that perceptual experience. But what's the explanation, exactly? After all not all states or episodes that have an intentional or representational content take an object in the sense that perception does, so how does the representational content of a perceptual experience determine that that experience takes an object. We are owed an explanation, but never given one.

Though I note this explanatory lacuna, I propose to follow my peers in ignoring it. Notice that representationalism, so characterized, is relatively weak position. It is not, for example, committed to perceptual phenomenology being exhaustively determined by the intentional or representational content of perception. Of course, insofar as the content of perception determines its object, then it is not without phenomenological significance. My seeing the pig crashing the wedding party differs in phenomenological character from my seeing the look of disappointment on the bride's face as she surveys the unfolding disaster.  And this has at least something to do with the fact that my perception in each case takes a different object. But that is not yet to claim that every aspect of phenomenology is determined by the content of the relevant perceptual experience. Perhaps, like Block's phenomenism, one may maintain that while the content of experience is phenomenologically relevant insofar as it determines the object of that experience, many aspects of phenomenology are determined, not by content, but by qualia. So our target is not restricted to representationalists, like Harman or Tye, who maintain that the representational content of experience exhaustively determines its phenomenological character. Our target includes, as well, positions such as Block's phenomenism, that attribute an object-determining intentional or representational content to experience without maintaining that the phenomenological character of that experience is exhaustively determined by that content.

This weak sense of representationalism, while widespread and indeed the prevailing orthodoxy, is not universally assented to. Thus, for example, many naive realists and disjunctivists maintain that while experience presents its object, it does so not by that object being represented in experience. So the minimal commitment here is to there being an irreducible presentational element to perceptual experience. Notice that perceptual experience may have an irreducible presentational element consistent with also, at the same time, having an intentional or representational content. What distinguishes these theorists from the representationalists that serve as our present target, is that they deny, and representationalists affirm, that the object of experience is determined by its intentional or representational content.

% section representationalism (end)

\section{Veridical Hallucination} % (fold)
\label{sec:veridical_hallucination}

The term veridical hallucination was first introduced in the context of psychical research and originally designated a form of precognitive dream. The dream, appropriately interpreted, was veridical insofar as it correctly represented events that had yet to transpire. But since the events have yet to happen, and the dreams required interpretation, it seemed inappropriate to describe the experience of such precognitive dreams as perceptions, so they were deemed, instead, veridical hallucinations. Bergson was, for a time, the president of Society of Psychical Research, whose members included Broad and James. However, when Bergson charges that for the realist as for the idealist perceptions are veridical hallucinations, the notion of veridical hallucination is not the one commonly in play in discussions of psychical research. Rather it is closer to the notion of veridical hallucination that figures in Grice's \citeyearpar{Grice:1962jw} essay. But what notion is that?

Intuitively, the idea seems easy enough to state. An experience is a veridical hallucination, when, despite being veridical, the veridical subject matter of that experience is not present in the awareness afforded by that experience. While the gloss requires explication, I do not think that it is terribly controversial as it stands, even if it may be controversial whether this or that example genuinely counts as an instance of veridical hallucination so glossed. However, as we shall see, there are complications in getting it to cohere with the views of either side of the fundamental debate in contemporary philosophy of perception concerning the representational character of perceptual experience.

Consider philosophers who deny that experience has a representational content, insisting instead that perception, at least, essentially involves an irreducible presentational element. An immediate problem for their accepting the present gloss is how to understand the veridicality of veridical hallucinations, for the most straightforward way to understand this is debarred from them. The most straightforward way to understand the veridicality would be to assign such experiences accuracy conditions or a representational content that are fulfilled in the circumstances of perception. But if no sensory experience has accuracy conditions or representational content, then veridicality of veridical hallucinations cannot be understood in terms of these experiences possessing veridical contents. The present difficulty is not very deep, however. Even if perceptual experience and sensory experience more generally cannot, strictly speaking, be veridical in the sense of correctly representing their object, typically accompanying such experiences are states that can straightforwardly be assessed for veridicality. 

Taking a stroll in Greenwich park, I turn, and look, and see an ancient chestnut tree. It is one of the sweet chestnut trees replanted there when the park was redesigned for Charles \textsc{ii} in the 1660s. Beneath it, among the scattered burrs, humans and squirrels vie with one another for chestnuts. The burrs that remain on the tree are large and brighter green than the surrounding foliage. Despite its manifest strength and stability, its trunk seems to flow in a wave like form. When I look upon the chestnut tree, being as it is, various things can seem true to me. As \citet{Price:1952ix} pointed out long ago, such seemings are not sensory appearances so much as epistemic reactions to what appears to us in sense experience. Looking upon the chestnut tree various judgments can see to be true. It can seem, for example, that the tree retains many of its burrs. This judgment seems true to me, since my experience seems to afford me awareness of aspects of the tree upon which the truth of the judgment depends---I see the bright green burrs set against the darker foliage. So the idea would be that a veridical hallucination is veridical in the sense that in undergoing that experience various judgments that seem true to the subject are in fact true. In this way the veridical character of veridical hallucination can be accommodated by a theorist that denies experience has accuracy conditions or a representational content more generally.

Representationalists, on the other hand, can accept the straightforward understanding of the veridicality of veridical hallucination. Insofar as they are willing to assign correctness conditions or representational content to sensory experience, such experiences will be veridical if they have veridical contents. So when a subject veridically hallucinates a scene the visual experience they undergo has a content that veridically represents that scene. The challenge for the representationalist does not consist in accommodating somehow the veridicality of veridical hallucination, but with accommodating the contrasting feature of the intuitive gloss, that the veridical subject matter of the visual experience is not present in the awareness, if any, afforded by that experience. Specifically, the challenge to the representationalist is to explain how something may be present in the awareness afforded by sense experience so as to give content to the denial which constitutes the contrasting feature of the intuitive gloss. Given the possibility of veridical hallucinations, having a veridical subject matter is insufficient for being aware of that subject matter. Veridical hallucinations are precisely veridical experiences that afford no awareness of their veridical subject matter. So being present in the awareness afforded by sense experience must be understood in some other way. But what way is that? That is the challenge that we shall be exploring.

% section veridical_hallucination (end)

\section{The Challenge} % (fold)
\label{sec:the_challenge}

How might the representationalist meet this challenge? I want to argue that the challenge could not be met, or, at least, not in a certain, otherwise attractive, way. What way is that?

It might naturally be thought that the difference between a perception of a scene and a veridical hallucination of that same scene consists in the presence of some condition that obtains in the former case but not in the latter case. Thus while the perceptual experience, like the veridical hallucination, at least as conceived by the representationalist, has a veridical content, there is some further condition that it satisfies that makes the perception more than merely a veridical hallucination. It is thus very tempting to think that what makes for a perception in distinction to a veridical hallucination is that the perceptual experience, while veridically representing its object, nonetheless meets some further condition. This, I maintain, is a way in which the challenge could not be met.

Not only is it natural for the representationalist to try to meet this challenge by specifying a further condition, one satisfied by perception of the scene if not a veridical hallucination of that same scene, but it is also very natural to appeal to a certain kind of condition. Specifically, it is natural to think that in the perceptual case, if not in the case of veridical hallucination, the perception causally or counterfactually depends upon its veridical subject matter. So a veridical hallucination, though veridical, won't causally or counterfactually depend upon the scene that it veridically represents. The underlying thought, here, is that perception is essentially a kind of sensitivity to the sensible environment, and this sensitivity is manifest in the causal or counterfactual condition that distinguishes perception from veridical hallucination.

So I will not only be considering representationalist responses to our challenge that proceed by specifying a condition that is satisfied by perception if not veridical hallucination, but I will consider only conditions of a certain kind, that perception somehow causally or counterfactually depends upon its veridical subject matter since the proposed condition purports to capture an essential element of perception, namely the sensitivity displayed to sensible aspects of the natural environment.

Actually, our examination of the representationalist response to this challenge will be restricted further still. I won't, in fact, consider causal conditions at all but will restrict our attention to a counterfactual formulation of the fugitive condition. This restriction is partly for convenience, but only in part. Seeing how putative counterfactual conditions fail are revealing of a more general problem. Or so I suggest.

Consider a counterfactual reduction of perceptual sensitivity. One might try to give such a reduction on the model of Nozick's \citeyearpar{Nozick:1981fk} tracking theory. Roughly speaking, perceptual sensitivity would be the counterfactual covariation of sense experience and the truth of a potentially known proposition \( p \) through a sphere of possibilities that extends to the nearest not-\( p \) world. Think of the proposition \( p \) as a proposition that could be known on the basis of undergoing the relevant sensory experience insofar as that experience is perceptual. So if the subject of that experience had the concepts necessary to entertain \( p \), and they were cognitively attending to their sense experience, then in perceiving what they perceived, they would be in a position to know that \( p \). So the proposition that \( p \) concerns the knowledge potential of undergoing that sense experience, not what is in fact known in undergoing that experience. 

On the Stalnaker-Lewis semantics for counterfactuals, the sphere of possibilities is determined by a conversationally salient similarity metric. The reductive ambitions of the account constrains admissible metrics, however. The relevant similarity metric could not be \emph{perceptual}, if sensitivity is to be specified independently of perceptual awareness. If the satisfaction of the counterfactual condition is what transforms a veridical experience into a perception, then if it is to be a substantive and informative condition, it must be specified independently of being a perception. But it is precisely this feature that gives rise to a pattern of counterexamples. Moreover, it is precisely this feature that reveals the general problems that will afflict not only the counterfactual condition, but any corresponding causal condition. Specifically, it is the vividness with which the semantics for counterfactuals can lay bare such reductive assumptions which is its chief and present virtue.

Now, it is true that, at least in paradigmatic cases, seeing something can survive small changes in the object, the circumstances of perception, or the perceiver, even where these changes occur along reductively admissible dimensions. But this counterfactual covariation is not true of perception generally.  As \citet{Johnston:2006uq} observes, small changes can sometimes result in a failure to perceive. Consider another experimental subject of Frankfurt's \citeyearpar{Frankfurt:1969kx} Dr Black. A device is surgically implanted that will cut off the flow of information from the optic nerve if a predefined target is not confined to some region of the visual field. Should the target exit that region---by the target moving with respect to the perceiver, or the perceiver moving with respect to the target, without corresponding adjustments---then the optic nerve will be temporarily disabled, for thirty seconds, say. After the device is implanted, the subject recovers from surgery with the target within the region. Small changes in the object or the perceiver---the target moving slightly, or the perceiver's gaze moving slightly---would result in temporary blindness. But that is consistent with the subject seeing the target upon regaining consciousness. 

Conversely, small changes can sometimes result in the subject seeing something that they previously could not. Consider an object hidden from view by a fragile camouflage that can only be sustained in narrowly defined viewing conditions. Small changes to these viewing conditions would reveal what the camouflage had previously hidden from view. The subject would see the object that they previously failed to see.

Schaffer summarizes the general problem well:
\begin{quote}
    Human perceptual competence forms a discontinuous scatter in logical space. \ldots\ The tracking theory identifies knowledge with counterfactual covariation of belief and truth through a sphere of possibilities. The contents of the sphere are determined by the similarity metric. Derailings occur because the similarity metric (on any reasonable interpretation) is completely out of alignment with our actual rough-and-ready perceptual capacities. The problem is systematic: the mismatch between the smoothness of logical space and the roughness of human perception is not likely to be fixed by a further epicycle. \citep[42]{Schaffer:2003vn}
\end{quote}
The present tracking theory identifies perceptual sensitivity with counterfactual covariation of sense experience and the truth of potentially known propositions through a sphere of possibilities. But the discontinuous scatter that human perceptual competence forms on the relevant sphere of possibilities presents the very same problem. Moreover, the discontinuous scatter is exactly what you would expect, if perceptual sensitivity were identical to, or constituted by, perceptual awareness. 

Not only is the tracking theory subject to a pattern of counterexample, but the very best case for the tracking theory undermines its reductive ambitions as well. In paradigm cases where seeing something can survive small changes, sense experience counterfactually covaries with the truth of a potentially known proposition \emph{because} sense experience affords the perceiver awareness of its environmental subject matter. Suppose my seeing the ancient chestnut tree is such a paradigm case. My seeing the bright green burs set against the darker foliage of the chestnut tree would survive small changes in the object, the circumstances of perception, or the perceiver. Thus I would still see the bright green burs even if they were slightly smaller, or the illumination were brighter, or I viewed it from a different vantage point. My sense experience counterfactually covaries with the truth of the proposition that the burs are bright green because my sense experience affords me awareness of the bright green of the burs, a particular that makes true that proposition. In the first instance, it is the presence of particulars and not the truth of propositions, that we track in vision. Since perceptual awareness of environmental particulars grounds the counterfactual covariation in paradigm cases, the counterfactual covariation could not be the basis of a reductive understanding of perceptual sensitivity.

According to Sellars, perceptual sensitivity is a reductively identifiable independent factor in perception given the explanatory role manifolds of ``sheer receptivity'' play in intuition. Against this, McDowell maintains that while perceptual experience must be guided from without if a realism, or at least an anti-idealism, is to be sustained, there is no need for manifolds of ``sheer receptivity'' to play this guiding role:
\begin{quote}
    \ldots\ once we understand how objects can be immediately present to \ldots\ sensory consciousness in intuition, we can take this need for external constraint to be met by perceived objects themselves. \citep[46]{McDowell:1998vn}
\end{quote}
The idea is that that perceptual sensitivity is, or is constituted by, objects in the mind-independent environment being present in sensory awareness in a way that precludes perceptual sensitivity being a reductively identifiable independent factor in perception and, hence, guidance by manifolds of ``sheer receptivity''.

The reason for the (likely) failure of the counterfactual condition should by now be clear. The motivation for such proposals is that perception is essentially a kind of sensitivity to the sensible environment, and this sensitivity is manifest in the causal or counterfactual condition that distinguishes perception from veridical hallucination. There is a kind of half truth to this thought. In the very best cases for the tracking theory, where the counterfactual condition holds, what explains that it does is the fact that the subject perceived what they did. Perception, in effect, is the mechanism that explains the counterfactual condition when it obtains. 

If we suppose that that's right, then we can see how the counterfactual condition is not fit for purpose (or, at least, its present purpose). Recall, the challenge to the representationalist was this. Intuitively, what distinguishes veridical hallucination from perception is that veridical hallucinations, while veridical, don't present their veridical subject matter in whatever awareness those experiences afford. While the representationalist has no problem, at least in principle, in understanding the veridical character of such hallucinations, they need to say what it is that the presentation of veridical subject matter is in the perceptual case. Clearly veridically representing its object is insufficient for perception to present that object in the sensory awareness it affords. Veridically representing its object is, at best, a necessary condition for perception's presentation of that object. And so the natural, if not inexorable, thought is that what distinguishes perception from veridical hallucination is the obtaining of a further condition. And a further natural, if not inexorable, thought is that the condition concerns the sensitivity of the perceiver to the veridical subject matter, a sensitivity that obtains in the perceptual if not the hallucinatory case. But if perceptual sensitivity is, as McDowell insists it must be, just the presentation of its object in the awareness afforded by perceptual experience, then it is no help whatsoever in distinguishing perception from veridical hallucination. When challenged to say what the difference is between perception and veridical hallucination, it is no answer to say, in effect, that perceptions are perceptions. True but unhelpful. But if that is really the best that the representationalist can do, then, I suggest, the Bergsonian charge is apt: For all that has been said, perceptual experience, as the representationalist conceives of it, no more than a veridical hallucination.

If the Bergsonian charge holds, then Putnam was right. (Or at least the Putnam of the Dewey lectures. Notoriously, Putnam contains multitudes.) There's a deep sense in which the prevailing representationalist orthodoxy is no real advance over the sense-data theory, representative realism, and phenomenalism. If perception is, as the representationalist conceives of it, no better than veridical hallucination, then purported intentional or representational content of perception is a veil between us and the objects destined to never be the objects of sensory awareness. Perhaps, it might be suggested that there is no real veil here. Perceptual experiences, as conceived by the representationalist, are, after all, veridical. But the resulting perceptual realism is anemic at best if the objects of perception are never the objects of sensory awareness. The representationalist's eyes are closed to the primordial and fundamental act of perception whereby we place ourselves in the very heart of things. A robust perceptual realism, such as James' natural realism that Putnam admired, requires the rejection of the prevailing representationalist orthodoxy.



% section the_challenge (end)

% Bibligography
\bibliographystyle{plainnat}
\bibliography{Philosophy}

\end{document}