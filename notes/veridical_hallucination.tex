%!TEX TS-program = xelatex 
%!TEX TS-options = -synctex=1 -output-driver="xdvipdfmx -q -E"
%!TEX encoding = UTF-8 Unicode
%
%  veridical_hallucination
%
%  Created by Mark Eli Kalderon on 2014-10-14.
%  Copyright (c) 2014. All rights reserved.
%

\documentclass[12pt]{article} 

% Definitions
\newcommand\mykeywords{veridical hallucination, perception, presentation, representation}
\newcommand\myauthor{Mark Eli Kalderon}

% Packages
\usepackage{geometry} \geometry{a4paper} 
\usepackage{url}
% \usepackage{txfonts}
\usepackage{color}
% \usepackage{enumerate}
\definecolor{gray}{rgb}{0.459,0.438,0.471}
\usepackage{setspace}
% \doublespace % Uncomment for doublespacing if necessary
% \usepackage{epigraph} % optional

% XeTeX
% \usepackage[cm-default]{fontspec}
\usepackage{xltxtra,xunicode}
\defaultfontfeatures{Scale=MatchLowercase,Mapping=tex-text}
\setmainfont{Hoefler Text}

% Bibliography
\usepackage[round]{natbib}

% Title Information
\title{Veridical Hallucination and Perceptual Presentation}
\author{\myauthor} 
\date{} % Leave blank for no date, comment out for most recent date

% PDF Stuff
\usepackage[plainpages=false, pdfpagelabels, bookmarksnumbered, backref, pdftitle={Form Without Matter}, pagebackref, pdfauthor={\myauthor}, pdfkeywords={\mykeywords}, xetex, colorlinks=true, citecolor=gray, linkcolor=gray, urlcolor=gray]{hyperref} 

%%% BEGIN DOCUMENT
\begin{document}

% Title Page
\maketitle
% \begin{abstract} % optional
% \noindent
% \end{abstract} 
% \vskip 2em \hrule height 0.4pt \vskip 2em
% \epigraph{\textsc{}{}

% Layout Settings
\setlength{\parindent}{1em}

% Main Content

\section{Bergson} % (fold)
\label{sec:bergson}

In \emph{Matter and Memory}, \citet{Bergson:1912pi} criticizes both the realist and the idealist though he seeks to retain insights from each. At first this can sound as if he is rejecting the very distinction in much the same way that \citet{Quine:1951fk} rejects the very distinction between the analytic and the synthetic. However, closer attention to what Bergson means specifically by ``realism'' reveals that realism and idealism do not constitute a partition on available alternatives. According to the idealist, the objects of perception are not independent of perceptual experience. Rather, the objects of perception are constituted by the awareness our perceptual experience affords us of them. With idealism so understood, it might reasonably be thought that realism just is its denial. So understood, realism would merely consist in the claim that the objects of perception exist and have the character that they do independently of the awareness afforded by our perceptual experience of them. Call this minimal realism. However, according to Bergson, there is more to realism, as he conceives of it, than minimal realism. Not only must the objects of perception exist and have their character independently of our perceptual experience of them, but they must also be the cause of that perception and unlike it in character. Bergson's realist is exemplified by Descartes. Matter, for Descartes, is a mode of extension and homogenous, but perception seems to present us with qualitative heterogeneity. Descartes maintained, in effect, that certain homogenous modes of extension cause in us internal sensations that display a qualitative heterogeneity. 

Bergson proposes to argue, in defense of common sense, that the realist is right to insist that the objects of perception exist and have their character independently of our perceptual experience of them, but, also, that Berkeley was right in insisting that so-called secondary qualities have as much reality as primary qualities. In siding with Berkeley in defense of common sense, Bergson is rejecting a characteristic doctrine of early modern philosophy, that some experiences do not resemble their causes, specifically, our experience of what Aristotle described as the proper sensibles---sensory objects available to one sense alone and about whose presence no error is possible, \emph{De Anima} \textsc{ii} 6 418\( ^{a} \)11–12----do not resemble their causes (on the significance of the transformation of the Aristotelian proper sensibles into secondary qualities in the seventeenth century see \citealt{Smith:1990sm,Winkler:2011aa}).

In chapter one of \emph{Matter and Memory}, Bergson presents a battery of arguments against the realist and the idealist. I want to focus here on a claim he makes towards the end of that chapter:
\begin{quote}
	But for realism as for idealism, perceptions are `veridical hallucinations,' states of the subject, projected outside himself; and the two doctrines differ merely in this: that in the one these states constitute reality, in the other they are sent forth to unite with it. \citep[73]{Bergson:1912pi}
\end{quote}

How is it that for realism as for idealism perceptions are merely veridical hallucinations? In the quotation, Bergson speaks of perceptual experiences as conceived by the realist and the idealist as states of the subject. Earlier on the same page, he elaborates. Both the realist and the idealist conceive of perceptual experience as an internal state of a subject, a conscious modification of the perceiving subject. Bracket, for the moment, why they should be thought to be saddled with such a commitment, though it is plausibly attributed to both Descartes and Berkeley. Focus instead on why Bergson thinks that perceptual experience, so conceived, is at best a mode of veridical hallucination. Bergson's thought, here, is that perception is a mode of awareness. And that in order for experience to afford us an awareness of an object that both exists and has its character independently of that awareness, that experience must, in some sense, place us in that object. However, conceiving of perceptual experience as an internal state precludes this. If we conceive of perceptual experience as ``an internal state, a mere modification of our personality'' then:
\begin{quote}
	our eyes are closed to the primordial and fundamental act of perception, ---the act, constituting pure perception, whereby we place ourselves in the very heart of things. \citep[73]{Bergson:1912pi}
\end{quote}
There is much to say about Bergson's evocative description of the primordial and fundamental act of perception as placing ourselves in the very heart of things. Unfortunately, this is not the place to go into it. (Bergson's doctrine, here, essentially involves his take on a dialectic involving Aristotle, Alexander of Aphrodisias, and Plotinus. Moreover, it involves his claim that the content of our perception is determined by our motor capacities.). Nevertheless, it is easy to see the general form of Bergson's worry. Internal states of a perceiver, conscious modifications of the perceiving subject, do not place the perceiver in objects external to their body. Thus such objects are not present in undergoing such conscious modifications of the perceiving subject. And if they are not present in undergoing such conscious modifications of the perceiving subject then such an experience affords no awareness of them. So even if the realist and the idealist can, at least by their own lights, construct for themselves a sense in which such internal states may be veridical, since such states afford no genuine awareness of the natural environment, they are at best, veridical hallucinations of that environment. That, at the very lest, is the general form of Bergson's worry.

I find Bergson's charge---that for realism and idealism, at least as he conceives of them, perceptions are veridical hallucinations---both striking and apt. Establishing this would require further explanation and defense. However, in this essay, I want to argue for a variant of Bergson's claim, that perceptions, as conceived by contemporary representationalists, are veridical hallucinations. I will thus be developing, in my own way, a theme from Putnam's Dewey Lectures.

According to a familiar story, analytic philosophy begins with a revolt against idealism, the rebellion beginning principally in Cambridge and subsequently spreading throughout the anglophone world. Cambridge realists, such as \cite{Russell:1912uq}, \citet{Moore:1953nx}, and \citet{Price:1932fk}, maintained that perception affords us with a sensory mode of awareness and that we our knowledgeable of the mind-independent environment by being aware of that environment. They also maintained that \emph{all} sense experience, and not just perception, involves this sensory mode of awareness. Cambridge realists were thus committed to a kind of experiential monism (in Snowdon's \citeyear{Snowdon:2008oz} terminology): All sense experience involves, as part of its nature, a sensory mode of awareness. Even subject to illusion or hallucination, there is something of which one is aware. And with that, they were an application of the argument from illusion, or hallucination, or conflicting appearances away from immaterial sense data and a representative realism that tended, over time, to devolve into a form of phenomenalism. Interestingly, the Aristotelian proper sensibles that had been transformed in the seventeenth century into secondary qualities, are, within the tradition of Cambridge realism, transformed again, sense data being their new avatar. Most contemporary analytic philosophers believe neither in sense data, representative realism, nor phenomenalism. While realistically inclined, these aspects of Cambridge realism have largely been abandoned. Instead of sense data, contemporary philosophers of perception are more likely to speak of accuracy conditions or the representational content of perception \citep[though see][]{Robinson:1994ms}. And while perception is deemed to have a representational content, they will insist that this is insufficient for a representative realism. Rather, one is directly aware of the environment by one's perception accurately representing that environment. Moreover, few remain with phenomenalist sympathies (though see \citealt{Foster:2000ny} and \citealt{Noe:2004fk}). Nevertheless, \citet{Putnam:1994kx} has suggested that this contemporary orthodoxy has perhaps more in common with the sense-data theory, representative realism, and phenomenalism than they care to admit. It struck Putnam, at least at the time of the Dewey Lectures, as insufficiently demanding by the standards of James' natural realism that he admired and, like Cambridge realism, ultimately unsustainable. Taking my cue from Bergson, I shall develop Putnam's worry by reflecting on veridical hallucination.

% section bergson (end)

\section{Representationalism} % (fold)
\label{sec:representationalism}

One of the fundamental debates in contemporary philosophy of perception concerns whether experience has a representational character. That debate can be seen to turn on the following questions:
\begin{enumerate}
	\item Does perceptual experience involve the presentation of its object?
	\item If it does, is perceptual presentation explicable in terms of representational content?
\end{enumerate}

Does perceptual experience present its object? Perhaps not. Perhaps perceptual experience involves a primitive form of objective representation as \citet{Burge:2010uq} maintains. Or perhaps to suppose that perception presents its object is to fun afoul of the Myth of the Given on some appropriate understanding of Sellar's \citeyearpar{Sellars:1956xp} myth. 



% section representationalism (end)

\section{Veridical Hallucination} % (fold)
\label{sec:veridical_hallucination}

The term veridical hallucination was first introduced in the context of psychical research and originally designated a form of precognitive dream. The dream, appropriately interpreted, was veridical insofar as it correctly represented events that had yet to transpire. But since the events have yet to happen, and the dreams required interpretation, it seemed inappropriate to describe the experience of such precognitive dreams as perceptions, so they were deemed veridical hallucinations. Bergson was, for a time, the president of Society of Psychical Research, whose members included Broad and James. However, when Bergson charges that for the realist as for the idealist perceptions are veridical hallucinations, the notion of veridical hallucination is not the one commonly in play in discussions of psychical research. Rather it is closer to the notion of veridical hallucination that figures in Grice's \citeyearpar{Grice:1962jw} essay. But what notion is that?

Intuitively, the idea seems easy enough to state. An experience is a veridical hallucination, when, despite being veridical, the veridical subject matter of that experience is not present in the awareness afforded by that experience. While the gloss requires explication, I do not think that it is terribly controversial as it stands, even if it may be controversial whether this or that example genuinely counts as an instance of veridical hallucination so glossed. However, as we shall see, there are complications in getting it to cohere with the views of either side of the fundamental debate in contemporary philosophy of perception concerning the representational character of perceptual experience.

Consider philosophers who deny that experience has a representational content, insisting instead that perception, at least, essentially involves an irreducible presentational element. An immediate problem for their accepting the present gloss is how to understand the veridicality of veridical hallucinations, for the most straightforward way to understand this is debarred from them. The most straightforward way to understand the veridicality would be to assign such experiences accuracy conditions or a representational content that are fulfilled in the circumstances of perception. But if no sensory experience has accuracy conditions or representational content, then veridicality of veridical hallucinations cannot be understood in terms of these experiences possessing veridical contents. The present difficulty is not very deep, however. Even if perceptual experience and sensory experience more generally cannot, strictly speaking, be veridical in the sense of correctly representing their object, typically accompanying such experiences are states that can straightforwardly be assessed for veridicality. 

Taking a stroll in Greenwich park, I turn, and look, and see an ancient chestnut tree. It is one of the sweet chestnut trees replanted there when the park was redesigned for Charles \textsc{ii} in the 1660s. Beneath it, among the scattered burrs, humans and squirrels vie with one another for chestnuts. The burrs that remain on the tree are large and brighter green than the surrounding foliage. Despite its manifest strength and stability, its trunk seems to flow in a wave like form. When I look upon the chestnut tree, being as it is, various things can seem true to me. As \citet{Price:1952ix} pointed out long ago, such seemings are not sensory appearances so much as epistemic reactions to what appears to us in sense experience. Looking upon the chestnut tree various judgments can see to be true. It can seem, for example, that the tree retains many of its burrs. This judgment seems true to me, since my experience seems to afford me awareness of aspects of the tree upon which the truth of the judgment depends---I see the bright green burrs set against the darker foliage. So the idea would be that a veridical hallucination is veridical in the sense that in undergoing that experience various judgments that seem true to the subject are in fact true. In this way the veridical character of veridical hallucination can be accommodated by a theorist that denies experience has accuracy conditions or a representational content more generally.

Representationalists, on the other hand, can accept the straightforward understanding of the veridicality of veridical hallucination. Insofar as they are willing to assign correctness conditions or representational content to sensory experience, such experiences will be veridical if they have veridical contents. So when a subject veridically hallucinates a scene the visual experience they undergo has a content that veridically represents that scene. The challenge for the representationalist does not consist in accommodating somehow the veridicality of veridical hallucination, but with accommodating the contrasting feature of the intuitive gloss, that the veridical subject matter of the visual experience is not present in the awareness, if any, afforded by that experience. Specifically, the challenge to the representationalist is to explain how something may be present in the awareness afforded by sense experience so as to give content to the denial which constitutes the contrasting feature of the intuitive gloss. Given the possibility of veridical hallucinations, having a veridical subject matter is insufficient for being aware of that subject matter. Veridical hallucinations are precisely veridical experiences that afford no awareness of their veridical subject matter. So being present in the awareness afforded by sense experience must be understood in some other way. But what way is that? That is the challenge that we shall be exploring.

% section veridical_hallucination (end)

\section{The Challenge} % (fold)
\label{sec:the_challenge}

% section the_challenge (end)

% Bibligography
\bibliographystyle{plainnat}
\bibliography{Philosophy}

\end{document}