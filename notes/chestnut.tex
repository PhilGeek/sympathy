%!TEX TS-program = xelatex 
%!TEX TS-options = -synctex=1 -output-driver="xdvipdfmx -q -E"
%!TEX encoding = UTF-8 Unicode
%
%  chestnut
%
%  Created by Mark Eli Kalderon on 2014-10-10.
%  Copyright (c) 2014. All rights reserved.
%

\documentclass[12pt]{article} 

% Definitions
\newcommand\mykeywords{active, passive, perception}
\newcommand\myauthor{Mark Eli Kalderon}

% Packages
\usepackage{geometry} \geometry{a4paper} 
\usepackage{url}
% \usepackage{txfonts}
\usepackage{color}
% \usepackage{enumerate}
\definecolor{gray}{rgb}{0.459,0.438,0.471}
\usepackage{setspace}
% \doublespace % Uncomment for doublespacing if necessary
% \usepackage{epigraph} % optional

% XeTeX
% \usepackage[cm-default]{fontspec}
\usepackage{xltxtra,xunicode}
\defaultfontfeatures{Scale=MatchLowercase,Mapping=tex-text}
\setmainfont{Hoefler Text}

% Bibliography
\usepackage[round]{natbib}

% Title Information
\title{A Puzzle Upon Viewing a Chestnut Tree}
\author{\myauthor} 
\date{} % Leave blank for no date, comment out for most recent date

% PDF Stuff
\usepackage[plainpages=false, pdfpagelabels, bookmarksnumbered, backref, pdftitle={Form Without Matter}, pagebackref, pdfauthor={\myauthor}, pdfkeywords={\mykeywords}, xetex, colorlinks=true, citecolor=gray, linkcolor=gray, urlcolor=gray]{hyperref} 

%%% BEGIN DOCUMENT
\begin{document}

% Title Page
\maketitle
% \begin{abstract} % optional
% \noindent
% \end{abstract} 
% \vskip 2em \hrule height 0.4pt \vskip 2em
% \epigraph{\textsc{}{}

% Layout Settings
\setlength{\parindent}{1em}

% Main Content


Among the many pleasures of Greenwich Park are its ancient chestnut trees. In the early autumn, humans and squirrels vie with one another in foraging among the fallen burrs. Standing before one of the sweet chestnut trees replanted there when the park was redesigned for Charles \textsc{ii} in the 1660s, an organism of impressive size and age presents itself. The majority of its burrs remain on the tree and are brighter green than the surrounding foliage. It is early evening, and the light is long and golden. The light both articulates the fine texture of the bark and sets off the overall flow of the trunk in dramatic relief. Despite its manifest strength and solidity, the twisted trunk appears to be flowing in a wave like form. I come to realize that I am witnessing an organic process, the growth of the trunk, occurring so slowly as to appear, from within my limited temporal perspective, to be frozen, static. The difference in the scale of our lives is striking. For a moment it induces in me a kind a kind of temporal vertigo.  Just as a radical difference in spatial scale can be vertiginous---(think of how small one can feel when viewing the Milky Way), a radical difference in temporal scale can be vertiginous as well. The scale of its life and the strength manifest in centuries of growth make the sweet chestnut tree a fit object of awe. I find myself musing that in a different cultural context, perhaps one more prone to animism, the tree might reasonably be reckoned a god. 

I turn, and look, and see the ancient chestnut tree. In the natural history of my so seeing, the chestnut tree, the object of my perception, figures prominently as a causal antecedent. The tree's trunk appears to flow because the trunk has that flowing structure prior to its so appearing. Its flowing structure, illuminated in the golden light of the early evening, causes me to see that structure. This observation is perhaps the source of the long tradition of thinking of perception as a passive capacity. At the very least, it is plausible that it is a materially necessary condition on a subject perceiving an object that the relevant sensory organs of the subject be acted upon, however mediately, by that object. Thus the chestnut tree must mediately alter my eyes---changing patterns of retinal stimulation, say---by altering the early evening's light in a complex subatomic process occurring at its surface.

% Modern philosophy had an interesting effect on the doctrine of perception's passivity. Insofar as perception is subject to mechanical explanation, it involves the subject's sensory organs being acted upon. But this is just the observation that encourages the thought that perception is passive. However, in rejecting active powers, modern philosophy leaves no room to draw the distinction between active and passive. And this has the effect of obscuring the consensus that had emerged, that perception is passive in the minimal sense that it is the effect of the perceiver's sense organs being acted upon, however mediately, by the object of perception.
%
% Despite the lingering effects of the modern paradigm, in particular the attrition of the active/passive distinction, it is nonetheless easy to be puzzled by perception's alleged passivity.

I turn, and look, and see the ancient chestnut tree. The flowing structure of its trunk mediately acts upon my eyes by acting upon the early evening illumination. And this is part of the natural history of my seeing the flowing structure of the tree's trunk. Not only must my eyes be mediately acted upon by the tree, but my seeing the tree in the golden light with its bright green burs and flowing trunk is an experience that I undergo. Even though my seeing of the tree is an experience that I undergo as a result of the tree mediately acting upon my eyes, my seeing of the tree is not something \emph{done} to me by the tree. It is \emph{I} that see the tree. It is I, and not the tree, that is doing the seeing. Even acknowledging the passive elements in the natural history and phenomenology of my perception (that the tree is the cause of my seeing it and that my seeing it is an experience that I undergo), there is a residual active element in my seeing. The tree may make itself seen, but my seeing of the tree is activity properly attributed to me and not the tree.

How are we to understand this?

Perhaps unsurprisingly, the puzzle is an ancient one. \citet[chapter 9.4.3]{Beere:2009vn} sees a version of the puzzle lying behind Aristotle's discussion of perception in \emph{De Anima} \textsc{ii} 5 (see \citealt[chapter 8]{Kalderon:2015fr} for discussion). Even by the standards of \emph{De Anima}, \emph{De Anima} \textsc{ii} 5 is notoriously difficult chapter to interpret and is thus, quite reasonably, a fit object of scholarly dispute (see, \emph{inter alia}, \citealt{Burnyeat:2002an}, \citealt{Heinaman:2007ys}, and \citealt{Bowin:2011uq}). However, despite Aristotle's hesitancy and incessant qualifications, I believe that the following clear idea can be extracted from his discussion (though I will develop it in my own manner). 

When I see the ancient chestnut tree, I exercise a natural capacity of mine, my natural capacity for sight. This is my capacity, a capacity that I possess, at least in the sense that the exercise of that capacity is something I do. Seeing is the exercise of sight and it is the animal that sees and not his eyes (nor his visual system, nor his brain). However, my natural capacity for sight, and indeed my perceptual capacities more generally, are distinguished from other capacities that I possess, in at least one important respect. 

Unlike my capacity for intentional action, the exercise of my perceptual capacities is not up to me. In the case of intentional action, there is a clear sense in which, right now, it is up to me to continue writing. If I wish I could push away the laptop and retire to the kitchen to make a cup of coffee. But right now, in my present circumstances, seeing what I do, is not up to me, at least not in this way. Of course I can change my circumstances and see other things, and it is up to me to do so, but I would merely make different things perceptually available by so altering my circumstances. The object of my perception must not only be present in the circumstances of perception but it must also act upon me in the appropriate way, by altering the relevant sense organs, however mediately, in order for me to exercise that capacity and so perceive it. Perception is, in Nietzsche's terminology, a \emph{reactive} capacity. It only ever acts by reacting to the presence of its object. Since sight is a natural capacity of mine, it is a capacity I posses, seeing is something I do. But seeing what I do is only possible insofar as the object of my perception is both present and the cause, in the appropriate manner, of my so seeing by acting, however mediately, upon my eyes.

% Bibligography
\bibliographystyle{plainnat}
\bibliography{Philosophy}

\end{document}