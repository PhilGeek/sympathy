%!TEX TS-program = xelatex 
%!TEX TS-options = -synctex=1 -output-driver="xdvipdfmx -q -E"
%!TEX encoding = UTF-8 Unicode
%
%  grasping
%
%  Created by Mark Eli Kalderon on 2014-10-13.
%  Copyright (c) 2014. All rights reserved.
%

\documentclass[12pt]{article} 

% Definitions
\newcommand\mykeywords{active, passive, perception}
\newcommand\myauthor{Mark Eli Kalderon}

% Packages
\usepackage{geometry} \geometry{a4paper} 
\usepackage{url}
% \usepackage{txfonts}
\usepackage{color}
% \usepackage{enumerate}
\definecolor{gray}{rgb}{0.459,0.438,0.471}
\usepackage{setspace}
% \doublespace % Uncomment for doublespacing if necessary
% \usepackage{epigraph} % optional

% XeTeX
% \usepackage[cm-default]{fontspec}
\usepackage{xltxtra,xunicode}
\defaultfontfeatures{Scale=MatchLowercase,Mapping=tex-text}
\setmainfont{Hoefler Text}

% Bibliography
\usepackage[round]{natbib}

% Title Information
\title{Grasping}
\author{\myauthor} 
\date{} % Leave blank for no date, comment out for most recent date

% PDF Stuff
\usepackage[plainpages=false, pdfpagelabels, bookmarksnumbered, backref, pdftitle={Form Without Matter}, pagebackref, pdfauthor={\myauthor}, pdfkeywords={\mykeywords}, xetex, colorlinks=true, citecolor=gray, linkcolor=gray, urlcolor=gray]{hyperref} 

%%% BEGIN DOCUMENT
\begin{document}

% Title Page
\maketitle
% \begin{abstract} % optional
% \noindent
% \end{abstract} 
% \vskip 2em \hrule height 0.4pt \vskip 2em
% \epigraph{\textsc{}{}

% Layout Settings
\setlength{\parindent}{1em}

% Main Content

\section{The Dawn of Understanding} % (fold)
\label{sec:grasping_and_the_dawn_of_understanding}

In a justly famous scene from \emph{2001: A Space Odyssey} set to Richard Strauss' \emph{Also Sprach Zarathustra}, a hominid ancestor, squatting among the skeletal remains of a boar, reaches out and tentatively grasps a femur. It is telling that this is how Stanley Kubrick chose to dramatize the initial transformation, induced by an alien obelisk, of our hominid ancestors that eventually gives rise to space-exploring humanity in the twenty-first century. Not only does our hominid ancestor grasp the femur, but they grasp as well an important application. Squatting among the skeletal remains, femur in hand, our hominid ancestor taps the boar's remains in exploratory manner. Each strike of the femur grows in force until finally, in a crescendo of activity, they smash the boar's skull into pieces. Our hominid ancestor has reached a crucial insight, that an implement such as the femur might kill a boar thus transforming them into prey. Moreover, the application generalizes. The femur might also be used as a weapon against competing groups of hominids. The acquired technology thus has political consequences. But what is presently important is the connection between grasping and cognition. We say we have grasped a situation when we have understood it. And philosophers are prone to speak of thinkers grasping the thoughts they think. Kubrick dramatizes the connection between grasping and cognition by having our hominid ancestor's grasping the femur among the boar's skeletal remains be the primal scene of a dawning understanding.

We have grasped a situation when we have understood it. We have a grip on it. If the understanding in question is practical, we might say that we have matters in hand. Nor are tactile metaphors confined to forms of higher cognition. They persist as well in our description of perceptual awareness. Not only do we speak of recognizing an object that we see as grasping the object present in our perceptual experience, but the presentation in experience is itself a kind of grasping. Perception puts us in contact with its object. In perceiving an object we apprehend it. The tactile metaphors for perceptual awareness tend to be modes of assimilation, and ingestion is a natural variant. Our hominid ancestor, looking up from the boar's remains, takes in the scene before them.

What makes tactile metaphors for perception apt? Tactile metaphors for perceptual awareness, even for non-tactile modes of awareness such as vision and audition, are primordial and persistent. Most contemporary philosophers of perception apply them unselfconsciously; that they do is a testament to the power of such metaphors. Understanding the power they have over us, understanding what makes them so compelling, we gain insight into the object of those metaphors. In understanding what makes grasping an apt metaphor for perception generally, we gain insight into the nature of perceptual presentation. Or so I shall argue.

We shall begin with a phenomenological investigation into the nature of grasping, a form of haptic touch. Once we have a better understanding of how grasping presents itself from within tactile experience, we will be in a better position to understand why grasping also presents itself as an exemplar of perceptual presentation. Grasping may be an exemplar of perceptual presentation, but that does not necessarily mean that all perception is a form of touch. One may grant that tactile metaphors for perceptual awareness are apt while eschewing any such reductive explanatory ambitions. Such ambitions were rife in classical Greek antiquity. Thus \citet[39]{Lindberg:1977aa} observes that in the ancient world ``the analogy of perception by contact in the sense of touch seemed to establish to nearly everybody’s satisfaction that contact was tantamount to sensation, and it was not apparent that further explanation was required.'' Aristotle criticizes this reductive explanatory ambition. Conceiving of non-tactile modes of perceptual awareness on the model of touch will seem explanatory insofar as touch is antecedently understood to be an unproblematic mode of perception. Aristotle's belaboring and not always completely resolving the \emph{aporiai} concerning touch in \emph{De Anima} \textsc{ii} undermines that assumption. And if further explanation is required, then we can no longer simply assume that contact is tantamount to sensation. Nevertheless, Aristotle accepts the aptness of the metaphor. Perception, for Aristotle, remains a mode of assimilation. So acceptance of the aptness of grasping as a metaphor for perceptual presentation need not carry with it commitment to any such reductive explanatory ambitions.

% section grasping_and_the_dawn_of_understanding (end)

\section{Haptic Perception} % (fold)
\label{sec:haptic_perception}

% section haptic_perception (end)

Grasping is a form of haptic touch. Haptic touch involves active exploration of the object of touch. This can involve a range of different exploratory activities. Thus to discern the texture of an object the perceiver may deploy lateral movement across its surface. Holding a stone in its hand, our hominid ancestor may feel the smoothness of the stone by rubbing its thumb across its surface. And its hardness may be felt by applying pressure to it. According to the taxonomy of \citet{Lederman:1987fr}, grasping is a distinctive exploratory activity that they describe as ``enclosure''. Grasping an object allows the perceiver to discern a different range of tangible properties. If texture is perceived by lateral motion and hardness by applying pressure, grasping or enclose makes volume and global shape available in tactile experience (though, of course, grasping is itself a way of applying pressure to an object and, hence, a way of perceiving the hardness of the object grasped). With enclosure, Lederman and Klatzky write:
\begin{quote}
	\ldots the hand maintains simultaneous contact with as much of the envelope of the object as possible. Often one can see an effort to mold the hand more precisely to object contours. Periods of static enclosure may alternate with shifts of the object in the hand(s). \citep[346--7]{Lederman:1987fr}
\end{quote}
The quoted passage brings out at least two important features of grasping, understood as a mode of haptic perception. 

First, grasping is an activity and so is spread over time. It has duration. Not only does our hominid ancestor tentatively reach out and grasp the boar's femur from amongst its skeletal remains, an event with duration, but its grasp must be actively maintained over a period of time. Maintaining simultaneous contact with the overall surface of a rigid body, or some nonsignificant portion of it, is a state sustained by activity. In this regard, it is like Ryle's \citeyearpar[149]{Ryle:1949qr} example of keeping the enemy at bay. Thus the state obtains for the duration of the sustaining activity. Moreover, in coming to perceive its overall shape and volume, the perceiver may shift the object in their hand. The tactile sense of an object's overall shape or volume is disclosed by such activity. And since activity has duration, it is disclosed over time. The presentation of the overall shape and volume of an object in tactile experience is itself spread over time like the activity that discloses it. One potential lesson, then, for the metaphysics of perceptual presentation, is that the presence of an object of perception may be disclosed over time. 

What explains the tendency, in grasping or enclosure, to shift the object periodically in one's hands? The full explanation involves, I suspect, the second feature of grasping or enclosure brought out by the passage. 

Lederman and Klatzky observe that there is a tendency for perceivers to exert effort to mold their hand more precisely to the contours of the object grasped. In grasping an object, the grasping hand in this way assimilates to the overall shape and volume of the object grasped. Consider grasping a solid, rigid body, such as a stone. In grasping a stone, our hominid ancestor extends their hand's activity, they tighten their grasp, until they can no more. Since the stone is solid, it resists penetration. Since it is rigid, it maintains its overall shape and volume even when in the hominid's grasp. (Contrast the way the overall shape and volume of an elastic body, such a sponge, deforms as it is squeezed.) With the stone in its grip, the hand of our hominid ancestor assimilates to the overall shape and volume of the stone. Of course, hands are unevenly shaped and imperfectly elastic. This means that an effort to mold one's hand to a rigid body thus disclosing its overall shape and volume will, most likely, be imperfectly realized. There may be some areas of the object's surface which the grasping hand does not conform to. The tendency to shift the grasped object in our hands compensates for this partial and imperfect disclosure. In shifting the object in one's hand, an area that the hand did not previously conform to may become accessible to touch. Successive grips may provide a better overall sense of the shape and volume of the rigid body. 

That the grasping hand assimilates to the overall shape and volume of the object grasped is potentially epistemically significant. The full case for this will have to wait, but we can begin to get a sense of why this might be so. A rigid, solid body has a certain overall shape and volume prior to being grasped. Moreover, it is sufficiently rigid and solid to maintain that overall shape and volume even when grasped. In making an effort to more precisely mold the hand to the contours of the rigid object, the hand thus takes on, to an approximate degree, the overall shape and volume of the object grasped. That is to say, the hand takes on a certain configuration determined by the hand's anatomy, the activity of the hand, and the overall shape of the object grasped. And the hand, so configured, encompasses a region of a certain volume itself determined by the hand and the volume of the object grasped. That is the point of making an effort to more precisely mold the hand to contours of the rigid object. In engaging in such haptic activity, in molding one's hand more precisely to the contours of the object, the overall shape and volume of the object had prior to being grasped, and maintained in being grasped. explains, in part, the hand's configuration in grasping the object and the force that needs to be exerted to maintain that configuration. Suppose that it is our hand's configuration in grasping and the force that needs to be exerted in maintaining that configuration that discloses the overall shape and volume of the object. (If so, at least in the present instance, haptic perception is dependent on proprioception and kinesthesis.) Since the object's overall shape and volume explains the hand's configuration and force, if the object eludes the hand's grasp, then that configuration and force would not have occurred. Perhaps the objectivity of grasping, understood as a mode of haptic perception, consists in the grasping hand's assimilating to the tangible features of the object had prior to grasping.

It might be objected that, at least for certain graspings, it is possible for the object to be absent and yet the hand to be in a duplicate configuration. However, a felt difference would remain. Maintaining the hand's configuration in the absence of the object requires different musculature activity since the perceiver can no longer rely on pressing against the rigid body in maintaining that configuration. The different pattern of activation of receptors in muscles and joints will result in a felt difference. Compare leaning against a wall with making as if to lean against a wall. Sustaining that posture in the absence of the supporting wall can be difficult to do. Miming is an acquired skill. As Jacques Tati demonstrates in \emph{Cours du Soir}, it can be taught and learned. So in the case of duplicate configuration, where the hand takes on the configuration it would have had if it were grasping the object, while the hand's configuration has been maintained in the absence of the object, there is a felt difference in the force exerted.

That the grasping hand assimilates to the contours of the object grasped is potentially epistemically significant. It is, then, if not the source of that haptic perception's objectivity, then its manifestation. If in grasping an object, the hand's configuration and force discloses the object's overall shape and volume, and that configuration and force would not have occurred in the absence of the object grasped, then our tactile experience would not be as it is when we haptically perceive if that object were in fact absent. While not yet proof against a Cartesian demon, one can begin to see the potential epistemic significance of the effort exerted in more precisely molding one's hand against the contours of the object grasped. It is the means by which certain tangible qualities of an external body are disclosed in touch. 

In the \emph{Theaetetus} 156 a--c, Socrates elaborates the Secret Doctrine of Protagoras by providing an account of perception as the contingent outcome of active and passive forces in conflict. Grasping as a mode of haptic perception can seem to approximate to this account. At the very least, the felt shape and volume of the object grasped is determined by conflicting forces. On the one hand, there is the force exerted in molding the hand more precisely to the contours of the rigid body. On the other hand, there are the self-maintaining forces of the rigid body itself. A solid, rigid body, such as a stone picked up by a hominid ancestor, is no mere sum of matter. It has a form or material structure determined by forces that are the categorical basis for its rigidity and solidity \citep{Johnston:2006js}. Haptic perception is the joint upshot of the force exerted by the grasping hand and the self-maintaining forces of the object grasped. There remains a difference, however, from the account elaborated by Socrates. The overall shape and volume of the object and our haptic perception of them are not ``twin births'' as Protagoras maintains. The forces that determine the object's rigidity and solidity are sufficient to maintain the object's overall shape and volume within the hand's grasp. So the perceived tangible qualities of the external body inhere in that body prior to being perceived; whereas in the account attributed to Protagoras, the perceived object comes into being with the perceiver's perception of it.

To fully appreciate the epistemic significance of the force of the hand's activity in more precisely molding to the contours of the object grasped, we need an account of how it is that assimilating to an external body discloses the tangible qualities that inhere in that body prior to haptic perception. The question here is a ``how possible'' question \citep{Cassam:2007lq}. How is objective haptic perception so much as possible?


\section{An Historical Interlude} % (fold)
\label{sec:an_historical_interlude}



% section an_historical_interlude (end)

\section{A puzzle} % (fold)
\label{sec:a_puzzle}



% section a_puzzle (end)

\section{section name} % (fold)
\label{sec:section_name}

% section section_name (end)

% Bibligography
\bibliographystyle{plainnat}
\bibliography{Philosophy}

\end{document}