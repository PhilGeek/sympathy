%!TEX TS-program = xelatex 
%!TEX TS-options = -synctex=1 -output-driver="xdvipdfmx -q -E"
%!TEX encoding = UTF-8 Unicode
%
%  grasping
%
%  Created by Mark Eli Kalderon on 2014-10-13.
%  Copyright (c) 2014. All rights reserved.
%

\documentclass[12pt]{article} 

% Definitions
\newcommand\mykeywords{active, passive, perception}
\newcommand\myauthor{Mark Eli Kalderon}

% Packages
\usepackage{geometry} \geometry{a4paper} 
\usepackage{url}
% \usepackage{txfonts}
\usepackage{color}
% \usepackage{enumerate}
\definecolor{gray}{rgb}{0.459,0.438,0.471}
\usepackage{setspace}
% \doublespace % Uncomment for doublespacing if necessary
% \usepackage{epigraph} % optional

% XeTeX
% \usepackage[cm-default]{fontspec}
\usepackage{xltxtra,xunicode}
\defaultfontfeatures{Scale=MatchLowercase,Mapping=tex-text}
\setmainfont{Hoefler Text}

% Bibliography
\usepackage[round]{natbib}

% Title Information
\title{Grasping}
\author{\myauthor} 
\date{} % Leave blank for no date, comment out for most recent date

% PDF Stuff
\usepackage[plainpages=false, pdfpagelabels, bookmarksnumbered, backref, pdftitle={Form Without Matter}, pagebackref, pdfauthor={\myauthor}, pdfkeywords={\mykeywords}, xetex, colorlinks=true, citecolor=gray, linkcolor=gray, urlcolor=gray]{hyperref} 

%%% BEGIN DOCUMENT
\begin{document}

% Title Page
\maketitle
% \begin{abstract} % optional
% \noindent
% \end{abstract} 
% \vskip 2em \hrule height 0.4pt \vskip 2em
% \epigraph{\textsc{}{}

% Layout Settings
\setlength{\parindent}{1em}

% Main Content


In a justly famous scene from \emph{2001 A Space Odyssey} set to Richard Strauss' \emph{Also Sprach Zarathustra}, a hominid ancestor, squatting among the skeletal remains of a boar, reaches out and tentatively grasps a femur. It is telling that this is how Stanley Kubrick chose to dramatize the initial transformation of our hominid ancestors, induced by an alien obelisk, that eventually gives rise to space exploring humanity in the twenty first century. Not only does our hominid ancestor grasp the femur, but they grasp as well an important application. Squatting among the skeletal remains of a boar, femur in hand, our hominid ancestor taps the boar's remains in exploratory manner. Each strike of the femur grows in force until they finally smash the skull of the boar with it. Our hominid ancestor has reached a crucial insight, that an implement such as the femur might kill a boar thus transforming them into prey. Moreover, the application generalizes. The femur might be used as a weapon against competing groups of hominids. The acquired technology thus has political consequences. But what is presently telling is the connection between grasping and cognition. We say we have grasped a situation when we have understood it. And philosophers are prone to speak of thinkers grasping the thoughts they think. Kubrick dramatizes the connection between grasping and cognition by having our hominid ancestor's grasping the femur among the boar's remains be the primal scene of a dawning understanding.

We have grasped a situation when we have understood it. We have a grip on it. If the understanding in question is practical, we might say that we have matters in hand. Nor are tactile metaphors confined to forms of higher cognition. They persist as well in our description of perceptual awareness. Seeing an object puts us in perceptual contact with it. Not only do we speak of recognizing an object that we see as grasping the object present in our perceptual experience, but the presentation in experience is itself a kind of grasping. Sometimes perception is described as a mode of apprehension, and Broad describes it as a mode of prehension.

% Bibligography
\bibliographystyle{plainnat}
\bibliography{Philosophy}

\end{document}