%!TEX TS-program = xelatex 
%!TEX TS-options = -synctex=1 -output-driver="xdvipdfmx -q -E"
%!TEX encoding = UTF-8 Unicode
%
%  grasping
%
%  Created by Mark Eli Kalderon on 2014-10-13.
%  Copyright (c) 2014. All rights reserved.
%

\documentclass[12pt]{article} 

% Definitions
\newcommand\mykeywords{active, passive, perception}
\newcommand\myauthor{Mark Eli Kalderon}

% Packages
\usepackage{geometry} \geometry{a4paper} 
\usepackage{url}
% \usepackage{txfonts}
\usepackage{color}
% \usepackage{enumerate}
\definecolor{gray}{rgb}{0.459,0.438,0.471}
\usepackage{setspace}
% \doublespace % Uncomment for doublespacing if necessary
% \usepackage{epigraph} % optional

% XeTeX
% \usepackage[cm-default]{fontspec}
\usepackage{xltxtra,xunicode}
\defaultfontfeatures{Scale=MatchLowercase,Mapping=tex-text}
\setmainfont{Hoefler Text}

% Bibliography
\usepackage[round]{natbib}

% Title Information
\title{Grasping}
\author{\myauthor} 
\date{} % Leave blank for no date, comment out for most recent date

% PDF Stuff
\usepackage[plainpages=false, pdfpagelabels, bookmarksnumbered, backref, pdftitle={Form Without Matter}, pagebackref, pdfauthor={\myauthor}, pdfkeywords={\mykeywords}, xetex, colorlinks=true, citecolor=gray, linkcolor=gray, urlcolor=gray]{hyperref} 

%%% BEGIN DOCUMENT
\begin{document}

% Title Page
\maketitle
% \begin{abstract} % optional
% \noindent
% \end{abstract} 
% \vskip 2em \hrule height 0.4pt \vskip 2em
% \epigraph{\textsc{}{}

% Layout Settings
\setlength{\parindent}{1em}

% Main Content

\section{The Dawn of Understanding} % (fold)
\label{sec:grasping_and_the_dawn_of_understanding}

In a justly famous scene from \emph{2001: A Space Odyssey}, set to Richard Strauss' \emph{Also Sprach Zarathustra}, a hominid ancestor, squatting among the skeletal remains of a boar, reaches out and tentatively grasps a femur. It is telling that this is how Stanley Kubrick chose to dramatize the initial transformation, induced by an alien obelisk, of our hominid ancestors, that eventually gives rise to space-exploring humanity in the twenty-first century. Not only does our hominid ancestor grasp the femur, but they grasp as well an important application. Squatting among the skeletal remains, femur in hand, our hominid ancestor taps the bones in exploratory manner. Each strike of the femur grows in force until finally, in a crescendo of activity, they smash the boar's skull to pieces. Our hominid ancestor has reached a crucial insight, that an implement, such as the femur, might transform boar into prey. Moreover, the application generalizes. The femur might also be used as a weapon against competing groups of hominids. The acquired technology thus has political consequences. What is presently important, however, is the connection between grasping and cognition. We say we have grasped a situation when we have understood it. And philosophers are prone to speak of thinkers grasping the thoughts they think. Kubrick dramatizes the connection between grasping and cognition by having our hominid ancestor's grasping the femur among the boar's skeletal remains be the primal scene of a dawning understanding.

We have \emph{grasped} a situation when we have understood it. We have a \emph{grip} on it. If the understanding in question is practical, we might say that we have matters \emph{in hand}. Nor are tactile metaphors confined to forms of higher cognition. They persist as well in our description of perceptual awareness. Not only do we speak of recognizing an object that we see as grasping the object present in our perceptual experience, but the presentation in experience is itself a kind of grasping. Perception puts us in \emph{contact} with its object. In perceiving an object we \emph{apprehend} it. The tactile metaphors for perceptual awareness tend to be modes of assimilation, and \emph{ingestion} is a natural variant. Our hominid ancestor, looking up from the boar's remains, \emph{takes in} the scene before them. If they see the obelisk, then, in a manner of speaking common among contemporary philosophers, the obelisk is the \emph{content} of our hominid ancestor's perception. But if the obelisk is the content of their perception then their perception of it is its container. To bring something into view so that it figures in the content of perception would be to contain it within that perception. But containment itself is a mode of assimilation. 

What makes tactile metaphors for perception apt? Tactile metaphors for perceptual awareness, even for non-tactile modes of awareness such as vision and audition, are primordial and persistent. Most contemporary philosophers of perception apply them unselfconsciously. That they do is a testament to the power of such metaphors. Understanding the power they have over us, understanding what makes them so compelling, we gain insight into the object of these metaphors. In understanding what makes grasping an apt metaphor for perception generally, we gain insight into the nature of sensory presentation. Or so I suggest.

We shall begin with a phenomenological investigation into the nature of grasping, a form of haptic touch. The investigation is phenomenological in that it seeks to uncover how grasping, understood as a mode of haptic perception, presents itself from within tactile experience. It is phenomenological because the object of investigation is restricted to perceptual appearances and not because of any methodology deployed in pursuing that investigation. The investigation thus need not involve ``bracketing'', nor need it confine itself to the deliverances of introspection in determining the nature of haptic appearance. In trying to understand how grasping, understood as a mode of haptic perception, presents itself from within tactile experience, we may avail ourselves of empirical and literary resources. Once we have a better understanding of how grasping presents itself from within tactile experience, we will be in a better position to understand why grasping also presents itself as an exemplar of sensory presentation more generally. 

Grasping may be an exemplar of sensory presentation, but it does not follow that all perception is a form of touch. One may grant that tactile metaphors for perceptual awareness are in some sense apt while eschewing any such reductive explanatory ambition. Such ambitions were rife in Greek antiquity. Thus \citet[39]{Lindberg:1977aa} observes that in the ancient world ``the analogy of perception by contact in the sense of touch seemed to establish to nearly everybody’s satisfaction that contact was tantamount to sensation, and it was not apparent that further explanation was required.'' Aristotle criticizes this explanatory strategy. Conceiving of non-tactile modes of perceptual awareness on the model of touch will seem explanatory insofar as touch is antecedently understood to be an unproblematic mode of perception. However, Aristotle's belaboring and not always completely resolving the \emph{aporiai} concerning touch in \emph{De Anima} 2 11 undermines that assumption \citep{Derrida:2005aa,Kalderon:2015fr}. And if further explanation is required, then we can no longer simply assume that contact is tantamount to sensation. Nevertheless, Aristotle accepts the aptness of the metaphor. Perception, for Aristotle, remains a mode of assimilation. The Stagirite defines perception as the assimilation of sensible form without the matter of the perceived particular (\emph{De Anima} 2 12 424\( ^{a} \)18–23, 2 5 418\( ^{a} \)3–6 ). So acceptance of the aptness of the metaphor carries with it no commitment to any such reductive explanatory ambition. Grasping may be apt metaphor for perception, even for non-tactile modes of perceptual awareness, without perception being reduced to a form of touch.

% section grasping_and_the_dawn_of_understanding (end)

\section{Haptic Perception} % (fold)
\label{sec:haptic_perception}

% section haptic_perception (end)

Grasping is a form of haptic touch. Haptic touch involves active exploration of the tangible object. This can involve a range of different stereotypical exploratory activities often combined in sequence. The different stereotypical exploratory activities are suited to presenting different ranges of tangible qualities. Thus to discern the texture of an object the perceiver may deploy lateral movement across its surface. Holding a stone in its hand, our hominid ancestor may feel the roughness of the stone by rubbing their thumb across its surface. And its hardness may be felt by applying pressure to it. According to the taxonomy of \citet{Lederman:1987fr}, grasping is a distinctive exploratory activity that they describe as ``enclosure''. Grasping an object allows the perceiver to discern a different range of tangible qualities. If texture is perceived by lateral motion and hardness by applying pressure, grasping or enclose makes volume and global shape available in tactile experience. Other stereotypical exploratory activities include: ``static contact''---passively resting one's hand on an externally supported object without an effort to mold to its contours to determine its temperature, ``unsupported holding''---holding the object without external support and without molding to determine the object's heft or weight often involving a ``weighing'' motion, ``contour following''---a smooth nonrepetative tracing of the contours of the object, ``part motion test''---moving a part of the object independently of the whole, and ``specific function test''---moving the object in such a way as to perform various functions. Though these stereotypical exploratory activities are optimized for determining a specific range of tangible qualities, they can determine other tangible qualities, though perhaps less well, with less tactual acuity. Thus, for example, grasping is itself a way of applying pressure to an object and, hence, a way of perceiving the hardness of the object grasped as well as other tangible qualities such as temperature, moistness, being metallic, and so on. Not only are these stereotypical exploratory activities optimized to determine a specific range of tangible qualities but they can be chained together to provide the perceiver with a more complete profile of the corporeal nature of the object under investigation.

With enclosure, Lederman and Klatzky write:
\begin{quote}
	\ldots the hand maintains simultaneous contact with as much of the envelope of the object as possible. Often one can see an effort to mold the hand more precisely to object contours. Periods of static enclosure may alternate with shifts of the object in the hand(s). \citep[346--7]{Lederman:1987fr}
\end{quote}
The quoted passage brings out several important features of grasping, understood as a mode of haptic perception. 

First, grasping a rigid, solid body involves the hand's maintaining simultaneous contact with as much of its overall surface as possible. Grasping is thus a kind of incorporation. Recall, what unites the various tactile metaphors for perception, even for non-tactile modes of perceptual awareness such as vision and audition, is that they tend to be modes of assimilation, and grasping exemplifies this pattern. It may not be as complete an incorporation as the variant, ingestion, but it remains a clear mode of assimilation nonetheless. In maintaining simultaneous contact with as much of its overall surface as possible, the hand assimilates to the contours of the object. As we shall see, that the grasping hand assimilates to the object grasped is a manifestation of the objectivity of that haptic perception. This is part of what it makes it an apt metaphor for perceptual presentation more generally.

Second, not only does the grasping hand assimilate to the overall shape and volume of the object grasped, but, as Lederman and Klatzky observe, effort is typically exerted to mold the hand more precisely to the object's contours. So grasping or enclosure involves not only the hand's configuration in maintaining simultaneous contact with the overall surface of the object, but the force of the hand's activity as well. Not only is this force exerted in achieving the end of molding the hand more perfectly to contours of the object grasped, but it is exerted as well in the end's achievement---maintaining simultaneous contact with the overall surface of the object requires continued effort to sustain. This is physiologically and phenomenologically significant. It is physiologically significant in that the activation of different sets of receptors are coordinated in haptic perception \citep[see][chapter 3, for discussion]{Fulkerson:2014ek}. Grasping or enclosure will involve not only cutaneous activation but also the distinct sets of activations involved in kinesthesis, motor control, and our sense of agency. Moreover, this is reflected in our phenomenology. We feel the force with which we grip the object as well as the object's overall shape and volume.

Third, there is tendency, in grasping or enclosure, to shift the object periodically in one's hands. What explains this? Begin with Lederman's and Klatzky's observation that there is a tendency for perceivers to exert effort to mold their hand more precisely to the contours of the object grasped. In grasping an object, the grasping hand in this way assimilates to the overall shape and volume of the object grasped. Consider grasping a solid, rigid body, such as a stone. In grasping a stone, our hominid ancestor extends their hand's activity, they tighten their grasp, until they can no more. Since the stone is solid, it resists penetration. Since it is rigid, it maintains its overall shape and volume even when in the hominid's grasp. Contrast the way the overall shape and volume of an elastic body, such a sponge, deforms as it is squeezed. With the stone in its grip, the hand of our hominid ancestor assimilates to the overall shape and volume of the stone. Of course, hands are unevenly shaped and imperfectly elastic. This means that an effort to mold one's hand to a rigid body thus disclosing its overall shape and volume will most likely be imperfectly realized. There may be some areas of the object's surface that the grasping hand does not conform to. Tactile perception is partial in something like Hilbert's \citeyearpar{Hilbert:1987jq} sense. The tendency to shift the grasped object in our hands compensates for this partial and imperfect disclosure. In shifting the object in one's hand, an area that the hand did not previously conform to may become accessible to touch. Successive grips and the manner in which the object moves in one's hands as one shifts between them may provide a better overall sense of the shape and volume of the rigid body. 

Allow me to make two further observations about this passage, though now about issues that are merely implicit.

First, grasping is an activity and so is spread over time. It has duration. Not only does our hominid ancestor tentatively reach out and grasp the boar's femur from amongst its skeletal remains---an event with duration---, but its grasp must be actively maintained over a period of time. Maintaining simultaneous contact with the overall surface of a rigid body, or some nonsignificant portion of it, is a state sustained by activity. In this regard, it is like Ryle's \citeyearpar[149]{Ryle:1949qr} example of keeping the enemy at bay. The state thus obtains for the duration of the sustaining activity. Moreover, in coming to perceive its overall shape and volume, the perceiver may shift the object in their hand. The tactile sense of an object's overall shape or volume is disclosed by such activity. And since activity has duration, it is disclosed over time. The presentation of the overall shape and volume of an object in tactile experience is itself spread over time like the activity that discloses it. One potential lesson, then, for the metaphysics of sensory presentation, is that the object of perception may be disclosed over time, that its presentation in perceptual experience may have duration.

Second, that the grasping hand assimilates to the overall shape and volume of the object grasped is potentially epistemically significant. The full case for this will have to wait, but we can begin to get a sense of why this might be so. A rigid, solid body has a certain overall shape and volume prior to being grasped. Moreover, it is sufficiently rigid and solid to maintain that overall shape and volume even when grasped. In making an effort to more precisely mold the hand to the contours of the rigid object, the hand thus takes on, to an approximate degree, the overall shape and volume of the object grasped. That is to say, the hand takes on a certain configuration determined by the hand's anatomy, the activity of the hand, and the overall shape of the object grasped. And the hand, so configured, encompasses a region of a certain volume itself determined by the hand and the volume of the object grasped. That is the point of making an effort to more precisely mold the hand to contours of the rigid object. In engaging in such haptic activity, in molding one's hand more precisely to the contours of the object, the overall shape and volume of the object had prior to being grasped, and maintained in being grasped, explains, in part, the hand's configuration in grasping the object and the force that needs to be exerted to maintain that configuration. Suppose that it is our hand's configuration in grasping and the force that needs to be exerted in maintaining that configuration that discloses the overall shape and volume of the object. If so, at least in the present instance, haptic perception is dependent, in some appropriate sense, upon proprioception, kinesthesis, our capacity for motor activity, and our sense of agency (for relevant discussion see \citealt{OShaughnessy:1989zp,OShaughnessy:1995ty,Martin:1992aa,Fulkerson:2014ek}). Since the object's overall shape and volume explains the hand's configuration and force, if the object eludes the hand's grasp, then that configuration and force would not have occurred. If the object is absent, there is nothing for the hand to assimilate to. Perhaps the objectivity of grasping, understood as a mode of haptic perception, consists in the grasping hand's assimilating to the tangible qualities of the object had prior to grasping.

Against this suggestion, it might be objected that, at least for certain graspings, it is possible for the object to be absent and yet the hand to be in a duplicate configuration. However, a felt difference would remain. Maintaining the hand's configuration in the absence of the object requires different musculature activity since the perceiver can no longer rely on pressing against the rigid body in maintaining that configuration. The different pattern of activation of receptors in muscles and joints will result in a felt difference. Compare leaning against a wall with making as if to lean against a wall. Sustaining that posture in the absence of the supporting wall can be difficult to do. Miming is an acquired skill. As Jacques Tati demonstrates in \emph{Cours du Soir}, it can be taught and learned. So in the case of duplicate configuration, where the hand takes on the configuration it would have had if it were grasping the object, while the hand's configuration has been maintained in the absence of the object, there is a felt difference in the force exerted.

That the grasping hand assimilates to the contours of the object grasped is potentially epistemically significant. It is, if not the source of that haptic perception's objectivity, then its manifestation. In grasping an object, the hand assimilates to the object's contours. If in grasping an object, the hand's configuration and force discloses the object's overall shape and volume, and that configuration and force would not have occurred in the absence of the object grasped, then our tactile experience would not be as it is when we haptically perceive if that object were in fact absent. While not yet proof against a Cartesian demon, one can begin to see the potential epistemic significance of the effort exerted in more precisely molding one's hand against the contours of the object grasped. It is the means by which certain tangible qualities of an external body are disclosed in our grasp. 

In the \emph{Theaetetus} 156 a--c, Socrates elaborates the Secret Doctrine of Protagoras by providing an account of perception as the contingent outcome of active and passive forces in conflict. Grasping as a mode of haptic perception can seem to approximate to that account. At the very least, the felt shape and volume of the object grasped is determined by conflicting forces. On the one hand, there is the force exerted in molding the hand more precisely to the contours of the rigid body. On the other hand, there are the self-maintaining forces of the rigid body itself. A solid, rigid body, such as a stone picked up by a hominid ancestor, is no mere sum of matter. It has a form or material structure determined by forces that are the categorical basis for its rigidity and solidity \citep{Johnston:2006js}. Haptic perception is the joint upshot of the force exerted by the grasping hand and the self-maintaining forces of the object grasped. There remains a crucial difference, however, from the account elaborated by Socrates. The overall shape and volume of the object and our haptic perception of them are not ``twin births'' as Protagoras maintains. The forces that determine the object's rigidity and solidity are sufficient to maintain the object's overall shape and volume within the hand's grasp. So the perceived tangible qualities of the external body inhere in that body prior to being perceived; whereas in the account attributed to Protagoras, the perceived object comes into being with the perceiver's perception of it. One might concede to Protagoras that the presentation of the object's overall shape and volume in tactile experience and the perceiver's feeling its overall shape and volume are, in fact, ``twin births''. It is at least the case that if overall shape and volume are not present in tactile experience then they are not felt, and if they are not felt, they are not present in tactile experience, at least not in that way. But not only is this consistent with perceptual realism, but, arguably, it is only intelligibly sustained against the background of a realist metaphysics. If a tangible quality's presentation in tactile experience is explained, in part, by that quality inhering in the object perceived, then the object must possess this quality prior to perception. There is a connection, then, between explanatory priority and objectivity (this, I argue, is Aristotle's view, \citealt{Kalderon:2015fr}). At least with grasping or enclosure understood as a mode of haptic perception, this perceptual realism is sustained by the force of the hand's activity in conflict with the self-maintaining forces of the object grasped.

\section{Active Wax} % (fold)
\label{sec:active_wax}

So far in our discussion of grasping or enclosure we have established at least one claim about the metaphysics of sensory presentation, that sensory presentation is of such a nature that its objects may be disclosed over time. \citet{Broad:1952kx} took this dynamical aspect of sensory presentation to be confined to haptic perception. This is, at best, an exaggeration. Since the objects of audition, sounds and their sources, are spread over time, then it is at least natural to think that their presentation in auditory experience is itself disclosed over time. Moreover, there is reason to think that the presentation in visual experience of color qualities may itself be spread over time, at least some of the time. Thus as Broackes observes: 
\begin{quote}
	in order to tell what colour an object is, we may try it out in a number of different lighting environments. It is not that we are trying to get it into one single `standard' lighting condition, at which point it will, so to speak, shine in its true colours. Rather, we are looking, in the way it handles a variety of different illuminations (all of which are more or less `normal'), for its constant capacity to modify light. \citep[215]{Broackes:1997pa}
\end{quote}
Notice perceived colors belong to a distinct ontological category than audibilia. Audibilia, sounds and their sources, may be particulars like perceived colors, but whereas perceived colors are quality instances, audibilia are events or processes. So the fact that sensory presentation is spread over time need not be a consequence of the temporal mode of being of its object. Thus our phenomenological investigation into grasping understood as a mode of haptic perception has made vivid at least one claim about the metaphysics of sensory presentation, that the presence of the object of perception may be disclosed over time in perceptual experience, that sensory presentation may have duration. Moreover, this holds not only for the sensory presentation at work in haptic perception but plausibly for the sensory presentation at work in other sensory modalities as well.

Though a small claim about the metaphysics of sensory presentation, it has significant consequences. To take but one example, consider the claim that our ordinary experience of the natural environment that partly constitutes the Manifest Image of Nature is nothing more than a Grand Illusion. When our hominid ancestor turns, and looks, and sees, they are seemingly presented with a richly detailed scene of the alien obelisk rising among the reddish rocks set against a cloudy dawn sky. And this is true of the experience of twenty-first century humanity as well. When we visually perceive something, we are seemingly presented with a richly detailed scene. However, empirical research into change and inattentional blindness has suggested to some philosophers that this aspect of our phenomenology is illusory. Our visual experience may present itself as the presentation of a richly detailed scene, but, in fact, at any given moment, we are at best visually presented with a detail of some fragment of that scene. For at least some cases, the reasoning for the Grand Illusion hypothesis may be resisted. For it seems to presuppose that experience only presents what could be present in experience at any given moment. But if perceptual experience may disclose its object over time, then the claim that visual perception presents a richly detailed scene is consistent with the claim that, at any given moment, visual perception at best presents a fragment of that scene, as long as the richly detailed scene is understood to be disclosed over time and not present at a moment. Some of the arguments, then, if not all of them, for the Grand Illusion hypothesis turn on denying this claim about the metaphysics of sensory presentation---that perception may disclose its object over time.

Our first claim about the metaphysics of sensory presentation involved a literal feature of grasping or enclosure. Grasping is a mode of haptic perception, and the presentation of its object is spread over time. That observation suffices to establish that sensory presentation may be a kind of disclosure with duration. Consider now another feature of grasping or enclosure, that the grasping hand assimilates to the rigid, solid body in its grasp. The hand's assimilating to the overall shape and volume of the object grasped is a manifestation, if not the source, of that haptic perception's objectivity. This, I suggested, is part of what makes grasping or enclosure an apt metaphor for sensory presentation more generally. It is important to get clearer about what this assimilation amounts to, and how it may be generalized, if assimilation is genuinely part of what makes grasping an apt metaphor for sensory presentation.

Grasping, understood as a mode of haptic perception, is, like the variant metaphor, ingestion, a kind of incorporation. This can suggest that the mode of assimilation is material---that it is a taking in, or incorporation, of a material body. Thus, for example, in eating an olive, the matter of the olive is taken in and presented to the organ of taste and thereby tasted (the organ of taste, understood as flavor, need not be thought to be confined to the tongue with its taste buds but arguably includes retronasal receptors as well). But while some forms of sensory perception involve material assimilation such as tasting, not all do. Vision and audition involve the material assimilation of no thing. So if the assimilation at work in grasping or enclosure is part of what makes it an apt metaphor for sensory presentation generally, it must be understood in some other way.

Perhaps, the assimilation at work in grasping or enclosure is not material but formal. In grasping or enclosure the hand assimilates to the contours of the object grasped. The interior of the hand thus approximates to the overall shape of the object, and the volume it encloses approximates to the object's volume. The shape of the interior of the hand is similar to the overall shape of the object, and the volume of the region it encloses is similar to the volume of that object. Perhaps, in this way, the hand assimilates the tangible form of the object grasped, by becoming similar to it. However, while our hand may be warmed when feeling the warmth of an object, our eyes do not become red when viewing a traditional English phone booth (though such a view has been attributed, incorrectly to my mind, to Aristotle, \citealt{Slakey:1961ss, Sorabji:1974fk,Everson:1997ep}). So it can seem that formal assimilation is no better off than material assimilation in this regard.

However, this latter problem for assimilation understood formally if not materially may be avoided by means of a small generalization. In grasping an object, where is the shape and volume that you feel? If grasping is a mode of haptic perception, then surely they are in the object that you grasp. Now, where is your haptic experience of that object? In your head? That answer seems so implausible on its face that only a philosopher could believe it. If anywhere, it seems more reasonable to suppose, at least initially, that it is closer to where the shape and volume are felt, in your handling of the object. Perhaps in trying to come to an understanding of formal assimilation at work in grasping or enclosure that may be generalized to other sensory modalities, we focussed too closely on the shape of the interior of the hand and the volume it encloses. If our haptic experience is where we handle the object grasped, perhaps the similarity obtains not only between the hand and certain tangible qualities of the object, but between the haptic experience and the tangible qualities presented in it. Haptic experience, like perceptual experience more generally has a conscious character. Perhaps, in grasping or enclosure, understood as a mode of haptic perception, the phenomenological character of haptic experience formally assimilates to the tangible qualities presented in it. And, arguably at least, this feature is generalizable to other sensory modalities as well---that in sensory perception quite generally, the phenomenological character of perceptual experience formally assimilates to the object presented in it.

Before considering whether that generalization partly grounds the aptness of grasping or enclosure as a metaphor for sensory presentation, even for non-tactile modes of perceptual awareness, let us look closer at formal assimilation at work in haptic perception. Earlier we noted that haptic perception, like perception generally, is partial. The partial character of grasping understood as a mode of haptic perception explained the tendency, observed by Lederman and Klatzky, for the perceiver to shift the object of haptic exploration periodically in their hands. Such behavior compensates for the partial and imperfect disclosure of the overall shape and volume of the object grasped. Successive grips and the manner in which the object moves in one's hands provide a better profile of the corporeal nature of the object under investigation. If the successive grips disclose different aspects of the object's overall shape and volume, then they provide something like different haptic perspectives on the object grasped. While talk of ``perspective'' derives from the case of vision, at the very least a clear analogue of that notion finds application in the haptic case. To the extent it does, then talk of ``haptic perspective'', while in a sense is visuocentric, is not perjoratively so (on visuocentrism in philosophy of perception see \citealt{OCallaghan:2007xy}). Suppose the rigid, solid body, is irregularly shaped, then it potentially feels different in successive grips. And in the case of contour following, different paths may be followed giving rise to different progressions of intensive sensation, themselves constituting different haptic perspectives on the constant contour of the object of haptic investigation. This perspective relativity bears on our understanding of the formal assimilation at work in grasping understood as a mode of haptic perception. In haptic perception, the tangible qualities of the object are presented to the perceiver's perspective on that object---the distinctive way they are handling that object in the given circumstances---and this is reflected in the conscious character of their haptic experience. So with respect to grasping or enclosure understood as a mode of haptic perception, the doctrine of formal assimilation should be understood as the claim that the phenomenological character of haptic experience formally assimilates to the tangible qualities presented to the perceiver's haptic perspective. 

Haptic experience only formally assimilates to the tangible object it presents relative to the perceiver's haptic perspective. The perspectival relativity of formal assimilation bears on another feature of the assimilation at work in grasping or enclosure. The assimilation is formal in that, not only the shape of the interior of the hand and the region it encloses is similar to the overall shape and volume of the object, but the haptic experience, its conscious qualitative character, is similar to the tangible object at least as it is presented to the perceiver's haptic perspective. However, this does not require that the similarity be exact. The perspective relativity of formal assimilation nicely brings this out. Thus an irregularly-shaped, rigid, solid body, thanks to the self-maintaining forces that constitute the categorical bases of its rigidity and solidity, maintains its overall shape and volume despite progressive handling and the successive grips with which it is held. But that same shape feels different with different grips. If the phenomenological character were wholly determined by the tangible qualities present in haptic experience, then we would be hard pressed to explain why this is so. 

In a remarkable passage, Robert Kilwardby writes:
\begin{quote}
	You will have some kind of simile for understanding this if you assume that there is a seal in front of the wax so that it touches it and that the wax has a life by which it turns itself towards the seal, and by pressing itself against it, makes itself like it.
\end{quote}
Kilwardby transform's the Peripatetic analogy by imagining life to inhere in the wax so that it is actively pressing against the seal and so taking its sensible form upon itself. The vitalist twist on the wax analogy accomplishes two things. First, in the active wax tacking upon itself the sensible form of the seal, the analogy makes intelligible how perception may be a non-material mode of assimilation, an internalization or mode of taking in. But importantly, what sense it provides to this non-material mode of assimilation is consistent with what is assimilated in this way existing and having its character independently of that perception. Indeed, it is the resistance to the wax's activity that discloses the sensible form of the object had prior to perception.

Regardless of Kilwardby's intent, the hand, the mobile and elastic instrument of haptic exploration, is the active wax in grasping or enclosure understood as a mode of haptic perception. It is the hand that is actively molding itself to the object in grasping or enclosure. And it is the hand that is thereby taking upon itself a configuration and enclosing a certain volume determined by the overall shape and volume of the object grasped. Further, I take it that it is at least part of Kikwardby's suggestion that it is the activity of the wax and the resistance it meets in pressing against the seal that discloses the shape of the seal had prior to perception. So if the hand is the active wax in grasping understood as a mode of haptic perception, then it is the force of the hand's activity and the resistance it meets in maintaining simultaneous contact with a non-significant portion of the object's overall surface that discloses the tangible qualities of the object had prior to the haptic encounter. Kilwardby's suggestion, then, narrowly confined to haptic presentation, is that the presentation of tangible qualities of objects external to the perceiver's body is due, at least in part, to the activity of the hand in grasping and the resistance it encounters. 

% section active_wax (end)

\section{A Puzzle} % (fold)
\label{sec:a_puzzle}

% To fully appreciate the epistemic significance of the force of the hand's activity in more precisely molding to the contours of the object grasped, we need an account of how it is that assimilating to an external body discloses the tangible qualities that inhere in that body prior to haptic perception. The question here is a ``how possible'' question \citep{Cassam:2007lq}. How is objective haptic perception so much as possible?

In discussing the objectivity of grasping, understood as a mode of haptic perception, we supposed that it is our hand's configuration in grasping and the force that needs to be exerted in maintaining that configuration that discloses the overall shape and volume of the object grasped. I believe this supposition to be both plausible and true, but once it is clearly stated, a puzzle immediately arises. 

Embodiment is a fundamental feature of animal existence and so a fundamental feature of the existence of primates like ourselves. An animal's awareness of its body is a mode of self-presentation. There may be more to an animal than is revealed in bodily awareness, bodily awareness nevertheless presents corporeal aspects of the animal whose awareness it is. Bodily awareness remains a mode of self-presentation even if its disclosure of the animal whose awareness it is is partial in this way. Let bodily awareness be understood broadly enough to comprise both proprioception and kinesthesis and potentially more besides. So bodily awareness affords the perceiver with, among other things, awareness of the configuration of their limbs as well as awareness of their motion. So understood, awareness of the hand's configuration in grasping and awareness of the force that needs to be exerted in maintaining that grasp are both modes of bodily awareness. And since bodily awareness is a kind of self-presentation so are awareness of the hand's configuration and awareness of the force exerted in maintaining it. 

Our puzzle now is this. How can a mode of self-presentation disclose the presence of some other thing? How can bodily awareness be leveraged into disclosing the presence of something external to the perceiver's body? What alchemy transmutes bodily sensation into tactile perception?

Our puzzle concerns whether grasping so much as could be a mode of haptic perception. Though our interest is presently restricted to grasping as a mode of haptic perception, we can, however, get a better sense of that puzzle by considering an analogous case. So consider felt temperature. Contrast two cases. In both cases you feel warm, and you feel warm to the same degree. But in the first case, you feel warm because of a fever, and in the second case, you feel warm because because of the ambient heat. In both cases, your body is warmed. They differ only in the source of the warmth, with whether the warmth of your body is internally or externally generated. And in both cases, you feel equally warm. Nevertheless, a phenomenological difference remains. In the second case, not only do you feel warm, but you feel, as well, the warmth in the ambient air. Indeed, the warmth you feel is in conformity with the warmth felt in the ambient air. What explains this phenomenological difference? How are tangible qualities felt in something external to the perceiver's body such that perceiver feels in conformity with such qualities?

The puzzle is not meant to underwrite skepticism about haptic perception or tactile perception more generally. We are taking it for granted that in grasping a stone, say, our hominid ancestor feels the overall shape and volume of that stone. We are taking it for granted that grasping is a mode of haptic perception that affords the perceiver awareness of tangible qualities that inhere in the object grasped. Our puzzle is not meant to underwrite skepticism about wether grasping is a genuine mode of haptic perception of the tangible qualities of external bodies so much as to underwrite a ``how-possible'' question \citep{Cassam:2007lq}. How is it that the configuration of the hand and the force exerted in maintaining that configuration disclose the overall shape and volume of the object grasped? How is objective haptic perception so much as possible? The puzzle, then, is at best proof of an explanatory lacuna rather than proof of the impossibility of objective haptic perception.

There is an aspect of grasping or enclosure that has so far remained implicit in our discussion of it but is crucial for refining our how-possible question in such a way as to point to an adequate solution. The perceiver, in exerting effort in more precisely molding their hand to the contours of the object grasped, encounters felt resistance to their efforts. It is because the self-maintaining forces of the body resist the hand's encroachment that the hand can assimilate to the body's contours. The forces that constitute the body's solidity ensure that the force of the grasping hand does not penetrate it. And the forces that constitute the body's rigidity ensure that it maintains its overall shape and volume despite the force of the hand's grasp. Maybe it is the hand's encounter with felt resistance that discloses the tangible qualities of an external body. The suggestion, here, is not merely that the puzzle overlooked the contribution of cutaneous activation to tactile awareness, but rather with how cutaneous activation interacts with kinesthesis and bodily awareness more generally in giving rise to the experience of an external limit to the body's activity

There is a long history connecting objectivity to felt resistance to touch. In the \emph{Sophist}, Plato recasts the Gigantomachy, the struggle for supremacy over the cosmos between the Olympian Gods and the Giants, as a metaphysical dispute. The Gods, or Friends of the Forms, insist that only imperceptible forms are most real. Against them, the Giants, the offspring of Gaia, insist that only material bodies exist:
\begin{quote}
	One party is trying to drag everything down to earth out of heaven and the unseen, literally grasping rocks and trees in their hands, for they lay hold upon every stock and stone and strenuously affirm that real existence belongs only to that which can be handled and offers resistance to the touch. (Plato, \emph{Sophist} 246a; Cornford in \citealt[990]{Hamilton:1989fk})
\end{quote}
For the Giants, felt resistance to touch has become a touchstone for reality. Only that which can be handled and offers resistance to touch is real. Even if one rejects the materialist metaphysics of the Giants, one can accept that the experience that grounds their materialist conviction is phenomenologically compelling. It would have to be to elicit such cosmic conviction. Grasping something which offers resistance to touch is a phenomenologically vivid and primitively compelling experience of what is external to us. 

The phenomenologically vivid and primitively compelling experience of felt resistance to touch will underwrite the dramatic episode involving Dr Johnson outside of a church in Harwich:
\begin{quote}
	After we came out of the church, we stood talking for some time together of Bishop Berkeley’s ingenious sophistry to prove the non-existence of matter, and that every thing in the universe is merely ideal. I observed, that though we are satisfied his doctrine is not true, it is impossible to refute it. I never shall forget the alacrity with which Johnson answered, striking his foot with mighty force against a large stone, ’till he rebounded from it, ``I refute it thus.'' This was a stout exemplification of the first truths of Pere Buffier, or the original principles of Reid and Beattie; without admitting which, we can no more argue in metaphysicks, than we can argue in mathematicks without axioms. To me it is inconceivable how Berkeley can be answered by pure reasoning \ldots \citep[\textsc{i} 471]{Boswell:1935fk}
\end{quote}
The reality of external matter was demonstrated in the resistance it offered to Dr Johnson’s foot, which rebounded despite its mighty force. It was a demonstration not in the sense of proof, since it is inconceivable how Berkeley can be answered in pure reasoning. Moreover, what was stoutly exemplified was metaphysically axiomatic, a first truth, but proof proceeds from axioms, it does not establish them. Rather Dr Johnson’s performance was a demonstration of first truths by showing or exhibiting them. (On the character of Johnson’s refutation of Berkeley see \citealt{Patey:1986uq}). Dr Johnson's demonstration, like the Giants' before him, draws its dramatic power from the phenomenologically vivid and primitively compelling experience of felt resistance to touch. And this remains true even if the dramatic power of that gesture is all but exhausted in the twentieth century clich\'{e} of the exasperated, table-pounding realist.

Campbell, in his contribution to \citet[71]{Campbell:2014aa}, argues, instead, that Dr Johnson's demonstration was essentially multimodal, depending not only upon the kicking of the stone but upon seeing it as well:
\begin{quote}
	It is important that Johnson's kicking the rock is a multimodal affair. It would not have had the same visceral impact if Johnson had rebounded off the thing while kicking it in the pitch dark. That would merely have established the presence of some force or another. (Campbell in \citealt[71]{Campbell:2014aa})
\end{quote}
There is more, however, to the experience of kicking a stone in the dark than Campbell allows. For example, despite the darkness, Dr Johnson, perhaps through the reverberation of his foot which rebound despite its mighty force, might discern that it was stone and not a log that he was kicking. The characteristic density of stone as opposed to wood might be felt in this manner. And if it is sufficiently cold, he might feel the coldness of the stone through the leather of his boot. So it is not true that all that kicking the stone in the dark presents is some force or another. It can present as well thermal and material qualities of the object kicked. Campbell underestimates the experience of kicking a stone in the dark in a further way. Not only would that experience establish the presence of some force or another, it would disclose the self-maintaining forces that constitute a rigid, solid body external to Dr Johnson's body. Earlier in \citet[26]{Campbell:2014aa}, however, and more plausibly to my mind, Campbell claimed that it was ``the obstinance of the rock, its resistance to the will'' that manifest its mind independence. But surely the obstinance of the rock, its resistance to the will, the effect of the rock's self-maintaining forces which reveals it to be mind-independent matter, was manifest in Dr Johnson's haptic encounter with it independently of being seen.

How does felt resistance to touch disclose tangible qualities inhering in external bodies prior to perception? In exerting effort to mold more precisely the hand to the contours of the object grasped, in assimilating to the object, the perceiver experiences felt resistance to touch, they experience a limit to the hand's activity. How does the limit to the hand's activity in grasping a solid, rigid body disclose its overall shape and volume? After all, not all limitations to the body's activity are due to its interaction with external bodies. There are internal limitations to the body's activity as well. We encounter an internal limitation to the body's activity due to fatigue or in an inability to touch one's toes. So not every experience of a limitation to the body's activities is due to the tangible qualities inhering in an external body prior to perception. So how is it that in grasping, or enclosure, the limitation to the hand's activity in molding more precisely to the contours of the object grasped and the consequent felt resistance to touch disclose that object's overall shape and volume? How does the experienced limitation to the hand's activity become, in haptic perception, an experience of the tangible qualities of an external body? How is it that by means of an experienced limitation to the hand's activity tangible qualities are felt in something external to the perceiver's body and felt in conformity to those qualities?

This then is the refined version of our how-possible question: How is it possible for felt resistance to the hand's activity in grasping or enclosure to disclose a rigid body's overall shape and volume? Earlier I claimed that the refinement of our question could point to an adequate solution. Indeed we have all but stated it. Though perhaps that can only be appreciated once the solution is clearly in view.

% section a_puzzle (end)

\section{Sympathy} % (fold)
\label{sec:sympathy}

When our hominid ancestor reaches out and picks up a rough-hewn stone, perhaps in preparation to skirmish with a competing group of hominids, they feel the overall shape and volume of the stone in their grasp. It is not the hand's shape, the configuration of the hand in grasping or enclosure, that they haptically perceive though they may be aware of it. It is the stone's shape that is disclosed in their grasp. They feel the overall shape and volume in the stone, and its overall shape and volume are tangible qualities of the stone that their hand is felt to conform to. I shall make a suggestion that will be the basis for an answer to our refined how-possible question. Specifically, feeling tangible qualities in something external to the perceiver's body and feeling in conformity with them can fruitfully be understood as due to the operation of sympathy. One obstacle to appreciating this concerns out present understanding of sympathy, where sympathy is a kind of emotional response to others, a kind of fellow-feeling akin to compassion or pity. The notion of sympathy that is being invoked as the principle governing haptic presentation is closer to the notion at work in Stoic physics, if more abstract and not at all reliant on on their vitalistic metaphysics. Felt resistance to touch, insofar as it is the presentation of an object external to the perceiver's body, is a sympathetic response to the force that resists the hand's activity. Recall our refined version of our how-possible question was this: How is it possible for felt resistance to the hand's activity in grasping or enclosure to disclose a rigid body's overall shape and volume? If feeling tangible qualities in something external to the perceiver's body and in conformity to them is due to the operation of sympathy then we have a basis for an answer. It is when the limit to hand's activity is a sympathetic response to as a countervailing force, as the hand's force encountering an alien force resisting it, one force in conflict with another, like it yet distinct from it, that the self maintaining forces of the body disclose that body's presence and tangible qualities to tactile awareness.

Earlier, the initial statement of the puzzle was motivated by considering the analogy of felt temperature. We contrasted two cases. In both cases you feel warm, and you feel warm to the same degree. But in the first case, you feel warm because of a fever, and in the second case, you feel warm because because of the ambient heat. There is also, importantly, a phenomenological difference between these cases. In the second case, not only do you feel warm, but you feel, as well, the warmth in the ambient air. Indeed, the warmth you feel is in conformity with the warmth felt in the ambient air. What explains the phenomenological difference is that in the second case, but not in the first, the felt warmth is a sympathetic response to the ambient heat, to the thermal properties of something external to the perceiver's body. In sympathetically responding to ambient heat, the warmth you feel becomes a way of feeling the warmth in something located outside of your body. Moreover, in sympathetically responding to ambient heat, the warmth you feel is in conformity with the warmth felt in the air.

The proposal is that presentation in haptic perception is governed by the principle of sympathy. There are two ways to understand this. The fist proceeds synthetically. That is, beginning with elements and principles understood independently of haptic perception, one constructs the notion of the presentation of tangible qualities of external bodies in tactile experience on their basis. So, for example, one might begin with bodily sensation and ``extend its reach'', so to speak, via the operation of sympathy to construct a notion of the presentation of tangible qualities of external bodies. So understood, haptic presentation would be the coordination of bodily sensations with the tangible qualities of external bodies via the operation of sympathy. The second way proceeds analytically. That is, beginning with the notion of the presentation of tangible qualities of external bodies in tactile experience, one analyses or decomposes that notion into constituent elements that must be present and principles that must be operative if the presentation of the tangible qualities of external bodies is so much as possible. 

The synthetic approach naturally, if not inexorably, motivates indirect realism about tactile perception. So consider again our toy model where we begin with bodily sensation and extend its reach through the operation of sympathy. Bodily sensation does not involve the presentation of tangible qualities of external bodies. It is, instead, a mode of self-presentation. Thanks to the operation of sympathy, in being presented with an aspect of our corporeal nature, we are mediately presented with the tangible quality of an external body. But haptic perception is not indirect in this way. When our hominid ancestor grasps a rough-hewn stone they feel its overall shape and volume in the stone. Moreover, the presentation of these tangible qualities in their haptic experience is not apparently mediated. Our hominid ancestor need not attend to their bodily sensations as a means of attending to the tangible qualities of external bodies, rather these are directly disclosed in haptic perception. Indeed, attending to the body and its activity draws attentive resources away from the object of tactile perception. Focus too much on the warmth you feel and you cease to feel the warmth in the air. It is because the tangible qualities of an external body is directly disclosed in haptic perception that grasping becomes, in the cosmology of the Giants, a touchstone for reality. Grasping could not play this rhetorical role if it were apparently mediated.

On the alternative, analytic approach, indirect realism is simply not a possibility. One begins with an irreducible unity, the presentation of the tangible qualities of external bodies in tactile experience, and then discern what intelligible structure it must display if it is so much as possible. Arguably at least, any notion of perceptual presentation necessarily involves a subject--object distinction (though see \citealt{Johnston:2007qy}). If an object is present in perceptual experience then not only is there the object of perception---what is present in that experience--but there is also a perceiver that undergoes that experience---the subject to whom the object is presented. If we allow for modes of self-presentation where the subject and object are the same entity, then the subject--object distinction arguably required by the presupposed unity is merely hyperintensional. So compare Plotinus' view, in the \emph{Fifth Ennead}, that intellection, the presentation of intelligible objects, the highest form of unity short of that displayed by the hyperontic One, requires the distinction between the act of intellect and its object. Nevertheless, the Intellect apprehends only itself insofar as it is an image of the One. So the subject--object distinction required for intelligible presentation is consistent with its being a mode of self-presentation and so hyperintensional. (For the puzzlement that results from not allowing modes of self-presentation see \citealt{Yrjonsuuri:2008aa}.)  

Notice that in proceeding analytically, the subject--object distinction is not something to overcome. Instead we are presupposing their unity in an episode or process of perceptual presentation. There is no need to bridge the gap between subject and object since we began with their unity in tactile perception and merely discern that their distinction, potentially hyperintensional, is intelligibly required. The need to bridge the gap between the subject and object constituted by their distinction only arises if their unity is not presupposed. Thus bridging the gap between subject an object by having bodily sensation be coordinated with tangible qualities of external bodies via the operation of sympathy and its attendant indirect realism only arises if their unity in perceptual presentation is not presupposed but something to be constructed from elements and principles antecedently understood. 

The analytic approach to perceptual presentation is comparable to Frege's approach towards thought, at least at certain stages of his career, on certain interpretations. Frege begins with a unity, a truth-evaluable thought, and discerns what intelligible structure it must display. Beginning with the thought, Frege analyzes or decomposes that thought into constituent elements that must be present and principles that must be operative if that thought is to be so much as truth-evaluable. The problem of the unity of the proposition simply does not arise for Frege, since he does not begin with independently understood elements and principles and tries to construct thoughts on their basis. Rather the unity of thought is explanatorily prior to the intelligible structure it must display if it is to be so much as truth-evaluable. Similarly, on the analytic approach, the unity of perceptual presentation is explanatorily prior to the intelligible structure it must display if it is so much as possible.

In grasping or enclosure the overall shape and volume of the object is directly disclosed in a perceiver's haptic encounter with it. Since I believe that perception quite generally involves an irreducible presentational element, I do not believe that the haptic presentation of the tangible qualities of external bodies could be constructed out of elements and principles understood independently of haptic perception. So I am debarred from the synthetic approach. Thus I proceed analytically. Presupposing the unity of haptic presentation, I try to determine the intelligible structure it must display if it is so much as possible. The claim that the presentation of tangible qualities of external bodies in haptic experience involves the operation of sympathy should be understood in this light. It is not the claim that one thing, the tangible qualities of external bodies, is mediately presented by another thing, the presentation of aspects of the subject's corporeal nature in bodily sensation. Rather, it is the claim that the presentation of tangible qualities of external bodies in haptic experience is an irreducible unity that is governed by the principle of sympathy. Feeling a tangible quality in an external body and in conformity with it just is the presentation of that quality in tactile experience and can be analytically explicated in terms of the operation of sympathy.

I observed earlier that our present conception of sympathy can be an obstacle to appreciating this. To overcome this limitation, as well as to introduce some claims about the operative notion of sympathy, it will be useful to consider briefly a select history. Specifically, I want to consider sympathy as a principle of action at a distance in Stoic physics, Plotinus' use of the Stoic notion in explaining vision, and Hume's use of sympathy in drawing the self--other distinction in social intercourse. 

% section sympathy (end)

% Bibligography
\bibliographystyle{plainnat}
\bibliography{Philosophy}

\end{document}